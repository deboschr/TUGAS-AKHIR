\setlength{\LTcapwidth}{\textwidth}
\renewcommand{\arraystretch}{1.4}

\begin{longtable}{
|>{\centering\arraybackslash}m{2.8cm}
|>{\centering\arraybackslash}m{2.8cm}
|>{\centering\arraybackslash}m{2.8cm}
|>{\centering\arraybackslash}m{2.8cm}
|>{\centering\arraybackslash}m{2.8cm}|}
\caption{Pengaruh Variasi \texttt{Interval} pada Hasil Pemeriksaan}
\vspace{-3mm}
\label{tab:hasil-pengujian-interval} \\
\hline
\textbf{Interval} &
\textbf{Durasi} &
\textbf{Total Tautan} &
\textbf{Tautan Halaman} &
\textbf{Tautan Rusak} \\
\hline
\endfirsthead

\multicolumn{5}{c}{Tabel~\ref{tab:hasil-pengujian-interval} dilanjutkan dari halaman sebelumnya}\\[4pt]
\hline
\textbf{Interval} &
\textbf{Durasi} &
\textbf{Total Tautan} &
\textbf{Tautan Halaman} &
\textbf{Tautan Rusak} \\
\hline
\endhead

\hline
\multicolumn{5}{|r|}{Bersambung ke halaman berikutnya} \\ \hline
\endfoot

\hline
\endlastfoot

% ====== ISI DATA DI SINI ======

0 & 14m 35s & 602 & 368 & 72 (Tabel~\ref{tab:percobaan-interval-0}) \\ \hline


500 & 12m 53s & 602 & 368 & 78 (Tabel~\ref{tab:percobaan-interval-500}) \\ \hline


1000 & 14m 27s & 602 & 368 & 76 (Tabel~\ref{tab:percobaan-interval-1000}) \\ \hline


1500 & 15m 22s & 602 & 368 & 79 (Tabel~\ref{tab:percobaan-interval-1500}) \\ \hline


2000 & 16m 51s & 602 & 368 & 78 (Tabel~\ref{tab:percobaan-interval-2000}) \\ \hline


\end{longtable}
