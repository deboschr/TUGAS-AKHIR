\setlength{\LTcapwidth}{\textwidth}
\renewcommand{\arraystretch}{1.4}

\begin{longtable}{
|>{\centering\arraybackslash}m{2.8cm}
|>{\centering\arraybackslash}m{2.8cm}
|>{\centering\arraybackslash}m{2.8cm}
|>{\centering\arraybackslash}m{2.8cm}
|>{\centering\arraybackslash}m{2.8cm}|}
\caption{Pengaruh Variasi \textit{Request Timeout} pada Hasil Pemeriksaan}
\vspace{-3mm}
\label{tab:hasil-pengujian-request-timeout} \\
\hline
\textbf{Request Timeout} &
\textbf{Durasi} &
\textbf{Total Tautan} &
\textbf{Tautan Halaman} &
\textbf{Tautan Rusak} \\
\hline
\endfirsthead

\multicolumn{5}{c}{Tabel~\ref{tab:hasil-pengujian-request-timeout} dilanjutkan dari halaman sebelumnya}\\[4pt]
\hline
\textbf{Request Timeout} &
\textbf{Durasi} &
\textbf{Total Tautan} &
\textbf{Tautan Halaman} &
\textbf{Tautan Rusak} \\
\hline
\endhead

\hline
\multicolumn{5}{|r|}{Bersambung ke halaman berikutnya} \\ \hline
\endfoot

\hline
\endlastfoot

% ====== ISI DATA DI SINI ======

5 & 14m 50s & 585 & 343 & 91 (Tabel~\ref{tab:percobaan-request-timeout-5}) \\ \hline

10 & 14m 01s & 602 & 368 & 79 (Tabel~\ref{tab:percobaan-request-timeout-10}) \\ \hline

15 & 14m 01s & 602 & 368 & 79 (Tabel~\ref{tab:percobaan-request-timeout-15}) \\ \hline

20 & 17m 16s & 602 & 368 & 76 (Tabel~\ref{tab:percobaan-request-timeout-20}) \\ \hline

25 & 17m 16s & 602 & 368 & 76 (Tabel~\ref{tab:percobaan-request-timeout-25}) \\ \hline

30 & 17m 16s & 602 & 368 & 76 (Tabel~\ref{tab:percobaan-request-timeout-30}) \\ \hline


\end{longtable}
