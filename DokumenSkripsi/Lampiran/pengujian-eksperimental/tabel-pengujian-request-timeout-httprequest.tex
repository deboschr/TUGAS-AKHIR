\setlength{\LTcapwidth}{\textwidth}
\renewcommand{\arraystretch}{1.4}

\begin{longtable}{
|>{\centering\arraybackslash}m{2.5cm}
|>{\centering\arraybackslash}m{2.5cm}
|>{\centering\arraybackslash}m{2.5cm}
|>{\centering\arraybackslash}m{2.5cm}
|>{\centering\arraybackslash}m{5cm}|}
\caption{Pengaruh Variasi \textit{Request Timeout} pada Hasil Pemeriksaan}
\vspace{-3mm}
\label{tab:hasil-pengujian-request-timeout} \\
\hline
\textbf{Request Timeout (detik)} &
\textbf{Durasi Pemeriksaan} &
\textbf{Jumlah Total Tautan} &
\textbf{Jumlah Tautan Halaman} &
\textbf{Jumlah Tautan Rusak} \\
\hline
\endfirsthead

\multicolumn{5}{c}{Tabel~\ref{tab:hasil-pengujian-request-timeout} dilanjutkan dari halaman sebelumnya}\\[4pt]
\hline
\textbf{Request Timeout (detik)} &
\textbf{Durasi Pemeriksaan} &
\textbf{Jumlah Total Tautan} &
\textbf{Jumlah Tautan Halaman} &
\textbf{Jumlah Tautan Rusak} \\
\hline
\endhead

\hline
\multicolumn{5}{|r|}{Bersambung ke halaman berikutnya} \\ \hline
\endfoot

\hline
\endlastfoot

% ====== ISI DATA DI SINI ======
5  & -- & -- & -- & -- \\ \hline
10 & -- & -- & -- & -- \\ \hline
15 & -- & -- & -- & -- \\ \hline
20 & -- & -- & -- & -- \\ \hline
25 & -- & -- & -- & -- \\ \hline

\end{longtable}
