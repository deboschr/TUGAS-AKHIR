\renewcommand{\arraystretch}{1.4}
\begin{table}[H]
\centering
\caption{Pengujian pada Jendela Notifikasi}
\label{tab:uji-notifikasi}
\begin{tabular}{|c|>{\raggedright\arraybackslash}p{6cm}|>{\raggedright\arraybackslash}p{6cm}|c|}
\hline
\textbf{Kasus} & \textbf{Skenario} & \textbf{Hasil Diharapkan} & \textbf{Hasil Uji} \\ \hline

1 &
Menampilkan notifikasi warning ketika pengguna menekan tombol Start tanpa memasukkan URL. &
Jendela notifikasi warning muncul dengan pesan yang sesuai. &
Sesuai \\ \hline

2 &
Menampilkan notifikasi error ketika proses pemeriksaan mengalami exception. &
Jendela notifikasi error muncul dan menampilkan pesan kesalahan. &
Sesuai \\ \hline

3 &
Menampilkan notifikasi success setelah proses ekspor berhasil. &
Jendela notifikasi success muncul dan mengonfirmasi bahwa ekspor berhasil. &
Sesuai \\ \hline

4 &
Menampilkan notifikasi information ketika fitur tertentu memerlukan informasi tambahan. &
Jendela notifikasi information muncul dan menampilkan pesan informasi yang relevan. &
Sesuai \\ \hline

5 &
Menutup jendela notifikasi menggunakan tombol Close. &
Jendela menutup dengan benar dan tidak memengaruhi tampilan utama aplikasi. &
Sesuai \\ \hline

6 &
Membuka dua notifikasi secara berurutan (misalnya warning lalu success). &
Setiap notifikasi muncul dengan benar sesuai urutan pemicu dan tidak saling menumpuk. &
Sesuai \\ \hline

7 &
Munculnya notifikasi warning ketika pengguna mencoba melakukan ekspor tanpa data. &
Jendela notifikasi warning muncul dan menjelaskan bahwa tidak ada data untuk diekspor. &
Sesuai \\ \hline

\end{tabular}
\end{table}
