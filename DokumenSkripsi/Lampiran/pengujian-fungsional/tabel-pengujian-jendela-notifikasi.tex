\renewcommand{\arraystretch}{1.4}
\begin{table}[H]
\centering
\caption{Pengujian pada Jendela Notifikasi}
\label{tab:pengujian-jendela-notifikasi}
\begin{tabular}{|c|>{\raggedright\arraybackslash}p{6cm}|>{\raggedright\arraybackslash}p{6cm}|c|}
\hline
\textbf{Kasus} & \textbf{Skenario} & \textbf{Hasil Diharapkan} & \textbf{Hasil Uji} \\ \hline

1 &
Menampilkan notifikasi \textit{warning} ketika pengguna menekan tombol Start tanpa memasukkan URL. &
Jendela notifikasi \textit{warning} muncul dengan pesan yang sesuai. &
Sesuai \\ \hline

2 &
Menampilkan notifikasi \textit{error} ketika proses pemeriksaan mengalami \textit{exception}. &
Jendela notifikasi \textit{error} muncul dan menampilkan pesan kesalahan. &
Sesuai \\ \hline

3 &
Menampilkan notifikasi \textit{success} setelah proses ekspor berhasil. &
Jendela notifikasi \textit{success} muncul dan mengonfirmasi bahwa ekspor berhasil. &
Sesuai \\ \hline


\end{tabular}
\end{table}
