\renewcommand{\arraystretch}{1.2}

\begin{longtable}{
|>{\centering\arraybackslash}p{1cm}
|>{\raggedright\arraybackslash}p{4.5cm} 
|>{\raggedright\arraybackslash}p{4.5cm}
|>{\raggedright\arraybackslash}p{4.5cm}|}
\caption{Hasil Pengujian Fungsional Jendela Detail Tautan}
\vspace{-3mm}
\label{tab:hasil-pengujian-fungsional-jendela-detail-tautan} \\

\hline
\multicolumn{1}{|c|}{\textbf{Kasus}} &
\multicolumn{1}{c|}{\textbf{Skenario}} &
\multicolumn{1}{c|}{\textbf{Hasil yang Diharapkan}} &
\multicolumn{1}{c|}{\textbf{Hasil Uji}} \\
\hline
\endfirsthead

\multicolumn{4}{c}{Tabel~\ref{tab:hasil-pengujian-fungsional-jendela-detail-tautan} dilanjutkan dari halaman sebelumnya}\\[4pt]

\hline
\multicolumn{1}{|c|}{\textbf{Kasus}} &
\multicolumn{1}{c|}{\textbf{Skenario}} &
\multicolumn{1}{c|}{\textbf{Hasil yang Diharapkan}} &
\multicolumn{1}{c|}{\textbf{Hasil Uji}} \\
\hline
\endhead

\hline
\multicolumn{4}{|r|}{Bersambung ke halaman berikutnya} \\ \hline
\endfoot

\hline
\endlastfoot

1 &
Menekan salah satu baris pada tabel. &
Jendela detail tautan terbuka dan menampilkan informasi lengkap sesuai baris yang di tekan. &
Saat baris di tekan, jendela detail muncul dan seluruh informasi (URL, \textit{final} URL, \textit{status code}, \textit{error}, dan sumber halaman) sesuai dengan data baris tersebut. \\ \hline

2 &
Mencoba membuka lebih dari satu jendela detail tautan. &
Aplikasi hanya menampilkan satu jendela detail tautan. &
Tidak bisa membuka jendela lain selain harus menutup jendela saat ini. \\ \hline

3 &
Menutup jendela detail tautan. &
Jendela detail tertutup sepenuhnya. &
Jendela detail menghilang dan tidak lagi tampil di layar. \\ \hline

4 &
Setelah menutup jendela detail tautan, menekan baris yang lain. &
Jendela detail terbuka kembali dengan informasi yang benar sesuai tautan yang baru di tekan. &
Jendela muncul kembali dan menampilkan informasi yang tepat untuk tautan tersebut, tidak menampilkan data dari tautan sebelumnya. \\ \hline



\end{longtable}