\renewcommand{\arraystretch}{1.4}
\begin{table}[H]
\centering
\caption{Pengujian pada Jendela Detail Tautan}
\label{tab:uji-detail-tautan}
\begin{tabular}{|c|>{\raggedright\arraybackslash}p{6cm}|>{\raggedright\arraybackslash}p{6cm}|c|}
\hline
\textbf{Kasus} & \textbf{Skenario} & \textbf{Hasil Diharapkan} & \textbf{Hasil Uji} \\ \hline

1 &
Membuka jendela detail untuk tautan yang valid. &
URL, final URL, dan jenis konten tampil lengkap; tidak ada pesan error. &
Sesuai \\ \hline

2 &
Membuka jendela detail untuk tautan yang mengalami error (misalnya timeout atau 404). &
Pesan error tampil pada kolom error; URL dan final URL tetap ditampilkan sesuai hasil pemeriksaan. &
Sesuai \\ \hline

3 &
Membuka jendela detail untuk tautan yang ditemukan di banyak halaman. &
Daftar halaman sumber menampilkan seluruh source lengkap dengan anchor text dan URL halaman. &
Sesuai \\ \hline

4 &
Membuka jendela detail untuk tautan yang hanya memiliki satu halaman sumber. &
Daftar halaman sumber menampilkan satu baris yang sesuai. &
Sesuai \\ \hline

5 &
Memilih tautan yang memiliki final URL berbeda akibat redirect. &
Final URL tampil sesuai hasil redirect, sementara URL utama tetap ditampilkan sebagai URL awal. &
Sesuai \\ \hline

6 &
Membuka jendela detail saat tidak ada pemeriksaan yang sedang berlangsung (data statis). &
Data tetap ditampilkan normal karena jendela detail hanya membaca data hasil pemeriksaan. &
Sesuai \\ \hline

7 &
Menutup dan membuka kembali jendela detail untuk tautan yang sama. &
Data yang ditampilkan konsisten dan tidak berubah. &
Sesuai \\ \hline

\end{tabular}
\end{table}
