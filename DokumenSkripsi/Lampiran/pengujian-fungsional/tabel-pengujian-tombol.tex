\renewcommand{\arraystretch}{1.7}

\begin{longtable}{
|>{\centering\arraybackslash}p{1cm}
|>{\raggedright\arraybackslash}p{4.5cm} 
|>{\raggedright\arraybackslash}p{4.5cm}
|>{\raggedright\arraybackslash}p{4.5cm}|}
\caption{Hasil Pengujian Fungsional Tombol}
\vspace{-3mm}
\label{tab:hasil-pengujian-fungsional-tombol} \\

\hline
\multicolumn{1}{|c|}{\textbf{Kasus}} &
\multicolumn{1}{c|}{\textbf{Skenario}} &
\multicolumn{1}{c|}{\textbf{Hasil yang Diharapkan}} &
\multicolumn{1}{c|}{\textbf{Hasil Uji}} \\
\hline
\endfirsthead

\multicolumn{4}{c}{Tabel~\ref{tab:hasil-pengujian-fungsional-tombol} dilanjutkan dari halaman sebelumnya}\\[4pt]

\hline
\multicolumn{1}{|c|}{\textbf{Kasus}} &
\multicolumn{1}{c|}{\textbf{Skenario}} &
\multicolumn{1}{c|}{\textbf{Hasil yang Diharapkan}} &
\multicolumn{1}{c|}{\textbf{Hasil Uji}} \\
\hline
\endhead

\hline
\multicolumn{4}{|r|}{Bersambung ke halaman berikutnya} \\ \hline
\endfoot

\hline
\endlastfoot


1 &
Menekan tombol \texttt{Start} ketika status aplikasi berada pada status \texttt{IDLE}. &
Proses crawling dimulai, data lama dibersihkan, dan status berubah menjadi \texttt{CHECKING}. &
Aplikasi merubah status menjadi \texttt{CHECKING} dan memulai proses pemeriksaan. \\ \hline

2 &
Menekan tombol \texttt{Start} ketika status aplikasi berada pada status \texttt{CHECKING}, \texttt{STOPPED} atau \texttt{COMPLETED}. &
Proses pemeriksaan dimulai ulang dan status berubah menjadi \texttt{CHECKING}. &
Aplikasi membersihkan data lama, merubah status menjadi \texttt{CHECKING} dan memulai ulang proses pemeriksaan. \\ \hline

3 &
Menekan tombol \texttt{Stop} ketika status aplikasi berada pada status \texttt{CHECKING}. &
Proses crawling dihentikan dan status berubah menjadi \texttt{STOPPED}. &
Aplikasi menghentikan proses pemeriksaan yang sedang berjalandan dan merubah status menjadi \texttt{STOPPED}. \\ \hline

4 &
Menekan tombol \texttt{Stop} ketika status aplikasi berada pada status \texttt{IDLE}, \texttt{STOPPED} atau \texttt{COMPLETED}. &
Tombol \texttt{Stop} tidak bisa ditekan. &
Tombol \texttt{Stop} tidak merespons karena dalam status \textit{disabled}. \\ \hline

5 &
Menekan tombol \texttt{Export} ketika status aplikasi berada pada status \texttt{IDLE} atau \texttt{CHECKING}. &
Tombol \texttt{Export} tidak bisa ditekan. &
Tombol \texttt{Export} tidak merespons karena dalam status \textit{disabled}. \\ \hline

6 &
Menekan tombol \texttt{Export} ketika status \texttt{STOPPED} atau \texttt{COMPLETED} dan data di tabel tersedia. &
Dialog penyimpanan berkas muncul dan proses ekspor dapat dilakukan. &
Aplikasi menampilkan dialog penyimpanan berkas, lalu menampilkan notifikasi SUCCESS saat selesai. \\ \hline

7 &
Menekan tombol \texttt{Export} ketika status \texttt{STOPPED} atau \texttt{COMPLETED} namun data di tabel tidak tersedia. &
Muncul notifikasi \texttt{WARNING} bahwa tidak ada data yang dapat diekspor. &
Aplikasi menampilkan notifikasi \texttt{WARNING} \textit{“There are no broken links to export.”}. \\ \hline

8 &
Menekan tombol \texttt{Export} lalu membatalkan dialog penyimpanan berkas &
Proses ekspor dibatalkan tanpa ada perubahan status aplikasi. &
Aplikasi tidak memproses ekspor \\ \hline



\end{longtable}
