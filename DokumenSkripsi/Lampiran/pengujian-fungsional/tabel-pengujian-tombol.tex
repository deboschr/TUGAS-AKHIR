\renewcommand{\arraystretch}{1.4}
\begin{table}[H]
\centering
\caption{Pengujian pada Tombol}
\label{tab:uji-tombol}
\begin{tabular}{|c|>{\raggedright\arraybackslash}p{6cm}|>{\raggedright\arraybackslash}p{6cm}|c|}
\hline
\textbf{Kasus} & \textbf{Skenario} & \textbf{Hasil Diharapkan} & \textbf{Hasil Uji} \\ \hline

1 &
Menekan tombol Start ketika field URL berisi URL valid. &
Proses pemeriksaan dimulai; status berubah menjadi CHECKING; tombol Stop aktif; tombol Export nonaktif. &
Sesuai \\ \hline

2 &
Menekan tombol Start ketika field URL kosong atau tidak valid. &
Window notifikasi warning terbuka; proses tidak dimulai; tombol Stop dan Export tetap nonaktif. &
Sesuai \\ \hline

3 &
Menekan tombol Stop ketika proses pemeriksaan belum berjalan (status IDLE). &
Tidak terjadi apa-apa; tidak ada status atau UI yang berubah. &
Sesuai \\ \hline

4 &
Menekan tombol Stop saat proses pemeriksaan sedang berjalan. &
Proses dihentikan; status berubah menjadi STOPPED; tombol Start aktif kembali; tombol Export tetap nonaktif. &
Sesuai \\ \hline

5 &
Menekan tombol Export ketika tidak ada data hasil pemeriksaan. &
Window notifikasi warning terbuka; tidak ada berkas yang diekspor. &
Sesuai \\ \hline

6 &
Menekan tombol Export setelah proses pemeriksaan selesai dan data telah tersedia. &
Jendela penyimpanan muncul; berkas hasil pemeriksaan berhasil diekspor sesuai format yang dipilih. &
Sesuai \\ \hline

7 &
Menekan tombol Start setelah proses sebelumnya dihentikan dengan tombol Stop. &
Proses pemeriksaan baru dimulai; status kembali menjadi CHECKING; UI reset dan tombol Export nonaktif. &
Sesuai \\ \hline

\end{tabular}
\end{table}
