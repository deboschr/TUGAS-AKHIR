\renewcommand{\arraystretch}{1.4}
\begin{table}[H]
\centering
\caption{Pengujian pada Summary}
\label{tab:pengujian-summary}
\begin{tabular}{|c|>{\raggedright\arraybackslash}p{6cm}|>{\raggedright\arraybackslash}p{6cm}|c|}
\hline
\textbf{Kasus} & \textbf{Skenario} & \textbf{Hasil Diharapkan} & \textbf{Hasil Uji} \\ \hline

1 &
Memulai proses pemeriksaan dengan URL valid. &
Status berubah menjadi \texttt{CHECKING} dan seluruh nilai Summary mulai dihitung. &
Sesuai \\ \hline

2 &
Proses pemeriksaan selesai tanpa dihentikan. &
Status berubah menjadi \texttt{COMPLETED}, seluruh nilai Summary menampilkan hasil akhir crawling. &
Sesuai \\ \hline

3 &
Menghentikan proses pemeriksaan dengan menekan tombol Stop. &
Status berubah menjadi \texttt{STOPPED}, nilai Summary tidak lagi bertambah. &
Sesuai \\ \hline

4 &
Menerapkan filter URL pada tabel hasil. &
Nilai Summary tidak berubah dan tetap menampilkan total data crawling sebenarnya. &
Sesuai \\ \hline

5 &
Menerapkan filter Status Code pada tabel hasil. &
Nilai Summary tetap, tidak dipengaruhi filter apa pun. &
Sesuai \\ \hline

6 &
Melakukan pemeriksaan ulang setelah proses sebelumnya dihentikan. &
Summary di-reset ke kondisi awal dan menampilkan nilai baru sesuai proses crawling berikutnya. &
Sesuai \\ \hline

\end{tabular}
\end{table}
