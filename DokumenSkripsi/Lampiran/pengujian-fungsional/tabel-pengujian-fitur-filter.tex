\renewcommand{\arraystretch}{1.4}

\begin{longtable}{
|>{\centering\arraybackslash}p{1cm}
|>{\raggedright\arraybackslash}p{4.5cm} 
|>{\raggedright\arraybackslash}p{4.5cm}
|>{\raggedright\arraybackslash}p{4.5cm}|}
\caption{Hasil Pengujian Funsional Fitur \textit{Filter}}
\vspace{-3mm}
\label{tab:pengujian-funsional-fitur-filter} \\

\hline
\multicolumn{1}{|c|}{\textbf{Kasus}} &
\multicolumn{1}{c|}{\textbf{Skenario}} &
\multicolumn{1}{c|}{\textbf{Hasil yang Diharapkan}} &
\multicolumn{1}{c|}{\textbf{Hasil Uji}} \\
\hline
\endfirsthead

\multicolumn{4}{c}{Tabel~\ref{tab:pengujian-funsional-fitur-filter} dilanjutkan dari halaman sebelumnya}\\[4pt]

\hline
\multicolumn{1}{|c|}{\textbf{Kasus}} &
\multicolumn{1}{c|}{\textbf{Skenario}} &
\multicolumn{1}{c|}{\textbf{Hasil yang Diharapkan}} &
\multicolumn{1}{c|}{\textbf{Hasil Uji}} \\
\hline
\endhead

\hline
\multicolumn{4}{r}{Bersambung ke halaman berikutnya} \\
\endfoot

\hline
\endlastfoot
1 &
Menerapkan \textit{filter} URL dengan kondisi \texttt{Equals} menggunakan sebuah URL tertentu (``https://godev.co/''). &
Tabel hanya menampilkan satu baris yang URL-nya sama persis dengan kata kunci \textit{filter}. &
Tabel hanya menampilkan baris yang URL-nya ``https://godev.co/''. \\ \hline

2 &
Menerapkan \textit{filter} URL dengan kondisi \texttt{Contains} menggunakan kata kunci tertentu (``informatika'') &
Tabel hanya menampilkan baris yang URL-nya mengandung kata kunci \textit{filter}. &
Tabel hanya menampilkan baris yang URL-nya mengandung teks ``informatika''. \\ \hline

3 &
Menerapkan \textit{filter} URL dengan kondisi \texttt{Starts With} menggunakan awal URL tertentu (``https://informatika''). &
Tabel hanya menampilkan baris yang URL-nya diawali dengan kunci \textit{filter}. &
Tabel hanya menampilkan baris yang URL-nya dimulai dengan ``https://informatika''. \\ \hline

4 &
Menerapkan \textit{filter} URL dengan kondisi \texttt{Ends With} menggunakan akhir URL tertentu (``.pdf''). &
Tabel hanya menampilkan baris yang URL-nya diakhiri dengan kunci \textit{filter}. &
Tabel hanya menampilkan baris yang URL-nya diakhiri dengan ``.pdf''. \\ \hline

5 &
Menerapkan \textit{filter} Kode Status dengan kondisi \texttt{Equals} menggunakan sebuah nilai tertentu (0). &
Tabel hanya menampilkan baris yang kode statusnya sama dengan kunci \textit{filter}. &
Tabel hanya menampilkan baris dengan error koneksi. \\ \hline

6 &
Menerapkan \textit{filter} Kode Status dengan kondisi \texttt{Greater Than} menggunakan sebuah nilai tertentu (404). &
Tabel hanya menampilkan baris yang kode statusnya lebih besar dari kunci \textit{filter}. &
Tabel hanya menampilkan baris dengan kode status yang lebih besar dari 404. \\ \hline

7 &
Menerapkan \textit{filter} Kode Status dengan kondisi \texttt{Less Than} menggunakan sebuah nilai tertentu (404). &
Tabel hanya menampilkan baris yang kode statusnya lebih kecil dari kunci \textit{filter}. &
Tabel hanya menampilkan baris dengan kode status yang lebih kecil dari 404. \\ \hline

8 &
Menerapkan \textit{filter} gabungan URL \texttt{Contains} (``informatika'') dan Kode Status \texttt{Equals} (404). &
Tabel menampilkan hanya baris yang memenuhi kedua kondisi sekaligus. &
Tabel hanya menampilkan baris yang URL-nya mengandung teks ``informatika'' dan \textit{error}-nya \textit{404 Not Found} \\ \hline

9 &
Menerapkan \textit{filter} ketika proses pemeriksaan masih berlangsung. &
Tabel hanya menampilkan baris sesuai \textit{filter} &
\textit{Filter} tetap bekerja secara \textit{real time } dan hasil tabel menyesuaikan setiap kali ada tautan baru masuk. \\ \hline

10 &
Menghapus penerapan \textit{filter}. &
Tabel kembali menampilkan seluruh tautan rusak. &
Tabel langsung menampilkan semua baris tautan rusak seperti sebelum \textit{filter} diterapkan. \\ \hline

\end{longtable}