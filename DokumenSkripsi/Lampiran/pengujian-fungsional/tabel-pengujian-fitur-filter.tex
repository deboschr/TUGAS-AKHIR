\renewcommand{\arraystretch}{1.4}
\begin{table}[H]
\centering
\caption{Pengujian pada Fitur Filter}
\label{tab:pengujian-fitur-filter}
\begin{tabular}{|c|>{\raggedright\arraybackslash}p{6cm}|>{\raggedright\arraybackslash}p{6cm}|c|}
\hline
\textbf{Kasus} & \textbf{Skenario} & \textbf{Hasil Diharapkan} & \textbf{Hasil Uji} \\ \hline

1 &
Menerapkan filter URL dengan kondisi ``equals'' dan nilai tertentu. &
Tabel hanya menampilkan tautan yang URL-nya persis sama dengan nilai filter. &
Sesuai \\ \hline

2 &
Menerapkan filter URL dengan kondisi ``contains'' dan kata kunci tertentu. &
Tabel hanya menampilkan tautan yang URL-nya mengandung kata kunci tersebut. &
Sesuai \\ \hline

3 &
Menerapkan filter URL dengan kondisi ``starts with'' dengan kata kunci tertentu. &
Tabel menampilkan tautan yang URL-nya diawali dengan teks sesuai filter. &
Sesuai \\ \hline

4 &
Menerapkan filter URL dengan kondisi ``ends with'' dengan kata kunci tertentu. &
Tabel menampilkan tautan yang URL-nya diakhiri dengan teks sesuai filter. &
Sesuai \\ \hline

5 &
Menerapkan filter Status Code dengan kondisi ``equals'' dengan kata kunci tertentu. &
Tabel hanya menampilkan tautan dengan status HTTP yang persis sama. &
Sesuai \\ \hline

6 &
Menerapkan filter Status Code dengan kondisi ``greater than'' dengan kata kunci tertentu. &
Tabel menampilkan tautan dengan status HTTP lebih besar dari nilai filter. &
Sesuai \\ \hline

7 &
Menerapkan filter Status Code dengan kondisi ``less than'' dengan kata kunci tertentu. &
Tabel menampilkan tautan yang status HTTP-nya lebih kecil dari nilai filter. &
Sesuai \\ \hline

8 &
Menggabungkan filter URL (contains) dan Status Code (equals). &
Tabel hanya menampilkan tautan yang memenuhi kedua filter. &
Sesuai \\ \hline

9 &
Menggabungkan filter URL (starts with) dan Status Code (greater than). &
Tabel menampilkan tautan yang URL-nya sesuai kondisi dan statusnya lebih besar dari nilai filter. &
Sesuai \\ \hline

10 &
Menerapkan filter sehingga tidak ada data yang sesuai. &
Tabel berubah menjadi kosong; pagination menyesuaikan menjadi satu halaman. &
Sesuai \\ \hline

11 &
Menghapus semua filter yang sedang aktif. &
Tabel kembali menampilkan seluruh data hasil pemeriksaan. &
Sesuai \\ \hline


\end{tabular}
\end{table}
