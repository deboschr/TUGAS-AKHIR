\renewcommand{\arraystretch}{1.4}
\begin{table}[H]
\centering
\caption{Pengujian pada Fitur Filter}
\label{tab:uji-filter}
\begin{tabular}{|c|>{\raggedright\arraybackslash}p{6cm}|>{\raggedright\arraybackslash}p{6cm}|c|}
\hline
\textbf{Kasus} & \textbf{Skenario} & \textbf{Hasil Diharapkan} & \textbf{Hasil Uji} \\ \hline

1 &
Menerapkan filter URL dengan kondisi “contains” dan kata kunci tertentu. &
Tabel hanya menampilkan tautan yang URL-nya mengandung kata kunci tersebut. &
Sesuai \\ \hline

2 &
Menerapkan filter URL dengan kondisi “starts with”. &
Tabel menampilkan tautan yang dimulai dengan teks sesuai input filter. &
Sesuai \\ \hline

3 &
Menerapkan filter berdasarkan status HTTP, misalnya hanya menampilkan tautan 404. &
Tabel hanya menampilkan tautan dengan status error sesuai pilihan. &
Sesuai \\ \hline

4 &
Menggabungkan filter URL dan filter status HTTP secara bersamaan. &
Tabel menampilkan hanya tautan yang memenuhi kedua kondisi filter. &
Sesuai \\ \hline

5 &
Menerapkan filter yang tidak sesuai dengan data sehingga hasil menjadi nol. &
Tabel menampilkan keadaan kosong; pagination menyesuaikan menjadi satu halaman. &
Sesuai \\ \hline

6 &
Menghapus filter yang sedang aktif. &
Seluruh tautan kembali ditampilkan sesuai hasil pemeriksaan lengkap. &
Sesuai \\ \hline

7 &
Mengganti kondisi filter dengan cepat (misal “contains” → “equals”). &
Tabel langsung memperbarui hasil sesuai kondisi filter baru tanpa delay atau error. &
Sesuai \\ \hline

8 &
Menerapkan filter URL saat proses pemeriksaan sudah selesai. &
Filter berjalan normal; nilai Summary tidak terpengaruh. &
Sesuai \\ \hline

\end{tabular}
\end{table}
