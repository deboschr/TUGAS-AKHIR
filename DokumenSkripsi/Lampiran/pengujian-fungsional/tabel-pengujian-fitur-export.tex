\renewcommand{\arraystretch}{1.4}
\begin{table}[H]
\centering
\caption{Pengujian pada Fitur Ekspor}
\label{tab:uji-export}
\begin{tabular}{|c|>{\raggedright\arraybackslash}p{6cm}|>{\raggedright\arraybackslash}p{6cm}|c|}
\hline
\textbf{Kasus} & \textbf{Skenario} & \textbf{Hasil Diharapkan} & \textbf{Hasil Uji} \\ \hline

1 &
Menekan tombol Export ketika tidak ada data hasil pemeriksaan. &
Window notifikasi warning terbuka dan tidak ada berkas yang diekspor. &
Sesuai \\ \hline

2 &
Menekan tombol Export setelah proses pemeriksaan selesai dengan data lengkap. &
Dialog penyimpanan muncul dan berkas Excel/JSON berhasil dibuat sesuai isi tabel. &
Sesuai \\ \hline

3 &
Menerapkan filter sehingga hanya sebagian data tampil di tabel, lalu menekan tombol Export. &
Berkas hasil ekspor hanya berisi data yang sesuai dengan filter aktif. &
Sesuai \\ \hline

4 &
Mengekspor dalam format Excel. &
Berkas Excel berhasil dibuat dan kolom-kolom terisi sesuai struktur yang ditentukan aplikasi. &
Sesuai \\ \hline

5 &
Mengekspor dalam format JSON. &
Berkas JSON berhasil dibuat dan memuat data tautan dalam format objek yang valid. &
Sesuai \\ \hline

6 &
Mengganti filter dan melakukan ekspor ulang. &
Berkas yang dihasilkan sesuai dengan tampilan tabel setelah filter baru diterapkan. &
Sesuai \\ \hline

7 &
Menjalankan proses pemeriksaan baru, lalu mengekspor hasilnya. &
Berkas ekspor berisi data hasil pemeriksaan terbaru, bukan data sebelumnya. &
Sesuai \\ \hline

\end{tabular}
\end{table}
