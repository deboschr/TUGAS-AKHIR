\renewcommand{\arraystretch}{1.4}

\begin{longtable}{
|>{\centering\arraybackslash}p{1cm}
|>{\raggedright\arraybackslash}p{4.5cm} 
|>{\raggedright\arraybackslash}p{4.5cm}
|>{\raggedright\arraybackslash}p{4.5cm}|}
\caption{Hasil Pengujian Fungsional Fitur \textit{Filter}}
\vspace{-3mm}
\label{tab:hasil-pengujian-fungsional-fitur-export} \\

\hline
\multicolumn{1}{|c|}{\textbf{Kasus}} &
\multicolumn{1}{c|}{\textbf{Skenario}} &
\multicolumn{1}{c|}{\textbf{Hasil yang Diharapkan}} &
\multicolumn{1}{c|}{\textbf{Hasil Uji}} \\
\hline
\endfirsthead

\multicolumn{4}{c}{Tabel~\ref{tab:hasil-pengujian-fungsional-fitur-export} dilanjutkan dari halaman sebelumnya}\\[4pt]

\hline
\multicolumn{1}{|c|}{\textbf{Kasus}} &
\multicolumn{1}{c|}{\textbf{Skenario}} &
\multicolumn{1}{c|}{\textbf{Hasil yang Diharapkan}} &
\multicolumn{1}{c|}{\textbf{Hasil Uji}} \\
\hline
\endhead

\hline
\multicolumn{4}{|r|}{Bersambung ke halaman berikutnya} \\ \hline
\endfoot

\hline
\endlastfoot

1 &
Menekan tombol \texttt{Export} ketika tabel kosong. &
Tidak ada berkas diekspor dan aplikasi menampilkan notifikasi bahwa tidak ada data yang dapat diekspor. &
Aplikasi menampilkan peringatan \textit{``There are no broken links to export.''} dan dialog penyimpanan tidak muncul. \\ \hline

2 &
Menekan tombol \texttt{Export} ketika tabel memiliki data. &
Dialog penyimpanan file muncul dan pengguna dapat memilih lokasi penyimpanan. &
Aplikasi menampilkan dialog penyimpanan file, dan proses ekspor berhasil menghasilkan file Excel. \\ \hline

3 &
Melakukan \textit{filter} pada tabel setelah ekspor pertama, lalu melakukan ekspor kembali. &
Ekspor berhasil dan data hasil ekspor sesuai dengan data terkini. &
File Excel kedua hanya berisi data hasil \textit{filter}. \\ \hline

4 &
Membatalkan dialog penyimpanan file (menekan tombol \texttt{Cancel}). &
Tidak ada berkas yang dibuat. &
Dialog langsung tertutup dan tidak ada berkas baru dihasilkan. \\ \hline


\end{longtable}