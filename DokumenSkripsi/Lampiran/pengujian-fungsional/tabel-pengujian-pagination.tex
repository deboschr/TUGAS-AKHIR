\renewcommand{\arraystretch}{1.4}

\begin{longtable}{
|>{\centering\arraybackslash}p{1cm}
|>{\raggedright\arraybackslash}p{4.5cm} 
|>{\raggedright\arraybackslash}p{4.5cm}
|>{\raggedright\arraybackslash}p{4.5cm}|}
\caption{Hasil Pengujian Funsional Pagination}
\vspace{-3mm}
\label{tab:pengujian-funsional-pagination} \\

\hline
\multicolumn{1}{|c|}{\textbf{Kasus}} &
\multicolumn{1}{c|}{\textbf{Skenario}} &
\multicolumn{1}{c|}{\textbf{Hasil yang Diharapkan}} &
\multicolumn{1}{c|}{\textbf{Hasil Uji}} \\
\hline
\endfirsthead

\multicolumn{4}{c}{Tabel~\ref{tab:pengujian-funsional-pagination} dilanjutkan dari halaman sebelumnya}\\[4pt]

\hline
\multicolumn{1}{|c|}{\textbf{Kasus}} &
\multicolumn{1}{c|}{\textbf{Skenario}} &
\multicolumn{1}{c|}{\textbf{Hasil yang Diharapkan}} &
\multicolumn{1}{c|}{\textbf{Hasil Uji}} \\
\hline
\endhead

\hline
\multicolumn{4}{r}{Bersambung ke halaman berikutnya} \\
\endfoot

\hline
\endlastfoot


1 &
Menekan tombol Start ketika status aplikasi berada pada status IDLE. &
Proses crawling dimulai, data lama dibersihkan, dan status berubah menjadi CHECKING. &
Aplikasi merubah status menjadi CHECKING dan memulai proses pemeriksaan. \\ \hline

2 &
Menekan tombol Start ketika status aplikasi berada pada status CHECKING, STOPPED atau COMPLETED. &
Proses pemeriksaan dimulai ulang dan status berubah menjadi CHECKING. &
Aplikasi membersihkan data lama, merubah status menjadi CHECKING dan memulai ulang proses pemeriksaan. \\ \hline

3 &
Menekan tombol Stop ketika status aplikasi berada pada status CHECKING. &
Proses crawling dihentikan dan status berubah menjadi STOPPED. &
Aplikasi menghentikan proses pemeriksaan yang sedang berjalandan dan merubah status menjadi STOPPED. \\ \hline

4 &
Menekan tombol Stop ketika status aplikasi berada pada status IDLE, STOPPED atau COMPLETED. &
Tombol Stop tidak bisa ditekan. &
Tombol Stop tidak merespons karena dalam status disabled. \\ \hline

5 &
Menekan tombol Export ketika status aplikasi berada pada status IDLE atau CHECKING. &
Tombol Export tidak bisa ditekan. &
Tombol Export tidak merespons karena dalam status disabled. \\ \hline

6 &
Menekan tombol Export ketika status STOPPED atau COMPLETED dan data di tabel tersedia. &
Dialog penyimpanan file muncul dan proses ekspor dapat dilakukan. &
Aplikasi menampilkan dialog penyimpanan file, lalu menampilkan notifikasi SUCCESS saat selesai. \\ \hline

7 &
Menekan tombol Export ketika status STOPPED atau COMPLETED namun data di tabel tidak tersedia. &
Muncul notifikasi WARNING bahwa tidak ada data yang dapat diekspor. &
Aplikasi menampilkan notifikasi WARNING \textit{“There are no broken links to export.”}. \\ \hline

8 &
Menekan tombol Export lalu membatalkan dialog penyimpanan file &
Proses ekspor dibatalkan tanpa ada perubahan status aplikasi. &
Aplikasi tidak memproses ekspor \\ \hline



\end{longtable}
