\renewcommand{\arraystretch}{1.4}
\begin{table}[H]
\centering
\caption{Pengujian pada Pagination}
\label{tab:pengujian-pagination}
\begin{tabular}{|c|>{\raggedright\arraybackslash}p{6cm}|>{\raggedright\arraybackslash}p{6cm}|c|}
\hline
\textbf{Kasus} & \textbf{Skenario} & \textbf{Hasil Diharapkan} & \textbf{Hasil Uji} \\ \hline

1 &
Menekan tombol Next ketika data lebih dari satu halaman. &
Halaman berpindah ke halaman berikutnya dan data sesuai halaman baru ditampilkan. &
Sesuai \\ \hline

2 &
Menekan tombol Prev ketika berada di halaman selain halaman pertama. &
Halaman berpindah ke halaman sebelumnya dan data sesuai halaman tersebut ditampilkan. &
Sesuai \\ \hline

3 &
Menekan tombol Prev ketika berada di halaman pertama. &
Tidak terjadi perpindahan halaman, tampilan tetap berada pada halaman pertama. &
Sesuai \\ \hline

4 &
Menekan tombol Next ketika berada di halaman terakhir. &
Tidak terjadi perpindahan halaman, tampilan tetap berada pada halaman terakhir. &
Sesuai \\ \hline

5 &
Menerapkan filter sehingga jumlah data menjadi kurang dari satu halaman. &
Kontrol pagination menyesuaikan, hanya satu halaman yang ditampilkan, tombol Next dan Prev nonaktif. &
Sesuai \\ \hline

6 &
Menghapus filter sehingga jumlah data kembali lebih dari satu halaman. &
Pagination aktif kembali, kontrol Next dan Prev berfungsi normal sesuai jumlah halaman. &
Sesuai \\ \hline

\end{tabular}
\end{table}
