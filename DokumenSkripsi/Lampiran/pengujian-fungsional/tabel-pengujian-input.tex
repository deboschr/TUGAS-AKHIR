\renewcommand{\arraystretch}{1.4}
\begin{table}[H]
\centering
\caption{Pengujian pada Input URL}
\label{tab:uji-input}
\begin{tabular}{|c|>{\raggedright\arraybackslash}p{6cm}|>{\raggedright\arraybackslash}p{6cm}|c|}
\hline
\textbf{Kasus} & \textbf{Skenario} & \textbf{Hasil Diharapkan} & \textbf{Hasil Uji} \\ \hline

1 &
Memasukkan URL dengan struktur lengkap: https://example.com/page?id=1 &
Input diterima dan proses pemeriksaan dimulai. &
Sesuai \\ \hline

2 &
Memasukkan URL tanpa skema: example.com/page &
Window notifikasi \textit{warning} terbuka. &
Sesuai \\ \hline

3 &
Memasukkan URL tanpa host: https:///page &
Window notifikasi \textit{warning} terbuka. &
Sesuai \\ \hline

4 &
Memasukkan URL dengan skema tidak valid: ftp://unpar.com &
Window notifikasi \textit{warning} terbuka. &
Sesuai \\ \hline

5 &
Memasukkan URL dengan sintaks tidak valid: http://exa mple..com &
Window notifikasi \textit{warning} terbuka. &
Sesuai \\ \hline

6 &
Tidak memasukkan URL atau hanya spasi &
Window notifikasi \textit{warning} terbuka. &
Sesuai \\ \hline

7 &
Memasukkan URL dengan dot-segment: https://unpar.com/.././index.html &
URL dinormalisasi dan proses pemeriksaan dimulai. &
Sesuai \\ \hline

\end{tabular}
\end{table}
