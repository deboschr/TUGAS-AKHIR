\renewcommand{\arraystretch}{1.4}
\begin{table}[H]
\centering
\caption{Pengujian pada Fitur Pemeriksaan}
\label{tab:pengujian-fitur-pemeriksaan}
\begin{tabular}{|c|p{6cm}|p{6cm}|c|}
\hline
\textbf{Kasus} & \textbf{Skenario} & \textbf{Hasil Diharapkan} & \textbf{Hasil Uji} \\ \hline

1 &
Menjalankan proses pemeriksaan. &
Halaman dikunjungi sesuai urutan FIFO. &
Sesuai \\ \hline

2 &
Memeriksa tautan dengan host yang sama. &
Setiap tautan dengan host sama diambil menggunakan metode GET dan berhasil diparse sebagai HTML. &
Sesuai \\ \hline

3 &
Memeriksa tautan dengan host berbeda. &
Setiap tautan dengan host berbeda diperiksa pertama kali memakai HEAD, lalu GET apabila HEAD gagal. &
Sesuai \\ \hline

4 &
Menjalankan pemeriksaan pada halaman yang memiliki banyak tautan. &
Jumlah tautan yang diperiksa tidak melebihi batas yang sudah ditentukan. &
Sesuai \\ \hline

5 &
Menjalankan pemeriksaan pada situs dengan tautan yang sama muncul berulang kali. &
Setiap URL hanya diperiksa satu kali; tidak ada hasil duplikat di tabel. &
Sesuai \\ \hline

6 &
Memeriksa banyak tautan dari host yang sama dalam waktu singkat. &
\textit{Rate limiting} per host aktif dan jeda permintaan antar tautan konsisten. &
Sesuai \\ \hline


\end{tabular}
\end{table}
