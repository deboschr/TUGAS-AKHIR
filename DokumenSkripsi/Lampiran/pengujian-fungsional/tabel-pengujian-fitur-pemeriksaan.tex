\renewcommand{\arraystretch}{1.4}
\begin{longtable}{
|>{\centering\arraybackslash}p{1cm}
|>{\raggedright\arraybackslash}p{4.5cm} 
|>{\raggedright\arraybackslash}p{4.5cm}
|>{\raggedright\arraybackslash}p{4.5cm}|}
\caption{Hasil Pengujian Fungsional Fitur Pemeriksaan}
\vspace{-3mm}
\label{tab:hasil-pengujian-fungsional-fitur-pemeriksaan} \\

\hline
\multicolumn{1}{|c|}{\textbf{Kasus}} &
\multicolumn{1}{c|}{\textbf{Skenario}} &
\multicolumn{1}{c|}{\textbf{Hasil yang Diharapkan}} &
\multicolumn{1}{c|}{\textbf{Hasil Uji}} \\
\hline
\endfirsthead

\multicolumn{4}{c}{Tabel~\ref{tab:hasil-pengujian-fungsional-fitur-pemeriksaan} dilanjutkan dari halaman sebelumnya}\\[4pt]

\hline
\multicolumn{1}{|c|}{\textbf{Kasus}} &
\multicolumn{1}{c|}{\textbf{Skenario}} &
\multicolumn{1}{c|}{\textbf{Hasil yang Diharapkan}} &
\multicolumn{1}{c|}{\textbf{Hasil Uji}} \\
\hline
\endhead

\hline
\multicolumn{4}{|r|}{Bersambung ke halaman berikutnya} \\ \hline
\endfoot

\hline
\endlastfoot

1 &
Memeriksa satu per satu tautan rusak yang ditemukan dengan menekan URL pada tabel. &
Aplikasi mengarahkan tautan untuk terbuka ke \textit{browser} dan menampilkan \textit{error} yang sama. &
Tautan terbuka ke \textit{browser} dan menunjukkan \textit{error} yang sama dengan hasil pemeriksaan aplikasi. \\ \hline

2 &
Mengekspor hasil pemeriksaan lalu memeriksa satu per satu daftar halaman sumber. &
Seluruh daftar URL halaman sumber memiliki \textit{host} yang sama dengan \textit{host} URL awal. &
Seluruh daftar URL halaman menunjukkan bahwa \textit{host}-nya ``informatika.unpar.ac.id'', sama dengan \textit{host} URL awal. \\ \hline

3 &
Membuat \textit{counter} untuk masing-masing URL yang ditemukan, lalu tampilkan jendela notifikasi jika \textit{counter} lebih dari satu. &
Aplikasi tidak membuka jendela notifikasi untuk menampilkan pesan tautan diperiksa lebih dari satu kali. &
Pemeriksaan berjalan sampai proses selesai tanpa menampilkan jendela notifikasi apapun. \\ \hline


4 &
Menambahkan log waktu pemeriksaan pada setiap permintaan HTTP. &
Tautan dengan \textit{host} yang sama menampilkan jeda waktu yang konsisten antar permintaan, sedangkan tautan dari \textit{host} yang berbeda tidak mengikuti jeda yang sama. &
Log menunjukkan bahwa permintaan ke \textit{host} ``informatika.unpar.ac.id'' muncul dengan selang waktu tetap, sementara permintaan ke \textit{host} lain dapat muncul di antara jeda tersebut. \\ \hline


\end{longtable}


% ###################################################################################
% ###################################################################################
% ###################################################################################

\renewcommand{\arraystretch}{1.5}
\begin{table}[H]
   \centering
   \caption{Ringkasan Proses Pemeriksaan}
   \vspace{3mm}
   \label{tab:ringkasan-proses-pemeriksaan-fitur-pemeriksaan}
   
   \begin{tabular}{|p{7cm}|p{5cm}|}
      \hline      
      \textbf{Status Pemeriksaan} & COMPLETED \\ \hline
      \textbf{Waktu Mulai Pemeriksaan} & 26-11-2025 06:41:36 \\ \hline
      \textbf{Waktu Selesai Pemeriksaan} & 26-11-2025 06:54:43 \\ \hline
      \textbf{Durasi Pemeriksaan} & 13~menit 6~detik \\ \hline
      \textbf{Jumlah Total Tautan} & 602 \\ \hline
      \textbf{Jumlah Tautan Halaman} & 302 \\ \hline
      \textbf{Jumlah Tautan Rusak} & 80 \\ \hline
   \end{tabular}
\end{table}

% ###################################################################################
% ###################################################################################
% ###################################################################################

\renewcommand{\arraystretch}{1.5}
\begin{longtable}{
   |>{\raggedright\arraybackslash}m{4cm}
   |>{\raggedright\arraybackslash}m{6cm}
   |>{\centering\arraybackslash}m{2cm}|}
\caption{Ringkasan Tautan Rusak}
\vspace{-3mm}
\label{tab:ringkasan-tautan-rusak-fitur-pemeriksaan} \\
\hline

{\centering\arraybackslash\textbf{Kategori}} &
{\centering\arraybackslash\textbf{Error}} &
{\centering\arraybackslash\textbf{Jumlah}} \\ \hline
\endfirsthead

\multicolumn{3}{c}{Tabel~\ref{tab:ringkasan-tautan-rusak-fitur-pemeriksaan} dilanjutkan dari halaman sebelumnya}\\[4pt]

\hline
{\centering\arraybackslash\textbf{Kategori}} &
{\centering\arraybackslash\textbf{Error}} &
{\centering\arraybackslash\textbf{Jumlah}} \\ \hline
\endhead

\hline
\multicolumn{3}{|r|}{Bersambung ke halaman berikutnya} \\
\hline
\endfoot

\hline
\endlastfoot

% ********************************************

\multirow{5}{*}{\textbf{Connection Error}}
 & ConnectException & 13 \\ \cline{2-3}
 & HttpConnectTimeoutException & 5 \\ \cline{2-3}
 & IOException & 1 \\ \cline{2-3}
 & IllegalArgumentException & 1 \\ \cline{2-3}
 & SSLHandshakeException & 5 \\ \hline

\multicolumn{2}{|c|}{{\centering\textbf{Total Connection Error}}} &
{\centering\arraybackslash\textbf{25}} \\ \hline

% ********************************************

\multirow{5}{*}{\textbf{4XX Client Error}}
 & 400 Bad Request & 2 \\ \cline{2-3}
 & 403 Forbidden & 12 \\ \cline{2-3}
 & 404 Not Found & 31 \\ \cline{2-3}
 & 410 Gone & 1 \\ \cline{2-3}
 & 429 Too Many Requests & 4 \\ \hline

\multicolumn{2}{|c|}{{\centering\textbf{Total 4XX Client Error}}} &
{\centering\arraybackslash\textbf{50}} \\ \hline

% ********************************************

\multirow{1}{*}{\textbf{5XX Server Error}}
 & 503 Service Unavailable & 1 \\ \hline

\multicolumn{2}{|c|}{{\centering\textbf{Total 5XX Server Error}}} &
{\centering\arraybackslash\textbf{1}} \\ \hline

% ********************************************

\multirow{2}{*}{\textbf{Non-Standard Error}}
 & 520 & 2 \\ \cline{2-3}
 & 999 & 2 \\ \hline

\multicolumn{2}{|c|}{{\centering\textbf{Total Non-Standard Error}}} &
{\centering\arraybackslash\textbf{4}} \\ \hline

\end{longtable}
