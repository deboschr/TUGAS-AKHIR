\renewcommand{\arraystretch}{1.4}
\begin{table}[H]
\centering
\caption{Pengujian pada Fitur Pemeriksaan}
\label{tab:uji-pemeriksaan}
\begin{tabular}{|c|>{\raggedright\arraybackslash}p{6cm}|>{\raggedright\arraybackslash}p{6cm}|c|}
\hline
\textbf{Kasus} & \textbf{Skenario} & \textbf{Hasil Diharapkan} & \textbf{Hasil Uji} \\ \hline

1 &
Memulai pemeriksaan dengan URL valid pada domain yang memiliki beberapa halaman internal. &
Crawler mengekstraksi semua tautan internal, menambahkannya ke antrian pemeriksaan, dan menampilkan hasil secara real-time. &
Sesuai \\ \hline

2 &
Memeriksa halaman yang hanya memiliki sedikit tautan. &
Proses selesai cepat; semua tautan tampil sesuai jumlah sebenarnya. &
Sesuai \\ \hline

3 &
Memeriksa halaman dengan banyak tautan eksternal. &
Tautan eksternal diperiksa menggunakan HEAD/GET; status tampil sesuai kondisi asli tautan. &
Sesuai \\ \hline

4 &
Memeriksa tautan internal yang rusak (misalnya 404). &
Tautan ditandai sebagai rusak; kolom error terisi kode atau pesan error yang benar. &
Sesuai \\ \hline

5 &
Memeriksa tautan eksternal yang timeout atau gagal diakses. &
Tautan diberi status error (misalnya timeout); hasil tampil real-time di tabel. &
Sesuai \\ \hline

6 &
Crawler menemukan tautan duplikat pada beberapa halaman. &
Tautan tidak diperiksa dua kali; hanya satu entri yang muncul pada daftar hasil. &
Sesuai \\ \hline

7 &
Menghentikan pemeriksaan ketika proses crawling masih berjalan. &
Pemeriksaan berhenti; tidak ada tautan baru yang ditambahkan; status berubah menjadi STOPPED. &
Sesuai \\ \hline

8 &
Menjalankan pemeriksaan baru setelah proses sebelumnya selesai. &
Semua hasil lama dibersihkan; hasil baru tampil real-time sesuai proses pemeriksaan berikutnya. &
Sesuai \\ \hline

\end{tabular}
\end{table}
