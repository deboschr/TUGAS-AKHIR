%_____________________________________________________________________________
%=============================================================================
% data.tex v11 (24-07-2020) dibuat oleh Lionov - Informatika FTIS UNPAR
%
% Perubahan pada versi 11 (24-07-2020)
%	- Penambahan enumitem dan nosep untuk semua list, untuk menghemat kertas
%   - Bagian V: penambahan opsi Daftar Kode Program dan Daftar Notasi
%   - Bagian XIV: menjadi Bagian XVI
%   - Bagian XV: menjadi Bagian XVII
%   - Bagian XIV yang baru: untuk pilihan jenis tanda tangan mahasiswa
%   - Bagian XV yang baru: untuk pilihan munculnya tanda tangan digital untuk
%     dosen/pejabat
%_____________________________________________________________________________
%=============================================================================

%=============================================================================
% 								PETUNJUK
%=============================================================================
% Ini adalah file data (data.tex)
% Masukkan ke dalam file ini, data-data yang diperlukan oleh template ini
% Cara memasukkan data dijelaskan di setiap bagian
% Data yang WAJIB dan HARUS diisi dengan baik dan benar adalah SELURUHNYA !!
% Hilangkan tanda << dan >> jika anda menemukannya
%=============================================================================

%_____________________________________________________________________________
%=============================================================================
% 								BAGIAN 0
%=============================================================================
% Entri untuk memperbaiki posisi "DAFTAR ISI" jika tidak berada di bagian 
% tengah halaman. Sayangnya setiap sistem menghasilkan posisi yang berbeda.
% Isilah dengan 0 atau 1 (e.g. \daftarIsiError{1}). 
% Pemilihan 0 atau 1 silahkan disesuaikan dengan hasil PDF yang dihasilkan.
%=============================================================================
\daftarIsiError{0}   
%\daftarIsiError{1}   
%=============================================================================

%_____________________________________________________________________________
%=============================================================================
% 								BAGIAN I
%=============================================================================
% Tambahkan package2 lain yang anda butuhkan di sini
%=============================================================================
\usepackage{booktabs} 
\usepackage{longtable}
\usepackage{amssymb}
\usepackage{todo}
\usepackage{verbatim} 		%multiline comment
\usepackage{pgfplots}
\usepackage{enumitem}
%\overfullrule=3mm % memperlihatkan overfull 
%=============================================================================

%_____________________________________________________________________________
%=============================================================================
% 								BAGIAN II
%=============================================================================
% Mode dokumen: menentukan halaman depan dari dokumen, apakah harus mengandung 
% prakata/pernyataan/abstrak dll (termasuk daftar gambar/tabel/isi) ?
% - final 		: hanya untuk buku skripsi, dicetak lengkap: judul ina/eng, 
%   			  pengesahan, pernyataan, abstrak ina/eng, untuk, kata 
%				  pengantar, daftar isi (daftar tabel dan gambar tetap 
%				  opsional dan dapat diatur), seluruh bab dan lampiran.
%				  Otomatis tidak ada nomor baris dan singlespacing
% - sidangakhir	: buku sidang akhir = buku final - (pengesahan + pernyataan +
%   			  untuk + kata pengantar)
%				  Otomatis ada nomor baris dan onehalfspacing 
% - sidang 		: untuk sidang 1, buku sidang = buku sidang akhir - (judul 
%				  eng + abstrak ina/eng)
%				  Otomatis ada nomor baris dan onehalfspacing
% - bimbingan	: untuk keperluan bimbingan, hanya terdapat bab dan lampiran
%   			  saja, bab dan lampiran yang hendak dicetak dapat ditentukan 
%				  sendiri (nomor baris dan spacing dapat diatur sendiri)
% Mode default adalah 'template' yang menghasilkan isian berwarna merah, 
% aktifkan salah satu mode di bawah ini :
%=============================================================================
%\mode{bimbingan} 		% untuk keperluan bimbingan
%\mode{sidang} 			% untuk sidang 1
%\mode{sidangakhir} 	% untuk sidang 2 / sidang pada Skripsi 2(IF)
%\mode{final} 			% untuk mencetak buku skripsi 
%=============================================================================
\mode{final}
% \mode{sidangakhir}

%_____________________________________________________________________________
%=============================================================================
% 								BAGIAN III
%=============================================================================
% Line numbering: penomoran setiap baris, nomor baris otomatis di-reset ke 1
% setiap berganti halaman.
% Sudah dikonfigurasi otomatis untuk mode final (tidak ada), mode sidang (ada)
% dan mode sidangakhir (ada).
% Untuk mode bimbingan, defaultnya ada (\linenumber{yes}), jika ingin 
% dihilangkan, isi dengan "no" (i.e.: \linenumber{no})
% Catatan:
% - jika nomor baris tidak kembali ke 1 di halaman berikutnya, compile kembali
%   dokumen latex anda
% - bagian ini hanya bisa diatur di mode bimbingan
%=============================================================================
%\linenumber{no} 
\linenumber{yes}
%=============================================================================

%_____________________________________________________________________________
%=============================================================================
% 								BAGIAN IV
%=============================================================================
% Linespacing: jarak antara baris 
% - single	: otomatis jika ingin mencetak buku skripsi, opsi yang 
%			     disediakan untuk bimbingan, jika pembimbing tidak keberatan 
%			     (untuk menghemat kertas)
% - onehalf	: otomatis jika ingin mencetak dokumen untuk sidang
% - double 	: jarak yang lebih lebar lagi, jika pembimbing berniat memberi 
%             catatan yg banyak di antara baris (dianjurkan untuk bimbingan)
% Catatan: bagian ini hanya bisa diatur di mode bimbingan
%=============================================================================
\linespacing{single}
%\linespacing{onehalf}
%\linespacing{double}
%=============================================================================

%_____________________________________________________________________________
%=============================================================================
% 								BAGIAN V
%=============================================================================
% Tidak semua skripsi memuat gambar, tabel, kode program, dan/atau notasi. 
% Untuk skripsi yang tidak memuat hal-hal tersebut, maka tidak diperlukan 
% Daftar Gambar, Daftar Tabel, Daftar Kode Program, dan/atau Daftar Notasi. 
% Sayangnya hal tsb sulit dilakukan secara manual karena membutuhkan 
% kedisiplinan pengguna template.  
% Jika tidak ingin menampilkan satu/lebih daftar-daftar tersebut (misalnya 
% untuk bimbingan), isi dengan "no" (e.g. \gambar{no})
%=============================================================================
\gambar{yes}
%\gambar{no}
\tabel{yes}
% \tabel{no} 
\kode{yes}
% \kode{no} 
%\notasi{yes}
\notasi{no}
%=============================================================================

%_____________________________________________________________________________
%=============================================================================
% 								BAGIAN VI
%=============================================================================
% Pada mode final, sidang dan sidangkahir, seluruh bab yang ada di folder "Bab"
% dengan nama file bab1.tex, bab2.tex s.d. bab9.tex akan dicetak terurut, 
% apapun isi dari perintah \bab.
% Pada mode bimbingan, jika ingin:
% - mencetak seluruh bab, isi dengan 'all' (i.e. \bab{all})
% - mencetak beberapa bab saja, isi dengan angka, pisahkan dengan ',' 
%   dan bab akan dicetak terurut sesuai urutan penulisan (e.g. \bab{1,3,2}). 
% Catatan: Jika ingin menambahkan bab ke-3 s.d. ke-9, tambahkan file 
% bab3.tex, bab4.tex, dst di folder "Bab". Untuk bab ke-10 dan 
% seterusnya, harus dilakukan secara manual dengan mengubah file skripsi.tex
% Catatan: bagian ini hanya bisa diatur di mode bimbingan
%=============================================================================
\bab{all}
%=============================================================================

%_____________________________________________________________________________
%=============================================================================
% 								BAGIAN VII
%=============================================================================
% Pada mode final, sidang dan sidangkhir, seluruh lampiran yang ada di folder 
% "Lampiran" dengan nama file lampA.tex, lampB.tex s.d. lampJ.tex akan dicetak 
% terurut, apapun isi dari perintah \lampiran.
% Pada mode bimbingan, jika ingin:
% - mencetak seluruh lampiran, isi dengan 'all' (i.e. \lampiran{all})
% - mencetak beberapa lampiran saja, isi dengan huruf, pisahkan dengan ',' 
%   dan lampiran akan dicetak terurut sesuai urutan (e.g. \lampiran{A,E,C}). 
% - tidak mencetak lampiran apapun, isi dengan "none" (i.e. \lampiran{none})
% Catatan: Jika ingin menambahkan lampiran ke-C s.d. ke-I, tambahkan file 
% lampC.tex, lampD.tex, dst di folder Lampiran. Untuk lampiran ke-J dan 
% seterusnya, harus dilakukan secara manual dengan mengubah file skripsi.tex
% Catatan: bagian ini hanya bisa diatur di mode bimbingan
%=============================================================================
\lampiran{all}
%=============================================================================

%_____________________________________________________________________________
%=============================================================================
% 								BAGIAN VIII
%=============================================================================
% Data diri dan skripsi/tugas akhir
% - namanpm		: Nama dan NPM anda, penggunaan huruf besar untuk nama harus 
%				  benar dan gunakan 10 digit npm UNPAR, PASTIKAN BAHWA 
%				  BENAR !!! (e.g. \namanpm{Jane Doe}{1992710001}
% - judul 		: Dalam bahasa Indonesia, perhatikan penggunaan huruf besar, 
%				  judul tidak menggunakan huruf besar seluruhnya !!! 
% - tanggal 	: isi dengan {tangga}{bulan}{tahun} dalam angka numerik, 
%				  jangan menuliskan kata (e.g. AGUSTUS) dalam isian bulan.
%			  	  Tanggal ini adalah tanggal dimana anda akan melaksanakan 
%				  sidang ujian akhir skripsi/tugas akhir
% - pembimbing	: pembimbing anda, lihat daftar dosen di file dosen.tex
%				  jika pembimbing hanya 1, kosongkan parameter kedua 
%				  (e.g. \pembimbing{\JND}{} ), \JND adalah kode dosen
% - penguji 	: para penguji anda, lihat daftar dosen di file dosen.tex
%				  (e.g. \penguji{\JHD}{\JCD} )
% !!Lihat singkatan pembimbing dan penguji anda di file dosen.tex!!
% Petunjuk: hilangkan tanda << & >>, dan isi sesuai dengan data anda
%=============================================================================
\namanpm{Ade Rimbo Spencher}{6182001060}
\tanggal{5}{1}{2026}         %isi bulan dengan angka
\pembimbing{\PAN}{}
\penguji{\RCP}{\HUH} 
%=============================================================================

%_____________________________________________________________________________
%=============================================================================
% 								BAGIAN IX
%=============================================================================
% Judul dan title : judul bhs indonesia dan inggris
% - judulINA: judul dalam bahasa indonesia
% - judulENG: title in english
% Petunjuk: 
% - hilangkan tanda << & >>, dan isi sesuai dengan data anda
% - langsung mulai setelah '{' awal, jangan mulai menulis di baris bawahnya
% - gunakan \texorpdfstring{\\}{} untuk pindah ke baris baru
% - judul TIDAK ditulis dengan menggunakan huruf besar seluruhnya !!
%=============================================================================
\judulINA{Pengembangan Aplikasi Desktop Pemeriksa Tautan Rusak pada Situs Web}
\judulENG{Development of a Desktop Application for Checking Broken Links on Websites}

% \judulINA{Perangkat Lunak Pemeriksa Tautan Rusak Situs Web}
% \judulENG{A Software Tool for Checking Broken Links on Websites}


%_____________________________________________________________________________
%=============================================================================
% 								BAGIAN X
%=============================================================================
% Abstrak dan abstract : abstrak bhs indonesia dan inggris
% - abstrakINA: abstrak bahasa indonesia
% - abstrakENG: abstract in english 
% Petunjuk: 
% - hilangkan tanda << & >>, dan isi sesuai dengan data anda
% - langsung mulai setelah '{' awal, jangan mulai menulis di baris bawahnya
%=============================================================================

\abstrakINA{
Keberadaan tautan rusak pada situs web dapat mengganggu aksesibilitas informasi, menurunkan kualitas pengalaman pengguna, serta berdampak negatif terhadap kredibilitas dan pemeliharaan situs web. Permasalahan ini umum terjadi pada berbagai jenis situs web dan sulit ditangani secara manual ketika jumlah halaman dan tautan yang dimiliki cukup besar. Oleh karena itu, penelitian ini bertujuan untuk mengembangkan sebuah aplikasi berbasis desktop yang mampu melakukan pemeriksaan tautan rusak pada situs web secara otomatis.
\\
Aplikasi yang dikembangkan menerima masukan berupa URL awal situs web, kemudian melakukan penelusuran halaman menggunakan mekanisme \textit{web crawling}. Pada setiap halaman yang dikunjungi, aplikasi melakukan ekstraksi tautan dari elemen HTML \texttt{<a>} dan memeriksa setiap tautan dengan mengirimkan permintaan HTTP untuk menentukan status aksesibilitasnya. Tautan yang teridentifikasi sebagai rusak diklasifikasikan berdasarkan jenis kesalahan yang terjadi dan ditampilkan secara langsung melalui antarmuka pengguna grafis. Selain itu, aplikasi dilengkapi dengan fitur ekspor hasil pemeriksaan ke berkas Excel untuk mempermudah membaca hasil pemeriksaan.
\\
Hasil pengujian menunjukkan bahwa aplikasi mampu melakukan pemeriksaan tautan rusak dengan akurat dan konsisten sesuai dengan skenario pengujian yang ditetapkan. Pengujian eksperimental memperlihatkan bahwa parameter waktu tunggu koneksi dan waktu tunggu respons HTTP berpengaruh terhadap hasil dan durasi pemeriksaan. Nilai waktu tunggu yang terlalu kecil menyebabkan meningkatnya kesalahan identifikasi tautan rusak, sedangkan nilai yang lebih besar menghasilkan hasil yang lebih stabil dengan konsekuensi durasi pemeriksaan yang lebih lama. Berdasarkan hasil pengujian fungsional dan eksperimental, dapat disimpulkan bahwa aplikasi yang dikembangkan telah memenuhi tujuan penelitian sebagai alat bantu pemeriksa tautan rusak pada situs web.
}

\abstrakENG{
The presence of broken links on websites can disrupt information accessibility, reduce the quality of user experience, and negatively affect website credibility and maintenance. This problem commonly occurs across various types of websites and is difficult to handle manually when a website contains a large number of pages and links. Therefore, this research aims to develop a desktop-based application capable of automatically checking broken links on websites.
\\
The developed application accepts an initial website URL as input and performs page traversal using a \textit{web crawling} mechanism. For each visited page, the application extracts hyperlinks from HTML \texttt{<a>} elements and examines each link by sending an HTTP request to determine its accessibility status. Links identified as broken are classified based on the type of error encountered and are displayed directly through a graphical user interface. In addition, the application provides an export feature that allows inspection results to be saved into an Excel file to facilitate result analysis.
\\
The testing results show that the application is able to perform broken link inspection accurately and consistently according to the defined testing scenarios. Experimental testing indicates that connection timeout and HTTP response timeout parameters have a direct impact on both the accuracy and duration of the inspection process. Timeout values that are too short increase the number of incorrectly identified broken links, whereas longer timeout values produce more stable results at the cost of longer inspection time. Based on the functional and experimental testing results, it can be concluded that the developed application fulfills the research objectives as a supporting tool for checking broken links on websites.
}


%=============================================================================

%_____________________________________________________________________________
%=============================================================================
% 								BAGIAN XI
%=============================================================================
% Kata-kata kunci dan keywords : diletakkan di bawah abstrak (ina dan eng)
% - kunciINA: kata-kata kunci dalam bahasa indonesia
% - kunciENG: keywords in english
% Petunjuk: hilangkan tanda << & >>, dan isi sesuai dengan data anda.
%=============================================================================
\kunciINA{Tautan Rusak, Tautan Halaman, \textit{Web Crawling}, Pemeriksaan Tautan, Aplikasi Desktop}
\kunciENG{Broken Links, Webpage Links, Web Crawling, Link Checking, Desktop Application}

%=============================================================================

%_____________________________________________________________________________
%=============================================================================
% 								BAGIAN XII
%=============================================================================
% Persembahan : kepada siapa anda mempersembahkan skripsi ini ...
% Petunjuk: hilangkan tanda << & >>, dan isi sesuai dengan data anda.
%=============================================================================
% \untuk{<<kepada siapa anda mempersembahkan skripsi ini\ldots?>>}
\untuk{Tugas akhir ini dipersembahkan untuk kedua orang tua tercinta.}

%=============================================================================

%_____________________________________________________________________________
%=============================================================================
% 								BAGIAN XIII
%=============================================================================
% Kata Pengantar: tempat anda menuliskan kata pengantar dan ucapan terima 
% kasih kepada yang telah membantu anda bla bla bla ....  
% Petunjuk: hilangkan tanda << & >>, dan isi sesuai dengan data anda.
%=============================================================================
% \prakata{<<Tuliskan kata pengantar dari anda di sini \ldots>>} 
\prakata{Puji syukur dipanjatkan kepada Tuhan Yang Maha Esa karena atas rahmat dan karunia beserta izinNya, penulis dapat menyelesaikan penulisan tugas akhir yang berjudul ``Pengembangan Aplikasi Desktop Pemeriksa Tautan Rusak pada Situs Web''. Penulis juga ingin mengucapkan terima kasih kepada pihak-pihak yang senantiasa memberikan dukungan serta bantuan selama pengerjaan tugas akhir ini. Penulis ingin mengucapkan terima kasih kepada:
\begin{enumerate}
    \item Orang tua yang senantiasa memberikan dukungan, doa, motivasi, dan semangat selama pengerjaan tugas akhir ini.
    \item Bapak \PAN selaku dosen pembimbing yang senantiasa memberikan bimbingan, motivasi, dan waktunya dari awal hingga selesai sehingga tugas akhir ini dapat diselesaikan dengan baik.
    \item Ibu \RCP dan \HUH selaku dosen penguji yang telah memberikan kritik dan saran yang membangun.
    \item Teman-teman dekat yang senantiasa memberikan dukungan dan semangat.
\end{enumerate}
\noindent
Penulis menyadari bahwa tugas akhir ini masih jauh dari sempurna. Oleh karena itu, penulis ingin mengucapkan permohonan maaf jika masih banyak kekurangan pada tugas akhir ini. Penulis berharap tugas akhir ini dapat bermanfaat bagi pembaca dan bagi pihak yang akan meneruskan penelitian ini.} 

%=============================================================================

%_____________________________________________________________________________
%=============================================================================
% 								BAGIAN XIV
%=============================================================================
% Jenis tandatangan di lembar pernyataan mahasiswa tentang plagiarisme.
% Ada 4 pilihan:
%   - digital   : diisi menggunakan digital signature (menggunakan pengolah
%                 pdf seperti Adobe Acrobat Reader DC).
%   - gambar    : diisi dengan gambar tandatangan mahasiswa (file tandatangan
%                 bertipe pdf/png/jpg). Dianjukan menggunakan warna biru.
%                 Letakkan gambar di folder "Gambar" dengan nama ttd.jpg/
%                 ttd.png/ttd.pdf (tergantung jenis file. Hapus file ttd.jpg
%                 yang digunakan sebagai contoh
%   - materai   : khusus bagi yang ingin mencetak buku dan menandatangani di 
%                 atas materai. Sama dengan pilihan ``digital'' dan dicetak.
%   - kosong    : tempat kosong ini bisa diisi dengan tanda tangan yang
%                 digambar langsung di atas pdf (fill&sign via acrobat, tanda
%                 tangan dapat dibuat dengan mouse atau stylus)
%=============================================================================
%\ttd{digital}
%\ttd{gambar}
%\ttd{materai}
%\ttd{kosong}
%=============================================================================
\ttd{gambar}

%_____________________________________________________________________________
%=============================================================================
% 								BAGIAN XV
%=============================================================================
% Pilihan tanda tangan digital untuk dosen/pejabat:
%   - no    : pdf TIDAK dapat ditandatangani secara digital, mengakomodasi 
%             yang akan menandatangani via ``menulis'' di file pdf
%   - yes   : pdf dapat ditandatangani secara digital
% 
% PERHATIAN: perubahan ini harus ditanyakan ke kaprodi/dosen koordinator, 
% apakah harus mengisi ``no" atau ``yes". Default = no 
% Untuk mahasiswa Informatika = yes
%=============================================================================
\ttddosen{yes}
% \ttddosen{no}
%=============================================================================

%_____________________________________________________________________________
%=============================================================================
% 								BAGIAN XVI
%=============================================================================
% Tambahkan hyphen (pemenggalan kata) yang anda butuhkan di sini 
%=============================================================================
\hyphenation{ma-te-ma-ti-ka}
\hyphenation{fi-si-ka}
\hyphenation{tek-nik}
\hyphenation{in-for-ma-ti-ka}
%=============================================================================

%_____________________________________________________________________________
%=============================================================================
% 								BAGIAN XVII
%=============================================================================
% Tambahkan perintah yang anda buat sendiri di sini 
%=============================================================================
\renewcommand{\vtemplateauthor}{lionov}
\pgfplotsset{compat=newest}
\setlist{nosep}
\usepackage{array}
\usepackage{multirow}
\usepackage{makecell}
\usepackage{longtable}
% \usepackage{caption}
\usepackage{amsmath}
\usepackage{algorithm}
\usepackage{algpseudocode}
\usepackage{setspace}

%=============================================================================

% Pewarnaan tambahan untuk kode program Java, FXML, dan CSS
\definecolor{gray}{rgb}{0.5,0.5,0.5}
\definecolor{bluekeywords}{rgb}{0.0,0.0,0.7}
\definecolor{greencomments}{rgb}{0.0,0.5,0.0}
\definecolor{redstrings}{rgb}{0.6,0,0}

% ------------------ JAVA ------------------
\lstdefinelanguage{Java}{
  morekeywords={abstract,assert,boolean,break,byte,case,catch,char,class,const,continue,
    default,do,double,else,enum,extends,false,final,finally,float,for,goto,if,implements,
    import,instanceof,int,interface,long,native,new,null,package,private,protected,public,
    return,short,static,strictfp,super,switch,synchronized,this,throw,throws,transient,true,
    try,void,volatile,while,var},
  sensitive=true,
  morecomment=[l]{//},
  morecomment=[s]{/*}{*/},
  morestring=[b]",
}

% ------------------ FXML (XML) ------------------
\lstdefinelanguage{FXML}{
  morekeywords={fx,fx:controller,xmlns,AnchorPane,GridPane,Pane,HBox,VBox,Label,Button,
    TextField,TextArea,ImageView,TableView,TableColumn},
  morestring=[b]",
  morecomment=[s]{<!--}{-->},
  commentstyle=\color{green!50!black}\itshape,
  moredelim=[s][\color{black}]{>}{<},
  sensitive=true,
}

% ------------------ JavaScript ------------------
\lstdefinelanguage{JavaScript}{
  morekeywords={
    break,case,catch,continue,debugger,default,delete,do,else,finally,for,function,
    if,in,instanceof,new,return,switch,this,throw,try,typeof,var,let,const,void,while,
    with,yield,async,await,export,import,from,class,extends,super
  },
  sensitive=true,
  morecomment=[l]{//},
  morecomment=[s]{/}{/},
  morestring=[b]",
  morestring=[b]',
  morestring=[b]`,
}

% ------------------ CSS ------------------
\lstdefinelanguage{CSS}{
  morekeywords={color,background,margin,padding,border,display,flex,align,justify,font,
    position,top,bottom,left,right,width,height,min-width,min-height,max-width,max-height,
    content,grid,transition,animation,transform,box-shadow,overflow,visibility,opacity},
  sensitive=true,
  morecomment=[s]{/*}{*/},
  morestring=[b]",
  alsoletter={:},
  moredelim=[s][\color{redstrings}]{:}{;},
}

% ------------------ Style umum ------------------
\lstset{
  backgroundcolor=\color{white},
  basicstyle=\fontfamily{fvm}\selectfont\footnotesize,
  keywordstyle=\color{bluekeywords}\bfseries,
  commentstyle=\color{greencomments}\itshape,
  stringstyle=\color{redstrings},
  numberstyle=\tiny\color{gray},
  numbers=left,
  stepnumber=1,
  numbersep=6pt,
  showstringspaces=false,
  breaklines=true,
  frame=single,
  rulecolor=\color{gray},
  captionpos=t,
  tabsize=4
}