\chapter{Landasan Teori}
\label{chap:020000-landasan-teori}
Bab ini membahas teori dan teknologi yang mendasari pengembangan perangkat lunak pemeriksa tautan rusak, mencakup protokol HTTP, URI, konsep dasar situs web, \textit{web crawling}, serta pustaka dan teknologi yang digunakan seperti Jsoup, Java HTTP Client API, dan JavaFX.

\section{HTTP}
\label{sec:020100-http}
% \textit{Hypertext Transfer Protocol} (HTTP) adalah protokol tingkat aplikasi yang menjadi dasar komunikasi pada arsitektur \textit{World Wide Web}. HTTP berjalan di atas protokol TCP dengan port standar 80. Untuk komunikasi yang membutuhkan perlindungan, digunakan dapat digunakan HTTPS, yaitu HTTP yang berjalan di atas \textit{Transport Layer Security} (TLS) pada port standar 443. Melalui HTTPS, komunikasi memperoleh jaminan kerahasiaan, autentikasi, dan integritas data.

HTTP bekerja dengan pola komunikasi berbasis \textit{client-server}, di mana sebuah \textit{client} mengirimkan \textit{request} dan \textit{origin server} menanggapinya dengan \textit{response}. Protokol ini bersifat \textit{stateless}, artinya setiap \textit{request} dapat dipahami secara terpisah tanpa ketergantungan pada interaksi sebelumnya, sehingga \textit{server} tidak diwajibkan menyimpan konteks antar \textit{request}. 

Tujuan dari sebuah \textit{request} adalah \textit{resource} yang disediakan oleh \textit{origin server} melalui URL. \textit{Origin server} adalah program yang dapat menghasilkan \textit{response} untuk sebuah \textit{resource}. HTTP tidak membatasi apa yang dimaksud dengan \textit{resource}, melainkan hanya menyediakan antarmuka untuk berinteraksi dengannya. Informasi mengenai \textit{resource} tersebut disampaikan dalam bentuk \textit{representation}, yaitu data beserta metadata yang mencerminkan keadaan \textit{resource} pada waktu tertentu dan dapat ditransmisikan melalui protokol.


\subsection{Metode HTTP}
\label{subsec:0201-metode-http}
Metode HTTP berfungsi untuk menunjukkan tujuan dari \textit{request} yang dibuat oleh \textit{client} dan hasil sukses apa yang diharapkan dari \textit{request} tersebut. Metode HTTP memiliki beberapa sifat umum, diantaranya adalah \textit{safe} dan \textit{idempotent}. Sebuah metode dikatakan \textit{safe} apabila semantik yang didefinisikan bersifat \textit{read-only}, yaitu \textit{client} tidak meminta dan tidak mengharapkan adanya perubahan keadaan pada \textit{origin server} akibat penerapan metode tersebut. Metode yang termasuk \textit{safe} adalah \texttt{GET}, \texttt{HEAD}, \texttt{OPTIONS}, dan \texttt{TRACE}. Sebuah metode disebut \textit{idempotent} apabila dampak yang dimaksudkan pada \textit{server} dari beberapa \textit{request} identik, sama dengan dampak dari satu \textit{request}. Metode \texttt{PUT}, \texttt{DELETE}, serta seluruh metode aman adalah \textit{idempotent}. 

\vspace{10mm}

Berikut ini adalah daftar beberapa metode HTTP:

\vspace{2mm}

\begin{itemize}[itemsep=5pt]
    \item \textbf{GET}: Metode ini digunakan untuk meminta transfer representasi terkini dari \textit{resource} target.
  
    \item \textbf{HEAD}: Metode ini identik dengan \texttt{GET}, tetapi \textit{server} tidak boleh mengirimkan konten dalam \textit{response}. \texttt{HEAD} digunakan untuk memperoleh metadata dari representasi yang dipilih tanpa harus mentransfer data representasi itu sendiri.
  
    \item \textbf{POST}: Metode ini digunakan untuk meminta \textit{resource} target memproses representasi yang disertakan dalam \textit{request} sesuai dengan semantik khusus yang dimiliki oleh \textit{resource} tersebut.
  
    \item \textbf{PUT}: Metode ini digunakan untuk meminta agar keadaan dari \textit{resource} target dibuat atau diganti dengan keadaan yang ditentukan oleh representasi yang disertakan dalam isi pesan \textit{request}.
  
    \item \textbf{DELETE}: Metode ini digunakan untuk meminta agar \textit{origin server} menghapus asosiasi antara \textit{resource} target dengan fungsionalitasnya saat ini.
  
    \item \textbf{OPTIONS}: Metode ini digunakan untuk meminta informasi mengenai opsi komunikasi yang tersedia bagi \textit{resource} target, baik pada \textit{origin server} maupun perantara. Metode ini memungkinkan \textit{client} mengetahui opsi dan/atau persyaratan yang terkait dengan sebuah \textit{resource}, atau kemampuan dari sebuah \textit{server}, tanpa menyiratkan adanya tindakan terhadap \textit{resource} tersebut.
  
\end{itemize}



\subsection{Kode Status HTTP}
\label{subsec:0201-kode-status-http}

Kode status HTTP adalah bagian dari baris awal pada \textit{response} \textit{server} yang menunjukkan hasil pemrosesan terhadap suatu \textit{request}. Kode ini terdiri dari tiga digit numerik dan dikelompokkan ke dalam lima kelas utama berdasarkan digit pertamanya: informasi (1xx), keberhasilan (2xx), pengalihan (3xx), kesalahan dari \textit{client} (4xx), dan kesalahan dari \textit{server} (5xx).

\subsubsection{\textit{Informational} 1xx}
\label{subsubsec:020104-infotmational-1xx}

Kode-kode pada kelas ini menunjukkan bahwa \textit{request} telah diterima dan sedang diproses, tetapi belum ada \textit{response} final. Berikut adalah daftar kode status pada kategori ini:

\begin{itemize}[itemsep=5pt]
    \item \textbf{100 (\textit{Continue})}: Menunjukkan bahwa \textit{server} telah menerima bagian awal dari \textit{request} dan tidak menemukan masalah pada tahap tersebut. \textit{Client} dapat melanjutkan pengiriman sisa \textit{request}, dan \textit{server} akan mengirimkan \textit{response} akhir setelah seluruh \textit{request} selesai diproses.
  
    \item \textbf{101 (\textit{Switching Protocols})}: Menunjukkan bahwa \textit{server} menyetujui permintaan \textit{client} untuk mengganti protokol aplikasi yang digunakan pada koneksi yang sama, sesuai dengan yang diminta dalam \textit{request}.
  
    \item \textbf{102 (\textit{Processing})}: Menunjukkan bahwa \textit{server} telah menerima \textit{request} secara lengkap, namun proses penanganannya masih berjalan dan belum dapat menghasilkan \textit{response} akhir.~\cite{RFC2518}
  
    \item \textbf{103 (\textit{Early Hints})}: Digunakan oleh \textit{server} untuk mengirimkan informasi awal kepada \textit{client} melalui \textit{header}, sebelum \textit{response} akhir tersedia, sehingga \textit{client} dapat mempersiapkan proses lanjutan lebih cepat.~\cite{RFC8297}
\end{itemize}

\subsubsection{\textit{Successful} 2xx}
\label{subsubsec:201004-successful-2xx}

Kode-kode pada kelas ini menunjukkan bahwa \textit{request} telah diterima, dipahami, dan diproses dengan sukses. Berikut adalah daftar kode status pada kategori ini:

\begin{itemize}[itemsep=5pt]
    \item \textbf{200 (\textit{OK})}: Menunjukkan bahwa \textit{request} berhasil diproses dan \textit{response} berisi hasil yang sesuai dengan jenis \textit{request} yang dikirimkan.
  
    \item \textbf{201 (\textit{Created})}: Menunjukkan bahwa \textit{request} berhasil diproses dan menghasilkan satu atau lebih \textit{resource} baru pada \textit{server}.
  
    \item \textbf{202 (\textit{Accepted})}: Menunjukkan bahwa \textit{request} telah diterima oleh \textit{server} untuk diproses, namun proses tersebut belum selesai pada saat \textit{response} dikirimkan.
  
    \item \textbf{203 (\textit{Non-Authoritative Information})}: Menunjukkan bahwa \textit{request} berhasil diproses, tetapi konten dalam \textit{response} telah mengalami perubahan oleh \textit{transforming proxy}. Akibatnya, informasi yang diterima oleh \textit{client} tidak sepenuhnya identik dengan data yang dikirimkan oleh \textit{origin server}.
  
    \item \textbf{204 (\textit{No Content})}: Menunjukkan bahwa \textit{request} berhasil diproses, tetapi \textit{server} tidak mengirimkan konten apa pun dalam \textit{response}.
  
    \item \textbf{205 (\textit{Reset Content})}: Menunjukkan bahwa \textit{server} telah berhasil memproses \textit{request} dan meminta agar \textit{user agent} mengembalikan tampilan dokumen ke kondisi awal, sebagaimana sebelum \textit{request} tersebut dikirimkan.
  
    \item \textbf{206 (\textit{Partial Content})}: Menunjukkan bahwa \textit{server} berhasil memenuhi \textit{request} dengan mengirimkan sebagian data dari \textit{resource} yang diminta. Kondisi ini biasanya terjadi ketika \textit{client} menggunakan \textit{range request} untuk mengambil bagian tertentu dari data, misalnya pada proses pengunduhan berkas berukuran besar.
  
    \item \textbf{207 (\textit{Multi-Status})}: Digunakan untuk mengirimkan informasi status dari beberapa operasi yang berbeda dalam satu \textit{request}, di mana setiap operasi dapat memiliki status yang berbeda.~\cite{RFC4918}
  
    \item \textbf{226 (IM \textit{Used})}: Menunjukkan bahwa \textit{server} berhasil memproses \textit{request} \texttt{GET}, dan representasi yang dikirimkan merupakan hasil dari satu atau lebih proses manipulasi terhadap \textit{resource} sebelum dikirimkan kepada \textit{client}.~\cite{RFC3229}
\end{itemize}

\subsubsection{\textit{Redirection} 3xx}
\label{subsubsec:020104-redirection-3xx}

Kode-kode pada kelas ini menunjukkan bahwa \textit{client} harus melakukan langkah tambahan untuk menyelesaikan \textit{request}, seperti mengikuti \textit{redirect}. Berikut adalah daftar kode status pada kategori ini:

\begin{itemize}[itemsep=8pt]
    \item \textbf{300 (\textit{Multiple Choices})}: Menunjukkan bahwa \textit{resource} yang diminta memiliki beberapa alternatif representasi atau lokasi. \textit{Server} menyediakan daftar pilihan tersebut agar \textit{user agent} atau pengguna dapat memilih representasi yang diinginkan.
    
    \item \textbf{301 (\textit{Moved Permanently})}: Menunjukkan bahwa \textit{resource} target telah dipindahkan secara permanen ke URI baru. Untuk \textit{request} selanjutnya, \textit{client} sebaiknya menggunakan URI baru tersebut.
  
    \item \textbf{302 (\textit{Found})}: Menunjukkan bahwa \textit{resource} target sementara tersedia pada URI yang berbeda. Karena sifatnya sementara, \textit{client} dianjurkan untuk tetap menggunakan URI asli pada \textit{request} berikutnya.
  
    \item \textbf{303 (\textit{See Other})}: Menunjukkan bahwa \textit{server} mengarahkan \textit{client} ke \textit{resource} lain yang tercantum dalam \textit{header} \texttt{Location}. Pengalihan ini digunakan untuk memperoleh \textit{response} alternatif terhadap \textit{request} yang dikirimkan.
  
    \item \textbf{304 (\textit{Not Modified})}: Menunjukkan bahwa \textit{resource} tidak mengalami perubahan sejak terakhir kali diakses oleh \textit{client}. Dalam kondisi ini, \textit{client} dapat menggunakan salinan data yang telah dimiliki tanpa perlu menerima ulang konten dari \textit{server}.
 
    \item \textbf{307 (\textit{Temporary Redirect})}: Menunjukkan bahwa \textit{resource} target sementara berada pada URI yang berbeda, dan \textit{client} sebaiknya mengulangi \textit{request} ke URI tersebut dengan metode yang sama.
  
    \item \textbf{308 (\textit{Permanent Redirect})}: Menunjukkan bahwa \textit{resource} target telah dipindahkan secara permanen ke URI baru, dan seluruh \textit{request} berikutnya sebaiknya diarahkan ke URI tersebut.
  
\end{itemize}


\subsubsection{\textit{Client Error} 4xx}
\label{subsubsec:020104-client-error-4xx}

Kode-kode pada kelas ini menunjukkan bahwa telah terjadi kesalahan di sisi \textit{client}. Berikut adalah daftar kode status pada kategori ini:

\begin{itemize}[itemsep=5pt]
    \item \textbf{400 (\textit{Bad Request})}: Menunjukkan bahwa \textit{request} tidak dapat diproses karena kesalahan dari sisi \textit{client}, misalnya format \textit{request} tidak valid, sintaks salah, atau parameter yang dikirim tidak sesuai.
  
    \item \textbf{401 (\textit{Unauthorized})}: Menunjukkan bahwa \textit{request} memerlukan autentikasi, tetapi \textit{client} tidak menyertakan kredensial yang sah atau belum melakukan autentikasi.
  
    \item \textbf{402 (\textit{Payment Required})}: Kode status ini disediakan untuk penggunaan di masa mendatang dan saat ini belum digunakan secara umum.
  
    \item \textbf{403 (\textit{Forbidden})}: Menunjukkan bahwa \textit{server} memahami \textit{request}, tetapi menolak untuk memberikan akses terhadap \textit{resource} yang diminta.
  
    \item \textbf{404 (\textit{Not Found})}: Menunjukkan bahwa \textit{resource} yang diminta tidak ditemukan pada \textit{server}.
  
    \item \textbf{405 (\textit{Method Not Allowed})}: Menunjukkan bahwa metode HTTP yang digunakan dikenali oleh \textit{server}, tetapi tidak diizinkan untuk \textit{resource} tersebut.
  
    \item \textbf{406 (\textit{Not Acceptable})}: Menunjukkan bahwa \textit{server} tidak dapat menyediakan representasi \textit{resource} yang sesuai dengan preferensi \textit{client}, sebagaimana ditentukan melalui \textit{header} negosiasi konten.
  
    \item \textbf{407 (\textit{Proxy Authentication Required})}: Menunjukkan bahwa \textit{client} harus melakukan autentikasi terlebih dahulu kepada \textit{proxy} sebelum \textit{request} dapat diproses.
  
    \item \textbf{408 (\textit{Request Timeout})}: Menunjukkan bahwa \textit{server} menghentikan pemrosesan karena \textit{request} tidak diterima secara lengkap dalam batas waktu yang ditentukan.
  
    \item \textbf{409 (\textit{Conflict})}: Menunjukkan bahwa \textit{request} tidak dapat diproses karena terjadi konflik dengan kondisi \textit{resource} saat ini.
  
    \item \textbf{410 (\textit{Gone})}: Menunjukkan bahwa \textit{resource} yang diminta sudah tidak tersedia lagi pada \textit{server} dan kemungkinan bersifat permanen.
  
    \item \textbf{411 (\textit{Length Required})}: Menunjukkan bahwa \textit{server} menolak \textit{request} karena tidak disertai \textit{header} \texttt{Content-Length}.
  
    \item \textbf{412 (\textit{Precondition Failed})}: Menunjukkan bahwa kondisi tertentu yang disertakan dalam \textit{request} tidak terpenuhi ketika diperiksa oleh \textit{server}.
  
    \item \textbf{413 (\textit{Content Too Large})}: Menunjukkan bahwa \textit{server} menolak memproses \textit{request} karena ukuran konten melebihi batas yang dapat diterima.
  
    \item \textbf{414 (URI \textit{Too Long})}: Menunjukkan bahwa \textit{server} menolak \textit{request} karena URI target terlalu panjang untuk diproses.
  
    \item \textbf{415 (\textit{Unsupported Media Type})}: Menunjukkan bahwa format konten pada \textit{request} tidak didukung oleh \textit{server} untuk \textit{resource} tersebut.
    
    \item \textbf{416 (\textit{Range Not Satisfiable})}: Menunjukkan bahwa \textit{server} tidak dapat memenuhi rentang data yang diminta melalui \textit{header} \texttt{Range}.
  
    \item \textbf{417 (\textit{Expectation Failed})}: Menunjukkan bahwa \textit{server} tidak dapat memenuhi ekspektasi yang dinyatakan dalam \textit{header} \texttt{Expect}.
  
    \item \textbf{421 (\textit{Misdirected Request})}: Menunjukkan bahwa \textit{request} dikirim ke \textit{server} yang tidak dapat memberikan \textit{response} yang sesuai untuk URI target.
  
    \item \textbf{422 (\textit{Unprocessable Content})}: Menunjukkan bahwa \textit{server} memahami format dan sintaks \textit{request}, tetapi tidak dapat memproses instruksi yang terkandung di dalamnya.
  
    \item \textbf{423 (\textit{Locked})}: Menunjukkan bahwa \textit{resource} yang diminta sedang dalam keadaan terkunci.~\cite{RFC4918}
  
    \item \textbf{424 (\textit{Failed Dependency})}: Menunjukkan bahwa \textit{request} tidak dapat diproses karena kegagalan pada operasi lain yang menjadi prasyarat.~\cite{RFC4918}
  
    \item \textbf{425 (\textit{Too Early})}: Menunjukkan bahwa \textit{server} menolak memproses \textit{request} karena berpotensi diproses ulang (\textit{replay}).~\cite{RFC8470}
  
    \item \textbf{426 (\textit{Upgrade Required})}: Menunjukkan bahwa \textit{server} hanya bersedia memproses \textit{request} jika \textit{client} menggunakan protokol yang berbeda.
  
    \item \textbf{428 (\textit{Precondition Required})}: Menunjukkan bahwa \textit{server} mengharuskan \textit{request} disertai kondisi tertentu sebelum dapat diproses.~\cite{RFC6585}
  
    \item \textbf{429 (\textit{Too Many Requests})}: Menunjukkan bahwa \textit{client} mengirim terlalu banyak \textit{request} dalam jangka waktu tertentu.~\cite{RFC6585}
  
    \item \textbf{431 (\textit{Request Header Fields Too Large})}: Menunjukkan bahwa \textit{server} menolak \textit{request} karena ukuran \textit{header} terlalu besar.~\cite{RFC6585}
  
    \item \textbf{451 (\textit{Unavailable For Legal Reasons})}: Menunjukkan bahwa akses terhadap \textit{resource} dibatasi karena alasan hukum.~\cite{RFC7725}
\end{itemize}


\subsubsection{\textit{Server Error} 5xx}
\label{subsubsec:020104-server-error-5xx}

Kode-kode pada kelas ini menunjukkan bahwa \textit{server} menyadari bahwa terjadi kesalahan di sisi \textit{server}. Berikut adalah daftar kode status pada kategori ini:

\vspace{3mm}

\begin{itemize}[itemsep=5pt]
    \item \textbf{500 (\textit{Internal Server Error})}: Menunjukkan bahwa \textit{server} mengalami kesalahan internal yang tidak terduga, sehingga tidak dapat memproses \textit{request} yang diterima.
  
    \item \textbf{501 (\textit{Not Implemented})}: Menunjukkan bahwa \textit{server} tidak mendukung fungsionalitas yang diperlukan untuk memproses \textit{request}. Kode ini biasanya muncul ketika \textit{server} tidak mengenali atau tidak mendukung metode HTTP yang digunakan.
  
    \item \textbf{502 (\textit{Bad Gateway})}: Menunjukkan bahwa \textit{server}, ketika bertindak sebagai \textit{gateway} atau \textit{proxy}, menerima \textit{response} yang tidak valid dari \textit{server} lain yang dihubunginya.
  
    \item \textbf{503 (\textit{Service Unavailable})}: Menunjukkan bahwa \textit{server} sementara waktu tidak dapat menangani \textit{request}. Kondisi ini biasanya terjadi karena \textit{server} sedang kelebihan beban atau dalam proses pemeliharaan. \textit{Server} dapat menyertakan \textit{header} \texttt{Retry-After} untuk memberi tahu kapan \textit{client} dapat mencoba kembali.
  
    \item \textbf{504 (\textit{Gateway Timeout})}: Menunjukkan bahwa \textit{server}, saat bertindak sebagai \textit{gateway} atau \textit{proxy}, tidak menerima \textit{response} tepat waktu dari \textit{server} lain yang diperlukan untuk menyelesaikan \textit{request}.
  
    \item \textbf{505 (\textit{HTTP Version Not Supported})}: Menunjukkan bahwa \textit{server} tidak mendukung versi HTTP yang digunakan dalam \textit{request}.
  
    \item \textbf{506 (\textit{Variant Also Negotiates})}: Menunjukkan adanya kesalahan konfigurasi pada \textit{server}. Kondisi ini terjadi ketika mekanisme negosiasi konten menghasilkan siklus, sehingga \textit{server} tidak dapat menentukan representasi akhir yang akan dikirimkan.~\cite{RFC2295}
  
    \item \textbf{507 (\textit{Insufficient Storage})}: Menunjukkan bahwa \textit{server} tidak memiliki ruang penyimpanan yang cukup untuk menyelesaikan pemrosesan \textit{request} dengan sukses.~\cite{RFC4918}
  
    \item \textbf{508 (\textit{Loop Detected})}: Menunjukkan bahwa \textit{server} menghentikan pemrosesan karena terdeteksi adanya perulangan tak berhingga (\textit{infinite loop}) saat menangani \textit{request}.~\cite{RFC5842}
  
    \item \textbf{511 (\textit{Network Authentication Required})}: Menunjukkan bahwa \textit{client} harus melakukan autentikasi terlebih dahulu untuk memperoleh akses ke jaringan sebelum \textit{request} dapat diproses.~\cite{RFC6585}
\end{itemize}



\section{URI}
\label{sec:020200-uri}
% \textit{Uniform Resource Identifier} (URI) adalah rangkaian karakter yang digunakan untuk mengidentifikasi suatu sumber daya, baik berupa entitas fisik maupun abstrak. URI menyediakan cara standar untuk merepresentasikan identitas sumber daya melalui sintaksis yang terstruktur. Terdapat dua bentuk utama URI, yaitu \textit{Uniform Resource Locator} (URL) dan \textit{Uniform Resource Name} (URN).  URL adalah URI yang menyertakan informasi lokasi dan mekanisme akses terhadap sumber daya, sedangkan URN adalah URI yang berfungsi sebagai nama tetap suatu sumber daya tanpa menyertakan lokasi penyimpanannya.


\subsection{Struktur URL}
\label{subsec:0202-struktur-url}
URL khusus untuk skema \texttt{http} atau \texttt{https} memiliki strukturnya tersendiri. Pada struktur URL ini terdapat tujuh komponen dan komponen yang ditandai dengan karakter kurung siku artinya komponen tersebut bersifat opsional. Berikut adalah struktur dari URL tersebut:

\vspace{2mm}

\begin{center}
\texttt{scheme ":" "//" [userinfo "@"] host [":" port] path-abempty ["?" query] ["\#" fragment]}
\end{center}

\vspace{2mm}

Komponen-komponen dalam URL tersebut dijelaskan sebagai berikut:
\begin{itemize}[itemsep=1mm]
  \item \textbf{\textit{Scheme}}\\
  Komponen ini bersifat wajib serta diakhiri dengan tanda titik dua (\texttt{:}) dan dua garis miring (\texttt{//}). \textit{Scheme} menunjukkan protokol yang digunakan untuk mengakses sumber daya. Pada URL khusus untuk protokol HTTP, nilai yang digunakan adalah \texttt{http} atau \texttt{https} yang didefinisikan dalam format penulisan \textit{lowercase}.

  \item \textbf{\textit{User Info}}\\
  Komponen ini bersifat opsional dan diakhiri dengan tanda at (\texttt{@}). \textit{User info} digunakan untuk menyertakan informasi identitas pengguna. Komponen ini jarang digunakan karena pertimbangan keamanan.

  \item \textbf{\textit{Host}}\\
  Komponen ini bersifat wajib dan menunjukkan lokasi \textit{server} tempat sumber daya berada. \textit{Host} dapat berupa nama domain dalam format penulisan \textit{lowercase}, alamat IPv4, atau alamat IPv6, dan digunakan oleh \textit{client} untuk menentukan tujuan koneksi.

  \item \textbf{\textit{Port}}\\
  Komponen ini bersifat opsional dan diawali dengan tanda titik dua (\texttt{:}). \textit{Port} menunjukkan nomor port pada \textit{host} yang digunakan untuk komunikasi. Jika tidak dituliskan, maka digunakan port standar sesuai dengan \textit{scheme}, yaitu port 80 untuk HTTP dan port 443 untuk HTTPS.

  \item \textbf{\textit{Path-abempty}}\\
  Komponen ini bersifat wajib pada URL HTTP dan diawali dengan tanda garis miring (\texttt{/}) atau dapat berupa \textit{string} kosong. \textit{Path-abempty} menunjukkan jalur menuju sumber daya pada \textit{server}. Jika tidak dituliskan, maka \textit{path} dianggap sebagai \textit{string} kosong.

  \item \textbf{\textit{Query}}\\
  Komponen ini bersifat opsional dan diawali dengan tanda tanya (\texttt{?}). \textit{Query} digunakan untuk menyertakan parameter tambahan dalam bentuk pasangan kunci dan nilai yang dipisahkan oleh karakter tertentu, selain itu \textit{query} juga digunakan untuk menyampaikan data ke \textit{server} tanpa mengubah struktur \textit{path}.

  \item \textbf{\textit{Fragment}}\\
  Komponen ini bersifat opsional dan diawali dengan tanda pagar (\texttt{\#}). \textit{Fragment} digunakan untuk merujuk ke bagian tertentu dari sumber daya. Informasi ini tidak dikirim ke \textit{server}, melainkan diproses oleh \textit{user agent} seperti \textit{browser}.
\end{itemize}


\subsection{Kategori Karakter dan \textit{Percent-Encoding}}
\label{subsec:0202-karakter-percent-encoding}
Kategori karakter yang dapat digunakan dalam URI dikelompokkan berdasarkan tingkat kebolehan penggunaannya serta aturan pengkodean yang menyertainya. Kategori karakter tersebut dijelaskan sebagai berikut:
\begin{itemize}[itemsep=1mm]
  \item \textbf{\textit{Unreserved characters}}\\
  Karakter ini dapat digunakan langsung tanpa pengkodean tambahan. Termasuk di dalamnya huruf alfabet A sampai Z dan a sampai z, digit angka 0 sampai 9, serta simbol \texttt{-}, \texttt{\_}, \texttt{.}, dan \texttt{\textasciitilde}.

  \item \textbf{\textit{Reserved characters}}\\
  Karakter ini memiliki fungsi khusus yang bergantung pada konteks penggunaannya dalam URI. Oleh karena itu, penggunaannya harus memperhatikan aturan struktur URI yang berlaku. \textit{Reserved characters} dibagi menjadi dua kelompok, yaitu \textit{general delimiters} dan \textit{subcomponent delimiters}. \textit{General delimiters} meliputi \texttt{:}, \texttt{/}, \texttt{?}, \texttt{\#}, \texttt{[}, \texttt{]}, dan \texttt{@}, yang digunakan untuk memisahkan komponen utama URI. Sementara itu, \textit{subcomponent delimiters} meliputi \texttt{!}, \texttt{\$}, \texttt{\&}, \texttt{'}, \texttt{(}, \texttt{)}, \texttt{*}, \texttt{+}, \texttt{,}, \texttt{;}, dan \texttt{=}, yang digunakan di dalam subkomponen URI. Apabila karakter-karakter ini digunakan di luar konteks yang diperbolehkan, maka karakter tersebut harus dikodekan agar tidak disalahartikan.
  
  \item \textbf{Karakter lain}\\
  Karakter non-ASCII, spasi, dan simbol lain yang tidak termasuk dalam kategori \textit{Unreserved characters} dan \textit{Reserved characters} tidak dapat digunakan secara langsung. Karakter ini harus dikodekan sebelum dapat dimasukkan ke dalam URI.
\end{itemize}

\vspace{2mm}

Pengkodean karakter dilakukan dengan mekanisme \textit{percent-encoding}. Mekanisme ini bekerja dengan menggantikan karakter tertentu dengan representasi nilai heksadesimalnya dalam format \texttt{\%HH}, di mana \texttt{HH} merupakan nilai ASCII dari karakter yang bersangkutan. Sebagai contoh, karakter spasi direpresentasikan sebagai \texttt{\%20}, tanda kutip ganda sebagai \texttt{\%22}, dan tanda pagar sebagai \texttt{\%23}. RFC~3986 juga menegaskan bahwa karakter \textit{unreserved} yang tidak memerlukan pengkodean sebaiknya tetap dituliskan dalam bentuk aslinya untuk menjaga keterbacaan URI. Selain itu, digit heksadesimal pada \textit{percent-encoding} harus ditulis menggunakan huruf besar guna menjaga konsistensi penulisan.


\subsection{Referensi Absolut dan Relatif}
\label{subsec:0202-referensi-url}
URL dapat dinyatakan dalam bentuk referensi absolut maupun relatif, bergantung pada kelengkapan komponen yang dicantumkan. URL absolut memuat seluruh komponen utama URI, termasuk \textit{scheme} dan \textit{host}, sehingga dapat diinterpretasikan secara mandiri tanpa memerlukan konteks tambahan. Contoh URL absolut adalah \texttt{https://unpar.ac.id/page.html}, yang secara eksplisit menunjukkan protokol dan lokasi sumber daya. Sebaliknya, URL relatif tidak mencantumkan \textit{scheme} atau \textit{host} dan hanya berisi sebagian komponen URI, seperti \textit{path}, \textit{query}, atau \textit{fragment}. Contoh URL relatif adalah \texttt{/images/logo.png}. Selain itu, terdapat pula \textit{same-document reference}, yaitu URL yang hanya memuat komponen \textit{fragment} untuk merujuk ke bagian tertentu dalam dokumen yang sama, misalnya \texttt{\#section2}.

Agar referensi relatif dapat digunakan dalam komunikasi berbasis HTTP, referensi tersebut harus terlebih dahulu diubah menjadi URL absolut melalui suatu mekanisme resolusi. Proses resolusi ini dilakukan dengan menentukan sebuah base URI, kemudian menggabungkannya dengan referensi relatif yang digunakan sehingga membentuk URL yang lengkap. Salah satu tahap penting dalam proses tersebut adalah penyederhanaan \textit{path}, termasuk penghapusan \textit{dot-segments} seperti \texttt{/./} dan \texttt{/../}, agar struktur \textit{path} menjadi lebih ringkas dan tidak menimbulkan ambiguitas. Base URI umumnya diperoleh dari URL dokumen yang memuat referensi relatif tersebut, atau dari elemen \texttt{<base>} apabila elemen tersebut didefinisikan di dalam dokumen HTML. Dengan mekanisme ini, setiap referensi relatif dapat diinterpretasikan secara konsisten dan diarahkan ke sumber daya yang sesuai.


\section{HTML}
\label{sec:020300-html}
% % ~\cite{powell:10:htmlcss}
Hypertext Markup Language (HTML) merupakan bahasa markup standar yang digunakan untuk menyusun dan menampilkan halaman web pada \textit{browser}. HTML didefinisikan sebagai himpunan elemen yang dituliskan dalam bentuk tag, di mana setiap elemen dapat memiliki atribut untuk memberikan informasi tambahan. Dokumen HTML tersusun secara hierarkis dan secara konseptual dipandang sebagai sebuah pohon struktur yang dikenal dengan \textit{Document Object Model} (DOM).

Sejak diperkenalkan pada awal 1990-an, HTML telah berkembang melalui beberapa versi. HTML 4.01 yang dirilis pada tahun 1999 menjadi salah satu versi yang banyak digunakan dan bertahan cukup lama. Setelah itu muncul XHTML sebagai reformulasi HTML dalam sintaks XML, namun penerapannya terbatas. Versi terbaru adalah HTML5 yang dikembangkan oleh Web Hypertext Application Technology Working Group (WHATWG) dan kemudian diadopsi oleh World Wide Web Consortium (W3C), dengan dukungan yang lebih baik terhadap elemen semantik, multimedia, serta penulisan sintaks yang lebih sederhana.

\subsection{Struktur Dasar HTML}
\label{subsec:0224-struktur-dasar-html}

Sebuah dokumen HTML terdiri dari beberapa bagian utama. Dokumen diawali dengan deklarasi \texttt{<!DOCTYPE html>} yang memberi tahu \textit{browser} mengenai standar yang digunakan. Seluruh isi dokumen dibungkus dalam elemen \texttt{<html>} yang menjadi akar dari semua elemen lain.  

Di dalam \texttt{<html>} terdapat dua bagian penting, yaitu \texttt{<head>} dan \texttt{<body>}. Bagian \texttt{<head>} memuat informasi tentang dokumen seperti judul, metadata, dan pemanggilan sumber daya eksternal. Bagian \texttt{<body>} berisi konten utama yang ditampilkan kepada pengguna, seperti teks, gambar, tautan, tabel, maupun elemen multimedia.  


\begin{lstlisting}[language=HTML, caption={Struktur dasar dokumen HTML}, label={lst:html-basic-structure}]
<!DOCTYPE html>
<html>
<head>
    <title>Contoh Halaman</title>
    <meta charset="UTF-8">
    <link rel="stylesheet" href="style.css">
    <script src="script.js"></script>
</head>
<body>
    <h1>Selamat Datang</h1>
    <p>Ini adalah contoh struktur dasar HTML.</p>
</body>
</html>
\end{lstlisting}

Kode~\ref{lst:html-basic-structure} menunjukan struktur dasar dari HTML, baris pertama berisi deklarasi \texttt{<!DOCTYPE html>}. Elemen \texttt{<html>} membungkus seluruh dokumen, sedangkan \texttt{<head>} berisi \texttt{<title>} sebagai judul halaman, \texttt{<meta charset="UTF-8">} untuk menetapkan karakter encoding, \texttt{<link>} untuk memanggil stylesheet eksternal, serta \texttt{<script>} untuk menyertakan berkas JavaScript. Bagian \texttt{<body>} menampilkan konten kepada pengguna, dalam contoh ini berupa judul dan paragraf.

\subsection{Elemen Semantik dan Non-Semantik}
\label{subsec:0224-elemen-semantik-dan-non-semantik}

HTML menyediakan dua jenis elemen utama, yaitu elemen semantik dan elemen non-semantik.

\begin{enumerate}
    \item \textbf{Elemen Semantik} \\
    Elemen semantik adalah elemen yang memiliki makna jelas bagi \textit{browser} maupun pembaca, karena nama elemen menggambarkan fungsinya. Contoh elemen semantik antara lain:
    \begin{itemize}
        \item \texttt{<header>}: untuk bagian kepala suatu halaman atau bagian.
        \item \texttt{<nav>}: untuk kumpulan tautan navigasi.
        \item \texttt{<section>}: untuk sebuah bagian tematik dalam dokumen.
        \item \texttt{<article>}: untuk konten yang berdiri sendiri, seperti artikel berita.
        \item \texttt{<aside>}: untuk konten samping, seperti catatan atau iklan.
        \item \texttt{<footer>}: untuk bagian kaki halaman.
    \end{itemize}

    \item \textbf{Elemen Non-Semantik} \\
    Elemen non-semantik adalah elemen yang tidak menggambarkan makna spesifik dari kontennya, melainkan digunakan untuk keperluan pemformatan atau pengelompokan. Contoh elemen non-semantik adalah:
    \begin{itemize}
        \item \texttt{<div>}: untuk pengelompokan blok konten.
        \item \texttt{<span>}: untuk pengelompokan teks dalam baris.
    \end{itemize}
\end{enumerate}  

\subsection{Atribut Global dan Spesifik}
\label{subsec:0224-atribut-global-dan-spesifik}

Setiap elemen HTML dapat memiliki atribut yang berfungsi untuk memberikan informasi tambahan atau mengatur perilaku elemen tersebut. Atribut terbagi menjadi dua kategori, yaitu atribut global dan atribut spesifik.

Atribut global dapat digunakan pada hampir semua elemen HTML. Atribut ini bersifat umum karena tidak terikat pada fungsi tertentu dari elemen. Contoh atribut global meliputi:
\begin{itemize}
    \item \texttt{id} : identifikasi unik untuk sebuah elemen.
    \item \texttt{class} : pengelompokan elemen dengan gaya atau fungsi yang sama.
    \item \texttt{style} : mendefinisikan gaya inline menggunakan CSS.
    \item \texttt{title} : menyediakan keterangan tambahan yang biasanya ditampilkan sebagai \textit{tooltip}.
\end{itemize}

Atribut spesifik adalah atribut yang hanya berlaku pada elemen tertentu sesuai dengan fungsinya. Tabel~\ref{tab:html-specific-attributes} menunjukkan beberapa contoh atribut spesifik beserta elemen tempat atribut tersebut digunakan.

\begin{table}[H]
\centering
\caption{Contoh atribut spesifik pada elemen HTML}
\label{tab:html-specific-attributes}
\begin{tabular}{|l|l|p{7cm}|}
\hline
\textbf{Elemen} & \textbf{Atribut} & \textbf{Keterangan} \\ \hline
\texttt{<a>} & \texttt{href} & Menentukan alamat tujuan tautan. \\ \hline
\texttt{<img>} & \texttt{src} & Menentukan lokasi berkas gambar. \\ \hline
\texttt{<form>} & \texttt{action} & Menentukan alamat tujuan pengiriman data formulir. \\ \hline
\texttt{<script>} & \texttt{src} & Menentukan sumber berkas JavaScript eksternal. \\ \hline
\texttt{<link>} & \texttt{href} & Menentukan lokasi stylesheet atau sumber daya terkait lainnya. \\ \hline
\end{tabular}
\end{table}


% \subsection{Elemen yang Mengandung URL}
% \label{subsec:0224-elemen-yang-mengandung-url}

% Beberapa elemen HTML memiliki atribut URL yang menghubungkan dokumen dengan sumber daya lain. Elemen-elemen ini berperan penting dalam membangun keterhubungan antarhalaman maupun dengan berkas eksternal. Tabel~\ref{tab:html-url-elements} menampilkan beberapa elemen tersebut.

% \begin{center}
% \begin{longtable}{|l|l|p{5cm}|}
% \caption{Elemen HTML yang mengandung atribut URL} \label{tab:html-url-elements} \\

% \hline \multicolumn{1}{|c|}{\textbf{Elemen}} & \multicolumn{1}{c|}{\textbf{Atribut}} & \multicolumn{1}{c|}{\textbf{Keterangan}} \\ \hline 
% \endfirsthead

% \multicolumn{3}{c}%
% {{\bfseries \tablename\ \thetable{} -- lanjutan dari halaman sebelumnya}} \\
% \hline \multicolumn{1}{|c|}{\textbf{Elemen}} & \multicolumn{1}{c|}{\textbf{Atribut}} & \multicolumn{1}{c|}{\textbf{Keterangan}} \\ \hline 
% \endhead

% \hline \multicolumn{3}{|r|}{{Bersambung ke halaman berikutnya}} \\ \hline
% \endfoot

% \hline \hline
% \endlastfoot

% \texttt{<a>} & \texttt{href} & Menentukan alamat tujuan tautan. \\ \hline
% \texttt{<area>} & \texttt{href} & Menentukan alamat tujuan pada \textit{image map}. \\ \hline
% \texttt{<link>} & \texttt{href} & Menghubungkan dokumen dengan sumber daya eksternal, seperti stylesheet atau ikon. \\ \hline
% \texttt{<script>} & \texttt{src} & Menentukan lokasi berkas JavaScript eksternal. \\ \hline
% \texttt{<img>} & \texttt{src} & Menentukan lokasi berkas gambar. \\ \hline
% \texttt{<iframe>} & \texttt{src} & Menyematkan halaman web lain dalam bingkai. \\ \hline
% \texttt{<frame>} & \texttt{src} & Menentukan sumber halaman yang ditampilkan dalam sebuah frame. \\ \hline
% \texttt{<embed>} & \texttt{src} & Menyematkan konten eksternal, seperti multimedia. \\ \hline
% \texttt{<object>} & \texttt{data} & Menyematkan objek eksternal, seperti PDF atau aplikasi kecil. \\ \hline
% \texttt{<source>} & \texttt{src} & Menentukan sumber alternatif untuk elemen \texttt{<audio>} atau \texttt{<video>}. \\ \hline
% \texttt{<track>} & \texttt{src} & Menentukan lokasi berkas teks untuk \textit{caption} atau \textit{subtitle}. \\ \hline
% \texttt{<audio>} & \texttt{src} & Menentukan sumber berkas audio. \\ \hline
% \texttt{<video>} & \texttt{src} & Menentukan sumber berkas video. \\ \hline
% \texttt{<form>} & \texttt{action} & Menentukan alamat tujuan pengiriman data formulir. \\ \hline
% \texttt{<input type="image">} & \texttt{src} & Menentukan lokasi berkas gambar untuk tombol kirim. \\ \hline
% \texttt{<button>} & \texttt{formaction} & Menentukan alamat tujuan pengiriman data formulir khusus untuk tombol tersebut. \\ \hline
% \texttt{<base>} & \texttt{href} & Menentukan URL dasar untuk semua URL relatif dalam dokumen. \\ \hline
% \texttt{<meta>} & \texttt{http-equiv="refresh"} & Dapat memuat URL untuk mengarahkan ulang halaman. \\ \hline
% \texttt{<ins>} & \texttt{cite} & Menentukan URL sumber untuk penambahan teks. \\ \hline
% \texttt{<del>} & \texttt{cite} & Menentukan URL sumber untuk penghapusan teks. \\ \hline
% \texttt{<q>} & \texttt{cite} & Menentukan URL sumber untuk kutipan singkat. \\ \hline
% \texttt{<blockquote>} & \texttt{cite} & Menentukan URL sumber untuk kutipan panjang. \\ \hline
% \texttt{<isindex>} & \texttt{action} & Menentukan alamat tujuan query pencarian (elemen usang). \\ \hline

% \end{longtable}
% \end{center}



\section{\textit{Web Crawling}}
\label{sec:020400-web-crawling}
% % ~\cite{liu:11:webdatamining}
\textit{Web crawling} adalah proses otomatis untuk mengambil dokumen dari web dan mengekstraksi tautan yang terkandung di dalamnya untuk penjelajahan lebih lanjut. Proses ini dijalankan oleh perangkat lunak yang disebut \textit{crawler} atau \textit{spider}, yang bekerja secara iteratif dengan memulai dari satu atau lebih URL awal (\textit{seed} URL) dan mengikuti \textit{hyperlink} yang ditemukan dalam halaman web.

\subsection{Algoritma \textit{Crawling}}
\label{subsec:0204-algoritme-crawling}

\textit{Web crawling} dapat dipandang sebagai proses eksplorasi graf, di mana simpul merepresentasikan halaman web dan sisi merepresentasikan tautan antar halaman. Untuk mengatur proses eksplorasi, \textit{crawler} mempertahankan sebuah struktur antrean yang disebut \textit{frontier}. Frontier adalah kumpulan URL yang telah ditemukan tetapi belum dikunjungi. URL baru yang diekstraksi dari suatu halaman ditambahkan ke frontier jika belum pernah dimasukkan sebelumnya, sedangkan URL yang sudah dikunjungi disimpan dalam sebuah himpunan terpisah untuk mencegah duplikasi.

Secara umum, proses \textit{crawling} dilakukan sebagai berikut:
\begin{enumerate}
  \item Ambil sebuah URL dari frontier sesuai strategi antrean yang digunakan.
  \item Lakukan permintaan HTTP untuk mengambil dokumen dari URL tersebut.
  \item Ekstrak tautan dari dokumen HTML yang diperoleh.
  \item Tambahkan tautan baru ke frontier jika belum pernah dikunjungi.
\end{enumerate}

Terdapat beberapa strategi yang dapat digunakan untuk menentukan urutan eksplorasi halaman web. Dua strategi yang umum adalah sebagai berikut:

\begin{itemize}
  \item \textbf{Breadth-First Crawling} \\
  Pada strategi ini, frontier diimplementasikan sebagai antrean FIFO (\textit{First-In First-Out}). URL yang pertama kali ditemukan akan dikunjungi terlebih dahulu. Pendekatan ini menghasilkan cakupan eksplorasi yang merata dan cenderung mengunjungi halaman-halaman populer lebih awal karena sifat distribusi \textit{power-law} pada web.

  \item \textbf{Na\"{i}ve Best-First Crawling} \\
  Pada strategi ini, frontier diimplementasikan sebagai \textit{priority queue}, di mana setiap URL diberi skor prioritas. Skor tersebut dapat ditentukan berdasarkan faktor topologis seperti jumlah tautan masuk, atau berdasarkan kesesuaian konten dengan suatu topik tertentu. URL dengan skor prioritas tertinggi akan dikunjungi lebih dahulu, sehingga strategi ini banyak digunakan dalam \textit{focused crawling}.
\end{itemize}


\subsection{Jenis-Jenis \textit{Crawler}}
\label{subsec:0204-jenis-crawler}

Berdasarkan tujuan dan ruang lingkup \textit{crawling}, terdapat beberapa tipe \textit{crawler} yang umum dibahas dalam literatur:

\begin{itemize}
  \item \textbf{\textit{Universal Crawler}} \\
  \textit{Universal crawler} dirancang untuk menjelajahi bagian besar dari web tanpa pembatasan topik. Jenis ini umumnya digunakan oleh mesin pencari berskala besar, sehingga membutuhkan infrastruktur terdistribusi dan mekanisme skalabilitas tinggi untuk menangani miliaran halaman web.  

  \item \textbf{\textit{Focused Crawler}} \\
  \textit{Focused crawler} ditujukan untuk mengambil hanya halaman-halaman yang relevan terhadap suatu topik tertentu. Seleksi dilakukan dengan mengevaluasi konten atau atribut halaman sebelum mengikuti tautan. Dengan pendekatan ini, efisiensi meningkat karena \textit{crawler} tidak mengunjungi halaman yang dianggap tidak relevan.  

  \item \textbf{\textit{Topical Crawler}} \\
  \textit{Topical crawler} merupakan variasi dari focused \textit{crawler} yang menekankan pada keterkaitan topik antar halaman yang saling terhubung. Prinsip \textit{topical locality} digunakan untuk memperkirakan bahwa halaman yang berdekatan dengan halaman relevan cenderung juga relevan terhadap topik yang sama. Strategi ini membantu \textit{crawler} memilih jalur penelusuran yang lebih sesuai dengan domain yang ditargetkan.  
\end{itemize}

\subsection{Tantangan Implementasi}
\label{subsec:0204-tantangan-crawling}

Pengembangan sistem \textit{web crawling} menghadapi sejumlah tantangan teknis yang telah banyak dibahas dalam literatur. Beberapa tantangan utama tersebut adalah sebagai berikut:

\begin{itemize}
    \item \textbf{\textit{Fetching}} \\
    Proses pengambilan konten dari halaman web harus dilakukan secara efisien. \textit{Crawler} perlu menetapkan batas waktu (\textit{timeout}) agar tidak menunggu terlalu lama pada halaman yang tidak responsif. Selain itu, diperlukan mekanisme pengendalian kecepatan permintaan (\textit{rate limiting}) dengan cara menyisipkan jeda (\textit{delay}) antar permintaan atau membatasi jumlah permintaan dalam satuan waktu. Hal ini mencegah beban berlebihan pada server tujuan serta mengurangi risiko pemblokiran alamat IP.

    \item \textbf{\textit{Parsing}} \\
    Konten HTML yang diperoleh dari web harus diurai (\textit{parse}) menjadi representasi struktur DOM untuk memungkinkan ekstraksi elemen seperti \texttt{<a href>}. Tantangan muncul karena halaman web sering kali tidak mematuhi standar HTML, misalnya memiliki tag yang tidak lengkap atau sintaks yang salah. Oleh karena itu, diperlukan parser yang toleran terhadap kesalahan dan mampu menangani berbagai \textit{encoding} karakter. Selain itu, konten dinamis berbasis JavaScript biasanya tidak tersedia langsung dalam HTML statis sehingga dapat terlewat oleh \textit{crawler} tradisional.

    \item \textbf{\textit{Link Extraction} dan \textit{Canonicalization}} \\
    Semua tautan yang ditemukan harus dinormalisasi agar tidak terjadi duplikasi akibat variasi penulisan URL. Normalisasi mencakup langkah-langkah seperti konversi nama host menjadi huruf kecil, penghapusan fragmen (\texttt{\#}), parameter sesi, serta penyeragaman penggunaan \textit{trailing slash}. Dengan demikian, \textit{crawler} dapat memastikan setiap sumber daya diidentifikasi secara unik dan tidak mengambil halaman yang sama berulang kali.

    \item \textbf{\textit{Spider Trap} dan \textit{Infinite Loops}} \\
    Beberapa situs web berisi pola tautan yang dapat membuat \textit{crawler} terjebak dalam perulangan tak terbatas, misalnya kalender dinamis yang menghasilkan jumlah halaman tanpa akhir atau URL dengan parameter yang terus berubah. Untuk mengatasi masalah ini, \textit{crawler} perlu membatasi kedalaman penelusuran (\textit{crawl depth}), menetapkan jumlah maksimum URL per domain, dan menerapkan deteksi pola berulang yang berpotensi menjadi jebakan.

    \item \textbf{\textit{Page Repository}} \\
    Sistem \textit{crawler} perlu menyimpan metadata setiap halaman yang sudah dikunjungi, termasuk URL, waktu kunjungan terakhir, status pengambilan, dan sidik jari konten (\textit{content hash}). Penyimpanan informasi ini mencegah kunjungan ulang yang tidak diperlukan dan mendukung penerapan \textit{scheduled recrawling} untuk memperbarui konten secara periodik.

    \item \textbf{\textit{Concurrency}} \\
    Untuk meningkatkan kecepatan \textit{crawling}, \textit{crawler} modern menjalankan beberapa permintaan secara paralel menggunakan banyak utas (\textit{threads}) atau proses. Akan tetapi, peningkatan jumlah permintaan simultan dapat menimbulkan risiko pembatasan atau pemblokiran oleh server target. Oleh karena itu, \textit{crawler} memerlukan strategi sinkronisasi, antrian tugas (\textit{task queue}), serta mekanisme pengendalian beban untuk menjaga efisiensi sekaligus menghindari gangguan pada server.
\end{itemize}


\subsection{Etika dan Kepatuhan}
\label{subsec:0204-etika-crawling}

Aktivitas \textit{web crawling} tidak hanya menghadapi tantangan teknis, tetapi juga harus memperhatikan aspek etika dan kepatuhan terhadap norma yang berlaku di lingkungan web. Beberapa prinsip yang umum diterapkan adalah sebagai berikut:

\begin{itemize}
  \item \textbf{\texttt{robots.txt}} \\
  File \texttt{robots.txt} digunakan dalam protokol \textit{Robots Exclusion Protocol} untuk menentukan bagian situs yang boleh atau tidak boleh diakses oleh \textit{crawler} File ini ditempatkan di direktori akar (\texttt{/}) dari sebuah situs web. Sebagai contoh:
\begin{verbatim}
User-agent: *
Disallow: /admin/
Allow: /public/
\end{verbatim}
  Instruksi tersebut menyatakan bahwa semua agen otomatis (ditandai dengan \texttt{*}) tidak diperbolehkan mengakses direktori \texttt{/admin/}, tetapi diperbolehkan mengakses \texttt{/public/}. Walaupun kepatuhan terhadap \texttt{robots.txt} bersifat sukarela, pengabaian aturan ini dapat dianggap sebagai pelanggaran etika.

  \item \textbf{\textit{User-Agent Identification}} \\
  Setiap permintaan HTTP memiliki \textit{header} \texttt{User-Agent} yang berfungsi mengidentifikasi \textit{crawler} Informasi ini memungkinkan administrator situs mengetahui identitas perangkat lunak yang melakukan \textit{crawling}. Sebagai praktik yang baik, \textit{crawler} mencantumkan nama, versi, serta alamat URL untuk dokumentasi atau kontak. Contoh:
\begin{verbatim}
User-Agent: MyCrawler/1.0 (+http://example.com/crawler-info)
\end{verbatim}
  Dengan adanya identitas ini, pemilik situs dapat memahami tujuan \textit{crawling} dan memiliki kesempatan untuk berkomunikasi dengan pengembang \textit{crawler} bila diperlukan.

  \item \textbf{Kepatuhan terhadap Kebijakan Situs} \\
  \textit{crawler} harus menghormati kebijakan akses dan tidak mengambil konten yang bersifat pribadi, rahasia, atau dilindungi hak cipta. Praktik seperti bypass autentikasi, pengambilan data berbayar, atau eksploitasi celah keamanan termasuk kategori penyalahgunaan yang melanggar hukum maupun etika.
\end{itemize}


\section{JavaFX}
\label{sec:020500-javafx}
% % \subsection{JavaFX}
% \label{subsec:0304-javafx}

Pemilihan JavaFX sebagai pustaka antarmuka pengguna didasarkan pada kebutuhan sistem yang harus disajikan dalam bentuk aplikasi desktop dengan tampilan sederhana, tabel yang mudah dibaca, serta kontrol proses yang jelas (lihat Subsubbab~\ref{subsec:0303-kebutuhan-non-fungsional}). Sebagaimana dijelaskan pada Subbab~\ref{sec:02-javafx}, JavaFX menawarkan paradigma \textit{scene graph} yang memungkinkan penyusunan antarmuka secara hierarkis dan konsisten. Paradigma ini memudahkan pengelolaan komponen visual yang diperlukan untuk menampilkan hasil pemeriksaan tautan secara \textit{streaming} (lihat Subsubbab~\ref{subsec:0303-kebutuhan-fungsional}).

Dalam implementasi, JavaFX akan digunakan melalui beberapa komponen utama berikut:

\begin{itemize}
  \item \textbf{Stage, Scene, dan Node}. Struktur dasar aplikasi akan mengikuti siklus hidup JavaFX dengan menurunkan kelas dari \texttt{Application}. Objek \texttt{Stage} akan berperan sebagai jendela utama, sedangkan \texttt{Scene} digunakan untuk menampung keseluruhan antarmuka. Komponen UI seperti \texttt{Button}, \texttt{Label}, dan \texttt{TextField} direpresentasikan sebagai turunan \texttt{Node} yang diorganisasi dalam kontainer tata letak, misalnya \texttt{VBox} atau \texttt{BorderPane}.

  \item \textbf{FXML dan Controller}. Untuk memisahkan logika aplikasi dari tampilan, struktur antarmuka didefinisikan secara deklaratif dalam berkas FXML. Atribut \texttt{fx:id} akan dipakai agar komponen UI dapat diakses dari kelas controller, sedangkan event handler seperti \texttt{onAction} digunakan untuk menangani interaksi pengguna, misalnya saat memulai atau menghentikan proses pemeriksaan tautan.

  \item \textbf{Komponen Tabel}. Hasil pemeriksaan tautan akan ditampilkan dalam bentuk tabel menggunakan \texttt{TableView}. Komponen ini mendukung penyajian data terstruktur dalam baris dan kolom, sehingga cocok untuk menampilkan daftar halaman yang diperiksa maupun daftar tautan rusak. Setiap kolom akan diikat dengan \texttt{Property} pada model data agar perubahan nilai dapat langsung tercermin pada tampilan.

  \item \textbf{Property dan Binding}. Untuk mendukung pembaruan data secara langsung, mekanisme \texttt{StringProperty}, \texttt{BooleanProperty}, dan \texttt{IntegerProperty} akan digunakan. Binding dua arah dimanfaatkan agar nilai pada komponen input dan model selalu konsisten, sedangkan binding satu arah memastikan perubahan status pemeriksaan langsung diperlihatkan pada label atau tabel.

  \item \textbf{Pengendalian Thread}. Karena proses pemeriksaan tautan berjalan secara paralel, pembaruan antarmuka pengguna harus dijalankan melalui \texttt{Platform.runLater()}. Hal ini menjamin sinkronisasi antara \textit{thread} pemeriksaan dengan \textit{thread} JavaFX, sehingga tampilan dapat diperbarui secara aman tanpa menimbulkan error \texttt{Not on FX application thread}.
\end{itemize}




\section{Jsoup}
\label{sec:020600-jsoup}
% Jsoup adalah sebuah pustaka Java yang berfungsi sebagai HTML \textit{parser} untuk memproses dokumen HTML sehingga dapat diolah dalam struktur yang lebih teratur. Dokumen yang diproses oleh Jsoup dapat berasal dari tiga sumber utama, yaitu halaman web yang diakses melalui URL, berkas yang tersimpan secara lokal, dan \texttt{String} yang berisi kode HTML. Pustaka ini dirancang agar toleran terhadap dokumen HTML yang tidak valid atau tidak terstruktur, sehingga hasil \textit{parsing} selalu dalam bentuk \textit{Document Object Model} (DOM) yang telah diperbaiki. Berikut adalah beberapa kelas yang tersedia pada pustaka Jsoup:

\vspace{-2mm}

\subsection{Kelas \texttt{Jsoup}}
\label{subsec:020601-kelas-jsoup}
Kelas \texttt{Jsoup} merupakan titik masuk utama untuk menggunakan pustaka Jsoup. Berikut adalah beberapa metode yang tersedia pada kelas ini:

\begin{itemize}
    \item \texttt{connect}: Metode ini digunakan untuk membuat koneksi HTTP menuju URL situs web tujuan.
    
    \item \texttt{parse}: Metode ini digunakan untuk melakukan \textit{parsing} pada \texttt{String} HTML menjadi objek \texttt{Document}.
    
    \item \texttt{parseBodyFragment}: Metode ini digunakan untuk melakukan \textit{parsing} pada potongan HTML parsial dan menghasilkan objek \texttt{Document}.
    
\end{itemize}


\subsection{Kelas \texttt{Document}}
\label{subsec:020602-kelas-document}
Kelas \texttt{Document} merupakan representasi dari keseluruhan dokumen HTML setelah proses \textit{parsing}. Kelas ini berfungsi sebagai akar struktur DOM dan menyediakan metode untuk membaca, menelusuri, atau memodifikasi isi dokumen. Berikut adalah beberapa metode yang tersedia pada kelas ini:

\begin{itemize}[itemsep=5pt]
    \item \texttt{title}: Metode ini digunakan untuk mendapatkan elemen \texttt{<title>} dari dokumen.
    
    \item \texttt{body}: Metode ini digunakan untuk mendapatkan elemen \texttt{<body>} dari dokumen.
    
    \item \texttt{select}: Metode ini digunakan untuk mencari elemen dalam dokumen berdasarkan CSS \textit{selector}.
    
    \item \texttt{baseUri}: Metode ini digunakan untuk mendapatkan URL dasar dari dokumen, jika sumbernya diambil melalui URL.
    
\end{itemize}

\vspace{5mm}

\subsection{Kelas \texttt{Element}}
\label{subsec:020603-kelas-element}
Kelas \texttt{Element} merupakan representasi satu elemen HTML dalam struktur DOM. Kelas ini memiliki \textit{tag}, atribut, konten teks, serta daftar elemen anak. Berikut adalah beberapa metode yang tersedia pada kelas ini:

\begin{itemize}[itemsep=5pt]
    \item \texttt{attr}: Metode ini digunakan untuk mengambil nilai suatu atribut.
    
    \item \texttt{text}: Metode ini digunakan untuk mendapatkan teks dalam elemen beserta elemen turunannya.
    
    \item \texttt{absUrl}: Metode ini digunakan untuk mendapatkan URL absolut dari sebuah atribut, seperti \texttt{href} atau \texttt{src}, berdasarkan URI dasar dokumen.
    
    \item \texttt{html}: Metode ini digunakan untuk memperoleh markup HTML bagian dalam elemen.
    
    \item \texttt{select}: Metode ini digunakan untuk mencari elemen turunan berdasarkan CSS \textit{selector}.
    
    \item \texttt{children}: Metode ini digunakan untuk mendapatkan daftar elemen anak dari sebuah elemen tertentu.
    
\end{itemize}

\vspace{5mm}

\subsection{Contoh Kode Program}
\label{subsec:0227-contoh-kode-program}

Kode~\ref{lst:jsoup-example} menunjukkan contoh penggunaan Jsoup untuk mengambil sebuah halaman web, memprosesnya menjadi objek \texttt{Document}, dan mengekstraksi sejumlah informasi dari dokumen tersebut. Contoh ini menggambarkan bagaimana kelas-kelas utama yang telah dijelaskan sebelumnya, yaitu \texttt{Jsoup}, \texttt{Connection}, \texttt{Connection.Response}, \texttt{Document}, \texttt{Elements}, dan \texttt{Element}, digunakan secara bersama-sama dalam sebuah program nyata. Hasil eksekusi dari kode ini berupa informasi judul halaman, URI dasar dokumen, serta daftar tautan beserta teks, URL absolut, dan atribut lain yang terkait.

\vspace{20mm}

\begin{lstlisting}[language=Java, caption={Contoh penggunaan Jsoup}, label={lst:jsoup-example}]
public class JsoupExample {
    public static void main(String[] args) throws Exception {
        Connection connection = Jsoup.connect("https://www.example.com").userAgent("BrokenLinkChecker 1.0").timeout(5000);
        Connection.Response response = connection.execute();

        System.out.println("Status Code : " + response.statusCode());
        System.out.println("Content-Type: " + response.headers().get("Content-Type"));

        Document doc = response.parse();

        System.out.println("Title   : " + doc.title());
        System.out.println("Base URI: " + doc.baseUri());

        Elements links = doc.select("a[href]");
        for (Element link : links) {
            System.out.println("Teks   : " + link.text());
            System.out.println("Href   : " + link.attr("href"));
            System.out.println("AbsURL : " + link.absUrl("href"));
        }
    }
}
\end{lstlisting}

\vspace{5mm}

Alur dari Kode~\ref{lst:jsoup-example} dimulai dengan pembuatan objek \texttt{Connection} melalui pemanggilan metode \texttt{Jsoup.connect()}, kemudian ditetapkan nilai \textit{User-Agent} dengan \texttt{userAgent()} dan batas waktu koneksi dengan \texttt{timeout()}. Permintaan dieksekusi melalui \texttt{execute()} untuk menghasilkan objek \texttt{Connection.Response}, dari mana dapat diperoleh kode status dengan \texttt{statusCode()} dan \textit{header} tertentu melalui \texttt{headers()}. Isi respons kemudian diproses menjadi objek \texttt{Document} menggunakan \texttt{parse()}, yang selanjutnya menyediakan informasi judul halaman dengan \texttt{title()} dan URI dasar dengan \texttt{baseUri()}. Setelah itu, method \texttt{select()} digunakan untuk memilih seluruh elemen \texttt{<a>} yang memiliki atribut \texttt{href}, menghasilkan koleksi \texttt{Elements}. Koleksi ini diiterasi, dan untuk setiap \texttt{Element} diperoleh teks melalui \texttt{text()}, nilai atribut melalui \texttt{attr("href")}, serta URL absolut melalui \texttt{absUrl("href")}. Dengan demikian, kode ini tidak hanya menampilkan daftar tautan, tetapi juga memperlihatkan cara menggunakan beberapa method penting yang telah dijelaskan sebelumnya.




\section{OkHttp}
\label{sec:020700-ok-http}
% \input{Bab/landasan-teori/ok-http}


\section{Java HTTP Client}
\label{sec:020700-java-http-client}
% \input{Bab/landasan-teori/java-http-client}


\section{Apache POI}
\label{sec:020800-apache-poi}
% Apache POI merupakan pustaka berbasis Java yang disediakan oleh The Apache Software Foundation untuk memanipulasi berbagai format berkas Microsoft Office. Dokumentasi resminya menyatakan bahwa proyek Apache POI bertujuan untuk menyediakan dan memelihara API Java yang mampu memproses format berkas yang didasarkan pada standar Office Open XML (OOXML) serta format OLE2 Compound Document. Dengan demikian, pustaka ini memungkinkan pengembang untuk membaca, membuat, dan memodifikasi berkas seperti Word, Excel, maupun PowerPoint melalui antarmuka pemrograman berbasis Java.

Untuk berkas spreadsheet, Apache POI menyediakan dua implementasi utama, yaitu HSSF dan XSSF. Komponen HSSF digunakan untuk menangani format Excel 97--2003 (XLS), sedangkan XSSF digunakan untuk format Excel 2007 ke atas (XLSX) yang berbasis OOXML. Selain kedua komponen tersebut, Apache POI juga menyediakan SS UserModel, yaitu API tingkat tinggi yang menyediakan antarmuka umum untuk bekerja dengan spreadsheet tanpa bergantung pada format file tertentu. Melalui antarmuka seperti \texttt{Workbook}, \texttt{Sheet}, \texttt{Row}, dan \texttt{Cell}, SS UserModel memungkinkan pengelolaan struktur spreadsheet secara konsisten pada kedua format yang berbeda tersebut.

Arsitektur modul Apache POI dirancang agar fleksibel dan dapat diperluas, sehingga setiap komponen memiliki tanggung jawab yang jelas terhadap format berkas tertentu. Dengan dukungan terhadap standar OOXML dan OLE2, POI menjadi salah satu pustaka Java yang paling banyak digunakan untuk integrasi dokumen Office dalam berbagai aplikasi, baik dalam konteks pembuatan laporan, pengolahan data terstruktur, maupun otomasi dokumen.



\section{Gson}
\label{sec:020900-gson}
% \input{Bab/landasan-teori/gson}


