Analisis mengenai tautan rusak pada penelitian ini didasarkan pada teori URI yang dijelaskan pada Subbab~\ref{sec:020200-uri} serta teori HTTP pada Subbab~\ref{sec:020100-http}. Tujuan utamanya adalah memahami faktor-faktor yang membuat tautan gagal diakses. Dengan melihat struktur URL, aturan sintaks, mekanisme resolusi, hingga kode status HTTP, kita dapat memahami dengan lebih jelas bagaimana tautan dinyatakan rusak dan apa saja indikator yang bisa digunakan untuk mengenalinya.


\vspace{-3mm}
\subsubsection*{Struktur URL}
\vspace{-3mm}
Pada Subbab~\ref{subsec:0202-struktur-url} dijelaskan bahwa URL terdiri atas komponen \textit{scheme}, \textit{host}, \textit{port}, \textit{path}, \textit{query}, dan \textit{fragment}. Beberapa komponen memiliki peran langsung terhadap keberhasilan akses. Kesalahan pada salah satu komponen utama dapat menghasilkan tautan yang tidak dapat diproses oleh agen pengguna atau tidak dapat ditemukan oleh server.

Faktor yang dapat menyebabkan tautan rusak meliputi:
\begin{itemize}
  \item \textbf{Scheme} tidak valid sehingga koneksi tidak dapat dibangun.
  \item \textbf{Host} tidak valid atau tidak terdaftar sehingga sumber daya tidak dapat ditemukan.
  \item \textbf{Port} tidak sesuai dengan layanan yang tersedia.
  \item \textbf{Path} tidak mengarah pada sumber daya sehingga menghasilkan respons 404 \textit{Not Found}.
  \item \textbf{Query} tidak sesuai sehingga server tidak dapat menyediakan representasi sumber daya.
\end{itemize}

Komponen-komponen tersebut menentukan arah dan tujuan akses. Ketidaktepatan nilai pada salah satunya cukup untuk membuat tautan tidak dapat digunakan.

\vspace{-3mm}
\subsubsection*{Validitas Sintaks dan Encoding}
Pada Subsubbab~\ref{subsec:0202-karakter-percent-encoding} dijelaskan bahwa URI memiliki aturan sintaks dan \textit{percent-encoding} yang harus dipenuhi agar dapat diproses dengan benar. Pelanggaran terhadap aturan ini menyebabkan URL tidak dapat ditafsirkan atau ditolak oleh server.

Kesalahan yang dapat menyebabkan tautan rusak antara lain:
\begin{itemize}
  \item Penggunaan karakter \textit{reserved} pada posisi yang tidak sesuai.
  \item Keberadaan spasi atau karakter non-ASCII yang tidak dikodekan.
  \item Format \textit{percent-encoding} yang tidak valid.
\end{itemize}

Selain itu, proses normalisasi URL dapat menghasilkan URL yang tidak valid apabila tidak dilakukan dengan benar. Contohnya:
\begin{itemize}
  \item \textit{Host} tidak distandardisasi ke bentuk huruf kecil.
  \item \textit{Port} default tetap dicantumkan sehingga URL tidak identik dengan bentuk kanonisnya.
  \item \textit{Path} tidak dibersihkan dari segmen \texttt{.} atau \texttt{..}, sehingga jalur yang dibentuk tidak sesuai.
\end{itemize}

\vspace{-3mm}
\subsubsection*{URL Relatif}
\vspace{-2mm}
Pada Subsubbab~\ref{subsec:0202-struktur-url} dijelaskan bahwa URL relatif tidak dapat digunakan secara langsung, melainkan harus digabungkan dengan \textit{base} URI untuk membentuk URL absolut. Kesalahan sering terjadi apabila \textit{base} URI tidak sesuai dengan struktur dokumen, sehingga proses resolusi menghasilkan URL absolut yang berbeda dari lokasi sumber daya yang sebenarnya. Selain itu, resolusi URL juga mewajibkan penghapusan \textit{dot-segments} seperti \texttt{.} dan \texttt{..}. Apabila segmen tersebut tidak dihapus, jalur yang dihasilkan menjadi tidak akurat. Akibatnya, URL relatif yang secara sintaks valid berubah menjadi tautan yang gagal diarahkan ke sumber daya yang benar. Situasi ini merupakan salah satu penyebab tautan rusak yang berasal dari kesalahan konstruksi URL absolut, bukan dari kesalahan struktur URL awal.

\vspace{-3mm}
\subsubsection*{Komunikasi HTTP}
\vspace{-2mm}
Pada Subbab~\ref{sec:020100-http} dijelaskan bahwa akses terhadap sumber daya tidak hanya ditentukan oleh validitas struktur URL, tetapi juga keberhasilan pertukaran pesan melalui HTTP. Tautan dapat dinyatakan rusak apabila proses komunikasi gagal meskipun URL telah ditulis dengan benar. Kegagalan semacam ini dapat disebabkan oleh beberapa kondisi, seperti kesalahan pemetaan domain yang mengakibatkan host tidak dapat ditemukan, penolakan koneksi ketika layanan tidak tersedia pada port tujuan, atau waktu tunggu yang habis sebelum server memberikan respons.

\vspace{-3mm}
\subsubsection*{Kode Status HTTP}
\vspace{-2mm}
Pada Subsubbab~\ref{subsec:0201-kode-status-http} dijelaskan bahwa kode status HTTP digunakan untuk menunjukkan hasil pemrosesan permintaan. Untuk keperluan penentuan tautan rusak, kategori kode status perlu ditetapkan dengan batas yang jelas agar setiap respons dapat dinilai secara konsisten.

Dalam penelitian ini ditetapkan bahwa seluruh kode pada kategori \textit{Client Error} 4xx dan \textit{Server Error} 5xx dianggap sebagai indikator kegagalan akses. Kategori \textit{Client Error} 4xx menunjukkan bahwa permintaan tidak berhasil diproses karena terdapat kekeliruan pada permintaan yang dikirimkan, seperti sumber daya yang tidak tersedia, permintaan yang tidak memenuhi syarat, atau pembatasan akses. Kategori \textit{Server Error} 5xx menunjukkan bahwa server tidak mampu menyediakan sumber daya akibat kegagalan pemrosesan atau gangguan pada layanan.



Seluruh kode pada dua kategori tersebut disertakan agar tidak ada kondisi gagal yang terlewat. Respons dari server dapat bervariasi antarplatform, sehingga pembatasan hanya pada kode tertentu dapat menyebabkan kegagalan yang sah tidak dicatat. Dengan menggunakan seluruh kategori \textit{Client Error} 4xx dan \textit{Server Error} 5xx, batasan mengenai definisi tautan rusak menjadi lebih jelas, yaitu setiap respons yang menyatakan bahwa sumber daya tidak dapat disediakan oleh permintaan maupun oleh server.

