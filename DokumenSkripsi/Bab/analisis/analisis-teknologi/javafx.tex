Pemilihan JavaFX sebagai pustaka antarmuka pengguna didasarkan pada kebutuhan sistem yang harus disajikan dalam bentuk aplikasi desktop dengan tampilan sederhana, tabel yang mudah dibaca, serta kontrol proses yang jelas (lihat Subsubbab~\ref{subsec:030302-kebutuhan-non-fungsional}). Sebagaimana dijelaskan pada Subbab~\ref{sec:020500-javafx}, JavaFX menawarkan paradigma \textit{scene graph} yang memungkinkan penyusunan antarmuka secara hierarkis dan konsisten. Paradigma ini memudahkan pengelolaan komponen visual yang diperlukan untuk menampilkan hasil pemeriksaan tautan secara \textit{streaming} (lihat Subsubbab~\ref{subsec:030301-kebutuhan-fungsional}).

Dalam implementasi, JavaFX akan digunakan melalui beberapa komponen utama berikut:

\begin{itemize}[itemsep=3mm]
  \item \textbf{Stage, Scene, dan Node}. Struktur dasar aplikasi akan mengikuti siklus hidup JavaFX dengan menurunkan kelas dari \texttt{Application}. Objek \texttt{Stage} akan berperan sebagai jendela utama, sedangkan \texttt{Scene} digunakan untuk menampung keseluruhan antarmuka. Komponen UI seperti \texttt{Button}, \texttt{Label}, dan \texttt{TextField} direpresentasikan sebagai turunan \texttt{Node} yang diorganisasi dalam kontainer tata letak, misalnya \texttt{VBox} atau \texttt{BorderPane}.

  \item \textbf{FXML dan Controller}. Untuk memisahkan logika aplikasi dari tampilan, struktur antarmuka didefinisikan secara deklaratif dalam berkas FXML. Atribut \texttt{fx:id} akan dipakai agar komponen UI dapat diakses dari kelas controller, sedangkan event handler seperti \texttt{onAction} digunakan untuk menangani interaksi pengguna, misalnya saat memulai atau menghentikan proses pemeriksaan tautan.

  \item \textbf{Komponen Tabel}. Hasil pemeriksaan tautan akan ditampilkan dalam bentuk tabel menggunakan \texttt{TableView}. Komponen ini mendukung penyajian data terstruktur dalam baris dan kolom, sehingga cocok untuk menampilkan daftar halaman yang diperiksa maupun daftar tautan rusak. Setiap kolom akan diikat dengan \texttt{Property} pada model data agar perubahan nilai dapat langsung tercermin pada tampilan.

  \item \textbf{Property dan Binding}. Untuk mendukung pembaruan data secara langsung, mekanisme \texttt{StringProperty}, \texttt{BooleanProperty}, dan \texttt{IntegerProperty} akan digunakan. Binding dua arah dimanfaatkan agar nilai pada komponen input dan model selalu konsisten, sedangkan binding satu arah memastikan perubahan status pemeriksaan langsung diperlihatkan pada label atau tabel.

  \item \textbf{Pengendalian Thread}. Karena proses pemeriksaan tautan berjalan secara paralel, pembaruan antarmuka pengguna harus dijalankan melalui \texttt{Platform.runLater()}. Hal ini menjamin sinkronisasi antara \textit{thread} pemeriksaan dengan \textit{thread} JavaFX, sehingga tampilan dapat diperbarui secara aman tanpa menimbulkan error \texttt{Not on FX application thread}.
\end{itemize}

\vspace{5mm}

Kode~\ref{lst:javafx-fxml} memperlihatkan contoh struktur antarmuka berbasis FXML yang menggunakan \texttt{BorderPane} sebagai \textit{root node} dan \texttt{HBox} sebagai kontainer tata letak.

\begin{lstlisting}[language=XML, caption={Contoh struktur dasar dokumen FXML}, label=lst:javafx-fxml]
<BorderPane xmlns:fx="http://javafx.com/fxml" fx:controller="Controller">
  <top>
    <HBox spacing="10">
      <Label text="Input:"/>
      <TextField fx:id="inputField"/>
      <Button fx:id="submitButton" text="Submit" onAction="#handleSubmit"/>
    </HBox>
  </top>
</BorderPane>
\end{lstlisting}

\vspace{5mm}

Kode~\ref{lst:javafx-fxml} menunjukkan bahwa elemen akar antarmuka adalah \texttt{BorderPane}. Pada bagian atasnya terdapat sebuah \texttt{HBox} dengan jarak antar elemen sebesar 10 piksel. Di dalam \texttt{HBox} didefinisikan sebuah \texttt{Label} untuk menampilkan teks, sebuah \texttt{TextField} yang diberi atribut \texttt{fx:id} agar dapat diakses dari \texttt{controller}, serta sebuah \texttt{Button} yang memiliki atribut \texttt{onAction} untuk memanggil metode penanganan kejadian pada \texttt{controller}.

Kode~\ref{lst:javafx-controller-binding} memperlihatkan contoh kelas \texttt{Controller} yang menggunakan \textit{property} dan \textit{binding} untuk menghubungkan \texttt{TextField} dengan \texttt{Label} melalui \texttt{StringProperty}.

\vspace{5mm}

\begin{lstlisting}[language=Java, caption={Contoh controller dengan \textit{property} dan \textit{binding}}, label=lst:javafx-controller-binding]
public class Controller {
    // TextField yang dihubungkan dari FXML
    @FXML
    private TextField inputField;

    // Label yang dihubungkan dari FXML
    @FXML
    private Label statusLabel;

    // Property untuk menyimpan dan mengelola nilai teks
    private StringProperty inputText = new SimpleStringProperty();

    // Dipanggil otomatis setelah FXML dimuat
    public void initialize() {
        // Binding dua arah antara TextField dan property
        inputField.textProperty().bindBidirectional(inputText);
        
        // Binding satu arah dari property ke Label
        statusLabel.textProperty().bind(inputText);
    }

    // Dipanggil saat Button ditekan
    @FXML
    public void handleSubmit() {
        // Memperbarui nilai property, perubahan langsung tercermin di Label
        inputText.set("Input diterima: " + inputText.get());
    }
}

\end{lstlisting}

\vspace{5mm}

Kode program pada Kode~\ref{lst:javafx-controller-binding} menunjukkan bahwa komponen \texttt{TextField} dan \texttt{Label} dihubungkan ke kelas \texttt{Controller} melalui anotasi \texttt{@FXML}. Sebuah \textit{property} berupa \texttt{StringProperty} digunakan untuk menyimpan nilai teks. Pada metode \texttt{initialize()}, property tersebut diikat secara dua arah dengan properti teks milik \texttt{TextField}, sehingga perubahan nilai pada salah satunya akan tercermin pada yang lain. Selanjutnya, properti teks milik \texttt{Label} diikat secara satu arah dengan property tersebut, sehingga nilai yang ditampilkan selalu sesuai. Ketika metode \texttt{handleSubmit()} dipanggil, nilai property diperbarui dan perubahan tersebut secara otomatis ditampilkan pada \texttt{Label} melalui mekanisme \textit{binding}.

