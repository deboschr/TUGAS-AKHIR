Pemilihan Apache POI sebagai pustaka untuk melakukan ekspor hasil pemeriksaan ke berkas Excel dilandasi kebutuhan sistem untuk menghasilkan keluaran dalam format \texttt{XLSX}. Format ini dipilih karena merupakan standar \textit{Office Open XML} (OOXML) yang umum digunakan dan didukung secara luas oleh perangkat lunak spreadsheet modern. Sebagaimana dijelaskan pada Subbab~\ref{sec:020800-apache-poi}, Apache POI menyediakan modul \texttt{XSSF} yang dirancang khusus untuk memproses berkas Excel dengan format OOXML, sehingga sesuai dengan kebutuhan aplikasi ini.

Dari sisi fungsional, penggunaan Apache POI mendukung kebutuhan untuk:
\begin{itemize}
  \item Menghasilkan berkas Excel berisi daftar tautan rusak serta informasi pendukung seperti sumber halaman dan kode status dan ringkasan hasil.
  \item Menjaga format keluaran yang konsisten dan mudah dibaca oleh pengguna.
\end{itemize}

Dalam implementasi, modul dan kelas berikut akan digunakan:
\begin{itemize}
  \item \texttt{XSSFWorkbook}\\
  Kelas ini merupakan implementasi \texttt{Workbook} untuk format \texttt{XLSX}. Objek \texttt{XSSFWorkbook} akan menjadi kontainer utama yang menyimpan seluruh \textit{sheet}, baris, dan sel yang akan ditulis ke berkas.

  \item \texttt{Sheet}\\
  Digunakan untuk membuat satu lembar kerja yang memuat tabel hasil pemeriksaan. Setiap baris pada \textit{sheet} akan mewakili satu entri data, misalnya satu tautan rusak.

  \item \texttt{Row} dan \texttt{Cell}\\
  Kedua kelas ini dipakai untuk membentuk tabel secara terstruktur. \texttt{Row} merepresentasikan baris dalam spreadsheet, sedangkan \texttt{Cell} digunakan untuk mengisi nilai dalam setiap kolom, seperti URL, sumber tautan, dan status HTTP.

  \item \texttt{FileOutputStream}\\
  Digunakan untuk menuliskan isi \texttt{Workbook} ke berkas fisik dalam format \texttt{XLSX}. Setelah proses penulisan selesai, \texttt{Workbook} ditutup agar sumber daya tidak bocor.

\end{itemize}

Dengan penggunaan \texttt{XSSF} dan SS UserModel, sistem dapat membentuk berkas \texttt{XLSX} tanpa perlu menangani detail internal format OOXML. Seluruh pembuatan \textit{sheet}, baris, dan sel dilakukan melalui antarmuka yang telah distandarkan oleh Apache POI, sehingga implementasi ekspor menjadi lebih sederhana.
