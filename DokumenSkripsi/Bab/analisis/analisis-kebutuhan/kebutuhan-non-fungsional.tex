
Selain fungsi inti, sistem juga harus memenuhi sejumlah kebutuhan non-fungsional agar pemeriksaan tautan dapat berjalan stabil, efisien, dan mudah digunakan. Kebutuhan non-fungsional yang ditetapkan adalah sebagai berikut:

\begin{enumerate}[itemsep=3pt]
    \item \textbf{Antarmuka pengguna}\\
    Sistem dibangun melalui aplikasi desktop berbasis JavaFX dengan tampilan yang sederhana, tabel yang mudah dibaca, serta kontrol proses yang jelas.
    
    \item \textbf{Pengendalian laju}\\
    Sistem membatasi jumlah permintaan dalam satuan waktu (\textit{rate limiting}) untuk mencegah \textit{server} tujuan terbebani dan mengurangi risiko pemblokiran.

    \item \textbf{Eksekusi paralel}\\
    Sistem mendukung eksekusi permintaan HTTP secara paralel khusus pada tahap pemeriksaan tautan agar proses lebih cepat.

    \item \textbf{Ketahanan terhadap HTML tidak valid}\\
    Sistem mampu mengurai dokumen HTML yang tidak sesuai standar dengan bantuan Jsoup, sehingga tautan tetap dapat diekstrak meskipun struktur dokumen bermasalah.

    \vspace{20mm}
    \item \textbf{Konsistensi identifikasi URL}\\
    Setiap URL harus dinormalisasi dan dicatat dalam penyimpanan terpusat dan unik untuk memastikan satu sumber daya hanya diperiksa sekali, sehingga hasil pemeriksaan konsisten dan tidak ada duplikasi.

    \item \textbf{Etika \textit{crawling}}\\
    Setiap permintaan HTTP harus menyertakan identitas aplikasi melalui \texttt{User-Agent}.

\end{enumerate}