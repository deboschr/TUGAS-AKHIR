
Selain fungsi inti, sistem juga harus memenuhi sejumlah kebutuhan non-fungsional agar pemeriksaan tautan dapat berjalan stabil, efisien, dan mudah digunakan. Kebutuhan non-fungsional yang ditetapkan adalah sebagai berikut:

\begin{enumerate}
    \item \textbf{Antarmuka pengguna}\\
    Sistem disajikan dalam bentuk aplikasi desktop berbasis JavaFX dengan tampilan yang sederhana, tabel yang mudah dibaca, serta kontrol proses yang jelas.

    \item \textbf{Kinerja pemeriksaan}\\
    Sistem mampu menangani jumlah tautan yang besar dengan tetap menjaga waktu tanggap yang wajar. Hal ini dicapai dengan penerapan batas waktu (\textit{timeout}) pada setiap permintaan.

    \item \textbf{Pengendalian laju}\\
    Sistem membatasi jumlah permintaan dalam satuan waktu (\textit{rate limiting}) untuk mencegah server tujuan terbebani dan mengurangi risiko pemblokiran.

    \item \textbf{Eksekusi paralel}\\
    Sistem mendukung eksekusi permintaan secara paralel khusus pada tahap pemeriksaan tautan agar proses lebih cepat.

    \item \textbf{Ketahanan terhadap HTML tidak valid}\\
    Sistem mampu mengurai dokumen HTML yang tidak sesuai standar dengan bantuan Jsoup, sehingga tautan tetap dapat diekstrak meskipun struktur dokumen bermasalah.

    \item \textbf{Konsistensi identifikasi URL}\\
    Setiap URL dinormalisasi dan dicatat dalam repository untuk memastikan satu sumber daya hanya diperiksa sekali, sehingga hasil pemeriksaan konsisten dan tidak ada duplikasi.

    \item \textbf{Etika \textit{crawling}}\\
    Setiap permintaan menyertakan identitas \texttt{User-Agent}. Sistem juga menyediakan pengaturan jeda permintaan untuk menjaga agar aktivitas pemeriksaan tidak membebani server.

\end{enumerate}


\vspace{30mm}