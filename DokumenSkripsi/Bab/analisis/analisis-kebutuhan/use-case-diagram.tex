Gambaran umum dari kebutuhan sistem dapat divisualisasikan dalam bentuk diagram \textit{use case} sebagaimana ditunjukkan pada Gambar~\ref{fig:use-case-diagram-applikasi}. Diagram tersebut memperlihatkan aktor utama yaitu pengguna yang berinteraksi langsung dengan perangkat lunak yang dikembangkan. Terdapat dua layanan inti yang dapat dilakukan pengguna, yaitu melakukan pemeriksaan tautan rusak pada sebuah situs web dan mengekspor hasil pemeriksaan ke dalam berkas eksternal. 


\begin{figure}[H]
    \centering
    \includegraphics[width=0.55\textwidth]{Gambar/030301-use-case-diagram.png}
    \caption{Use case diagram aplikasi}
    \label{fig:use-case-diagram-applikasi}
\end{figure}

Berikut adalah penjelasan untuk setiap fitur berdasarkan Gambar~\ref{fig:use-case-diagram-applikasi}:
\begin{enumerate}
    \item \textbf{Memeriksa Tautan Rusak}\\  
    Fitur ini digunakan untuk melakukan pemeriksaan terhadap seluruh tautan yang terdapat dalam sebuah situs web untuk mendeteksi tautan rusak (\textit{broken links}).  
    
    \begin{itemize}[itemsep=1mm]
        \item Nama: Memeriksa Tautan Rusak
        \item Aktor: Pengguna
        \item Deskripsi: Memeriksa tautan rusak yang ada pada sebuah situs web.
        \item Kondisi awal: Aplikasi telah dijalankan dan pengguna berada pada jendela utama.
        \item Kondisi akhir: Daftar tautan rusak ditampilkan di tabel hasil pemeriksaan.
        \item Skenario utama: Ditampilkan dalam tabel \ref{tab:skenario-01}
        \begin{table}[H]
            \centering
            \caption{Skenario memeriksa tautan rusak}
            \vspace{6pt}
            \begin{tabular}{|p{0.5cm} |p{6cm}| p{6cm}|}\hline
                No & Aksi Aktor & Reaksi Sistem \\ \hline
                1 & Pengguna memasukkan URL ke dalam kolom masukan dan menekan tombol \textit{Check}. & Sistem melakukan proses \textit{web crawling}, mengekstrak seluruh tautan yang ditemukan dan mengecek kode status HTTP dari setiap tautan. \\ \hline
                2 & & Sistem menampilkan daftar tautan rusak yang ditemukan dan ringkasan dari proses pengecekan. \\ \hline
            \end{tabular}
            \label{tab:skenario-01}
        \end{table}

    \end{itemize}
    

    \item \textbf{Menghentikan pemeriksaan}\\
    Fitur ini digunakan untuk menghentikan proses pemeriksaan tautan rusak ketika proses pemeriksaan sedang berlangsung.
    \begin{itemize}[itemsep=1mm]
        \item Nama: Menghentikan Pemeriksaan
        \item Aktor: Pengguna
        \item Deskripsi: Mengentikan proses pemeriksaan.
        \item Kondisi awal: Proses pemeriksaan sedang berlangsung (status aplikasi \textit{checking}).
        \item Kondisi akhir: Proses pemeriksaan berhenti dan status menjadi \textit{stopped}.
        \item Skenario utama: Ditampilkan dalam tabel \ref{tab:skenario-02}

        \begin{table}[H]
            \centering
            \caption{Skenario menghentikan pemeriksaan}
            \vspace{6pt}
            \begin{tabular}{|p{0.5cm} |p{6cm}| p{6cm}|}\hline
                No & Aksi Aktor & Reaksi Sistem \\ \hline
                1 & Pengguna menekan tombol \textit{stop} & Sistem mengantikan proses \textit{crawling} dan pemeriksaan.\\ \hline
                2 & & Sistem memperbarui status pemeriksaan menjadi \textit{stopped}. \\ \hline
            \end{tabular}
            \label{tab:skenario-02}
        \end{table}
    \end{itemize}


    \item \textbf{\textit{Filter} Hasil}\\
    Fitur ini digunakan untuk menyaring hasil pemeriksaan sehingga pengguna dapat menemukan tautan rusak dengan jenis kesalahan atau URL tertentu.
    \begin{itemize}[itemsep=1mm]
        \item Nama: \textit{Filter} Hasil
        \item Aktor: Pengguna
        \item Deskripsi: Melakukan penyaringan pada hasil pemeriksaan.
        \item Kondisi awal: Terdapat hasil pemeriksaan pada tabel hasil.
        \item Kondisi akhir: Tabel hasil menampilkan daftar tautan rusak berdasarkan ketentuan penyaringan.
        \item Skenario utama: Ditampilkan dalam tabel \ref{tab:skenario-03}
        \begin{table}[H]
            \centering
            \caption{Skenario filter hasil}
            \vspace{6pt}
            \begin{tabular}{|p{0.5cm} |p{6cm}| p{6cm}|}\hline
                No & Aksi Aktor & Reaksi Sistem \\ \hline
                1 & Pengguna memilih opsi pencarian & \\ \hline
                2 & Pengguna memasukkan kata kunci pencarian & Sistem menyaring hasil pemeriksaan sesuai masukan pengguna \\ \hline
                3 & & Sistem menampilkan hasil penyaringan ke tabel hasil. \\ \hline
            \end{tabular}
            \label{tab:skenario-03}
        \end{table}
    \end{itemize}
    
    \vspace{5mm}

    \item \textbf{Ekspor Hasil}\\
    Fitur ini digunakan untuk mengekspor hasil pemeriksaan tautan rusak ke dalam berkas eksternal untuk keperluan dokumentasi atau analisis lebih lanjut.
    \begin{itemize}[itemsep=1mm]
        \item Nama: Ekspor Hasil
        \item Aktor: Pengguna
        \item Deskripsi: Ekspor hasil pemeriksaan tautan rusak ke berkas dalam format Excel.
        \item Kondisi awal: Pemeriksaan tautan telah selesai dan hasilnya telah ditampilkan di tabel.
        \item Kondisi akhir: Berkas hasil ekspor tersimpan di perangkat pengguna.
        \item Skenario utama: Ditampilkan dalam tabel \ref{tab:skenario-04}
        
        \vspace{-5mm}

        \begin{table}[H]
            \centering
            \caption{Skenario ekspor hasil}
            \vspace{6pt}
            \begin{tabular}{|p{0.5cm} |p{6cm}| p{6cm}|}\hline
                No & Aksi Aktor & Reaksi Sistem \\ \hline
                1 & Pengguna menekan tombol \textit{Export}. & Sistem membuka dialog penyimpanan berkas. \\ \hline
                2 & Pengguna menentukan nama dan lokasi penyimpanan berkas. & Sistem memproses data hasil pemeriksaan menjadi format \texttt{xlsx}. \\ \hline
                3 & & Sistem menyimpan berkas hasil ekspor di lokasi yang telah dipilih. \\ \hline
            \end{tabular}
            \label{tab:skenario-04}
        \end{table}
    \end{itemize}
\end{enumerate}


