Jsoup adalah sebuah pustaka Java yang berfungsi sebagai HTML \textit{parser} untuk memproses dokumen HTML sehingga dapat diolah dalam struktur yang lebih teratur. Dokumen yang diproses oleh Jsoup dapat berasal dari tiga sumber utama, yaitu halaman web yang diakses melalui URL, berkas yang tersimpan secara lokal, dan \texttt{String} yang berisi kode HTML. Pustaka ini dirancang agar toleran terhadap dokumen HTML yang tidak valid atau tidak terstruktur, sehingga hasil \textit{parsing} selalu dalam bentuk \textit{Document Object Model} (DOM) yang telah diperbaiki. Berikut adalah beberapa kelas yang tersedia pada pustaka Jsoup:


\subsection{Kelas \texttt{Jsoup}}
\label{subsec:020601-kelas-jsoup}
Kelas \texttt{Jsoup} merupakan titik masuk utama untuk menggunakan pustaka Jsoup. Berikut adalah beberapa metode yang tersedia pada kelas ini:

\begin{itemize}[itemsep=0mm]
    \item \texttt{connect}\\
    Metode ini digunakan untuk membuat koneksi HTTP menuju URL situs web tujuan.
    
    \item \texttt{parse}\\
    Metode ini digunakan untuk melakukan \textit{parsing} pada \texttt{String} HTML menjadi objek \texttt{Document}.
    
    \item \texttt{parseBodyFragment}\\
    Metode ini digunakan untuk melakukan \textit{parsing} pada potongan HTML parsial dan menghasilkan objek \texttt{Document}.
\end{itemize}


\subsection{Kelas \texttt{Document}}
\label{subsec:020602-kelas-document}
Kelas \texttt{Document} merupakan representasi dari keseluruhan dokumen HTML setelah proses \textit{parsing}. Kelas ini berfungsi sebagai akar struktur DOM dan menyediakan metode untuk membaca, menelusuri, atau memodifikasi isi dokumen. Berikut adalah beberapa metode yang tersedia pada kelas ini:

\begin{itemize}[itemsep=0mm]
    \item \texttt{title}\\
    Metode ini digunakan untuk mendapatkan elemen \texttt{<title>} dari dokumen.
    
    \item \texttt{body}\\
    Metode ini digunakan untuk mendapatkan elemen \texttt{<body>} dari dokumen.
    
    \item \texttt{select}\\
    Metode ini digunakan untuk mencari elemen dalam dokumen berdasarkan CSS \textit{selector}.
    
    \item \texttt{baseUri}\\
    Metode ini digunakan untuk mendapatkan URL dasar dari dokumen, jika sumbernya diambil melalui URL.
    
\end{itemize}


\subsection{Kelas \texttt{Element}}
\label{subsec:020603-kelas-element}
Kelas \texttt{Element} merupakan representasi satu elemen HTML dalam struktur DOM. Kelas ini memiliki \textit{tag}, atribut, konten teks, serta daftar elemen anak. Berikut adalah beberapa metode yang tersedia pada kelas ini:

\begin{itemize}[itemsep=0mm]
    \item \texttt{attr}\\
    Metode ini digunakan untuk mengambil nilai suatu atribut.
    
    \item \texttt{text}\\
    Metode ini digunakan untuk mendapatkan teks dalam elemen beserta elemen turunannya.
    
    \item \texttt{absUrl}\\
    Metode ini digunakan untuk mendapatkan URL absolut dari sebuah atribut, seperti \texttt{href} atau \texttt{src}, berdasarkan URI dasar dokumen.
    
    \item \texttt{html}\\
    Metode ini digunakan untuk memperoleh markup HTML bagian dalam elemen.
    
    \item \texttt{children}\\
    Metode ini digunakan untuk mendapatkan daftar elemen anak dari sebuah elemen tertentu.
\end{itemize}