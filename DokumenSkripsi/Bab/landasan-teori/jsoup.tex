Jsoup adalah sebuah pustaka Java yang berfungsi sebagai HTML \textit{parser} untuk memproses dokumen HTML sehingga dapat diolah dalam struktur yang lebih teratur. Dokumen yang diproses oleh Jsoup dapat berasal dari tiga sumber utama, yaitu halaman web yang diakses melalui URL, berkas yang tersimpan secara lokal, dan \texttt{String} yang berisi kode HTML. Pustaka ini dirancang agar toleran terhadap dokumen HTML yang tidak valid atau tidak terstruktur, sehingga hasil \textit{parsing} selalu dalam bentuk \textit{Document Object Model} (DOM) yang telah diperbaiki. Berikut adalah beberapa kelas yang tersedia pada pustaka Jsoup:

\vspace{-2mm}

\subsection{Kelas \texttt{Jsoup}}
\label{subsec:020601-kelas-jsoup}
Kelas \texttt{Jsoup} merupakan titik masuk utama untuk menggunakan pustaka Jsoup. Berikut adalah beberapa metode yang tersedia pada kelas ini:

\begin{itemize}
    \item \texttt{connect}: Metode ini digunakan untuk membuat koneksi HTTP menuju URL situs web tujuan.
    
    \item \texttt{parse}: Metode ini digunakan untuk melakukan \textit{parsing} pada \texttt{String} HTML menjadi objek \texttt{Document}.
    
    \item \texttt{parseBodyFragment}: Metode ini digunakan untuk melakukan \textit{parsing} pada potongan HTML parsial dan menghasilkan objek \texttt{Document}.
    
\end{itemize}


\subsection{Kelas \texttt{Document}}
\label{subsec:020602-kelas-document}
Kelas \texttt{Document} merupakan representasi dari keseluruhan dokumen HTML setelah proses \textit{parsing}. Kelas ini berfungsi sebagai akar struktur DOM dan menyediakan metode untuk membaca, menelusuri, atau memodifikasi isi dokumen. Berikut adalah beberapa metode yang tersedia pada kelas ini:

\begin{itemize}[itemsep=5pt]
    \item \texttt{title}: Metode ini digunakan untuk mendapatkan elemen \texttt{<title>} dari dokumen.
    
    \item \texttt{body}: Metode ini digunakan untuk mendapatkan elemen \texttt{<body>} dari dokumen.
    
    \item \texttt{select}: Metode ini digunakan untuk mencari elemen dalam dokumen berdasarkan CSS \textit{selector}.
    
    \item \texttt{baseUri}: Metode ini digunakan untuk mendapatkan URL dasar dari dokumen, jika sumbernya diambil melalui URL.
    
\end{itemize}

\vspace{5mm}

\subsection{Kelas \texttt{Element}}
\label{subsec:020603-kelas-element}
Kelas \texttt{Element} merupakan representasi satu elemen HTML dalam struktur DOM. Kelas ini memiliki \textit{tag}, atribut, konten teks, serta daftar elemen anak. Berikut adalah beberapa metode yang tersedia pada kelas ini:

\begin{itemize}[itemsep=5pt]
    \item \texttt{attr}: Metode ini digunakan untuk mengambil nilai suatu atribut.
    
    \item \texttt{text}: Metode ini digunakan untuk mendapatkan teks dalam elemen beserta elemen turunannya.
    
    \item \texttt{absUrl}: Metode ini digunakan untuk mendapatkan URL absolut dari sebuah atribut, seperti \texttt{href} atau \texttt{src}, berdasarkan URI dasar dokumen.
    
    \item \texttt{html}: Metode ini digunakan untuk memperoleh markup HTML bagian dalam elemen.
    
    \item \texttt{select}: Metode ini digunakan untuk mencari elemen turunan berdasarkan CSS \textit{selector}.
    
    \item \texttt{children}: Metode ini digunakan untuk mendapatkan daftar elemen anak dari sebuah elemen tertentu.
    
\end{itemize}

\vspace{5mm}

\subsection{Contoh Kode Program}
\label{subsec:0227-contoh-kode-program}

Kode~\ref{lst:jsoup-example} menunjukkan contoh penggunaan Jsoup untuk mengambil sebuah halaman web, memprosesnya menjadi objek \texttt{Document}, dan mengekstraksi sejumlah informasi dari dokumen tersebut. Contoh ini menggambarkan bagaimana kelas-kelas utama yang telah dijelaskan sebelumnya, yaitu \texttt{Jsoup}, \texttt{Connection}, \texttt{Connection.Response}, \texttt{Document}, \texttt{Elements}, dan \texttt{Element}, digunakan secara bersama-sama dalam sebuah program nyata. Hasil eksekusi dari kode ini berupa informasi judul halaman, URI dasar dokumen, serta daftar tautan beserta teks, URL absolut, dan atribut lain yang terkait.

\vspace{20mm}

\begin{lstlisting}[language=Java, caption={Contoh penggunaan Jsoup}, label={lst:jsoup-example}]
public class JsoupExample {
    public static void main(String[] args) throws Exception {
        Connection connection = Jsoup.connect("https://www.example.com").userAgent("BrokenLinkChecker 1.0").timeout(5000);
        Connection.Response response = connection.execute();

        System.out.println("Status Code : " + response.statusCode());
        System.out.println("Content-Type: " + response.headers().get("Content-Type"));

        Document doc = response.parse();

        System.out.println("Title   : " + doc.title());
        System.out.println("Base URI: " + doc.baseUri());

        Elements links = doc.select("a[href]");
        for (Element link : links) {
            System.out.println("Teks   : " + link.text());
            System.out.println("Href   : " + link.attr("href"));
            System.out.println("AbsURL : " + link.absUrl("href"));
        }
    }
}
\end{lstlisting}

\vspace{5mm}

Alur dari Kode~\ref{lst:jsoup-example} dimulai dengan pembuatan objek \texttt{Connection} melalui pemanggilan metode \texttt{Jsoup.connect()}, kemudian ditetapkan nilai \textit{User-Agent} dengan \texttt{userAgent()} dan batas waktu koneksi dengan \texttt{timeout()}. Permintaan dieksekusi melalui \texttt{execute()} untuk menghasilkan objek \texttt{Connection.Response}, dari mana dapat diperoleh kode status dengan \texttt{statusCode()} dan \textit{header} tertentu melalui \texttt{headers()}. Isi respons kemudian diproses menjadi objek \texttt{Document} menggunakan \texttt{parse()}, yang selanjutnya menyediakan informasi judul halaman dengan \texttt{title()} dan URI dasar dengan \texttt{baseUri()}. Setelah itu, method \texttt{select()} digunakan untuk memilih seluruh elemen \texttt{<a>} yang memiliki atribut \texttt{href}, menghasilkan koleksi \texttt{Elements}. Koleksi ini diiterasi, dan untuk setiap \texttt{Element} diperoleh teks melalui \texttt{text()}, nilai atribut melalui \texttt{attr("href")}, serta URL absolut melalui \texttt{absUrl("href")}. Dengan demikian, kode ini tidak hanya menampilkan daftar tautan, tetapi juga memperlihatkan cara menggunakan beberapa method penting yang telah dijelaskan sebelumnya.

