Apache POI merupakan pustaka Java yang dikembangkan oleh The Apache Software Foundation untuk memanipulasi berbagai format berkas Microsoft Office seperti Word, Excel, maupun PowerPoint. Pustaka ini dapat memproses berkas dengan standar OLE 2 Compound Document (OLE2) maupun Office Open XML (OOXML), sehingga memungkinkan pemrosesan dokumen Microsoft Office generasi lama maupun generasi yang baru.


OLE2 merupakan standar penyimpanan biner yang digunakan oleh Microsoft Office untuk berkas-berkas generasi lama seperti DOC, XLS, dan PPT. Standar ini mendefinisikan struktur penyimpanan berbentuk \textit{compound file}, yaitu sistem berkas yang terdiri atas \textit{storages} dan \textit{streams}, di mana \textit{streams} menyimpan data utama dokumen dan \textit{storages} menjadi kontainer hierarkis yang memuat elemen-elemen lain di dalamnya~\cite{ms:cfb}. Sementara itu, OOXML merupakan standar yang digunakan oleh Microsoft Office untuk berkas-berkas generasi baru seperti DOCX, XLSX, dan PPTX. Standar ini mendefinisikan dokumen sebagai kumpulan berkas XML yang dikemas dalam struktur ZIP~\cite{ecma:376}.


Struktur proyek Apache POI terdiri dari beberapa modul yang masing-masing dirancang untuk menangani format berkas tertentu dalam ekosistem Microsoft Office. Berikut adalah modul-modul tersebut:
\begin{itemize}
    \item \textbf{\texttt{POIFS}}: Modul ini digunakan untuk membaca dan menulis struktur penyimpanan pada standar OLE2 sebagai kumpulan \textit{storages} dan \textit{streams}.
    
    \item \textbf{\texttt{HSSF}}: Modul ini digunakan untuk memproses isi berkas Microsoft Excel dengan format XLS pada standar OLE2.
    
    \item \textbf{\texttt{XSSF}}: Modul ini digunakan untuk memproses berkas Microsoft Excel dengan format XLSX pada standar OOXML.

    \item \textbf{\texttt{HWPF}}: Modul ini digunakan untuk memproses isi berkas Microsoft Word dengan format DOC pada standar OLE2.
    
    \item \textbf{\texttt{XWPF}}: Modul ini digunakan untuk memproses berkas Microsoft Word dengan format DOCX pada standar OOXML.
    
    \item \textbf{\texttt{HSLF}}: Modul ini digunakan untuk memproses isi berkas Microsoft PowerPoint dengan format PPT pada standar OLE2.
    
    \item \textbf{\texttt{XSLF}}: Modul ini digunakan untuk memproses berkas Microsoft PowerPoint dengan format PPTX pada standar OOXML.
    
\end{itemize}

Untuk mendukung pemrosesan berkas \textit{spreadsheet} pada modul \texttt{HSSF} dan \texttt{XSSF}, Apache POI menyediakan sebuah API bernama SS UserModel melalui paket \texttt{org.apache.poi.ss.usermodel}. API ini mendefinisikan elemen-elemen umum \textit{spreadsheet} seperti \texttt{Workbook}, \texttt{Sheet}, \texttt{Row}, dan \texttt{Cell}. Melalui API ini, pemrosesan \textit{spreadsheet} dapat dilakukan menggunakan antarmuka yang sama pada kedua modul, tanpa perlu menangani perbedaan struktur antara berkas dengan format XLS dan XLSX secara langsung.
