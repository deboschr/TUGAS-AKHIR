Apache POI merupakan pustaka berbasis Java yang disediakan oleh The Apache Software Foundation untuk memanipulasi berbagai format berkas Microsoft Office. Dokumentasi resminya menyatakan bahwa proyek Apache POI bertujuan untuk menyediakan dan memelihara API Java yang mampu memproses format berkas yang didasarkan pada standar Office Open XML (OOXML) serta format OLE2 Compound Document. Dengan demikian, pustaka ini memungkinkan pengembang untuk membaca, membuat, dan memodifikasi berkas seperti Word, Excel, maupun PowerPoint melalui antarmuka pemrograman berbasis Java.

Untuk berkas spreadsheet, Apache POI menyediakan dua implementasi utama, yaitu HSSF dan XSSF. Komponen HSSF digunakan untuk menangani format Excel 97--2003 (XLS), sedangkan XSSF digunakan untuk format Excel 2007 ke atas (XLSX) yang berbasis OOXML. Selain kedua komponen tersebut, Apache POI juga menyediakan SS UserModel, yaitu API tingkat tinggi yang menyediakan antarmuka umum untuk bekerja dengan spreadsheet tanpa bergantung pada format file tertentu. Melalui antarmuka seperti \texttt{Workbook}, \texttt{Sheet}, \texttt{Row}, dan \texttt{Cell}, SS UserModel memungkinkan pengelolaan struktur spreadsheet secara konsisten pada kedua format yang berbeda tersebut.

Arsitektur modul Apache POI dirancang agar fleksibel dan dapat diperluas, sehingga setiap komponen memiliki tanggung jawab yang jelas terhadap format berkas tertentu. Dengan dukungan terhadap standar OOXML dan OLE2, POI menjadi salah satu pustaka Java yang paling banyak digunakan untuk integrasi dokumen Office dalam berbagai aplikasi, baik dalam konteks pembuatan laporan, pengolahan data terstruktur, maupun otomasi dokumen.
