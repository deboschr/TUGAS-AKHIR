Java HTTP Client API adalah sebuah API dari Java yang digunakan untuk mengirim \textit{request} dan menerima \textit{response} melalui protokol HTTP. API ini menyediakan dua model komunikasi, yaitu sinkron dan asinkron. Komunikasi sinkron berarti eksekusi program akan menunggu hingga \textit{response} dari \textit{server} diterima sepenuhnya sebelum melanjutkan instruksi berikutnya. Sebaliknya, komunikasi asinkron menggunakan kelas \texttt{CompletableFuture}, yang memungkinkan hasil komputasi diperoleh di masa mendatang tanpa harus menunggu proses selesai. 

Objek pada kelas \texttt{HttpClient} juga memiliki karakteristik yang penting untuk penggunaan di lingkungan \textit{multithread}. Objek ini bersifat \textit{immutable}, artinya konfigurasi tidak dapat diubah setelah dibuat, serta bersifat \textit{thread-safe}, artinya dapat diakses secara bersamaan oleh beberapa \textit{thread} tanpa menimbulkan \textit{race condition}.

\subsection{Kelas \texttt{HttpClient}}
\label{subsec:020701-kelas-httpclient}
Kelas \texttt{HttpClient} merupakan komponen inti dalam Java HTTP Client API yang digunakan untuk mengirim \textit{request} HTTP dan menerima \textit{response} dari \textit{server}. Objek dari kelas ini dapat dibuat dengan menggunakan metode \texttt{newHttpClient} dengan konfigurasi bawaan yang siap pakai, atau bisa menggunakan metode \texttt{newBuilder} apabila ingin membuat objek dengan konfigurasi yang dapat disesuaikan.

Berikut adalah beberapa metode yang tersedia pada \texttt{HttpClient}:
\begin{itemize}[itemsep=5pt]
    \item \texttt{version}\\
    Metode ini digunakan untuk menetapkan versi protokol HTTP yang digunakan oleh \textit{client}. Pengaturan versi ini memungkinkan aplikasi untuk menentukan apakah komunikasi dilakukan menggunakan HTTP/1.1 atau HTTP/2 sesuai dengan dukungan \textit{server} dan kebutuhan sistem.
    
    \item \texttt{connectTimeout}\\
    Metode ini digunakan untuk menetapkan batas waktu maksimal dalam proses pembentukan koneksi ke \textit{server}. Apabila koneksi tidak berhasil dibangun dalam rentang waktu yang ditentukan, permintaan akan dihentikan dan dianggap gagal.
    
    \item \texttt{cookieHandler}\\
    Metode ini digunakan untuk mengatur mekanisme pengelolaan \textit{cookie} pada komunikasi HTTP. Dengan konfigurasi ini, \textit{client} dapat menyimpan dan mengirim kembali \textit{cookie} yang diberikan oleh \textit{server} pada permintaan berikutnya.

    \item \texttt{followRedirects}\\
    Metode ini digunakan untuk mengatur kebijakan penanganan \textit{redirect} yang diberikan oleh \textit{server}. Nilai \texttt{ALWAYS} digunakan untuk selalu mengikuti \textit{redirect}, \texttt{NEVER} untuk tidak mengikuti \textit{redirect} sama sekali, sedangkan \texttt{NORMAL} mengikuti \textit{redirect} kecuali dari URL dengan \textit{scheme} HTTPS ke HTTP.

    \item \texttt{authenticator}\\
    Metode ini digunakan untuk menangani proses autentikasi ketika \textit{server} memerlukan kredensial akses. Mekanisme ini memungkinkan \textit{client} untuk secara otomatis merespons permintaan autentikasi yang diberikan oleh \textit{server}.
\end{itemize}

\subsection{Kelas \texttt{HttpRequest}}
\label{subsec:020702-kelas-httprequest}
Kelas \texttt{HttpRequest} merepresentasikan sebuah \textit{request} HTTP yang akan dikirimkan ke \textit{server} menggunakan \texttt{HttpClient}. Kelas ini digunakan untuk mendefinisikan informasi yang berkaitan dengan permintaan HTTP, seperti URI tujuan, metode HTTP, \textit{header}, serta batas waktu permintaan. Untuk membangun objek kelas ini, digunakan metode \texttt{newBuilder}, sehingga konfigurasi \textit{request} dapat disesuaikan sebelum dikirim, tanpa memengaruhi pengaturan global pada objek \texttt{HttpClient}.

\vspace{3mm}

Berikut adalah beberapa metode yang tersedia pada \texttt{HttpRequest}:
\begin{itemize}[itemsep=2pt]
    \item \texttt{uri}\\
    Metode ini digunakan untuk menentukan alamat tujuan dari \textit{request} HTTP dalam bentuk objek \texttt{URI}.
    
    \item \texttt{header}\\
    Metode ini digunakan untuk menetapkan satu pasangan \textit{name-value} sebagai \textit{header}, namun jika ingin menetapkan beberapa \textit{header} sekaligus maka dapat digunakan metode \texttt{headers}.
    
    \item \texttt{timeout}\\
    Metode ini menerima parameter berupa objek \texttt{Duration} untuk menentukan batas waktu maksimal pemrosesan \textit{request}.

    \item \texttt{version}\\
    Metode ini digunakan untuk menentukan versi protokol HTTP yang akan digunakan pada proses \textit{request}.
    
    \item \texttt{method}\\
    Metode ini menetapkan jenis operasi HTTP yang akan dilakukan. Selain melalui metode ini, penetapan jenis operasi juga dapat dilakukan menggunakan metode yang lebih eksplisit, seperti \texttt{GET}, \texttt{POST}, \texttt{PUT}, dan \texttt{DELETE}.
    
\end{itemize}


\subsection{Kelas \texttt{HttpResponse}}
\label{subsec:020703-kelas-httpresponse}
Kelas \texttt{HttpResponse} merepresentasikan \textit{response} yang diterima setelah sebuah \textit{request} dieksekusi oleh \texttt{HttpClient}. Kelas ini menggunakan parameter generik \texttt{<T>} yang menunjukkan tipe data dari isi \textit{response body}. Nilai generik ini ditentukan oleh \texttt{HttpResponse.BodyHandler} yang digunakan saat mengirim \textit{request}.

Berikut adalah beberapa metode yang tersedia pada \texttt{HttpResponse}:
\begin{itemize}[itemsep=2pt]
    \item \texttt{statusCode}\\
    Metode ini digunakan untuk mendapatkan kode status HTTP yang dikirim oleh \textit{server}.
    
    \item \texttt{headers}\\
    Metode ini digunakan untuk mendapatkan daftar seluruh \textit{header} dari \textit{response}.
    
    \item \texttt{body}\\
    Metode ini digunakan untuk mendapatkan isi \textit{response body} dengan tipe data generik \texttt{<T>}.
    
    \item \texttt{previousResponse}\\
    Metode ini digunakan untuk mendapatkan \textit{response} sebelumnya apabila terjadi \textit{redirect}.
    
    \item \texttt{sslSession}\\
    Metode ini digunakan untuk mendapatkan informasi sesi SSL jika koneksi dilakukan melalui protokol HTTPS.
    
    \item \texttt{uri}\\
    Metode ini digunakan untuk mendapatkan URL tujuan akhir dari \textit{request}, termasuk setelah terjadi \textit{redirect}.
    
    \item \texttt{version}\\
    Metode ini digunakan untuk mendapatkan versi protokol HTTP yang digunakan pada komunikasi.

\end{itemize}

\vspace{50mm}