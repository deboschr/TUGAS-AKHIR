\section[URI]{URI~\cite{RFC3986}}
\label{sec:020200-uri}

\textit{Uniform Resource Identifier} (URI) adalah rangkaian karakter yang digunakan untuk mengidentifikasi suatu sumber daya, baik berupa entitas fisik maupun abstrak. URI menyediakan cara standar untuk merepresentasikan identitas sumber daya melalui sintaksis yang terstruktur dan dapat diterapkan dalam berbagai skema. Terdapat dua bentuk utama URI, yaitu \textit{Uniform Resource Locator} (URL) dan \textit{Uniform Resource Name} (URN). URL menunjukkan lokasi sumber daya serta mekanisme aksesnya, sedangkan URN berfungsi sebagai nama tetap yang tidak bergantung pada lokasi penyimpanan. Selain itu, URI dapat dibedakan menjadi bentuk hierarkis dan opak. URI hierarkis mengikuti struktur umum yang terdiri atas komponen \textit{authority} dan \textit{path}, sedangkan URI opak tidak mengikuti struktur tersebut dan diawali oleh skema yang langsung diikuti data, seperti pada skema \texttt{mailto}.

\subsection{Struktur Sintaks URI Hierarkis}
\label{subsec:0202-struktur-url}

URI hierarkis memiliki bentuk umum sebagai berikut:
\begin{center}
\texttt{scheme:[//authority]path[?query][\#fragment]}
\end{center}

Komponen-komponennya dijelaskan sebagai berikut:
\begin{itemize}
  \item \textbf{Scheme}\\
  Bagian ini berada di awal URI dan diakhiri dengan tanda titik dua (\texttt{:}). Scheme menyatakan mekanisme interpretasi atau protokol yang digunakan, seperti \texttt{http}, \texttt{https}, \texttt{ftp}, atau \texttt{mailto}.
  
  \item \textbf{Authority}\\
  Bagian ini bersifat opsional dan diawali dengan dua garis miring (\texttt{//}). Authority terdiri atas \textit{userinfo}, \textit{host}, dan \textit{port}. \textit{Userinfo} merupakan informasi tambahan yang dapat diberikan sebelum host. \textit{Host} dapat berupa nama domain atau alamat IP, sedangkan \textit{port} menunjukkan nomor port layanan. Jika port tidak dituliskan, port bawaan dari suatu scheme digunakan.
  
  \item \textbf{Path}\\
  Bagian ini menunjukkan jalur menuju suatu sumber daya. Path terdiri atas beberapa segmen yang dipisahkan oleh garis miring (\texttt{/}). Di dalamnya dapat terdapat \textit{dot-segments}, yaitu segmen khusus (\texttt{.} dan \texttt{..}) yang digunakan dalam mekanisme penyederhanaan jalur.
  
  \item \textbf{Query}\\
  Bagian ini bersifat opsional dan diawali dengan tanda tanya (\texttt{?}). Isi query berupa data tambahan yang interpretasinya bergantung pada skema atau aplikasi yang menggunakannya. Format umum pasangan \texttt{key=value} dengan pemisah \texttt{\&} merupakan konvensi yang banyak dipakai, meskipun tidak ditentukan dalam standar URI.
  
  \item \textbf{Fragment}\\
  Bagian ini bersifat opsional dan diawali dengan tanda pagar (\texttt{\#}). Fragment merujuk pada bagian tertentu dalam representasi sumber daya dan tidak dikirimkan kepada server.
\end{itemize}

\subsection{Kategori Karakter dan \textit{Percent-Encoding}}
\label{subsec:0202-karakter-percent-encoding}

Standar URI menetapkan beberapa kategori karakter yang menentukan cara representasinya dalam sebuah URI.
\begin{itemize}
  \item \textbf{Unreserved characters}\\
  Karakter ini dapat dituliskan secara langsung tanpa pengkodean. Termasuk di dalamnya huruf alfabet A--Z dan a--z, angka 0--9, serta karakter \texttt{-}, \texttt{\_}, \texttt{.}, dan \texttt{\textasciitilde}.
  
  \item \textbf{Reserved characters}\\
  Karakter ini memiliki fungsi khusus tergantung konteks. Karakter pemisah umum (general delimiters) meliputi \texttt{:}, \texttt{/}, \texttt{?}, \texttt{\#}, \texttt{[}, \texttt{]}, dan \texttt{@}. Karakter pemisah subkomponen (subcomponent delimiters) meliputi \texttt{!}, \texttt{\$}, \texttt{\&}, \texttt{'}, \texttt{(}, \texttt{)}, \texttt{*}, \texttt{+}, \texttt{,}, \texttt{;}, dan \texttt{=}. Jika karakter ini digunakan di luar konteks yang diperbolehkan, maka harus dikodekan.
  
  \item \textbf{Karakter lain}\\
  Karakter non-ASCII, spasi, atau simbol lain yang tidak termasuk dalam dua kategori sebelumnya harus direpresentasikan melalui mekanisme pengkodean.
\end{itemize}

Pengkodean karakter dilakukan melalui \textit{percent-encoding}, yaitu menggantikan suatu karakter dengan representasi heksadesimalnya dalam format \texttt{\%HH}. Karakter unreserved sebaiknya tetap dituliskan secara langsung tanpa dikodekan. Selain itu, digit heksadesimal dalam representasi \texttt{HH} harus dituliskan dalam huruf besar.

\subsection{Referensi Absolut dan Relatif}
\label{subsec:0202-referensi-url}

URI dapat berupa referensi absolut maupun referensi relatif. Referensi absolut mencantumkan scheme beserta komponen lain sehingga dapat ditafsirkan tanpa informasi tambahan. Referensi relatif tidak mencantumkan scheme dan dapat berupa path, query, atau fragment. Terdapat pula same-document reference, yaitu referensi yang hanya berisi fragment untuk merujuk pada bagian tertentu dalam dokumen yang sama.

Proses resolusi URI relatif terhadap suatu base URI dilakukan dengan menggabungkan komponen-komponennya, kemudian menyederhanakan hasilnya melalui mekanisme penghapusan \textit{dot-segments}. Tahap ini memastikan bahwa jalur yang dihasilkan berada dalam bentuk yang ringkas dan tidak ambigu.
