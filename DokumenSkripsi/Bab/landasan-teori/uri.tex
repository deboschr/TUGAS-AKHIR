\textit{Uniform Resource Identifier} (URI) adalah rangkaian karakter yang digunakan untuk mengidentifikasi suatu sumber daya, baik berupa entitas fisik maupun abstrak. URI menyediakan cara standar untuk merepresentasikan identitas sumber daya melalui sintaksis yang terstruktur. Terdapat dua bentuk utama URI, yaitu \textit{Uniform Resource Locator} (URL) dan \textit{Uniform Resource Name} (URN).  URL adalah URI yang menyertakan informasi lokasi dan mekanisme akses terhadap sumber daya, sedangkan URN adalah URI yang berfungsi sebagai nama tetap suatu sumber daya tanpa menyertakan lokasi penyimpanannya.


\subsection{Struktur URL}
\label{subsec:0202-struktur-url}
URL khusus untuk skema \texttt{http} atau \texttt{https} memiliki strukturnya tersendiri. Pada struktur URL ini terdapat tujuh komponen dan komponen yang ditandai dengan karakter kurung siku artinya komponen tersebut bersifat opsional. Berikut adalah struktur dari URL tersebut:

\vspace{5mm}

\begin{center}
\texttt{scheme ":" "//" [userinfo "@"] host [":" port] path-abempty ["?" query] ["\#" fragment]}
\end{center}

\vspace{5mm}

Komponen-komponen dalam URL tersebut dijelaskan sebagai berikut:
\begin{itemize}[itemsep=5pt]
  \item \textbf{\textit{Scheme}}\\
  Komponen ini bersifat wajib serta diakhiri dengan tanda titik dua (\texttt{:}) dan dua garis miring (\texttt{//}). \textit{Scheme} menunjukkan protokol yang digunakan untuk mengakses sumber daya. Pada URL khusus untuk protokol HTTP, nilai yang digunakan adalah \texttt{http} atau \texttt{https} yang didefinisikan dalam format penulisan \textit{lowercase}.

  \item \textbf{\textit{User Info}}\\
  Komponen ini bersifat opsional dan diakhiri dengan tanda at (\texttt{@}). \textit{User info} digunakan untuk menyertakan informasi identitas pengguna. Komponen ini jarang digunakan karena pertimbangan keamanan.

  \item \textbf{\textit{Host}}\\
  Komponen ini bersifat wajib dan menunjukkan lokasi \textit{server} tempat sumber daya berada. \textit{Host} dapat berupa nama domain dalam format penulisan \textit{lowercase}, alamat IPv4, atau alamat IPv6, dan digunakan oleh \textit{client} untuk menentukan tujuan koneksi.

  \item \textbf{\textit{Port}}\\
  Komponen ini bersifat opsional dan diawali dengan tanda titik dua (\texttt{:}). \textit{Port} menunjukkan nomor port pada \textit{host} yang digunakan untuk komunikasi. Jika tidak dituliskan, maka digunakan port standar sesuai dengan \textit{scheme}, yaitu port 80 untuk HTTP dan port 443 untuk HTTPS.

  \item \textbf{\textit{Path-abempty}}\\
  Komponen ini bersifat wajib pada URL HTTP dan diawali dengan tanda garis miring (\texttt{/}) atau dapat berupa \textit{string} kosong. \textit{Path-abempty} menunjukkan jalur menuju sumber daya pada \textit{server}. Jika tidak dituliskan, maka \textit{path} dianggap sebagai \textit{string} kosong.

  \item \textbf{\textit{Query}}\\
  Komponen ini bersifat opsional dan diawali dengan tanda tanya (\texttt{?}). \textit{Query} digunakan untuk menyertakan parameter tambahan dalam bentuk pasangan kunci dan nilai yang dipisahkan oleh karakter tertentu, selain itu \textit{query} juga digunakan untuk menyampaikan data ke \textit{server} tanpa mengubah struktur \textit{path}.

  \item \textbf{\textit{Fragment}}\\
  Komponen ini bersifat opsional dan diawali dengan tanda pagar (\texttt{\#}). \textit{Fragment} digunakan untuk merujuk ke bagian tertentu dari sumber daya. Informasi ini tidak dikirim ke \textit{server}, melainkan diproses oleh \textit{user agent} seperti \textit{browser}.
\end{itemize}


\vspace{5mm}

\subsection{Kategori Karakter dan \textit{Percent-Encoding}}
\label{subsec:0202-karakter-percent-encoding}

Kategori karakter yang dapat digunakan dalam URI dikelompokkan beserta aturan pengkodeannya sebagai berikut:
\begin{itemize}[itemsep=5pt]
  \item \textbf{\textit{Unreserved characters}}\\
  Karakter ini dapat digunakan langsung tanpa pengkodean tambahan. Termasuk di dalamnya huruf alfabet A sampai Z dan a sampai z, digit angka 0 sampai 9, serta simbol \texttt{-}, \texttt{\_}, \texttt{.}, dan \texttt{\textasciitilde}.
  
  \item \textbf{\textit{Reserved characters}}\\
  Karakter ini memiliki fungsi khusus tergantung pada konteks penggunaannya. 
  \textit{Reserved characters} dibagi menjadi dua kelompok yaitu \textit{general delimiters} dan \textit{subcomponent delimiters}. \textit{General delimiters} meliputi \texttt{:}, \texttt{/}, \texttt{?}, \texttt{\#}, \texttt{[}, \texttt{]}, dan \texttt{@}. \textit{Subcomponent delimiters} meliputi \texttt{!}, \texttt{\$}, \texttt{\&}, \texttt{'}, \texttt{(}, \texttt{)}, \texttt{*}, \texttt{+}, \texttt{,}, \texttt{;}, dan \texttt{=}. Jika karakter ini digunakan di luar konteks yang diperbolehkan, maka harus dikodekan.
  
  \item \textbf{Karakter lain}\\
  Karakter non-ASCII, spasi, dan simbol lain yang tidak termasuk dalam kategori \textit{Unreserved characters} dan \textit{Reserved characters} tidak dapat digunakan secara langsung. Karakter ini harus dikodekan sebelum dapat dimasukkan ke dalam URI.
\end{itemize}

\vspace{3mm}

Pengkodean karakter dilakukan dengan mekanisme \textit{percent-encoding}. Mekanisme ini menggantikan karakter dengan representasi heksadesimalnya dalam format \texttt{\%HH}, di mana \texttt{HH} adalah nilai ASCII karakter tersebut. Sebagai contoh, spasi ditulis sebagai \texttt{\%20}, tanda kutip ganda sebagai \texttt{\%22}, dan tanda pagar sebagai \texttt{\%23}. RFC~3986 juga mengatur bahwa karakter \textit{unreserved} yang tidak perlu dikodekan sebaiknya dituliskan dalam bentuk aslinya. Selain itu, digit heksadesimal dalam \textit{percent-encoding} harus ditulis dengan huruf besar.


\subsection{Referensi Absolut dan Relatif}
\label{subsec:0202-referensi-url}
URL dapat berupa referensi absolut maupun relatif. URL absolut mencantumkan seluruh komponen utama, termasuk \textit{scheme} dan \textit{host}, sehingga dapat diinterpretasikan secara mandiri tanpa bergantung pada konteks lain. Contoh URL absolut adalah \texttt{https://unpar.ac.id/page.html}. Sebaliknya, URL relatif tidak mencantumkan \textit{scheme} atau \textit{host} dan hanya menyertakan bagian seperti \textit{path}, \textit{query}, atau \textit{fragment}. Contoh URL relatif adalah \texttt{/images/logo.png}. Selain itu, terdapat pula \textit{same-document reference}, yaitu URL yang hanya berisi \textit{fragment} untuk merujuk ke bagian tertentu dalam dokumen yang sama, contohnya \texttt{\#section2}.

Agar referensi relatif dapat digunakan dalam komunikasi berbasis HTTP, referensi tersebut harus terlebih dahulu diubah menjadi URL absolut melalui suatu mekanisme resolusi. Proses resolusi dilakukan dengan menentukan sebuah base URI, kemudian menggabungkannya dengan referensi relatif yang digunakan untuk membentuk URL yang lengkap. Salah satu tahap penting dalam proses ini adalah penyederhanaan \textit{path}, termasuk penghapusan \textit{dot-segments} seperti \texttt{/./} dan \texttt{/../}, sehingga struktur \textit{path} menjadi ringkas dan tidak ambigu. Base URI umumnya diperoleh dari URL dokumen yang memuat referensi relatif tersebut, atau dari elemen \texttt{<base>} apabila didefinisikan dalam dokumen HTML. Mekanisme ini memastikan bahwa setiap referensi relatif dapat diinterpretasikan secara konsisten dan diarahkan ke sumber daya yang tepat.
