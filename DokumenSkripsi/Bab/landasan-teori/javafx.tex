JavaFX merupakan \textit{framework} antarmuka pengguna grafis untuk bahasa pemrograman Java. \textit{Framework} tersedia sebagai API publik dan dipandang sebagai penerus Swing dalam pengembangan antarmuka pengguna grafis pada platform Java. JavaFX menyediakan berbagai fitur untuk membangun antarmuka pengguna, antara lain dukungan data binding, kemampuan mendefinisikan antarmuka menggunakan kode Java maupun FXML sebagai bahasa markah berbasis XML, serta dukungan pengaturan tampilan dan gaya visual menggunakan CSS.

Antarmuka pengguna dalam JavaFX dibangun berdasarkan konsep \textit{scene graph}, yaitu struktur hierarkis yang merepresentasikan seluruh elemen antarmuka sebagai simpul (\texttt{Node}) dalam sebuah struktur pohon. Struktur ini memiliki satu simpul akar yang dimuat ke dalam objek \texttt{Scene}, kemudian ditampilkan melalui jendela aplikasi yang disebut \texttt{Stage}. Setiap \texttt{Node} dapat berupa elemen antarmuka seperti \texttt{Button}, \texttt{Label}, dan \texttt{TextField}, maupun berupa kontainer tata letak seperti \texttt{VBox}, \texttt{HBox}, dan \texttt{GridPane}, yang digunakan untuk menyusun elemen-elemen tersebut secara hierarkis.


\subsection{Siklus Hidup Aplikasi}
Setiap aplikasi JavaFX diturunkan dari kelas abstrak \texttt{javafx.application.Application}. Kelas ini mendefinisikan tiga metode utama yang membentuk siklus hidup aplikasi, yaitu:  

\begin{enumerate}
    \item \texttt{init()}\\
    Metode ini dipanggil pertama kali sebelum antarmuka pengguna dibuat. Umumnya digunakan untuk melakukan inisialisasi data awal atau membuka koneksi yang dibutuhkan.  

    \item \texttt{start(Stage stage)}\\
    Merupakan titik masuk utama aplikasi JavaFX. Pada tahap ini, objek \texttt{Scene} dibuat, elemen antarmuka pengguna dikonfigurasi, dan scene ditampilkan pada \texttt{Stage} utama.

    \item \texttt{stop()}\\
    Metode ini dipanggil ketika aplikasi ditutup. Biasanya digunakan untuk membersihkan sumber daya, menutup koneksi, atau menyimpan data sebelum aplikasi berhenti.  
\end{enumerate}


\subsection{Stage, Scene, dan Node}
JavaFX menyusun antarmuka pengguna berdasarkan hierarki yang dikenal dengan \textit{scene graph}. Tiga komponen utama dalam hierarki ini adalah sebagai berikut:

\begin{itemize}[itemsep=2pt]
    \item \texttt{Stage}\\
    \texttt{Stage} merupakan \textit{top-level container} dalam JavaFX yang merepresentasikan sebuah jendela aplikasi. Setiap aplikasi JavaFX memiliki setidaknya satu \textit{primary stage} yang disediakan secara otomatis oleh \textit{JavaFX runtime} dan diteruskan sebagai parameter ke dalam metode \texttt{start}. Objek \texttt{Stage} berfungsi sebagai wadah untuk menampilkan sebuah objek \texttt{Scene}.

    
    \item \texttt{Scene}\\
    \texttt{Scene} merupakan kontainer yang menampung seluruh struktur antarmuka pengguna dalam JavaFX. Setiap \texttt{Scene} memiliki tepat satu \textit{root node} yang menjadi akar dari struktur hierarkis antarmuka pengguna. Struktur ini berisi seluruh elemen antarmuka yang akan dirender dan ditampilkan pada sebuah \texttt{Stage}.
    
    \item \texttt{Node}\\
    \texttt{Node} merupakan kelas dasar (\textit{base class}) bagi seluruh elemen visual dalam JavaFX. Setiap elemen antarmuka pengguna, baik berupa kontrol antarmuka maupun kontainer tata letak, merupakan turunan dari kelas \texttt{Node}. Node disusun secara hierarkis dalam sebuah struktur yang disebut \textit{scene graph}, yang merepresentasikan keseluruhan antarmuka pengguna yang dimuat ke dalam sebuah \texttt{Scene}.
\end{itemize}


\subsection{Kontrol dan Tata Letak Antarmuka JavaFX}

JavaFX menyediakan berbagai elemen antarmuka pengguna yang terdiri atas kontrol antarmuka (\textit{controls}) dan kontainer tata letak (\textit{layout panes}) untuk membangun aplikasi interaktif. Beberapa elemen antarmuka yang umum digunakan antara lain sebagai berikut:

\begin{itemize}[itemsep=2pt]
    \item \texttt{TextField}, untuk menerima input teks satu baris.
    \item \texttt{TextArea}, untuk menerima input teks dalam bentuk multi-baris.
    \item \texttt{ComboBox}, untuk menampilkan pilihan dalam bentuk daftar tarik turun.
    \item \texttt{Button}, untuk memicu suatu aksi tertentu.
    \item \texttt{Label}, untuk menampilkan teks.
    \item \texttt{TableView}, untuk menyajikan data dalam bentuk baris dan kolom.
    \item \texttt{VBox}, \texttt{HBox}, dan \texttt{GridPane}, sebagai kontainer tata letak vertikal, horizontal, dan grid.
\end{itemize}

Elemen-elemen antarmuka tersebut dapat didefinisikan secara deklaratif menggunakan FXML, yaitu bahasa markah berbasis XML yang digunakan untuk mendeskripsikan struktur antarmuka pengguna secara terpisah dari logika aplikasi. Dalam berkas FXML, sebuah kelas \texttt{controller} dapat dihubungkan melalui atribut \texttt{fx:controller}, sedangkan elemen antarmuka yang perlu diakses dari kode program diberi pengenal menggunakan atribut \texttt{fx:id}. Selain itu, \textit{event handling} dapat didefinisikan secara deklaratif melalui atribut seperti \texttt{onAction}.

% Kode~\ref{lst:javafx-fxml} memperlihatkan contoh struktur antarmuka berbasis FXML yang menggunakan \texttt{BorderPane} sebagai \textit{root node} dan \texttt{HBox} sebagai kontainer tata letak.

% \begin{lstlisting}[language=XML, caption={Contoh struktur dasar dokumen FXML}, label=lst:javafx-fxml]
% <BorderPane xmlns:fx="http://javafx.com/fxml" fx:controller="Controller">
%   <top>
%     <HBox spacing="10">
%       <Label text="Input:"/>
%       <TextField fx:id="inputField"/>
%       <Button fx:id="submitButton" text="Submit" onAction="#handleSubmit"/>
%     </HBox>
%   </top>
% </BorderPane>
% \end{lstlisting}

% \vspace{3mm}

% Kode~\ref{lst:javafx-fxml} menunjukkan bahwa elemen akar antarmuka adalah \texttt{BorderPane}. Pada bagian atasnya terdapat sebuah \texttt{HBox} dengan jarak antar elemen sebesar 10 piksel. Di dalam \texttt{HBox} didefinisikan sebuah \texttt{Label} untuk menampilkan teks, sebuah \texttt{TextField} yang diberi atribut \texttt{fx:id} agar dapat diakses dari \texttt{controller}, serta sebuah \texttt{Button} yang memiliki atribut \texttt{onAction} untuk memanggil metode penanganan kejadian pada \texttt{controller}.


\subsection{\textit{Controller}, \textit{Property}, dan \textit{Binding}}

Dalam JavaFX, kelas \texttt{controller} digunakan sebagai penghubung antara antarmuka pengguna yang didefinisikan melalui FXML dengan logika aplikasi. Elemen antarmuka yang dideklarasikan di dalam berkas FXML dapat diakses dari kelas \texttt{controller} dengan menggunakan anotasi \texttt{@FXML}. Interaksi pengguna, seperti penekanan tombol, dapat ditangani melalui metode \textit{event handler} yang didefinisikan di dalam \texttt{controller}.

Selain itu, JavaFX menyediakan mekanisme \textit{property} untuk merepresentasikan data yang bersifat dinamis. \textit{Property} memungkinkan suatu nilai dipantau sehingga perubahan nilai tersebut dapat dideteksi dan dipropagasikan ke bagian lain dari aplikasi. Salah satu implementasi property yang umum digunakan adalah \texttt{StringProperty}, yang digunakan untuk menyimpan dan mengamati perubahan nilai bertipe \texttt{String}.

Untuk menghubungkan property dengan elemen antarmuka pengguna, JavaFX mendukung konsep \textit{binding}. \textit{Binding} memungkinkan nilai suatu property terhubung dengan property lain sehingga perubahan pada satu sisi akan secara otomatis memengaruhi sisi yang terikat. \textit{Binding} dapat bersifat satu arah (\textit{unidirectional}), di mana nilai mengalir dari sumber ke target, maupun dua arah (\textit{bidirectional}), di mana perubahan pada salah satu \textit{property} akan memperbarui \textit{property} lainnya.

% Kode~\ref{lst:javafx-controller-binding} memperlihatkan contoh kelas \texttt{Controller} yang menggunakan \textit{property} dan \textit{binding} untuk menghubungkan \texttt{TextField} dengan \texttt{Label} melalui \texttt{StringProperty}.

% \vspace{3mm}

% \begin{lstlisting}[language=Java, caption={Contoh controller dengan \textit{property} dan \textit{binding}}, label=lst:javafx-controller-binding]
% public class Controller {
%     @FXML private TextField inputField;
%     @FXML private Label statusLabel;
%     private StringProperty inputText = new SimpleStringProperty();

%     public void initialize() {
%         inputField.textProperty().bindBidirectional(inputText);
%         statusLabel.textProperty().bind(inputText);
%     }

%     @FXML
%     public void handleSubmit() {
%         inputText.set("Input diterima: " + inputText.get());
%     }
% }
% \end{lstlisting}


% Kode program pada Kode~\ref{lst:javafx-controller-binding} menunjukkan bahwa komponen \texttt{TextField} dan \texttt{Label} dihubungkan ke kelas \texttt{Controller} melalui anotasi \texttt{@FXML}. Sebuah \textit{property} berupa \texttt{StringProperty} digunakan untuk menyimpan nilai teks. Pada metode \texttt{initialize()}, property tersebut diikat secara dua arah dengan properti teks milik \texttt{TextField}, sehingga perubahan nilai pada salah satunya akan tercermin pada yang lain. Selanjutnya, properti teks milik \texttt{Label} diikat secara satu arah dengan property tersebut, sehingga nilai yang ditampilkan selalu sesuai. Ketika metode \texttt{handleSubmit()} dipanggil, nilai property diperbarui dan perubahan tersebut secara otomatis ditampilkan pada \texttt{Label} melalui mekanisme \textit{binding}.

