\textit{Hypertext Transfer Protocol} (HTTP) adalah protokol tingkat aplikasi yang menjadi dasar komunikasi pada arsitektur \textit{World Wide Web}. HTTP berjalan di atas protokol TCP dengan port standar 80. Untuk komunikasi yang membutuhkan perlindungan, digunakan dapat digunakan HTTPS, yaitu HTTP yang berjalan di atas \textit{Transport Layer Security} (TLS) pada port standar 443. Melalui HTTPS, komunikasi memperoleh jaminan kerahasiaan, autentikasi, dan integritas data.

HTTP bekerja dengan pola komunikasi berbasis \textit{client-server}, di mana sebuah \textit{client} mengirimkan \textit{request} dan \textit{origin server} menanggapinya dengan \textit{response}. Protokol ini bersifat \textit{stateless}, artinya setiap \textit{request} dapat dipahami secara terpisah tanpa ketergantungan pada interaksi sebelumnya, sehingga \textit{server} tidak diwajibkan menyimpan konteks antar \textit{request}. 

Tujuan dari sebuah \textit{request} adalah \textit{resource} yang disediakan oleh \textit{origin server} melalui URL. \textit{Origin server} adalah program yang dapat menghasilkan \textit{response} untuk sebuah \textit{resource}. HTTP tidak membatasi apa yang dimaksud dengan \textit{resource}, melainkan hanya menyediakan antarmuka untuk berinteraksi dengannya. Informasi mengenai \textit{resource} tersebut disampaikan dalam bentuk \textit{representation}, yaitu data beserta metadata yang mencerminkan keadaan \textit{resource} pada waktu tertentu dan dapat ditransmisikan melalui protokol.


\subsection{Metode HTTP}
\label{subsec:0201-metode-http}
Metode HTTP berfungsi untuk menunjukkan tujuan dari \textit{request} yang dibuat oleh \textit{client} dan hasil sukses apa yang diharapkan dari \textit{request} tersebut. Metode HTTP memiliki beberapa sifat umum, diantaranya adalah \textit{safe} dan \textit{idempotent}. Sebuah metode dikatakan \textit{safe} apabila semantik yang didefinisikan bersifat \textit{read-only}, yaitu \textit{client} tidak meminta dan tidak mengharapkan adanya perubahan keadaan pada \textit{origin server} akibat penerapan metode tersebut. Metode yang termasuk \textit{safe} adalah \texttt{GET}, \texttt{HEAD}, \texttt{OPTIONS}, dan \texttt{TRACE}. Sebuah metode disebut \textit{idempotent} apabila dampak yang dimaksudkan pada \textit{server} dari beberapa \textit{request} identik, sama dengan dampak dari satu \textit{request}. Metode \texttt{PUT}, \texttt{DELETE}, serta seluruh metode aman adalah \textit{idempotent}. 

\vspace{10mm}

Berikut ini adalah daftar beberapa metode HTTP:

\vspace{2mm}

\begin{itemize}[itemsep=5pt]
    \item \textbf{GET}: Metode ini digunakan untuk meminta transfer representasi terkini dari \textit{resource} target.
  
    \item \textbf{HEAD}: Metode ini identik dengan \texttt{GET}, tetapi \textit{server} tidak boleh mengirimkan konten dalam \textit{response}. \texttt{HEAD} digunakan untuk memperoleh metadata dari representasi yang dipilih tanpa harus mentransfer data representasi itu sendiri.
  
    \item \textbf{POST}: Metode ini digunakan untuk meminta \textit{resource} target memproses representasi yang disertakan dalam \textit{request} sesuai dengan semantik khusus yang dimiliki oleh \textit{resource} tersebut.
  
    \item \textbf{PUT}: Metode ini digunakan untuk meminta agar keadaan dari \textit{resource} target dibuat atau diganti dengan keadaan yang ditentukan oleh representasi yang disertakan dalam isi pesan \textit{request}.
  
    \item \textbf{DELETE}: Metode ini digunakan untuk meminta agar \textit{origin server} menghapus asosiasi antara \textit{resource} target dengan fungsionalitasnya saat ini.
  
    \item \textbf{OPTIONS}: Metode ini digunakan untuk meminta informasi mengenai opsi komunikasi yang tersedia bagi \textit{resource} target, baik pada \textit{origin server} maupun perantara. Metode ini memungkinkan \textit{client} mengetahui opsi dan/atau persyaratan yang terkait dengan sebuah \textit{resource}, atau kemampuan dari sebuah \textit{server}, tanpa menyiratkan adanya tindakan terhadap \textit{resource} tersebut.
  
\end{itemize}



\subsection{Kode Status HTTP}
\label{subsec:0201-kode-status-http}

Kode status HTTP adalah bagian dari baris awal pada \textit{response} \textit{server} yang menunjukkan hasil pemrosesan terhadap suatu \textit{request}. Kode ini terdiri dari tiga digit numerik dan dikelompokkan ke dalam lima kelas utama berdasarkan digit pertamanya: informasi (1xx), keberhasilan (2xx), pengalihan (3xx), kesalahan dari \textit{client} (4xx), dan kesalahan dari \textit{server} (5xx).

\subsubsection{\textit{Informational} 1xx}
\label{subsubsec:020104-infotmational-1xx}

Kode-kode pada kelas ini menunjukkan bahwa \textit{request} telah diterima dan sedang diproses, tetapi belum ada \textit{response} final. Berikut adalah daftar kode status pada kategori ini:

\begin{itemize}[itemsep=5pt]
    \item \textbf{100 (\textit{Continue})}: Menunjukkan bahwa \textit{server} telah menerima bagian awal dari \textit{request} dan tidak menemukan masalah pada tahap tersebut. \textit{Client} dapat melanjutkan pengiriman sisa \textit{request}, dan \textit{server} akan mengirimkan \textit{response} akhir setelah seluruh \textit{request} selesai diproses.
  
    \item \textbf{101 (\textit{Switching Protocols})}: Menunjukkan bahwa \textit{server} menyetujui permintaan \textit{client} untuk mengganti protokol aplikasi yang digunakan pada koneksi yang sama, sesuai dengan yang diminta dalam \textit{request}.
  
    \item \textbf{102 (\textit{Processing})}: Menunjukkan bahwa \textit{server} telah menerima \textit{request} secara lengkap, namun proses penanganannya masih berjalan dan belum dapat menghasilkan \textit{response} akhir.~\cite{RFC2518}
  
    \item \textbf{103 (\textit{Early Hints})}: Digunakan oleh \textit{server} untuk mengirimkan informasi awal kepada \textit{client} melalui \textit{header}, sebelum \textit{response} akhir tersedia, sehingga \textit{client} dapat mempersiapkan proses lanjutan lebih cepat.~\cite{RFC8297}
\end{itemize}

\subsubsection{\textit{Successful} 2xx}
\label{subsubsec:201004-successful-2xx}

Kode-kode pada kelas ini menunjukkan bahwa \textit{request} telah diterima, dipahami, dan diproses dengan sukses. Berikut adalah daftar kode status pada kategori ini:

\begin{itemize}[itemsep=5pt]
    \item \textbf{200 (\textit{OK})}: Menunjukkan bahwa \textit{request} berhasil diproses dan \textit{response} berisi hasil yang sesuai dengan jenis \textit{request} yang dikirimkan.
  
    \item \textbf{201 (\textit{Created})}: Menunjukkan bahwa \textit{request} berhasil diproses dan menghasilkan satu atau lebih \textit{resource} baru pada \textit{server}.
  
    \item \textbf{202 (\textit{Accepted})}: Menunjukkan bahwa \textit{request} telah diterima oleh \textit{server} untuk diproses, namun proses tersebut belum selesai pada saat \textit{response} dikirimkan.
  
    \item \textbf{203 (\textit{Non-Authoritative Information})}: Menunjukkan bahwa \textit{request} berhasil diproses, tetapi konten dalam \textit{response} telah mengalami perubahan oleh \textit{transforming proxy}. Akibatnya, informasi yang diterima oleh \textit{client} tidak sepenuhnya identik dengan data yang dikirimkan oleh \textit{origin server}.
  
    \item \textbf{204 (\textit{No Content})}: Menunjukkan bahwa \textit{request} berhasil diproses, tetapi \textit{server} tidak mengirimkan konten apa pun dalam \textit{response}.
  
    \item \textbf{205 (\textit{Reset Content})}: Menunjukkan bahwa \textit{server} telah berhasil memproses \textit{request} dan meminta agar \textit{user agent} mengembalikan tampilan dokumen ke kondisi awal, sebagaimana sebelum \textit{request} tersebut dikirimkan.
  
    \item \textbf{206 (\textit{Partial Content})}: Menunjukkan bahwa \textit{server} berhasil memenuhi \textit{request} dengan mengirimkan sebagian data dari \textit{resource} yang diminta. Kondisi ini biasanya terjadi ketika \textit{client} menggunakan \textit{range request} untuk mengambil bagian tertentu dari data, misalnya pada proses pengunduhan berkas berukuran besar.
  
    \item \textbf{207 (\textit{Multi-Status})}: Digunakan untuk mengirimkan informasi status dari beberapa operasi yang berbeda dalam satu \textit{request}, di mana setiap operasi dapat memiliki status yang berbeda.~\cite{RFC4918}
  
    \item \textbf{226 (IM \textit{Used})}: Menunjukkan bahwa \textit{server} berhasil memproses \textit{request} \texttt{GET}, dan representasi yang dikirimkan merupakan hasil dari satu atau lebih proses manipulasi terhadap \textit{resource} sebelum dikirimkan kepada \textit{client}.~\cite{RFC3229}
\end{itemize}

\subsubsection{\textit{Redirection} 3xx}
\label{subsubsec:020104-redirection-3xx}

Kode-kode pada kelas ini menunjukkan bahwa \textit{client} harus melakukan langkah tambahan untuk menyelesaikan \textit{request}, seperti mengikuti \textit{redirect}. Berikut adalah daftar kode status pada kategori ini:

\begin{itemize}[itemsep=8pt]
    \item \textbf{300 (\textit{Multiple Choices})}: Menunjukkan bahwa \textit{resource} yang diminta memiliki beberapa alternatif representasi atau lokasi. \textit{Server} menyediakan daftar pilihan tersebut agar \textit{user agent} atau pengguna dapat memilih representasi yang diinginkan.
    
    \item \textbf{301 (\textit{Moved Permanently})}: Menunjukkan bahwa \textit{resource} target telah dipindahkan secara permanen ke URI baru. Untuk \textit{request} selanjutnya, \textit{client} sebaiknya menggunakan URI baru tersebut.
  
    \item \textbf{302 (\textit{Found})}: Menunjukkan bahwa \textit{resource} target sementara tersedia pada URI yang berbeda. Karena sifatnya sementara, \textit{client} dianjurkan untuk tetap menggunakan URI asli pada \textit{request} berikutnya.
  
    \item \textbf{303 (\textit{See Other})}: Menunjukkan bahwa \textit{server} mengarahkan \textit{client} ke \textit{resource} lain yang tercantum dalam \textit{header} \texttt{Location}. Pengalihan ini digunakan untuk memperoleh \textit{response} alternatif terhadap \textit{request} yang dikirimkan.
  
    \item \textbf{304 (\textit{Not Modified})}: Menunjukkan bahwa \textit{resource} tidak mengalami perubahan sejak terakhir kali diakses oleh \textit{client}. Dalam kondisi ini, \textit{client} dapat menggunakan salinan data yang telah dimiliki tanpa perlu menerima ulang konten dari \textit{server}.
 
    \item \textbf{307 (\textit{Temporary Redirect})}: Menunjukkan bahwa \textit{resource} target sementara berada pada URI yang berbeda, dan \textit{client} sebaiknya mengulangi \textit{request} ke URI tersebut dengan metode yang sama.
  
    \item \textbf{308 (\textit{Permanent Redirect})}: Menunjukkan bahwa \textit{resource} target telah dipindahkan secara permanen ke URI baru, dan seluruh \textit{request} berikutnya sebaiknya diarahkan ke URI tersebut.
  
\end{itemize}


\subsubsection{\textit{Client Error} 4xx}
\label{subsubsec:020104-client-error-4xx}

Kode-kode pada kelas ini menunjukkan bahwa telah terjadi kesalahan di sisi \textit{client}. Berikut adalah daftar kode status pada kategori ini:

\begin{itemize}[itemsep=5pt]
    \item \textbf{400 (\textit{Bad Request})}: Menunjukkan bahwa \textit{request} tidak dapat diproses karena kesalahan dari sisi \textit{client}, misalnya format \textit{request} tidak valid, sintaks salah, atau parameter yang dikirim tidak sesuai.
  
    \item \textbf{401 (\textit{Unauthorized})}: Menunjukkan bahwa \textit{request} memerlukan autentikasi, tetapi \textit{client} tidak menyertakan kredensial yang sah atau belum melakukan autentikasi.
  
    \item \textbf{402 (\textit{Payment Required})}: Kode status ini disediakan untuk penggunaan di masa mendatang dan saat ini belum digunakan secara umum.
  
    \item \textbf{403 (\textit{Forbidden})}: Menunjukkan bahwa \textit{server} memahami \textit{request}, tetapi menolak untuk memberikan akses terhadap \textit{resource} yang diminta.
  
    \item \textbf{404 (\textit{Not Found})}: Menunjukkan bahwa \textit{resource} yang diminta tidak ditemukan pada \textit{server}.
  
    \item \textbf{405 (\textit{Method Not Allowed})}: Menunjukkan bahwa metode HTTP yang digunakan dikenali oleh \textit{server}, tetapi tidak diizinkan untuk \textit{resource} tersebut.
  
    \item \textbf{406 (\textit{Not Acceptable})}: Menunjukkan bahwa \textit{server} tidak dapat menyediakan representasi \textit{resource} yang sesuai dengan preferensi \textit{client}, sebagaimana ditentukan melalui \textit{header} negosiasi konten.
  
    \item \textbf{407 (\textit{Proxy Authentication Required})}: Menunjukkan bahwa \textit{client} harus melakukan autentikasi terlebih dahulu kepada \textit{proxy} sebelum \textit{request} dapat diproses.
  
    \item \textbf{408 (\textit{Request Timeout})}: Menunjukkan bahwa \textit{server} menghentikan pemrosesan karena \textit{request} tidak diterima secara lengkap dalam batas waktu yang ditentukan.
  
    \item \textbf{409 (\textit{Conflict})}: Menunjukkan bahwa \textit{request} tidak dapat diproses karena terjadi konflik dengan kondisi \textit{resource} saat ini.
  
    \item \textbf{410 (\textit{Gone})}: Menunjukkan bahwa \textit{resource} yang diminta sudah tidak tersedia lagi pada \textit{server} dan kemungkinan bersifat permanen.
  
    \item \textbf{411 (\textit{Length Required})}: Menunjukkan bahwa \textit{server} menolak \textit{request} karena tidak disertai \textit{header} \texttt{Content-Length}.
  
    \item \textbf{412 (\textit{Precondition Failed})}: Menunjukkan bahwa kondisi tertentu yang disertakan dalam \textit{request} tidak terpenuhi ketika diperiksa oleh \textit{server}.
  
    \item \textbf{413 (\textit{Content Too Large})}: Menunjukkan bahwa \textit{server} menolak memproses \textit{request} karena ukuran konten melebihi batas yang dapat diterima.
  
    \item \textbf{414 (URI \textit{Too Long})}: Menunjukkan bahwa \textit{server} menolak \textit{request} karena URI target terlalu panjang untuk diproses.
  
    \item \textbf{415 (\textit{Unsupported Media Type})}: Menunjukkan bahwa format konten pada \textit{request} tidak didukung oleh \textit{server} untuk \textit{resource} tersebut.
    
    \item \textbf{416 (\textit{Range Not Satisfiable})}: Menunjukkan bahwa \textit{server} tidak dapat memenuhi rentang data yang diminta melalui \textit{header} \texttt{Range}.
  
    \item \textbf{417 (\textit{Expectation Failed})}: Menunjukkan bahwa \textit{server} tidak dapat memenuhi ekspektasi yang dinyatakan dalam \textit{header} \texttt{Expect}.
  
    \item \textbf{421 (\textit{Misdirected Request})}: Menunjukkan bahwa \textit{request} dikirim ke \textit{server} yang tidak dapat memberikan \textit{response} yang sesuai untuk URI target.
  
    \item \textbf{422 (\textit{Unprocessable Content})}: Menunjukkan bahwa \textit{server} memahami format dan sintaks \textit{request}, tetapi tidak dapat memproses instruksi yang terkandung di dalamnya.
  
    \item \textbf{423 (\textit{Locked})}: Menunjukkan bahwa \textit{resource} yang diminta sedang dalam keadaan terkunci.~\cite{RFC4918}
  
    \item \textbf{424 (\textit{Failed Dependency})}: Menunjukkan bahwa \textit{request} tidak dapat diproses karena kegagalan pada operasi lain yang menjadi prasyarat.~\cite{RFC4918}
  
    \item \textbf{425 (\textit{Too Early})}: Menunjukkan bahwa \textit{server} menolak memproses \textit{request} karena berpotensi diproses ulang (\textit{replay}).~\cite{RFC8470}
  
    \item \textbf{426 (\textit{Upgrade Required})}: Menunjukkan bahwa \textit{server} hanya bersedia memproses \textit{request} jika \textit{client} menggunakan protokol yang berbeda.
  
    \item \textbf{428 (\textit{Precondition Required})}: Menunjukkan bahwa \textit{server} mengharuskan \textit{request} disertai kondisi tertentu sebelum dapat diproses.~\cite{RFC6585}
  
    \item \textbf{429 (\textit{Too Many Requests})}: Menunjukkan bahwa \textit{client} mengirim terlalu banyak \textit{request} dalam jangka waktu tertentu.~\cite{RFC6585}
  
    \item \textbf{431 (\textit{Request Header Fields Too Large})}: Menunjukkan bahwa \textit{server} menolak \textit{request} karena ukuran \textit{header} terlalu besar.~\cite{RFC6585}
  
    \item \textbf{451 (\textit{Unavailable For Legal Reasons})}: Menunjukkan bahwa akses terhadap \textit{resource} dibatasi karena alasan hukum.~\cite{RFC7725}
\end{itemize}


\subsubsection{\textit{Server Error} 5xx}
\label{subsubsec:020104-server-error-5xx}

Kode-kode pada kelas ini menunjukkan bahwa \textit{server} menyadari bahwa terjadi kesalahan di sisi \textit{server}. Berikut adalah daftar kode status pada kategori ini:

\vspace{3mm}

\begin{itemize}[itemsep=5pt]
    \item \textbf{500 (\textit{Internal Server Error})}: Menunjukkan bahwa \textit{server} mengalami kesalahan internal yang tidak terduga, sehingga tidak dapat memproses \textit{request} yang diterima.
  
    \item \textbf{501 (\textit{Not Implemented})}: Menunjukkan bahwa \textit{server} tidak mendukung fungsionalitas yang diperlukan untuk memproses \textit{request}. Kode ini biasanya muncul ketika \textit{server} tidak mengenali atau tidak mendukung metode HTTP yang digunakan.
  
    \item \textbf{502 (\textit{Bad Gateway})}: Menunjukkan bahwa \textit{server}, ketika bertindak sebagai \textit{gateway} atau \textit{proxy}, menerima \textit{response} yang tidak valid dari \textit{server} lain yang dihubunginya.
  
    \item \textbf{503 (\textit{Service Unavailable})}: Menunjukkan bahwa \textit{server} sementara waktu tidak dapat menangani \textit{request}. Kondisi ini biasanya terjadi karena \textit{server} sedang kelebihan beban atau dalam proses pemeliharaan. \textit{Server} dapat menyertakan \textit{header} \texttt{Retry-After} untuk memberi tahu kapan \textit{client} dapat mencoba kembali.
  
    \item \textbf{504 (\textit{Gateway Timeout})}: Menunjukkan bahwa \textit{server}, saat bertindak sebagai \textit{gateway} atau \textit{proxy}, tidak menerima \textit{response} tepat waktu dari \textit{server} lain yang diperlukan untuk menyelesaikan \textit{request}.
  
    \item \textbf{505 (\textit{HTTP Version Not Supported})}: Menunjukkan bahwa \textit{server} tidak mendukung versi HTTP yang digunakan dalam \textit{request}.
  
    \item \textbf{506 (\textit{Variant Also Negotiates})}: Menunjukkan adanya kesalahan konfigurasi pada \textit{server}. Kondisi ini terjadi ketika mekanisme negosiasi konten menghasilkan siklus, sehingga \textit{server} tidak dapat menentukan representasi akhir yang akan dikirimkan.~\cite{RFC2295}
  
    \item \textbf{507 (\textit{Insufficient Storage})}: Menunjukkan bahwa \textit{server} tidak memiliki ruang penyimpanan yang cukup untuk menyelesaikan pemrosesan \textit{request} dengan sukses.~\cite{RFC4918}
  
    \item \textbf{508 (\textit{Loop Detected})}: Menunjukkan bahwa \textit{server} menghentikan pemrosesan karena terdeteksi adanya perulangan tak berhingga (\textit{infinite loop}) saat menangani \textit{request}.~\cite{RFC5842}
  
    \item \textbf{511 (\textit{Network Authentication Required})}: Menunjukkan bahwa \textit{client} harus melakukan autentikasi terlebih dahulu untuk memperoleh akses ke jaringan sebelum \textit{request} dapat diproses.~\cite{RFC6585}
\end{itemize}
