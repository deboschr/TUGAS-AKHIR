Berdasarkan rangkaian pengujian eksperimental dan perbandingan dengan perangkat lunak serupa, diperoleh beberapa saran yang dapat dijadikan acuan untuk pengembangan lebih lanjut:

\begin{enumerate}
   \item Pengujian eksperimental menunjukkan adanya inkonsistensi hasil pada beberapa percobaan yang dipengaruhi oleh kondisi jaringan internet. Untuk menjaga konsistensi hasil pemeriksaan, disarankan agar sistem dilengkapi mekanisme pemeriksaan ulang khusus untuk tautan yang gagal karena \textit{timeout}, sehingga kesalahan yang disebabkan kondisi jaringan dapat diminimalkan dan hasil pemeriksaan menjadi lebih akurat.

   \item Untuk meningkatkan kelengkapan dan akurasi identifikasi tautan rusak, sistem dapat diperluas agar tidak hanya mengekstraksi tautan dari elemen \texttt{<a>}, tetapi juga dari elemen HTML lain yang memuat URL, seperti \texttt{<img>} (\texttt{src}), \texttt{<script>} (\texttt{src}), \texttt{<link>} (\texttt{href}), dan elemen serupa. Perluasan cakupan ini bertujuan mengurangi jumlah tautan yang terlewat sehingga hasil pemeriksaan menjadi lebih komprehensif.
\end{enumerate}