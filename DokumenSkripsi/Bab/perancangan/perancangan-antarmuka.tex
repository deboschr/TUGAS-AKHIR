\textit{Perancangan antarmuka} pengguna dibuat dalam bentuk desain fidelitas tinggi dengan tujuan untuk memberikan gambaran visual yang mendekati tampilan akhir aplikasi berbasis JavaFX. Perancangan ini mempertimbangkan aspek keterbacaan, konsistensi elemen antarmuka, serta kemudahan interaksi bagi pengguna dalam menjalankan proses pemeriksaan tautan rusak.

\begin{enumerate}
    \item \textbf{Jendela Utama} \\
    Jendela utama merupakan tampilan awal yang muncul ketika aplikasi dijalankan. Halaman ini memfasilitasi pengguna untuk memasukkan URL situs web yang akan diperiksa, memulai atau menghentikan proses pemeriksaan, serta menampilkan hasil secara langsung. Elemen-elemen penting pada jendela ini ditunjukkan pada Gambar~\ref{fig:rancangan-jendela-utama} dan dijelaskan sebagai berikut:
    
    \vspace{2mm}
    \begin{enumerate}
        
        \item \textbf{Kolom Masukan URL}: Digunakan untuk memasukkan alamat situs web yang akan diperiksa.
        
        \item \textbf{Tombol Start}: Digunakan untuk memulai proses pemeriksaan tautan.
        
        \item \textbf{Tombol Stop}: Digunakan untuk menghentikan proses pemeriksaan.
        
        \item \textbf{Checking Status}: Menunjukkan status terkini dari proses pemeriksaan.
        
        \item \textbf{Total Links}: Menunjukkan jumlah total tautan yang ditemukan selama proses pemeriksaan.
        
        \item \textbf{Webpage Links}: Menunjukkan jumlah tautan yang merupakan halaman situs web.
        
        \item \textbf{Broken Links}: Menunjukkan jumlah tautan rusak yang ditemukan.
        
        \item \textbf{Opsi Filter URL}: Digunakan untuk menentukan jenis pencarian pada kolom URL. Opsi yang tersedia adalah:
        \begin{itemize}
            \item \textbf{\textit{Equals}}: Menampilkan hasil dengan URL yang sama persis.
            \item \textbf{\textit{Contains}}: Menampilkan hasil yang mengandung teks tertentu pada URL.
            \item \textbf{\textit{Starts With}}: Menampilkan hasil dengan URL yang diawali teks tertentu.
            \item \textbf{\textit{Ends With}}: Menampilkan hasil dengan URL yang diakhiri teks tertentu.
        \end{itemize}
        
        \item \textbf{Kolom Masukan Filter URL}: Digunakan untuk memasukkan kata kunci pencarian URL sesuai dengan opsi \textit{filter} yang dipilih sebelumnya.
        
        \item \textbf{Opsi Filter Status Code}: digunakan untuk menyaring hasil berdasarkan kategori kode status HTTP. Opsi yang tersedia adalah:
        \begin{itemize}
            \item \textbf{Equals}: menampilkan tautan dengan kode status tertentu.
            \item \textbf{Greater Than}: menampilkan tautan dengan kode status lebih besar dari nilai yang dimasukkan.
            \item \textbf{Less Than}: menampilkan tautan dengan kode status lebih kecil dari nilai yang dimasukkan.
        \end{itemize}
        
        \item \textbf{Kolom Masukan Filter Status Code}: digunakan untuk memasukkan nilai numerik kode status HTTP sesuai dengan jenis filter yang dipilih pada poin sebelumnya.
        
        \item \textbf{Tombol Export}: digunakan untuk mengekspor hasil pemeriksaan ke dalam berkas Excel.
        
        \item \textbf{Tabel Hasil}: menampilkan daftar hasil pemeriksaan dalam bentuk tabel dengan dua kolom utama:
        \begin{enumerate}
            \item \textbf{Kolom Status}: menampilkan kode status HTTP dan \textit{reason phrase}.
            \item \textbf{Kolom URL}: menampilkan alamat tautan rusak.
        \end{enumerate}
        
        \item \textbf{Pagination}: digunakan untuk menavigasi hasil pemeriksaan jika jumlah tautan yang ditemukan cukup banyak.
    
    \end{enumerate}



    \item \textbf{Jendela Detail Broken Link} \\
    Jendela ini muncul ketika pengguna memilih salah satu tautan rusak pada tabel hasil. Elemen-elemen penting pada jendela ini ditunjukkan pada Gambar~\ref{fig:rancangan-jendela-broken-link} dan dijelaskan sebagai berikut:
    
    \begin{enumerate}
        
        \item \textbf{Kolom URL}: menampilkan alamat lengkap dari tautan rusak yang sedang diperiksa.
        
        \item \textbf{Kolom \textit{Final} URL}: menampilkan alamat terakhir tautan rusak hasil dari \textit{redirection}.
        
        \item \textbf{Kolom \textit{Content Type}}: menampilkan format data dari sumber daya tautan.
        
        \item \textbf{Kolom Status}: menampilkan kode status HTTP dan \textit{reason phrase}.
        
        \item \textbf{Tabel \textit{Webpage Link}}: berisi daftar halaman yang menjadi sumber dimana tautan rusak ditemukan, dengan dua kolom utama:
        \begin{enumerate}
            \item \textbf{Anchor Text}: Teks tautan yang muncul pada halaman sumber.
            \item \textbf{Source Webpage}: URL halaman tempat tautan rusak ditemukan.
        \end{enumerate}
    \end{enumerate}

\end{enumerate}

\begin{figure}[H]
    \centering
    \includegraphics[width=0.85\textwidth]{Gambar/040301-jendela-utama.png}
    \caption{Rancangan Halaman Utama}
    \label{fig:rancangan-jendela-utama}
\end{figure}

\begin{figure}[H]
    \centering
    \includegraphics[width=0.85\textwidth]{Gambar/040302-jendela-broken-link.png}
    \caption{Rancangan Jendela Broken Link}
    \label{fig:rancangan-jendela-broken-link}
\end{figure}