Perancangan antarmuka pengguna digunakan untuk memberikan gambaran visual komponen antarmuka yang tersedia pada aplikasi desktop pemeriksa tautan rusak yang dikembangkan. Pada aplikasi ini akan dibuat dua jendela, yaitu jendela utama dan jendela detail tautan. 

\vspace{1mm}

Berikut adalah penjelasan lebih detail untuk masing-masing jendela:
\begin{enumerate}
    \item \textbf{Jendela Utama} \\
    Jendela utama merupakan antarmuka pengguna yang muncul ketika aplikasi baru dijalankan. Melalui jendela ini, pengguna dapat memasukkan URL situs web yang menjadi target pemeriksaan, memulai dan menghentikan proses pemeriksaan, melihat daftar hasil pemeriksaan, melihat ringkasan hasil pemeriksaan, serta menyaring dan mengekspor hasil pemeriksaan. Rancangan jendela ini terdapat pada Gambar~\ref{fig:rancangan-antarmuka-jendela-utama}.
    
    \vspace{1mm}

    Berikut adalah penjelasan komponen-komponen antarmuka yang ada pada jendela ini:

    \begin{enumerate}[itemsep=1mm]

        \item \textbf{Kolom Masukan URL}: Digunakan untuk memasukkan URL situs web sebagai titik awal pemeriksaan.
        
        \item \textbf{Tombol Start}: Digunakan untuk memulai proses pemeriksaan.
        
        \item \textbf{Tombol Stop}: Digunakan untuk menghentikan proses pemeriksaan.
        
        \item \textbf{Status}: Menunjukkan status proses pemeriksaan.
        
        \item \textbf{All Links}: Menunjukkan jumlah total tautan yang ditemukan.
        
        \item \textbf{Webpage Links}: Menunjukkan jumlah tautan halaman situs web yang ditemukan.
        
        \item \textbf{Broken Links}: Menunjukkan jumlah tautan rusak yang ditemukan.
        
        \item \textbf{Opsi \textit{Filter} URL}: Digunakan untuk menentukan jenis pencarian berdasarkan URL. Opsi yang tersedia adalah:
        
        \begin{itemize}[itemsep=1mm]
            \item \textbf{\textit{Equals}}: Menampilkan tautan rusak dengan URL yang sama persis dengan kata kunci yang dimasukkan.
            
            \item \textbf{\textit{Contains}}: Menampilkan tautan rusak dengan URL yang mengandung kata kunci yang dimasukkan.
            
            \item \textbf{\textit{Starts With}}: Menampilkan tautan rusak dengan URL yang diawali dengan kata kunci yang dimasukkan.
            
            \item \textbf{\textit{Ends With}}: Menampilkan tautan rusak dengan URL yang diakhiri dengan kata kunci yang dimasukkan.
        \end{itemize}
        
        \item \textbf{Kolom Masukan \textit{Filter} URL}: Digunakan untuk memasukkan kata kunci pencarian URL sesuai dengan opsi \textit{filter} yang dipilih sebelumnya.
        
        \item \textbf{Opsi \textit{Filter} Status Code}: Digunakan untuk menentukan jenis pencarian berdasarkan kode status HTTP. Opsi yang tersedia adalah:
        \begin{itemize}[itemsep=1mm]
            \item \textbf{\textit{Equals}}: Menampilkan tautan rusak dengan kode status sama persis dengan kata kunci yang dimasukkan.
            
            \item \textbf{\textit{Greater Than}}: Menampilkan tautan rusak dengan kode status lebih besar dari kata kunci yang dimasukkan.
            
            \item \textbf{\textit{Less Than}}: Menampilkan tautan rusak dengan kode status lebih kecil dari kata kunci yang dimasukkan.
        \end{itemize}
        
        \item \textbf{Kolom Masukan \textit{Filter} Status Code}: Digunakan untuk memasukkan kata kunci pencarian kode status HTTP sesuai dengan opsi \textit{filter} yang dipilih sebelumnya.
        
        \item \textbf{Ringkasan Tabel Hasil}: Digunakan untuk menampilkan informasi halaman aktif dan jumlah halaman pada \textit{pagination}, serta informasi rentang baris yang ditampilkan pada tabel dan jumlah total baris pada tabel. 
        
        \item \textbf{Tombol Export}: Digunakan untuk menjalankan proses ekspor isi tabel ke berkas Excel.
        
        \item \textbf{Tabel Hasil}: Digunakan untuk menampilkan daftar tautan rusak. Tabel ini memiliki dua kolom sebagai berikut:
        \begin{enumerate}[itemsep=3pt]
            \item \textbf{Error}: Menunjukkan pesan kesalahan tautan dari tautan rusak.
            \item \textbf{URL}: Menunjukkan URL dari tautan rusak.
        \end{enumerate}
        
        \item \textbf{\textit{Pagination}}: Digunakan untuk melakukan navigasi tabel jika jumlah tautan yang ditemukan cukup banyak.
    
    \end{enumerate}


    \item \textbf{Jendela Detail Tautan} \\
    Jendela ini menampilkan informasi lengkap dari sebuah tautan yang dipilih oleh pengguna dengan menekan salah satu baris pada tabel hasil di jendela utama. Rancangan jendela ini terdapat pada Gambar~\ref{fig:rancangan-antarmuka-jendela-detail-tautan}. 
    
    \vspace{1mm}

    Berikut adalah penjelasan komponen-komponen antarmuka yang ada pada jendela ini:
    \begin{enumerate}[itemsep=2mm]
        \item \textbf{Kolom URL}: Menunjukkan URL asli dari tautan.
        \item \textbf{Kolom \textit{Final} URL}: Menunjukkan URL terakhir setelah proses \textit{redirect}.
        \item \textbf{Kolom \textit{Content Type}}: Menunjukkan jenis konten dari tautan.
        \item \textbf{Kolom \textit{Error}}: Menunjukkan pesan \textit{error} jika tautan gagal diperiksa.
        \item \textbf{Tabel \textit{Webpage Link}}: Menampilkan daftar halaman yang menjadi sumber tautan rusak ditemukan. Tabel ini memiliki dua kolom sebagai berikut:
        \begin{enumerate}[itemsep=1mm]
            \item \textbf{\textit{Anchor Text}}: Menunjukkan teks tautan yang muncul pada halaman sumber.
            \item \textbf{\textit{Webpage URL}}: Menunjukkan URL halaman tempat tautan rusak ditemukan.
        \end{enumerate}
    \end{enumerate}

\end{enumerate}

\vspace{5mm}

\begin{figure}[H]
    \centering
    \includegraphics[width=0.85\textwidth]{Gambar/040301-rancangan-jendela-utama.png}
    \caption{Rancangan antarmuka pengguna jendela utama}
    \label{fig:rancangan-antarmuka-jendela-utama}
\end{figure}

\begin{figure}[H]
    \centering
    \includegraphics[width=0.85\textwidth]{Gambar/040302-rancangan-jendela-detail-tautan.png}
    \caption{Rancangan antarmuka pengguna jendela detail tautan}
    \label{fig:rancangan-antarmuka-jendela-detail-tautan}
\end{figure}