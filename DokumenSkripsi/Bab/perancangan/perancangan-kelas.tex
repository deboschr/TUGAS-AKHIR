
Rancangan kelas aplikasi ditampilkan dalam bentuk diagram kelas pada Gambar~\ref{fig:class-diagram}, penjelasan dari setiap kelas adalah sebagai berikut:

\begin{enumerate}
    \item \textbf{Kelas Application}\\
    Kelas ini bertanggung jawab untuk memulai aplikasi JavaFX, menyiapkan jendela utama, serta membuka jendela lain seperti detail tautan dan notifikasi.
    \begin{itemize}
        \item \texttt{main}: Metode utama untuk menjalankan aplikasi.
        \item \texttt{start}: Metode untuk menginisialisasi aplikasi.
        \item \texttt{openMainWindow}: Metode untuk membuka jendela utama aplikasi.
        \item \texttt{openLinkWindow}: Metode untuk membuka jendela detail tautan.
        \item \texttt{openNotificationWindow}: Metode untuk membuka jendela notifikasi.
    \end{itemize}

    \item \textbf{Kelas MainController}\\
    Kelas ini bertanggung jawab untuk mengendalikan logika antarmuka jendela utama aplikasi.
    \begin{itemize}
        \item \texttt{crawler}: Atribut ini menyimpan objek dari kelas \texttt{Crawler} yang menangani proses \textit{crawling}.
        \item \texttt{summary}: Atribut ini menyimpan objek dari kelas \texttt{Summary} untuk menampilkan rinkasan proses \textit{crawling}.
        \item \texttt{allLinks}: Atribut ini menyimpan daftar semua tautan yang ditemukan selama proses crawling.
        \item \texttt{webpageLinks}: Atribut ini menyimpan daftar tautan dari halaman situs web.
        \item \texttt{brokenLinks}: Atribut ini menyimpan daftar tautan yang terdeteksi rusak.
        \item \texttt{onStartClick}: Metode untuk menangani \textit{event} ketika tombol \textit{start} ditekan, untuk memulai proses pemeriksaan.
        \item \texttt{onStopClick}: Metode untuk menangani \textit{event} ketika tombol \textit{stop} ditekan, untuk menghentikan proses pemeriksaan.
        \item \texttt{onExportClick}: Metode untuk menangani \textit{event} ketika tombol \textit{export} ditekan, untuk mengekspor isi tabel.
        \item \texttt{setSummaryCard}: Metode untuk mengatur data yang akan ditampilkan pada ringkasan.
        \item \texttt{setFilterCard}: Metode untuk mengatur logika filter pada tampilan hasil.
        \item \texttt{setTableView}: Metode untuk menentukan data yang akan ditampilkan pada tabel.
        \item \texttt{setPagination}: Metode untuk mengatur logika \textit{pagination} untuk tampilan tabel \textit{broken link}.
    \end{itemize}

    \item \textbf{Kelas LinkController}\\
    Kelas ini bertanggung jawab untuk mengendalikan logika antarmuka jendela detail tautan.
    \begin{itemize}
        \item \texttt{webpageLinks}: Atribut ini menyimpan peta yang berisi pasangan \textit{webpage link} dan \textit{anchor text}
        \item \texttt{setFieldValue}: Metode untuk menetapkan nilai \textit{field} pada jendela detail tautan.
        \item \texttt{setTableView}: Metode untuk menentukan data yang akan ditampilkan pada tabel \textit{webpage link}.
    \end{itemize}
    
    \item \textbf{Kelas NotificationController}\\
    Kelas ini bertanggung jawab untuk mengendalikan logika antarmuka jendela notifikasi. Metode pada kelas ini adalah \texttt{setNotification} yang bertugas untuk menetapkan nilai pada jendela notifikasi.

    \item \textbf{Kelas Crawler}\\
    Kelas ini bertanggung jawab untuk menjalankan proses \textit{web crawling}.
    \begin{itemize}
        \item \texttt{rootHost}: Atribut ini menyimpan \textit{host} dari URL awal yang digunakan untuk membedakan tautan internal dan eksternal.
        \item \texttt{isStopped}: Atribut ini menyimpan \textit{boolean} yang menandakan apakah proses \textit{crawling} dihentikan oleh pengguna.
        \item \texttt{frontier}: Atribut ini menyimpan antrean halaman yang akan dikunjungi.
        \item \texttt{repositories}: Atribut ini menyimpan daftar seluruh tautan yang ditemukan untuk memastikan setiap tautan dikunjungi satu kali.
        \item \texttt{rateLimiters}: Atribut ini menyimpan peta yang berisi pasangan \textit{host} dan objek kelas \texttt{RateLimiter} untuk membatasi kecepatan \textit{fetching} per-\textit{host}.
        \item \texttt{linkConsumer}: Atribut ini menyimpan fungsi \textit{callback} untuk mengirim tautan yang ditemukan ke antarmuka pengguna.
        \item \texttt{OK\_HTTP}: Atribut ini menyimpan objek klien HTTP yang digunakan untuk \textit{fetching} URL.
        \item \texttt{start}: Metode untuk memulai proses \textit{web crawling}.
        \item \texttt{stop}: Metode untuk mengentikan proses \textit{web crawling}.
        \item \texttt{fetchLink}: Metode untuk memeriksa dan melakukan \textit{parse} dokumen HTML.
        \item \texttt{extractLink}: Metode untuk mengekstraksi seluruh tautan dari sebuah dokumen HTML.
    \end{itemize}

    \item \textbf{Kelas Link}\\
    Kelas ini merepresentasikan tautan yang ditemukan dalam proses \textit{crawling}.
    \begin{itemize}
        \item \texttt{url}: Atribut ini menyimpan URL asli dari tautan.
        \item \texttt{finalUrl}: Atribut ini menyimpan URL terakhir setelah proses \textit{redirect}.
        \item \texttt{statusCode}: Atribut ini menyimpan kode status HTTP hasil pemeriksaan.
        \item \texttt{contentType}: Atribut ini menyimpan jenis konten dari tautan.
        \item \texttt{error}: Atribut ini menyimpan pesan error jika tautan gagal diperiksa.
        \item \texttt{connections}: Atribut ini menyimpan peta hubungan antara tautan ini dengan tautan lain beserta \textit{anchor text}-nya.
    \end{itemize}

    \item \textbf{Kelas Summary}\\
    Kelas ini merepresentasikan ringkasan dari proses \textit{crawling}.
    \begin{itemize}
        \item \texttt{status}: Attribute ini menyimpan objek dari enum Status untuk menampilkan status proses \textit{crawling}.
        \item \texttt{allLinkCount}: Attribute ini menyimpan jumlah seluruh tautan yang ditemukan.
        \item \texttt{webpageLinkCount}: Attribute ini menyimpan jumlah tautan halaman situs web.
        \item \texttt{brokenLinkCount}: Attribute ini menyimpan jumlah tautan rusak.
    \end{itemize}

    \item \textbf{Enum Status}\\
    Enum ini merepresentasikan status dari proses \textit{crawling}.
    \begin{itemize}
        \item \texttt{IDLE}: Proses belum dimulai.
        \item \texttt{CHECKING}: Proses pemeriksaan sedang berlangsung.
        \item \texttt{STOPPED}: Proses pemeriksaan dihentikan oleh pengguna.
        \item \texttt{COMPLETED}: Proses pemeriksaan selesai dilakukan.
    \end{itemize}

    \item \textbf{Kelas Exporter}\\
    Kelas ini bertanggung jawab untuk mengekspor hasil pemeriksaan ke berbagai format file.
    \begin{itemize}
        \item \texttt{toExcel}: Metode untuk mengekspor hasil ke file Excel.
        \item \texttt{toJSON}: Metode untuk mengekspor hasil ke file JSON.
        \item \texttt{toCSV}: Metode untuk mengekspor hasil ke file CSV.
    \end{itemize}
    
    \item \textbf{Kelas RateLimiter}\\
    Kelas ini bertanggung jawab untuk membatasi kecepatan HTTP \textit{request}.
    \begin{itemize}
        \item \texttt{INTERVAL}: Atribut ini menyimpan jarak minimum antar HTTP \textit{request}.
        \item \texttt{lastRequestTime}: Atribut ini menyimpan waktu terakhir HTTP \textit{request} dilakukan.
        \item \texttt{delay}: Metode untuk memberikan jeda antar HTTP \textit{request}.
    \end{itemize}
    
    \item \textbf{Kelas UrlHandler}\\
    Kelas ini bertanggung jawab untuk menyediakan utilitas untuk manipulasi URL.
    \begin{itemize}
        \item \texttt{getHost}: Metode untuk mengambil nama \textit{host} dari URL.
        \item \texttt{normalizeUrl}: Metode untuk menormalkan struktur URL.
        \item \texttt{normalizePath}: Metode untuk menormalkan struktur \textit{path} URL.
    \end{itemize}
    
    \item \textbf{Kelas HttpStatus}\\
    Kelas ini bertanggung jawab untuk menyediakan peta kode status HTTP ke \textit{reason phrase} yang sesuai.
    \begin{itemize}
        \item \texttt{ERROR\_STATUS\_MAP}: Atribut ini menyimpan peta yang berisi pasangan kode status HTTP dan \textit{reason phrase}.
        \item \texttt{getErrorStatus}: Metode untuk mengambil kombinasi kode status HTTP dan \textit{reason phrase}.
    \end{itemize}

\end{enumerate}


\begin{figure}[H]
    \centering
    \includegraphics[width=1\textwidth]{Gambar/040100-class-diagram.png}
    \caption{Diagram kelas aplikasi}
    \label{fig:class-diagram}
\end{figure}