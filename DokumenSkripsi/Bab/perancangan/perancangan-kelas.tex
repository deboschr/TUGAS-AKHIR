Perancangan kelas digunakan untuk menggambarkan struktur kode program aplikasi sebelum proses implementasi dilakukan. Perancangan ini menunjukkan hubungan antarkelas, atribut, serta metode yang dimiliki oleh setiap kelas. Hasil pada perancangan ini ditunjukkan melalui diagram kelas pada Gambar~\ref{fig:class-diagram}. Berikut adalah penjelasan dari diagram kelas tersebut:

\vspace{5mm}
\begin{enumerate}

    % ######################################################################
    \item \textbf{Kelas \texttt{Application}}\\
    Kelas \texttt{Application} merupakan titik awal aplikasi JavaFX dan bertugas untuk membuka jendela aplikasi. Kelas ini memiliki atribut \texttt{mainStage} yang menyimpan referensi ke jendela utama dan digunakan untuk menjadikan jendela lain sebagai turunan dari jendela utama. Berikut adalah penjelasan lebih rinci untuk metode dan relasi pada kelas ini:

    \vspace{2mm}
    \textbf{Metode:}
    \begin{itemize}

        \item \texttt{start}: Metode ini digunakan untuk memulai aplikasi. Masukan dari metode ini adalah sebuah objek \texttt{Stage} yang digunakan untuk membuat jendela utama.
        
        \item \texttt{openMainWindow}: Metode ini digunakan untuk membuka jendela utama.
        
        \item \texttt{openLinkWindow}: Metode ini digunakan untuk membuka jendela detail tautan. Masukan dari metode ini adalah sebuah objek \texttt{Link} yang ingin ditampilkan pada jendela detail tautan.
        
        \item \texttt{openNotifWindow}: Metode ini digunakan untuk membuka jendela notifikasi. Masukan dari metode ini adalah jenis pesan dan teks pesan yang ingin ditampilkan pada jendela notifikasi.
        
    \end{itemize}

    \vspace{2mm}
    \textbf{Relasi:}
    \begin{itemize}
        \item Relasi \textit{dependency} dengan kelas \texttt{LinkController}.
        \item Relasi \textit{dependency} dengan kelas \texttt{NotifController}.
    \end{itemize}

    \vspace{15mm}
    % ######################################################################
    \item \textbf{Kelas \texttt{MainController}}\\
    Kelas \texttt{MainController} bertugas untuk mengatur logika antarmuka, mengendalikan interaksi pengguna, serta mengelola data yang ditampilkan pada jendela utama. Berikut adalah penjelasan lebih rinci untuk atribut, metode, dan relasi pada kelas ini:

    \vspace{2mm}
    \textbf{Atribut:}
    \begin{itemize}

        \item \texttt{crawler}: Atribut ini menyimpan objek dari kelas \texttt{Crawler} yang digunakan untuk menjalankan proses pemeriksaan.
        
        \item \texttt{summary}: Atribut ini menyimpan objek dari kelas \texttt{Summary} yang digunakan untuk mencatat ringkasan hasil pemeriksaan.
        
        \item \texttt{allLinks}: Atribut ini menyimpan daftar seluruh tautan yang ditemukan dalam proses pemeriksaan.
        
        \item \texttt{webpageLinks}: Atribut ini menyimpan daftar tautan halaman yang ditemukan dalam proses pemeriksaan.
        
        \item \texttt{brokenLinks}: Atribut ini menyimpan daftar tautan rusak yang ditemukan dalam proses pemeriksaan.
        
    \end{itemize}

    \vspace{2mm}
    \textbf{Metode:}
    \begin{itemize}

        \item \texttt{initialize}: Metode ini digunakan untuk menyiapkan komponen antarmuka.
        
        \item \texttt{onStartClick}: Metode ini digunakan untuk menangani aksi pengguna ketika tombol \textit{Start} ditekan dan akan memulai proses pemeriksaan.
        
        \item \texttt{onStopClick}: Metode ini digunakan untuk menangani aksi pengguna ketika tombol \textit{Stop} ditekan dan akan menghentikan proses pemeriksaan.
        
        \item \texttt{onExportClick}: Metode ini digunakan untuk menangani aksi pengguna ketika tombol \textit{Export} ditekan dan akan memulai proses ekspor.
        
    \end{itemize}

    \vspace{2mm}
    \textbf{Relasi:}
    \begin{itemize}
        \item Relasi \textit{generalization} dengan kelas \texttt{LinkReceiver}.
        \item Relasi \textit{dependency} dengan kelas \texttt{Application}.
        \item Relasi \textit{dependency} dengan kelas \texttt{UrlHandler}.
        \item Relasi \textit{dependency} dengan kelas \texttt{Exporter}.
        \item Relasi \textit{dependency} dengan enumerasi \texttt{Status}.
        \item Relasi \textit{association} berkardinalitas \textit{one-to-many} dengan kelas \texttt{Link}.
        \item Relasi \textit{association} berkardinalitas \textit{one-to-one} dengan kelas \texttt{Crawler}.
        \item Relasi \textit{association} berkardinalitas \textit{one-to-one} dengan kelas \texttt{Summary}.
    \end{itemize}

    \vspace{4mm}
    % ######################################################################
    \item \textbf{Kelas \texttt{LinkController}}\\ 
    Kelas \texttt{LinkController} bertugas untuk mengatur logika antarmuka, mengendalikan interaksi pengguna, serta mengelola data yang ditampilkan pada jendela detail tautan. Kelas ini memiliki satu atribut, yaitu \texttt{link}, yang digunakan untuk menyimpan objek dari kelas \texttt{Link} yang akan ditampilkan pada jendela detail tautan. Selain itu, kelas ini memiliki satu metode \texttt{initialize} yang berfungsi untuk menyiapkan komponen antarmuka. Dengan adanya atribut \texttt{link}, maka kelas \texttt{LinkController} memiliki relasi \textit{association} berkardinalitas \textit{one-to-one} dengan kelas \texttt{Link}.


    \vspace{8mm}
    % ######################################################################
    \item \textbf{Kelas \texttt{NotifController}}\\
    Kelas \texttt{NotifController} bertugas untuk mengatur logika antarmuka, mengendalikan interaksi pengguna, serta mengelola data yang ditampilkan pada jendela notifikasi. Kelas ini memiliki atribut \texttt{type} dan \texttt{message}, yang masing-masing merupakan nilai \texttt{String} dan digunakan untuk menyimpan jenis pesan serta isi pesan yang akan ditampilkan. Kelas ini juga memiliki metode \texttt{initialize} yang berfungsi untuk menyiapkan komponen antarmuka.

    \vspace{1mm}
    % ######################################################################
    \item \textbf{Kelas \texttt{Exporter}}\\
    Kelas \texttt{Exporter} bertugas untuk melakukan ekspor hasil pemeriksaan ke berkas Excel (\texttt{xlsx}). Berikut adalah penjelasan lebih rinci untuk atribut, metode, dan relasi pada kelas ini:

    \textbf{Atribut:}
    \begin{itemize}
        
        \item \texttt{summary}: Atribut ini menyimpan ringkasan hasil pemeriksaan yang akan diekspor.
        
        \item \texttt{brokenLinks}: Atribut ini menyimpan daftar tautan rusak yang akan diekspor.
        
    \end{itemize}

    \textbf{Metode:}
    \begin{itemize}
        
        \item \texttt{save}: Metode ini digunakan untuk membuat dan menyimpan berkas Excel ke dalam penyimpanan lokal komputer. Masukan dari metode ini adalah objek \texttt{File} yang menunjukkan lokasi berkas hasil ekspor.
        
        \item \texttt{writeSummaryTable}: Metode ini digunakan untuk menulis tabel ringkasan hasil pemeriksaan pada berkas Excel yang akan diekspor. Masukan dari metode ini adalah objek \texttt{Sheet} tempat tabel ditulis. 
        
        \item \texttt{writeBrokenLinkTable}: Metode ini digunakan untuk menulis tabel tautan rusak pada berkas Excel yang akan diekspor. Masukan dari metode ini adalah objek \texttt{Sheet} tempat tabel ditulis. 
        
    \end{itemize}

    \textbf{Relasi:}
    \begin{itemize}
        \item Relasi \textit{association} berkardinalitas \textit{one-to-many} dengan kelas \texttt{Link}.
        \item Relasi \textit{association} berkardinalitas \textit{one-to-one} dengan kelas \texttt{Summary}.
        \item Relasi \textit{dependency} dengan kelas \texttt{ErrorHandler}.
    \end{itemize}


    \vspace{1mm}
    % ######################################################################
    \item \textbf{Kelas \texttt{RateLimiter}}\\
    Kelas \texttt{RateLimiter} bertugas untuk membatasi frekuensi permintaan HTTP agar pemeriksaan tidak dilakukan terlalu cepat yang dapat membebani \textit{server} situs web tujuan. Kelas ini memiliki metode \texttt{delay} yang digunakan untuk mengatur jarak waktu antarpermintaan HTTP. Berikut adalah penjelasan lebih rinci untuk atribut pada kelas ini:

    \begin{itemize}

        \item \texttt{INTERVAL}: Atribut ini menyimpan rentang waktu untuk melakukan permintaan HTTP.
        
        \item \texttt{lastRequestTime}: Atribut ini menyimpan waktu terakhir permintaan HTTP dilakukan.
        
    \end{itemize}


    \vspace{1mm}
    % ######################################################################
    \item \textbf{Kelas \texttt{UrlHandler}}\\
    Kelas \texttt{UrlHandler} bertugas untuk menyediakan metode yang menangani segala kebutuhan terkait URL. Berikut adalah penjelasan lebih rinci untuk metode pada kelas ini:
    \begin{itemize}
        
        \item \texttt{normalizeUrl}: Metode ini digunakan untuk melakukan normalisasi sebuah URL dengan menetapkan aturan-aturan tertentu. Masukan dari metode ini adalah sebuah URL, sedangkan keluarannya adalah URL yang sudah dinormalisasi.

        \item \texttt{getHost}: Metode ini digunakan untuk mendapatkan \textit{host} dari sebuah URL. Masukan dari metode ini adalah sebuah URL, sedangkan keluarannya adalah \textit{host} dari URL tersebut.
        
    \end{itemize}
    
    
    % ######################################################################
    \item \textbf{Kelas \texttt{Crawler}}\\
    Kelas \texttt{Crawler} bertugas untuk menjalankan proses \textit{web crawling} dan melakukan pemeriksaan pada setiap tautan yang ditemukan dalam proses tersebut. Berikut adalah penjelasan lebih rinci untuk atribut, metode, dan relasi pada kelas ini:

    \textbf{Atribut:}
    \begin{itemize}

        \item \texttt{MAX\_LINKS}: Atribut ini menyimpan batas maksimum jumlah tautan yang diperiksa untuk mencegah proses pemeriksaan berlangsung terlalu lama.

        \item \texttt{rootHost}: Atribut ini menyimpan \textit{host} dari URL awal dan digunakan untuk menentukan apakah sebuah tautan adalah tautan halaman situs web atau bukan.
        
        \item \texttt{linkSender}: Atribut ini menyimpan sebuah fungsi \textit{callback} bertipe \texttt{Consumer} yang digunakan untuk mengirimkan objek \texttt{Link} kembali ke kelas \texttt{MainController} setiap kali sebuah tautan selesai diperiksa.
        
        \item \texttt{frontier}: Atribut ini digunakan untuk menyimpan antrean halaman situs web yang akan diperiksa. Struktur data yang digunakan adalah \texttt{Queue} agar halaman situs web diproses menggunakan urutan \textit{First-In-First-Out} (FIFO).
        
        \item \texttt{repositories}: Atribut ini menyimpan daftar URL unik yang ditemukan selama proses pemeriksaan dan digunakan untuk memastikan tidak ada duplikasi dalam pemeriksaan. Struktur data yang digunakan adalah \texttt{Map}, dengan \textit{key} berupa URL dan \textit{value} berupa objek \texttt{Link} yang mewakili URL tersebut.
        
        \item \texttt{rateLimiters}: Atribut ini menyimpan daftar objek \texttt{RateLimiter} yang digunakan untuk membatasi frekuensi permintaan HTTP ke setiap \textit{host} URL. Struktur data yang digunakan adalah \texttt{Map}, dengan \textit{key} berupa \textit{host} URL dan \textit{value} berupa objek \texttt{RateLimiter} yang terkait dengan \textit{host} tersebut.
        
    \end{itemize}

    \textbf{Metode:}
    \begin{itemize}

        \item \texttt{start}: Metode ini digunakan untuk menjalankan proses \textit{web crawling} dengan algoritma BFS. Masukan dari metode ini adalah URL dari situs web yang ingin diperiksa.
        
        \item \texttt{stop}: Metode ini digunakan untuk menghentikan proses \textit{web crawling}.
        
        \item \texttt{checkLink}: Metode ini digunakan untuk melakukan pemeriksaan pada tautan dengan melakukan permintaan HTTP dan melakukan \textit{parsing} pada \textit{response body} ke dokumen HTML. Masukan dari metode ini adalah tautan target pemeriksaan dan sebuah penanda untuk melakukan \textit{parsing}. Keluaran dari metode ini adalah sebuah dokumen HTML jika \textit{parsing} dilakukan.
        
        \item \texttt{extractLink}: Metode ini digunakan untuk melakukan ekstraksi tautan dari sebuah dokumen HTML. Masukan dari metode ini adalah sebuah dokumen HTML, sedangkan keluarannya adalah daftar tautan pada HTML tersebut beserta dengan teks pada tautan tersebut (\textit{anchor text}).
        
    \end{itemize}

    \textbf{Relasi:}
    \begin{itemize}
        \item Relasi \textit{association} berkardinalitas \textit{one-to-one} dengan kelas \texttt{LinkReceiver}.
        \item Relasi \textit{association} berkardinalitas \textit{one-to-many} dengan kelas \texttt{Link}.
        \item Relasi \textit{association} berkardinalitas \textit{one-to-many} dengan kelas \texttt{RateLimiter}.
        \item Relasi \textit{dependency} dengan kelas \texttt{UrlHandler}.
        \item Relasi \textit{dependency} dengan kelas \texttt{ErrorHandler}.
    \end{itemize}



    \vspace{10mm}
    % ######################################################################
    \item \textbf{Kelas \texttt{ErrorHandler}}\\
    Kelas \texttt{ErrorHandler} bertugas untuk menyediakan metode yang menangani pesan kesalahan. Berikut adalah penjelasan lebih rinci untuk atribut dan metode pada kelas ini:

    \vspace{1mm}
    \textbf{Atribut:}
    \begin{itemize}
        
        \item \texttt{HTTP_ERROR\_MAP}: Atribut ini menyimpan daftar pesan kesalahan berdasarkan kode status HTTP. Struktur data yang digunakan adalah \texttt{Map}, dengan \textit{key} berupa kode status HTTP dan \textit{value} berupa pesan kesalahan yang sesuai dengan kode status tersebut.
        
    \end{itemize}

    \vspace{1mm}
    \textbf{Metode:}
    \begin{itemize}
        
        \item \texttt{getExceptionError}: Metode ini digunakan untuk memperoleh pesan kesalahan yang lebih mudah dipahami berdasarkan \texttt{exception} yang diberikan. Masukan dari metode ini adalah sebuah objek \texttt{Throwable} yang mewakili kesalahan yang terjadi, sedangkan keluarannya berupa teks yang mewakili kesalahan tersebut.
        
        \item \texttt{getHttpError}: Metode ini digunakan untuk mendapatkan pesan kesalahan pada HTTP yang diambil dari atribut \texttt{HTTP_ERROR\_MAP}. Masukan dari metode ini adalah sebuah kode status HTTP, sedangkan keluarannya adalah pesan kesalahan yang sesuai dengan kode status tersebut.
        
        \item \texttt{isHttpError}: Metode ini digunakan untuk menentukan apakah sebuah masukan berupa bilangan bulat merupakan kode status HTTP.
        
    \end{itemize}

    \vspace{3mm}
    % ######################################################################
    \item \textbf{Kelas \texttt{Link}}\\
    Kelas \texttt{Link} merupakan model yang digunakan untuk merepresentasikan tautan yang ditemukan selama proses pemeriksaan. Kelas ini memiliki metode \texttt{addRelation} yang digunakan untuk menambahkan relasi pada tautan saat ini. Masukan dari metode ini adalah tautan lain yang ingin ditambahkan sebagai relasi dan teks pada tautan (\textit{anchor text}). Berikut adalah penjelasan lebih rinci untuk atribut dan relasi pada kelas ini:

    \vspace{1mm}
    \textbf{Atribut:}
    \begin{itemize}

        \item \texttt{url}: Atribut ini menyimpan URL asli dari tautan.
        
        \item \texttt{finalUrl}: Atribut ini menyimpan URL akhir setelah \textit{redirect}.
        
        \item \texttt{statusCode}: Atribut ini menyimpan kode status HTTP.
        
        \item \texttt{contentType}: Atribut ini menyimpan tipe konten dari \textit{response body} HTTP.
        
        \item \texttt{error}: Atribut ini menyimpan pesan kesalahan tautan yang gagal diperiksa.
        
        \item \texttt{isWebpage}: Atribut ini digunakan untuk menandai apakah tautan saat ini merupakan tautan halaman atau bukan.
        
        \item \texttt{webpageSources}: Atribut ini menyimpan daftar tautan halaman yang menjadi sumber dari tautan ini ditemukan. Struktur data yang digunakan adalah \texttt{Map}, dengan \textit{key} berupa tautan halaman sumber dan \textit{value} berupa teks pada tautan (\textit{anchor text}).
        
    \end{itemize}

    \vspace{1mm}
    \textbf{Relasi:}
    \begin{itemize}
        
        \item Relasi \textit{association} berkardinalitas \textit{many-to-many} dengan kelas \texttt{Link}.
        
        \item Relasi \textit{dependency} dengan kelas \texttt{ErrorHandler}.
        
    \end{itemize}

    \vspace{10mm}

    % ######################################################################
    \item \textbf{Kelas \texttt{Summary}}\\
    Kelas \texttt{Summary} merupakan model yang digunakan untuk merepresentasikan data ringkasan dari proses pemeriksaan. Kelas ini memiliki relasi \textit{association} berkardinalitas \textit{one-to-one} dengan enumerasi \texttt{Status}. Berikut adalah penjelasan lebih rinci untuk atribut pada kelas ini:

    \begin{itemize}

        \item \texttt{status}: Atribut ini menyimpan enumerasi \texttt{Status} yang menunjukkan status terakhir dari proses pemeriksaan.
        
        \item \texttt{allLinksCount}: Atribut ini menyimpan jumlah total tautan.
        
        \item \texttt{webpageLinksCount}: Atribut ini menyimpan jumlah tautan halaman.
        
        \item \texttt{brokenLinksCount}: Atribut ini menyimpan jumlah tautan rusak.
        
        \item \texttt{startTime}: Atribut ini menyimpan waktu mulai proses pemeriksaan.
        
        \item \texttt{endTime}: Atribut ini menyimpan waktu selesai proses pemeriksaan.
    \end{itemize}


    % ######################################################################
    \item \textbf{Enumerasi \texttt{Status}}\\
    Enumerasi \texttt{Status} merupakan model yang digunakan untuk mendefinisikan status dari proses pemeriksaan. Enumerasi ini memiliki empat konstanta, yaitu \texttt{IDLE} yang menunjukkan bahwa belum ada aktivitas pada sistem, \texttt{CHECKING} yang menunjukkan bahwa proses pemeriksaan sedang berlangsung, \texttt{STOPPED} yang menunjukkan bahwa proses pemeriksaan dihentikan oleh pengguna, serta \texttt{COMPLETED} yang menunjukkan bahwa proses pemeriksaan telah selesai.

\end{enumerate}

\vspace{10mm}

\begin{figure}[H]
    \centering
    \includegraphics[width=1\textwidth]{Gambar/040100-class-diagram.png}
    \caption{Diagram kelas aplikasi}
    \label{fig:class-diagram}
\end{figure}
