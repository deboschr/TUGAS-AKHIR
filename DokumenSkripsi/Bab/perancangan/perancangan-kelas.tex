Struktur kelas pada aplikasi \textit{Broken Link Checker} dibagi menjadi tiga \textit{package} utama, yaitu \texttt{Network}, \texttt{Web Crawling}, dan \texttt{GUI Manager}. \textit{Package} \texttt{Network} berisi pustaka eksternal yang digunakan untuk melakukan pengambilan dan pemrosesan konten web. \textit{Package} \texttt{Web Crawling} berisi kelas inti yang mengatur proses pemeriksaan tautan, sedangkan \textit{package} \texttt{GUI Manager} menangani antarmuka pengguna dengan JavaFX. Diagram kelas aplikasi ini dapat dilihat pada Gambar~\ref{fig:class-diagram}.

Berikut adalah penjelasan dari diagram pada Gambar~\ref{fig:class-diagram}:

\begin{enumerate}
    \item \textbf{Kelas Jsoup dan OkHttp}\\
    Kedua kelas ini termasuk dalam \textit{package} \texttt{Network} dan merupakan representasi dari pustaka eksternal yang digunakan oleh aplikasi.
    \begin{itemize}
        \item \texttt{Jsoup} : Digunakan untuk mengambil dan mengurai dokumen HTML pada halaman \textit{same-host} menggunakan parser toleran kesalahan.
        \item \texttt{OkHttp} : Digunakan untuk melakukan permintaan HTTP.
    \end{itemize}

    \item \textbf{Kelas Link}\\
    Kelas \texttt{Link} merepresentasikan satu entitas tautan (URL) yang ditemukan selama proses \textit{crawling}. 
    Objek \texttt{Link} dapat mewakili halaman web maupun tautan eksternal yang diperiksa statusnya.
    \begin{itemize}
        \item \texttt{url} : Menyimpan alamat asli dari tautan.
        \item \texttt{finalUrl} : Menyimpan URL akhir setelah mengikuti pengalihan (jika ada).
        \item \texttt{statusCode} : Menyimpan kode status HTTP dari hasil pemeriksaan.
        \item \texttt{contentType} : Menyimpan tipe konten dari respons HTTP.
        \item \texttt{error} : Menyimpan keterangan atau pesan kesalahan (jika terjadi).
        \item \texttt{accessTime} : Menyimpan waktu akses terakhir dalam format \texttt{Instant}.
        \item \texttt{connections} : Menyimpan relasi ke objek \texttt{Link} lain beserta konteks kemunculannya.
    \end{itemize}

\begin{figure}[H]
    \centering
    \includegraphics[width=1\textwidth]{Gambar/040100-class-diagram.png}
    \caption{Diagram kelas aplikasi}
    \label{fig:class-diagram}
\end{figure}

    \item \textbf{Kelas Crawler}\\
    Kelas \texttt{Crawler} termasuk dalam \textit{package} \texttt{Web Crawling} dan merupakan inti dari proses \textit{web crawling}. Kelas ini bertugas mengelola antrean URL, mengambil konten halaman, mengekstrak tautan, serta mengirim hasil pemeriksaan secara \textit{streaming} ke antarmuka pengguna.
    \begin{itemize}
        \item \texttt{rootHost} : Atribut bertipe \textit{String} yang menyimpan nama host dari URL awal.
        \item \texttt{frontier} : Atribut bertipe \textit{Queue<Link>} yang digunakan sebagai antrean URL yang akan diproses.
        \item \texttt{repository} : Atribut bertipe \textit{Set<String>} yang berisi daftar URL yang sudah pernah dikunjungi untuk mencegah duplikasi pemeriksaan.
        \item \texttt{checkingStatusConsumer}, \texttt{webpageLinkConsumer}, \texttt{brokenLinkConsumer}, \texttt{linkConsumer} : Atribut bertipe \textit{Consumer} yang digunakan sebagai saluran komunikasi ke \texttt{Controller}.
        \item \texttt{OK\_HTTP} : Atribut statis bertipe \texttt{OkHttpClient} sebagai klien HTTP utama.
        \item \texttt{USER\_AGENT} : Atribut statis bertipe \textit{String} untuk identitas permintaan HTTP.
        \item \texttt{TIMEOUT} : Atribut statis bertipe \textit{int} yang menentukan batas waktu koneksi.
        \item \texttt{start(seedUrl : String)} : Metode untuk memulai proses \textit{crawling} dari URL awal.
        \item \texttt{stop()} : Metode untuk menghentikan proses \textit{crawling}.
        \item \texttt{fetchUrl(url : String, isParseDoc : boolean)} : Metode untuk mengambil konten halaman menggunakan \texttt{Jsoup} atau \texttt{OkHttp}.
        \item \texttt{normalizeUrl(url : String)} : Metode untuk menormalkan URL agar memiliki format yang konsisten.
        \item \texttt{extractUrl(doc : Document)} : Metode untuk mengekstrak semua tautan dari suatu dokumen HTML.
        \item \texttt{getHost(url : String)} : Metode untuk memperoleh host dari sebuah URL.
        \item \texttt{sendData(consumer : Consumer<T>, data : T)} : Metode generik untuk mengirim data hasil \textit{crawl} ke \texttt{Controller}.
    \end{itemize}

    \item \textbf{Kelas HttpStatus}\\
    Kelas \texttt{HttpStatus} berfungsi sebagai kelas utilitas yang memetakan kode status HTTP dengan keterangan resminya.
    \begin{itemize}
        \item \texttt{STATUS\_MAP} : Atribut statis bertipe \textit{Map<Integer, String>} yang menyimpan pasangan antara kode status HTTP dan keterangan (\textit{reason phrase}).
        \item \texttt{getStatus(statusCode : int)} : Metode statis untuk mendapatkan deskripsi teks dari suatu kode status HTTP.
    \end{itemize}

    \item \textbf{Kelas Controller}\\
    Kelas \texttt{Controller} termasuk dalam \textit{package} \texttt{GUI Manager} dan berperan sebagai pengendali antarmuka pengguna (UI). Kelas ini mengatur interaksi pengguna dengan aplikasi, memulai proses pemeriksaan tautan, dan memperbarui tampilan hasil secara langsung.
    \begin{itemize}
        \item \texttt{crawler} : Atribut bertipe \texttt{Crawler} yang digunakan untuk menjalankan proses pemeriksaan tautan.
        \item \texttt{summaryCard} : Atribut bertipe \texttt{Summary} yang merepresentasikan model data untuk kartu ringkasan.
        \item \texttt{totalLinks}, \texttt{webpageLinks}, \texttt{brokenLinks} : Koleksi bertipe \texttt{ObservableList} yang menampung hasil pemeriksaan untuk ditampilkan ke tabel.
        \item \texttt{initialize()} : Metode yang dijalankan otomatis saat antarmuka dimuat untuk melakukan inisialisasi awal.
        \item \texttt{onStartClick()} : Metode yang menangani event ketika tombol \textit{Start} ditekan dan memulai proses pemeriksaan tautan.
        \item \texttt{onStopClick()} : Metode yang digunakan untuk menghentikan proses pemeriksaan yang sedang berjalan.
        \item \texttt{onExportClick()} : Metode untuk mengekspor hasil pemeriksaan (belum diimplementasikan).
        \item \texttt{validateSeedUrl(rawUrl : String)} : Metode untuk memvalidasi dan menormalkan URL awal sesuai aturan RFC 3986.
        \item \texttt{normalizePath(path : String)} : Metode untuk membersihkan segmen path dari karakter titik (dot-segment).
        \item \texttt{showAlert(message : String)} : Metode utilitas untuk menampilkan pesan informasi di UI.
    \end{itemize}

    \item \textbf{Kelas Summary}\\
    Kelas \texttt{Summary} merepresentasikan model data yang menyimpan informasi ringkasan proses pemeriksaan tautan. Kelas ini berelasi dengan \texttt{Controller} untuk menampilkan nilai ke label pada antarmuka pengguna.
    \begin{itemize}
        \item \texttt{checkingStatus} : Atribut bertipe \texttt{enum} yang memiliki nilai \texttt{IDLE}, \texttt{CHECKING}, \texttt{STOPPED}, dan \texttt{COMPLETED}.
        \item \texttt{totalLink} : Menyimpan jumlah total tautan yang diperiksa.
        \item \texttt{webpageLink} : Menyimpan jumlah halaman web yang berhasil di-\textit{crawl}.
        \item \texttt{brokenLink} : Menyimpan jumlah tautan rusak yang ditemukan.
    \end{itemize}

    \item \textbf{Kelas Application}\\
    Kelas \texttt{Application} merupakan titik masuk (\textit{entry point}) dari program JavaFX. Kelas ini memuat berkas \texttt{FXML} utama dan menginisialisasi objek \texttt{Controller}.
    \begin{itemize}
        \item \texttt{main()} : Metode utama untuk menjalankan aplikasi.
        \item \texttt{start(stage : Stage)} : Metode untuk memuat antarmuka pengguna dan menampilkan jendela utama aplikasi.
    \end{itemize}
\end{enumerate}


