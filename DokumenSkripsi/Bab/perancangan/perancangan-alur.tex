Perancangan alur digunakan untuk menggambarkan interaksi antarobjek pada aplikasi yang dikembangkan selama proses pemeriksaan tautan berlangsung. Hasil pada perancangan ini ditujukan melalui \textit{sequence diagram} pada Gambar~\ref{fig:sequence-diagram}.

\vspace{1mm}

Berikut adalah penjelasan dari \textit{sequence diagram} tersebut:
\begin{enumerate}[itemsep=1.4mm]
    \item (Langkah 1) Pengguna menekan tombol \textit{Start} sehingga metode \texttt{onStartClick()} pada kelas \texttt{MainController} dipanggil.
    
    \item (Langkah 2) \texttt{MainController} meminta \texttt{UrlHandler} untuk menormalisasi URL masukan melalui metode \texttt{normalizeUrl()}.
    
    \item (Langkah 3) \texttt{UrlHandler} mengembalikan URL hasil normalisasi kepada \texttt{MainController}.
    
    \item (Langkah 4) Jika URL tidak valid, \texttt{MainController} menampilkan pesan kesalahan kepada pengguna.
    
    \item (Langkah 5) Jika URL valid, \texttt{MainController} memulai proses pemeriksaan dengan memanggil metode \texttt{start()} pada objek \texttt{Crawler}.
    
    \item (Langkah 6) \texttt{Crawler} membuat objek \texttt{Link} baru untuk merepresentasikan URL awal.
    
    \item (Langkah 7) Objek \texttt{Link} awal tersebut dimasukkan oleh \texttt{Crawler} ke dalam antrean \texttt{frontier} urutan paling belakang.
    
    \item (Langkah 8) \texttt{Crawler} mengambil tautan urutan paling depan dari \texttt{frontier}.
    
    \item (Langkah 9) \texttt{Crawler} meminta \texttt{RateLimiter} untuk melakukan \textit{rate limiting} sebelum permintaan HTTP dilakukan dengan memanggil metode \texttt{delay()}.
    
    \item (Langkah 10) \texttt{Crawler} melakukan permintaan HTTP untuk pemeriksaan tautan dengan memanggil metode \texttt{checkLink()}.
    
    \item (Langkah 11) \texttt{Crawler} mengirimkan tautan yang telah diperiksa ke \texttt{MainController} melalui metode \texttt{receive()}.
    
    \item (Langkah 12) \texttt{MainController} menampilkan tautan rusak kepada pengguna.
    
    \item (Langkah 13) \texttt{Crawler} memanggil metode \texttt{extractLink()} untuk mengekstraksi tautan dari dokumen HTML.
    
    \item (Langkah 14) Untuk setiap tautan yang berhasil diekstraksi dari halaman tersebut, \texttt{Crawler} memanggil \texttt{normalizeUrl()} pada \texttt{UrlHandler} untuk menormalisasi URL tautan.
    
    \item (Langkah 15) \texttt{UrlHandler} mengembalikan URL hasil normalisasi kepada \texttt{Crawler}.
    
    \item (Langkah 16) \texttt{Crawler} membuat objek \texttt{Link} baru berdasarkan URL hasil normalisasi.
    
    \item (Langkah 17) Jika tautan merupakan halaman dari situs web target, maka \texttt{Crawler} memasukkannya ke \texttt{frontier} urutan paling belakang.
    
    \item (Langkah 18) \texttt{Crawler} meminta \texttt{RateLimiter} untuk melakukan \textit{rate limiting} sebelum permintaan HTTP dilakukan dengan memanggil metode \texttt{delay()}.
    
    \item (Langkah 19) \texttt{Crawler} melakukan permintaan HTTP untuk pemeriksaan tautan dengan memanggil metode \texttt{checkLink()}.
    
    \item (Langkah 20) \texttt{Crawler} mengirimkan tautan yang telah diperiksa ke \texttt{MainController} melalui metode \texttt{receive()}.

    \item (Langkah 21) \texttt{MainController} menampilkan tautan rusak kepada pengguna.
    
    \item (Langkah 22) Ketika seluruh tautan dalam \texttt{frontier} sudah selesai diproses, \texttt{Crawler} mengirimkan sinyal bahwa proses pemeriksaan telah selesai.
    
    \item (Langkah 23) \texttt{MainController} menampilkan status bahwa proses pemeriksaan telah selesai kepada pengguna.
    
\end{enumerate}


\begin{figure}[H]
    \centering
    \includegraphics[width=0.95\textwidth]{Gambar/040200-sequence-diagram.png}
    \caption{Sequence diagram proses pemeriksaan tautan}
    \label{fig:sequence-diagram}
\end{figure}