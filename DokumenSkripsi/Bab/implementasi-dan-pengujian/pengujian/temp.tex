% Pada pengujian ini akan dilakukan eksplorasi terhadap parameter-parameter operasional yang memengaruhi performa dan hasil pemeriksaan. Tujuan dari eksplorasi ini adalah untuk mendapatkan nilai terbaik pada setiap parameter sehingga durasi pemeriksaan dan hasil pemeriksaan dapat optimal. Meskipun demikian, hasil pemeriksaan akan menjadi prioritas utama dalam menentukan nilai terbaik untuk setiap parameter. Setelah didapatkan nilai terbaik pada setiap parameter, akan dilakukan eksplorasi lanjutan dengan membandingkan hasil pemeriksaan pada perangkat lunak yang dikembangkan dengan perangkat lunak serupa.

\subsubsection{Hasil Pengujian Eksperimental}
\label{subsubsec:05020301-hasil-pengujian-eksperimental}
Setiap pengujian akan difokuskan pada satu parameter dan parameter lain akan dibuat konstan. Selain itu, seluruh percobaan akan menggunakan subjek yang sama, yaitu situs web Informatika UNPAR\footnote{\url{https://informatika.unpar.ac.id} (Diakses pada 6 Desember 2025)}. Pendekatan ini dilakukan agar pengaruh masing-masing parameter dapat diamati secara terpisah dan memastikan bahwa setiap variasi hasil benar-benar disebabkan oleh perubahan pada nilai parameter yang sedang diuji. Pada setiap percobaan dilakukan evaluasi terhadap durasi pemeriksaan, jumlah total tautan, jumlah tautan halaman, serta jumlah tautan rusak yang ditemukan.

Berikut adalah daftar parameter yang akan dieksplorasi:
\begin{itemize}
   % [itemsep=4pt]
   \item \textbf{\texttt{INTERVAL} pada \texttt{RateLimiter}}: Parameter ini menentukan jarak antarpermintaan HTTP pada \textit{host} yang sama.
   
   \item \textbf{\texttt{CONNECTION\_TIMEOUT} pada \texttt{HttpClient}}: Parameter ini menentukan batas waktu pembentukan koneksi dalam permintaan HTTP.
   
   \item \textbf{\texttt{REQUEST\_TIMEOUT} pada \texttt{HttpRequest}}: Parameter ini menentukan batas waktu menunggu respons \textit{server} dalam permintaan HTTP
\end{itemize}

\begin{enumerate}
   % ######################################################################################
   % ######################################################################################
   \item Pengujian pada Parameter \texttt{INTERVAL} \\
   Pengujian ini dilakukan untuk melihat pengaruh variasi nilai \texttt{INTERVAL} pada \texttt{RateLimiter} terhadap performa dan hasil pemeriksaan. Eksplorasi pada parameter ini menggunakan nilai dalam satuan milidetik (\textit{milisecond}), dengan nilai percobaan yaitu 0, 500, 1000, 1500, dan 2000. Dalam pengujian ini parameter \texttt{CONNECTION\_TIMEOUT} akan bernilai konstan 20~detik dan \texttt{REQUEST\_TIMEOUT} bernilai konstan 20~detik.
   
   \setlength{\LTcapwidth}{\textwidth}
\renewcommand{\arraystretch}{1.4}

\begin{longtable}{
|>{\centering\arraybackslash}m{2.2cm}
|>{\centering\arraybackslash}m{2.2cm}
|>{\centering\arraybackslash}m{2.8cm}
|>{\centering\arraybackslash}m{2.8cm}
|>{\centering\arraybackslash}m{3.5cm}|}
\caption{Hasil Eksplorasi Parameter \texttt{INTERVAL}}
\vspace{-3mm}
\label{tab:hasil-pengujian-interval} \\
\hline
\textbf{Interval} &
\textbf{Durasi} &
\textbf{Total Tautan} &
\textbf{Tautan Halaman} &
\textbf{Tautan Rusak} \\
\hline
\endfirsthead

\multicolumn{5}{c}{Tabel~\ref{tab:hasil-pengujian-interval} dilanjutkan dari halaman sebelumnya}\\[4pt]
\hline
\textbf{Interval} &
\textbf{Durasi} &
\textbf{Total Tautan} &
\textbf{Tautan Halaman} &
\textbf{Tautan Rusak} \\
\hline
\endhead

\hline
\multicolumn{5}{|r|}{Bersambung ke halaman berikutnya} \\ \hline
\endfoot

\hline
\endlastfoot

% ====== ISI DATA DI SINI ======

0 & -- & -- & -- & -- (Lampiran~\ref{tab:percobaan-interval-0}) \\ \hline


500 & -- & -- & -- & -- (Lampiran~\ref{tab:percobaan-interval-500}) \\ \hline


1000 & 14m 38s & 603 & 369 & 75 (Lampiran~\ref{tab:percobaan-interval-1000}) \\ \hline


1500 & 15m 11s & 603 & 369 & 75 (Lampiran~\ref{tab:percobaan-interval-1500}) \\ \hline


2000 & 16m 34s & 603 & 369 & 75 (Lampiran~\ref{tab:percobaan-interval-2000}) \\ \hline


\end{longtable}
   
   
   Hasil dari pengujian ini ditampilkan pada Tabel~\ref{tab:hasil-pengujian-interval}, sedangkan durasi pemeriksaan pada setiap percobaan adalah sebagai berikut:
   \begin{itemize}
      % [itemsep=2pt]
      \item Percobaan dengan nilai 0 milidetik berdurasi 13 menit 6 detik.
      \item Percobaan dengan nilai 500 milidetik berdurasi 13 menit 24 detik.
      \item Percobaan dengan nilai 1000 milidetik berdurasi 14 menit 30 detik.
      \item Percobaan dengan nilai 1500 milidetik berdurasi 15 menit 11 detik.
      \item Percobaan dengan nilai 2000 milidetik berdurasi 16 menit 34 detik.
   \end{itemize}


   Berdasarkan hasil pengujian yang telah didapatkan, diperoleh beberapa temuan sebagai berikut:
   \begin{enumerate}
      \item Secara umum, seluruh kategori \textit{error} memiliki jumlah yang sama pada setiap percobaan, meskipun terdapat perbedaan kecil pada percobaan dengan nilai 500~milidetik, yaitu jumlah SSL \textit{Error} yang lebih rendah dan jumlah \textit{Timeout} yang lebih tinggi dibandingkan percobaan lainnya. Setelah ditelusuri, perbedaan ini disebabkan oleh satu tautan, yaitu \url{https://www.klikhotel.com/}, yang pada seluruh percobaan lain teridentifikasi sebagai SSL \textit{Error}, namun pada percobaan 500~milidetik menjadi \textit{Timeout}. Setelah dilakukan pemeriksaan secara terpisah pada tautan ini, didapati bahwa hasil pemeriksaan menghasilkan \textit{error} dengan kategori SSL \textit{Error}.

      \item Semakin besar nilai \texttt{INTERVAL}, semakin lama waktu yang dibutuhkan untuk menyelesaikan pemeriksaan. Peningkatan durasi ini bersifat konsisten dan linear, serta tidak menghasilkan perubahan pada jumlah hasil pemeriksaan.
   \end{enumerate}

   Berdasarkan temuan tersebut, dapat disimpulkan bahwa parameter \texttt{INTERVAL} tidak memberikan pengaruh terhadap akurasi hasil pemeriksaan, karena seluruh variasi nilai \texttt{INTERVAL} menghasilkan jumlah tautan rusak yang relatif identik. Dengan demikian, pemilihan nilai terbaik dilakukan berdasarkan efisiensi waktu, yaitu nilai interval yang memberikan durasi pemeriksaan tercepat. Nilai tersebut adalah \textbf{0~milidetik} dengan durasi pemeriksaan 13 menit 6 detik.

   % ######################################################################################
   % ######################################################################################
   \item Pengujian pada Parameter \texttt{CONNECTION\_TIMEOUT} \\
   Pengujian ini dilakukan untuk melihat pengaruh variasi nilai \texttt{CONNECTION\_TIMEOUT} pada \texttt{HttpClient} terhadap performa dan hasil pemeriksaan. Eksplorasi pada parameter ini menggunakan nilai dalam satuan detik, dengan nilai percobaan yaitu 5, 10, 15, 20, dan 25. Dalam pengujian ini parameter \texttt{INTERVAL} akan bernilai konstan 0~milidetik sesuai dengan pengujian sebelumnya dan \texttt{REQUEST\_TIMEOUT} bernilai konstan 20~detik. Hasil pengujian ini ditampilkan pada Tabel~\ref{tab:hasil-pengujian-connection-timeout}.

   \setlength{\LTcapwidth}{\textwidth}
\renewcommand{\arraystretch}{1.4}

\begin{longtable}{
|>{\centering\arraybackslash}m{2.2cm}
|>{\centering\arraybackslash}m{2.2cm}
|>{\centering\arraybackslash}m{2.8cm}
|>{\centering\arraybackslash}m{2.8cm}
|>{\centering\arraybackslash}m{3.5cm}|}
\caption{Hasil Eksplorasi Parameter \texttt{CONNECTION\_TIMEOUT}}
\vspace{-3mm}
\label{tab:hasil-pengujian-connection-timeout} \\
\hline
\textbf{Connection Timeout} &
\textbf{Durasi} &
\textbf{Total Tautan} &
\textbf{Tautan Halaman} &
\textbf{Tautan Rusak} \\
\hline
\endfirsthead

\multicolumn{5}{c}{Tabel~\ref{tab:hasil-pengujian-connection-timeout} dilanjutkan dari halaman sebelumnya}\\[4pt]
\hline
\textbf{Connection Timeout} &
\textbf{Durasi} &
\textbf{Total Tautan} &
\textbf{Tautan Halaman} &
\textbf{Tautan Rusak} \\
\hline
\endhead

\hline
\multicolumn{5}{|r|}{Bersambung ke halaman berikutnya} \\ \hline
\endfoot

\hline
\endlastfoot

% ====== ISI DATA DI SINI ======

5 & 14m 50s & 585 & 343 & 91 (Lampiran~\ref{tab:percobaan-connection-timeout-5}) \\ \hline

10 & 14m 01s & 602 & 368 & 79 (Lampiran~\ref{tab:percobaan-connection-timeout-10}) \\ \hline

15 & 14m 01s & 602 & 368 & 79 (Lampiran~\ref{tab:percobaan-connection-timeout-15}) \\ \hline

20 & 17m 16s & 602 & 368 & 76 (Lampiran~\ref{tab:percobaan-connection-timeout-20}) \\ \hline

25 & 17m 16s & 602 & 368 & 76 (Lampiran~\ref{tab:percobaan-connection-timeout-25}) \\ \hline



\end{longtable}


   Pada pengujian ini didapatkan durasi pemeriksaan pada setiap percobaan sebagai berikut:
   \begin{itemize}[itemsep=2pt]
      \item Percobaan dengan nilai 5 detik berdurasi 12 menit 27 detik.
      \item Percobaan dengan nilai 10 detik berdurasi 12 menit 59 detik.
      \item Percobaan dengan nilai 15 detik berdurasi 13 menit 55 detik.
      \item Percobaan dengan nilai 20 detik berdurasi 14 menit 18 detik.
      \item Percobaan dengan nilai 25 detik berdurasi 13 menit 24 detik.
   \end{itemize}

   Berdasarkan hasil pengujian yang telah didapatkan, diperoleh beberapa temuan sebagai berikut:
   \begin{enumerate}[itemsep=3pt]
      \item Nilai \texttt{CONNECTION\_TIMEOUT} memiliki pengaruh terhadap hasil pemeriksaan. Hal ini terlihat dari variasi jumlah \textit{Timeout}, yang cenderung menurun pada nilai \texttt{CONNECTION\_TIMEOUT} yang lebih besar. Nilai \texttt{CONNECTION\_TIMEOUT} yang terlalu kecil menyebabkan beberapa permintaan gagal terselesaikan dalam batas waktu, sehingga menghasilkan jumlah \textit{Timeout} yang lebih tinggi.

      \item Beberapa kategori \textit{error} lain, seperti SSL \textit{Error}, 403 \textit{Forbidden}, dan 404 \textit{Not Found}, menunjukkan pola yang lebih stabil pada nilai \texttt{CONNECTION\_TIMEOUT} yang lebih besar. Hal ini mengindikasikan bahwa nilai \texttt{CONNECTION\_TIMEOUT} yang terlalu rendah dapat menyebabkan pemeriksaan berhenti sebelum respons dari \textit{server} diterima sepenuhnya.

      \item Durasi pemeriksaan cenderung meningkat seiring bertambahnya nilai \texttt{CONNECTION\_TIMEOUT}, meskipun terdapat anomali pada nilai 25 detik yang menghasilkan durasi pemeriksaan lebih cepat dibandingkan nilai 15 dan 20 detik. Perbedaan ini diduga disebabkan oleh faktor eksternal seperti kestabilan jaringan internet.
   
      \item Pada kategori SSL \textit{Error}, nilai 5 dan 10 detik menghasilkan 5 kasus, sedangkan nilai lainnya menghasilkan 6 kasus. Setelah ditelusuri, satu tautan yang seharusnya teridentifikasi sebagai SSL \textit{Error} justru berpindah ke kategori \textit{Timeout} pada nilai \texttt{CONNECTION\_TIMEOUT} 5 dan 10 detik.

      \item Pada kategori 404 \textit{Not Found}, jumlah \textit{error} meningkat dari 27 (nilai 5 dan 10 detik) menjadi 28 (nilai 15 detik) dan kemudian 29 (nilai 20 dan 25 detik). Perbedaan ini muncul karena satu tautan yang seharusnya menghasilkan respons 404 masuk ke kategori \textit{Timeout} pada nilai \texttt{CONNECTION\_TIMEOUT} yang lebih rendah, sedangkan satu tautan tambahan baru dapat terdeteksi sebagai 404 ketika halaman sumbernya berhasil dimuat pada nilai \texttt{CONNECTION\_TIMEOUT} yang lebih besar.

      \item Pada kategori \textit{Timeout}, nilai 20 dan 25 detik menghasilkan 3 kasus yang sama. Setelah dilakukan pemeriksaan manual, keempat tautan tersebut terbukti memiliki waktu muat yang memang lambat. Sebaliknya, tambahan jumlah \textit{Timeout} pada nilai 5, 10, dan 15 detik berasal dari tautan yang tidak terlalu lambat, namun membutuhkan waktu muat yang lebih berat, sehingga gagal menyelesaikan koneksi pada batas waktu yang lebih kecil.

   \end{enumerate}

   Berdasarkan temuan tersebut, nilai \texttt{CONNECTION\_TIMEOUT} yang terlalu kecil menghasilkan jumlah \textit{Timeout} yang lebih tinggi dan menyebabkan pemeriksaan menjadi kurang akurat. Sebaliknya, nilai \texttt{CONNECTION\_TIMEOUT} yang terlalu besar tidak memberikan peningkatan berarti dan hanya menambah durasi pemeriksaan. Sesuai dengan prinsip pengujian bahwa akurasi hasil pemeriksaan lebih diprioritaskan dibandingkan durasi eksekusi, maka pemilihan nilai terbaik harus mempertimbangkan keakuratan hasil terlebih dahulu. Dengan demikian, nilai terbaik untuk parameter ini adalah \textbf{20 detik}, karena memberikan hasil yang lebih stabil dan masih sama dengan hasil pada pengujian \texttt{INTERVAL} dengan nilai 0~milidetik.

   % ######################################################################################
   % ######################################################################################
   \item Pengujian pada Parameter \texttt{REQUEST\_TIMEOUT}
   Pengujian ini dilakukan untuk melihat pengaruh variasi nilai \texttt{REQUEST\_TIMEOUT} pada \texttt{HttpRequest} terhadap performa dan hasil pemeriksaan. Eksplorasi pada parameter ini menggunakan nilai dalam satuan detik, dengan nilai percobaan yaitu 5, 10, 15, 20, dan 25. Dalam pengujian ini parameter \texttt{INTERVAL} akan bernilai konstan 0~milidetik dan \texttt{CONNECTION\_TIMEOUT} bernilai konstan 20~detik sesuai dengan penetapan nilai terbaik sebelumnya. Hasil pengujian ini ditampilkan pada Tabel~\ref{tab:hasil-pengujian-request-timeout}.

   \setlength{\LTcapwidth}{\textwidth}
\renewcommand{\arraystretch}{1.4}

\begin{longtable}{
|>{\centering\arraybackslash}m{2.8cm}
|>{\centering\arraybackslash}m{2.8cm}
|>{\centering\arraybackslash}m{2.8cm}
|>{\centering\arraybackslash}m{2.8cm}
|>{\centering\arraybackslash}m{2.8cm}|}
\caption{Pengaruh Variasi \textit{Request Timeout} pada Hasil Pemeriksaan}
\vspace{-3mm}
\label{tab:hasil-pengujian-request-timeout} \\
\hline
\textbf{Request Timeout} &
\textbf{Durasi} &
\textbf{Total Tautan} &
\textbf{Tautan Halaman} &
\textbf{Tautan Rusak} \\
\hline
\endfirsthead

\multicolumn{5}{c}{Tabel~\ref{tab:hasil-pengujian-request-timeout} dilanjutkan dari halaman sebelumnya}\\[4pt]
\hline
\textbf{Request Timeout} &
\textbf{Durasi} &
\textbf{Total Tautan} &
\textbf{Tautan Halaman} &
\textbf{Tautan Rusak} \\
\hline
\endhead

\hline
\multicolumn{5}{|r|}{Bersambung ke halaman berikutnya} \\ \hline
\endfoot

\hline
\endlastfoot

% ====== ISI DATA DI SINI ======

5 & 14m 50s & 585 & 343 & 91 (Tabel~\ref{tab:percobaan-request-timeout-5}) \\ \hline

10 & 14m 01s & 602 & 368 & 79 (Tabel~\ref{tab:percobaan-request-timeout-10}) \\ \hline

15 & 14m 01s & 602 & 368 & 79 (Tabel~\ref{tab:percobaan-request-timeout-15}) \\ \hline

20 & 17m 16s & 602 & 368 & 76 (Tabel~\ref{tab:percobaan-request-timeout-20}) \\ \hline

25 & 17m 16s & 602 & 368 & 76 (Tabel~\ref{tab:percobaan-request-timeout-25}) \\ \hline

30 & 17m 16s & 602 & 368 & 76 (Tabel~\ref{tab:percobaan-request-timeout-30}) \\ \hline


\end{longtable}


   Pada pengujian ini didapatkan durasi pemeriksaan pada setiap percobaan sebagai berikut:
   \begin{itemize}[itemsep=2pt]
      \item Percobaan dengan nilai 5 detik berdurasi 7 menit 42 detik.
      \item Percobaan dengan nilai 10 detik berdurasi 12 menit 21 detik.
      \item Percobaan dengan nilai 15 detik berdurasi 14 menit 6 detik.
      \item Percobaan dengan nilai 20 detik berdurasi 13 menit 53 detik.
      \item Percobaan dengan nilai 25 detik berdurasi 13 menit 50 detik.
   \end{itemize}

   Berdasarkan hasil pengujian yang telah didapatkan, diperoleh beberapa temuan sebagai berikut:
   \begin{enumerate}
      \item Nilai \texttt{REQUEST\_TIMEOUT} memiliki pengaruh yang jelas terhadap hasil pemeriksaan. Hal ini terlihat dari jumlah \textit{Timeout} yang sangat tinggi pada nilai 5 detik (22 kasus), kemudian menurun drastis pada nilai yang lebih besar hingga mencapai 3 kasus pada nilai 20 dan 25 detik. Nilai \texttt{REQUEST\_TIMEOUT} yang terlalu kecil menyebabkan permintaan gagal sebelum respons \textit{server} diterima, sehingga menghasilkan \textit{Timeout} yang tidak mencerminkan kondisi tautan yang sebenarnya.


      \item Pada kategori SSL \textit{Error}, nilai 5 detik menghasilkan 5 kasus, berbeda dengan nilai lainnya yang menunjukkan 6 kasus. Setelah ditelusuri, satu tautan yang seharusnya menghasilkan SSL \textit{Error} justru berpindah ke kategori \textit{Timeout}. Tautan tersebut tidak lambat dalam pembentukan koneksi, tetapi lambat dalam memberikan respons setelah koneksi berhasil terbentuk, sehingga dengan nilai timeout yang kecil tautan tersebut gagal menyelesaikan proses dan tercatat sebagai \textit{Timeout}.

      \item Pada kategori \textit{Timeout}, jumlah kasus pada nilai 5, 10, dan 15 detik lebih tinggi dibandingkan nilai 20 dan 25 detik. Setelah dilakukan penelusuran, tidak semua tambahan kasus tersebut merupakan tautan yang benar-benar lambat. Beberapa tautan hanya membutuhkan waktu pemuatan yang lebih berat, sehingga pada nilai timeout yang kecil pemeriksaan tidak sempat menerima respons. Kondisi jaringan yang tidak stabil selama pemeriksaan juga memperbesar kemungkinan terjadinya \textit{Timeout} pada nilai timeout yang rendah.

      \item Pada kategori 404 \textit{Not Found}, ditemukan bahwa satu tautan yang lambat pada pengujian \texttt{CONNECTION\_TIMEOUT} dengan nilai 5 dan 10 detik kembali teridentifikasi sebagai \textit{Timeout} pada pengujian ini. Hal ini menunjukkan bahwa tautan tersebut lambat baik dalam proses pembentukan koneksi maupun dalam proses pengiriman respons, sehingga gagal menghasilkan status \texttt{404} pada nilai timeout yang kecil.

      \item Pada kategori 403 \textit{Forbidden}, jumlah kasus pada nilai 5, 10, dan 15 detik lebih rendah dibandingkan nilai 20 dan 25 detik. Setelah dilakukan penelusuran, beberapa tautan yang seharusnya menghasilkan respons 403 gagal dimuat pada nilai timeout yang kecil. Hal ini menyebabkan tautan tersebut tidak sempat menerima respons 403 dari \textit{server} dan masuk ke kategori \textit{Timeout}. Fenomena ini juga muncul pada pengujian \texttt{CONNECTION\_TIMEOUT}, tetapi dijelaskan pada bagian ini agar tidak terjadi pengulangan.
   \end{enumerate}

   Sesuai dengan prinsip pengujian bahwa akurasi hasil pemeriksaan lebih diprioritaskan dibandingkan durasi eksekusi, maka penentuan nilai terbaik harus didasarkan pada konsistensi hasil yang diperoleh. Berdasarkan data pada Tabel~\ref{tab:hasil-pengujian-request-timeout}, nilai \texttt{REQUEST\_TIMEOUT} yang terlalu kecil seperti 5 dan 10 detik menghasilkan jumlah \textit{Timeout} yang jauh lebih tinggi, yaitu masing-masing 22 dan 12 kasus, sehingga tidak mencerminkan kondisi tautan yang sebenarnya. Mulai pada nilai 20 detik, jumlah \textit{Timeout} menurun dan stabil pada 4 kasus, serta kategori \textit{error} lainnya menunjukkan pola yang konsisten. Durasi pemeriksaan pada nilai 20 detik juga masih berada dalam rentang yang baik, yaitu 13 menit 53 detik. Oleh karena itu, berdasarkan stabilitas hasil pemeriksaan, nilai terbaik untuk parameter ini adalah \textbf{20 detik}.
   
\end{enumerate}


\subsubsection{Kesimpulan Pengujian Eksperimental}
\label{subsubsec:05020302-kesimpulan-pengujian-eksperimental}
Berdasarkan rangkaian pengujian yang telah dilakukan terhadap tiga parameter operasional, yaitu \texttt{INTERVAL}, \texttt{CONNECTION\_TIMEOUT}, dan \texttt{REQUEST\_TIMEOUT}, diperoleh kesimpulan bahwa setiap parameter memiliki pengaruh yang berbeda terhadap durasi pemeriksaan dan hasil identifikasi tautan rusak. Sesuai dengan prinsip pengujian bahwa akurasi hasil pemeriksaan lebih diprioritaskan dibandingkan durasi eksekusi, penetapan nilai terbaik pada setiap parameter dilakukan berdasarkan konsistensi hasil serta minimnya \textit{error} yang tidak mencerminkan kondisi tautan yang sebenarnya.


Pada pengujian \texttt{INTERVAL}, seluruh variasi nilai menghasilkan jumlah tautan rusak yang relatif identik sehingga parameter ini tidak berpengaruh terhadap akurasi pemeriksaan, dan nilai terbaik ditetapkan pada \textbf{0~milidetik} karena memberikan durasi tercepat. Pada pengujian \texttt{CONNECTION\_TIMEOUT}, nilai yang terlalu kecil menghasilkan banyak \textit{error} \textit{Timeout}, sedangkan nilai yang lebih besar memberikan hasil yang lebih stabil, sehingga nilai terbaik adalah \textbf{20 detik}. Pola yang sama juga ditemukan pada pengujian \texttt{REQUEST\_TIMEOUT}, di mana nilai yang rendah menyebabkan kegagalan permintaan HTTP dan meningkatnya jumlah \textit{error} \textit{Timeout}, sementara hasil mulai stabil pada nilai 20 detik. Dengan demikian, nilai terbaik untuk parameter ini juga adalah \textbf{20 detik}.