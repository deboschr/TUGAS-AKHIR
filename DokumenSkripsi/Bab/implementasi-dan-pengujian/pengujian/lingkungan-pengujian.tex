Lingkungan pengujian menjelaskan konfigurasi perangkat komputer dan jaringan yang digunakan dalam proses pengujian aplikasi desktop pemeriksa tautan rusak. Berikut merupakan rincian lingkungan pengujian yang digunakan:


\begin{enumerate}
   \item \textbf{Lingkungan Perangkat Komputer}\\
   Pengujian dilakukan menggunakan sebuah komputer dengan spesifikasi sebagai berikut:
   \begin{itemize}
      \item Komputer: Lenovo Legion 5 (Model 82B5)
      \item \textit{Processor}: AMD Ryzen 5 4600H (12 \textit{Cores}, 16 \textit{Threads}), 3.0\,GHz
      \item \textit{Random Access Memory} (RAM): 16\,GB DDR4
      \item \textit{Solid State Drive} (SSD): SSD 512\,GB
      \item Sistem Operasi: Windows 11 64-bit
   \end{itemize}

   \item \textbf{Lingkungan Jaringan}\\
   Seluruh pengujian dilakukan menggunakan \textit{tethering} Wi-Fi dari ponsel sebagai sumber koneksi internet, dengan konfigurasi sebagai berikut:
   \begin{itemize}
      \item Jenis Jaringan: Seluler 4G/LTE
      \item Provider Internet: Telkomsel
      \item Rata-rata Kecepatan Unduh: 58{,}49\,Mbps
      \item Rata-rata Kecepatan Unggah: 46{,}63\,Mbps
      \item Rata-rata Latensi: 23\,ms
   \end{itemize}
\end{enumerate}
