Pada pengujian ini akan dilakukan eksplorasi terhadap terhadap parameter-parameter operasional yang memengaruhi performa dan hasil pemeriksaan. Tujuan dari eksplorasi ini adalah untuk mendapatkan nilai terbaik pada setiap parameter sehingga durasi pemeriksaan dan hasil pemeriksaan dapat optimal. Meskipun demikin, hasil pemeriksaan akan lebih diprioritaskan dalam menentukan nilai terbaik pada masing-masing parameter. Setelah didapatkan nilai terbaik pada setiap paramater, akan dilakukan eksplorasi lanjutan dengan membandingan hasil pemeriksaan pada perangkat lunak yang dikembangkan dengan perangkat lunak serupa.

Setiap pengujian akan difokuskan pada satu parameter dan paramater lain akan dibuat konstan. Selain itu, seluruh percobaan akan menggunakan subjek yang sama, yaitu situs web Program Studi Informatika Universitas Katolik Parahyangan yang beralamat pada \url{https://informatika.unpar.ac.id}. Pendekatan ini dilakukan agar pengaruh masing-masing parameter dapat diamati secara terpisah dan memastikan bahwa setiap variasi hasil benar-benar disebabkan oleh perubahan pada nilai parameter yang sedang diuji. Pada setiap percobaan dilakukan evaluasi terhadap duriasi pemeriksaan, jumlah total tautan, jumlah tautan halaman, serta jumlah tautan rusak yang ditemukan.

\noindent
Berikut adalah daftar parameter yang akan di eksplorasi:
\begin{itemize}[itemsep=4pt]
   \item \textbf{\texttt{INTERVAL} pada \texttt{RateLimiter}}: Parameter ini menentukan jarak antarpermintaan HTTP pada \textit{host} yang sama.
   
   \item \textbf{\texttt{CONNECTION\_TIMEOUT} pada \texttt{HttpClient}}: Parameter ini menentukan batas waktu pembentukan koneksi dalam permintaan HTTP.
   
   \item \textbf{\texttt{REQUEST\_TIMEOUT} pada \texttt{HttpRequest}}: Parameter ini menentukan batas waktu menunggu respons \textit{server} dalam permintaan HTTP

\end{itemize}


\subsubsection{Pengujian 1: \texttt{INTERVAL} pada \texttt{RateLimiter}}
\label{subsubsec:05020301-pengujian-1-interval-rate-limiter}
Pengujian ini dilakukan untuk melihat pengaruh variasi nilai \texttt{INTERVAL} pada \texttt{RateLimiter} terhadap performa dan hasil pemeriksaan. Eksplorasi pada parameter ini menggunakan nilai dalam satuan milidetik, dengan nilai percobaan yaitu 0, 500, 1000, 1500, dan 2000. Dalam pengujian ini parameter \texttt{CONNECTION\_TIMEOUT} akan bernilai konstan 20~detik dan \texttt{REQUEST\_TIMEOUT} bernilai konstan 20~detik. Hasil dari pengujian ini ditampilkan pada Tabel~\ref{tab:hasil-pengujian-interval}.

Berdasarkan hasil pengujian yang ditampilkan pada Tabel~\ref{tab:hasil-pengujian-interval}, berikut adalah temuan yang didapatkan:
\begin{enumerate}
   \item A
   \item B
   \item C
\end{enumerate}

\setlength{\LTcapwidth}{\textwidth}
\renewcommand{\arraystretch}{1.4}

\begin{longtable}{
|>{\centering\arraybackslash}m{2.2cm}
|>{\centering\arraybackslash}m{2.2cm}
|>{\centering\arraybackslash}m{2.8cm}
|>{\centering\arraybackslash}m{2.8cm}
|>{\centering\arraybackslash}m{3.5cm}|}
\caption{Hasil Eksplorasi Parameter \texttt{INTERVAL}}
\vspace{-3mm}
\label{tab:hasil-pengujian-interval} \\
\hline
\textbf{Interval} &
\textbf{Durasi} &
\textbf{Total Tautan} &
\textbf{Tautan Halaman} &
\textbf{Tautan Rusak} \\
\hline
\endfirsthead

\multicolumn{5}{c}{Tabel~\ref{tab:hasil-pengujian-interval} dilanjutkan dari halaman sebelumnya}\\[4pt]
\hline
\textbf{Interval} &
\textbf{Durasi} &
\textbf{Total Tautan} &
\textbf{Tautan Halaman} &
\textbf{Tautan Rusak} \\
\hline
\endhead

\hline
\multicolumn{5}{|r|}{Bersambung ke halaman berikutnya} \\ \hline
\endfoot

\hline
\endlastfoot

% ====== ISI DATA DI SINI ======

0 & -- & -- & -- & -- (Lampiran~\ref{tab:percobaan-interval-0}) \\ \hline


500 & -- & -- & -- & -- (Lampiran~\ref{tab:percobaan-interval-500}) \\ \hline


1000 & 14m 38s & 603 & 369 & 75 (Lampiran~\ref{tab:percobaan-interval-1000}) \\ \hline


1500 & 15m 11s & 603 & 369 & 75 (Lampiran~\ref{tab:percobaan-interval-1500}) \\ \hline


2000 & 16m 34s & 603 & 369 & 75 (Lampiran~\ref{tab:percobaan-interval-2000}) \\ \hline


\end{longtable}





% ###########################################################################
% ###########################################################################
% ###########################################################################
\subsubsection{Pengujian 2: \texttt{CONNECTION\_TIMEOUT} pada \texttt{HttpClient}}
\label{subsubsec:05020302-pengujian-2-connection-timeout-httpclient}
Pengujian ini dilakukan untuk melihat pengaruh variasi nilai \texttt{CONNECTION\_TIMEOUT} pada \texttt{HttpClient} terhadap performa dan hasil pemeriksaan. Eksplorasi pada parameter ini menggunakan nilai dalam satuan detik, dengan nilai percobaan yaitu 5, 10, 15, 20, dan 25. Dalam pengujian ini parameter \texttt{INTERVAL} akan bernilai konstan 1500~milidetik dan \texttt{REQUEST\_TIMEOUT} bernilai konstan 20~detik. Hasil dari pengujian ini ditampilkan pada Tabel~\ref{tab:hasil-pengujian-connection-timeout}.

Berdasarkan hasil pengujian yang ditampilkan pada Tabel~\ref{tab:hasil-pengujian-connection-timeout}, berikut adalah temuan yang didapatkan:
\begin{enumerate}
   \item A
   \item B
   \item C
\end{enumerate}

\setlength{\LTcapwidth}{\textwidth}
\renewcommand{\arraystretch}{1.4}

\begin{longtable}{
|>{\centering\arraybackslash}m{2.2cm}
|>{\centering\arraybackslash}m{2.2cm}
|>{\centering\arraybackslash}m{2.8cm}
|>{\centering\arraybackslash}m{2.8cm}
|>{\centering\arraybackslash}m{3.5cm}|}
\caption{Hasil Eksplorasi Parameter \texttt{CONNECTION\_TIMEOUT}}
\vspace{-3mm}
\label{tab:hasil-pengujian-connection-timeout} \\
\hline
\textbf{Connection Timeout} &
\textbf{Durasi} &
\textbf{Total Tautan} &
\textbf{Tautan Halaman} &
\textbf{Tautan Rusak} \\
\hline
\endfirsthead

\multicolumn{5}{c}{Tabel~\ref{tab:hasil-pengujian-connection-timeout} dilanjutkan dari halaman sebelumnya}\\[4pt]
\hline
\textbf{Connection Timeout} &
\textbf{Durasi} &
\textbf{Total Tautan} &
\textbf{Tautan Halaman} &
\textbf{Tautan Rusak} \\
\hline
\endhead

\hline
\multicolumn{5}{|r|}{Bersambung ke halaman berikutnya} \\ \hline
\endfoot

\hline
\endlastfoot

% ====== ISI DATA DI SINI ======

5 & 14m 50s & 585 & 343 & 91 (Lampiran~\ref{tab:percobaan-connection-timeout-5}) \\ \hline

10 & 14m 01s & 602 & 368 & 79 (Lampiran~\ref{tab:percobaan-connection-timeout-10}) \\ \hline

15 & 14m 01s & 602 & 368 & 79 (Lampiran~\ref{tab:percobaan-connection-timeout-15}) \\ \hline

20 & 17m 16s & 602 & 368 & 76 (Lampiran~\ref{tab:percobaan-connection-timeout-20}) \\ \hline

25 & 17m 16s & 602 & 368 & 76 (Lampiran~\ref{tab:percobaan-connection-timeout-25}) \\ \hline



\end{longtable}



% ###########################################################################
% ###########################################################################
% ###########################################################################
\subsubsection{Pengujian 3: \texttt{REQUEST\_TIMEOUT} pada \texttt{HttpRequest}}
\label{subsubsec:05020303-pengujian-3-request-timeout-httprequest}
Pengujian ini dilakukan untuk melihat pengaruh variasi nilai \texttt{REQUEST\_TIMEOUT} pada \texttt{HttpRequest} terhadap performa dan hasil pemeriksaan. Eksplorasi pada parameter ini menggunakan nilai dalam satuan detik, dengan nilai percobaan yaitu 5, 10, 15, 20, dan 25. Dalam pengujian ini parameter \texttt{INTERVAL} akan bernilai konstan 1500~milidetik dan \texttt{CONNECTION\_TIMEOUT} bernilai konstan 20~detik. Hasil dari pengujian ini ditampilkan pada Tabel~\ref{tab:hasil-pengujian-request-timeout}.

Berdasarkan hasil pengujian yang ditampilkan pada Tabel~\ref{tab:hasil-pengujian-request-timeout}, berikut adalah temuan yang didapatkan:
\begin{enumerate}
   \item A
   \item B
   \item C
\end{enumerate}

\setlength{\LTcapwidth}{\textwidth}
\renewcommand{\arraystretch}{1.4}

\begin{longtable}{
|>{\centering\arraybackslash}m{2.8cm}
|>{\centering\arraybackslash}m{2.8cm}
|>{\centering\arraybackslash}m{2.8cm}
|>{\centering\arraybackslash}m{2.8cm}
|>{\centering\arraybackslash}m{2.8cm}|}
\caption{Pengaruh Variasi \textit{Request Timeout} pada Hasil Pemeriksaan}
\vspace{-3mm}
\label{tab:hasil-pengujian-request-timeout} \\
\hline
\textbf{Request Timeout} &
\textbf{Durasi} &
\textbf{Total Tautan} &
\textbf{Tautan Halaman} &
\textbf{Tautan Rusak} \\
\hline
\endfirsthead

\multicolumn{5}{c}{Tabel~\ref{tab:hasil-pengujian-request-timeout} dilanjutkan dari halaman sebelumnya}\\[4pt]
\hline
\textbf{Request Timeout} &
\textbf{Durasi} &
\textbf{Total Tautan} &
\textbf{Tautan Halaman} &
\textbf{Tautan Rusak} \\
\hline
\endhead

\hline
\multicolumn{5}{|r|}{Bersambung ke halaman berikutnya} \\ \hline
\endfoot

\hline
\endlastfoot

% ====== ISI DATA DI SINI ======

5 & 14m 50s & 585 & 343 & 91 (Tabel~\ref{tab:percobaan-request-timeout-5}) \\ \hline

10 & 14m 01s & 602 & 368 & 79 (Tabel~\ref{tab:percobaan-request-timeout-10}) \\ \hline

15 & 14m 01s & 602 & 368 & 79 (Tabel~\ref{tab:percobaan-request-timeout-15}) \\ \hline

20 & 17m 16s & 602 & 368 & 76 (Tabel~\ref{tab:percobaan-request-timeout-20}) \\ \hline

25 & 17m 16s & 602 & 368 & 76 (Tabel~\ref{tab:percobaan-request-timeout-25}) \\ \hline

30 & 17m 16s & 602 & 368 & 76 (Tabel~\ref{tab:percobaan-request-timeout-30}) \\ \hline


\end{longtable}



% ###########################################################################
% ###########################################################################
% ###########################################################################
\subsubsection{Kesimpulan Hasil Pengujian}
\label{subsubsec:05020306-kesimpulan-hasil-pengujian}

\subsubsection{Perbandingan dengan Perangkat Lunak Serupa}
\label{subsubsec:05020304-perbandingan-dengan-perangkat-lunak-serupa}

\setlength{\LTcapwidth}{\textwidth}
\renewcommand{\arraystretch}{1.4}

\begin{longtable}{
|>{\centering\arraybackslash}m{5cm}
|>{\centering\arraybackslash}m{3cm}
|>{\centering\arraybackslash}m{3cm}
|>{\centering\arraybackslash}m{3cm}|}
\caption{Pengujian pada Informatika UNPAR}
\vspace{-3mm}
\label{tab:hasil-pengujian-if-unpar} \\
\hline
\textbf{Error} &
\textbf{Broken Link Scanner} &
\textbf{Broken Link Checker} &
\textbf{Dead Link Checker} \\
\hline
\endfirsthead

\multicolumn{4}{c}{Tabel~\ref{tab:hasil-pengujian-if-unpar} dilanjutkan dari halaman sebelumnya}\\[4pt]
\hline
\textbf{Error} &
\textbf{Broken Link Scanner} &
\textbf{Broken Link Checker} &
\textbf{Dead Link Checker} \\
\hline
\endhead

\hline
\multicolumn{4}{|r|}{Bersambung ke halaman berikutnya} \\ \hline
\endfoot

\hline
\endlastfoot

% ====== ISI DATA DI SINI ======

Host Not Found & 14 & 11 & 7 \\ \hline

I/O Error & 1 & -- & -- \\ \hline

Invalid URL & 1 & -- & -- \\ \hline

SSL Error & 6 & -- & 5 \\ \hline

Timeout & 4 & 3 & 2 \\ \hline

400 Bad Request & 2 & 1 & 1 \\ \hline

403 Forbidden & 12 & -- & 3 \\ \hline

404 Not Found & 29 & 23 & 35 \\ \hline

405 Method Not Allowed & -- & -- & 1 \\ \hline

410 Gone & 1 & -- & -- \\ \hline

429 Too Many Requests & -- & -- & 3 \\ \hline

502 Bad Gateway & 1 & 1 & 1 \\ \hline

520 & 2 & 1 & 2 \\ \hline

999 & 2 & -- & -- \\ \hline



\end{longtable}


\setlength{\LTcapwidth}{\textwidth}
\renewcommand{\arraystretch}{1.9}

\begin{longtable}{
|>{\centering\arraybackslash}m{6cm}
|>{\centering\arraybackslash}m{3cm}
|>{\centering\arraybackslash}m{3cm}
|>{\centering\arraybackslash}m{3cm}|}
\caption{Hasil pengujian perbandingan pada Informatika UNPAS}
\vspace{-3mm}
\label{tab:hasil-pengujian-if-unpas} \\
\hline
\textbf{Error} &
\textbf{Broken Link Scanner} &
\textbf{Broken Link Checker} &
\textbf{Dead Link Checker} \\
\hline
\endfirsthead

\multicolumn{4}{c}{Tabel~\ref{tab:hasil-pengujian-if-unpas} dilanjutkan dari halaman sebelumnya}\\[4pt]
\hline
\textbf{Error} &
\textbf{Broken Link Scanner} &
\textbf{Broken Link Checker} &
\textbf{Dead Link Checker} \\
\hline
\endhead

\hline
\multicolumn{4}{|r|}{Bersambung ke halaman berikutnya} \\ \hline
\endfoot

\hline
\endlastfoot

% ====== ISI DATA DI SINI ======

Host Not Found & 34 & 44 & 33 \\ \hline

Timeout & 2 & 2 & 2 \\ \hline

SSL Error & 1 & 0 & 1 \\ \hline

Too many redirections & 0 & 0 & 11 \\ \hline

400 Bad Request & 0 & 0 & 1 \\ \hline

404 Not Found & 8 & 20 & 20 \\ \hline

403 Forbidden & 15 & 0 & 6 \\ \hline

405 Method Not Allowed & 0 & 0 & 2 \\ \hline

429 Too Many Requests & 0 & 0 & 1 \\ \hline

502 Bad Gateway & 1 & 1 & 1 \\ \hline

522 & 0 & 1 & 0 \\ \hline


\end{longtable}

\setlength{\LTcapwidth}{\textwidth}
\renewcommand{\arraystretch}{1.4}

\begin{longtable}{
|>{\centering\arraybackslash}m{6cm}
|>{\centering\arraybackslash}m{3cm}
|>{\centering\arraybackslash}m{3cm}
|>{\centering\arraybackslash}m{3cm}|}
\caption{Hasil pengujian pada Informatika UNPAD}
\vspace{-3mm}
\label{tab:hasil-pengujian-if-unpad} \\
\hline
\textbf{Error} &
\textbf{Broken Link Scanner} &
\textbf{Broken Link Checker} &
\textbf{Dead Link Checker} \\
\hline
\endfirsthead

\multicolumn{4}{c}{Tabel~\ref{tab:hasil-pengujian-if-unpad} dilanjutkan dari halaman sebelumnya}\\[4pt]
\hline
\textbf{Error} &
\textbf{Broken Link Scanner} &
\textbf{Broken Link Checker} &
\textbf{Dead Link Checker} \\
\hline
\endhead

\hline
\multicolumn{4}{|r|}{Bersambung ke halaman berikutnya} \\ \hline
\endfoot

\hline
\endlastfoot

% ====== ISI DATA DI SINI ======

Host Not Found & 10 & 10 & 11 \\ \hline

Connection Failed & 4 & 0 & 2 \\ \hline

Timeout & 10 & 2 & 2 \\ \hline

Invalid URL & 5 & 0 & 0 \\ \hline

SSL Error & 2 & 0 & 0 \\ \hline

Too many redirections & 0 & 0 & 2 \\ \hline

400 Bad Request & 0 & 0 & 1 \\ \hline

401 Unauthorized & 11 & 0 & 11 \\ \hline

403 Forbidden & 6 & 0 & 4 \\ \hline

404 Not Found & 214 & 20 & 397 \\ \hline

405 Method Not Allowed & 0 & 0 & 2 \\ \hline

429 Too Many Requests & 0 & 0 & 1 \\ \hline

502 Bad Gateway & 1 & 0 & 0 \\ \hline

503 Service Unavailable & 1 & 1 & 1 \\ \hline

520 & 1 & 2 & 2 \\ \hline

522 & 5 & 1 & 0 \\ \hline

530 & 3 & 3 & 3 \\ \hline

999 & 1 & 0 & 1 \\ \hline



\end{longtable}

\setlength{\LTcapwidth}{\textwidth}
\renewcommand{\arraystretch}{1.4}

\begin{longtable}{
|>{\centering\arraybackslash}m{6cm}
|>{\centering\arraybackslash}m{3cm}
|>{\centering\arraybackslash}m{3cm}
|>{\centering\arraybackslash}m{3cm}|}
\caption{Pengujian pada Informatika UNIKOM}
\vspace{-3mm}
\label{tab:hasil-pengujian-if-unikom} \\
\hline
\textbf{Error} &
\textbf{Broken Link Scanner} &
\textbf{Broken Link Checker} &
\textbf{Dead Link Checker} \\
\hline
\endfirsthead

\multicolumn{4}{c}{Tabel~\ref{tab:hasil-pengujian-if-unikom} dilanjutkan dari halaman sebelumnya}\\[4pt]
\hline
\textbf{Error} &
\textbf{Broken Link Scanner} &
\textbf{Broken Link Checker} &
\textbf{Dead Link Checker} \\
\hline
\endhead

\hline
\multicolumn{4}{|r|}{Bersambung ke halaman berikutnya} \\ \hline
\endfoot

\hline
\endlastfoot

% ====== ISI DATA DI SINI ======

-- & -- & -- & -- \\ \hline

-- & -- & -- & -- \\ \hline

-- & -- & -- & -- \\ \hline

-- & -- & -- & -- \\ \hline

-- & -- & -- & -- \\ \hline

-- & -- & -- & -- \\ \hline



\end{longtable}
