Pada pengujian ini akan dilakukan eksplorasi terhadap parameter-parameter operasional yang memengaruhi performa dan hasil pemeriksaan. Tujuan dari eksplorasi ini adalah untuk mendapatkan nilai terbaik pada setiap parameter sehingga durasi pemeriksaan dan hasil pemeriksaan dapat optimal. Meskipun demikian, hasil pemeriksaan akan menjadi prioritas utama dalam menentukan nilai terbaik untuk setiap parameter. Setelah didapatkan nilai terbaik pada setiap parameter, akan dilakukan eksplorasi lanjutan dengan membandingkan hasil pemeriksaan pada perangkat lunak yang dikembangkan dengan perangkat lunak serupa.

Setiap pengujian akan difokuskan pada satu parameter dan parameter lain akan dibuat konstan. Selain itu, seluruh percobaan akan menggunakan subjek yang sama, yaitu situs web Program Studi Informatika Universitas Katolik Parahyangan\footnote{\url{https://informatika.unpar.ac.id} (Diakses pada 6 Desember 2025)}. Pendekatan ini dilakukan agar pengaruh masing-masing parameter dapat diamati secara terpisah dan memastikan bahwa setiap variasi hasil benar-benar disebabkan oleh perubahan pada nilai parameter yang sedang diuji. Pada setiap percobaan dilakukan evaluasi terhadap durasi pemeriksaan, jumlah total tautan, jumlah tautan halaman, serta jumlah tautan rusak yang ditemukan.

\noindent
Berikut adalah daftar parameter yang akan dieksplorasi:
\begin{itemize}[itemsep=4pt]
   \item \textbf{\texttt{INTERVAL} pada \texttt{RateLimiter}}: Parameter ini menentukan jarak antarpermintaan HTTP pada \textit{host} yang sama.
   
   \item \textbf{\texttt{CONNECTION\_TIMEOUT} pada \texttt{HttpClient}}: Parameter ini menentukan batas waktu pembentukan koneksi dalam permintaan HTTP.
   
   \item \textbf{\texttt{REQUEST\_TIMEOUT} pada \texttt{HttpRequest}}: Parameter ini menentukan batas waktu menunggu respons \textit{server} dalam permintaan HTTP
\end{itemize}


\subsubsection{Pengujian 1: \texttt{INTERVAL} pada \texttt{RateLimiter}}
\label{subsubsec:05020301-pengujian-1-interval-rate-limiter}
Pengujian ini dilakukan untuk melihat pengaruh variasi nilai \texttt{INTERVAL} pada \texttt{RateLimiter} terhadap performa dan hasil pemeriksaan. Eksplorasi pada parameter ini menggunakan nilai dalam satuan milidetik (\textit{milisecond}), dengan nilai percobaan yaitu 0, 500, 1000, 1500, dan 2000. Dalam pengujian ini parameter \texttt{CONNECTION\_TIMEOUT} akan bernilai konstan 20~detik dan \texttt{REQUEST\_TIMEOUT} bernilai konstan 20~detik. 


Hasil dari pengujian ini ditampilkan pada Tabel~\ref{tab:hasil-pengujian-interval}, sedangkan durasi pemeriksaan pada setiap percobaan adalah sebagai berikut:
\begin{itemize}[itemsep=2pt]
   \item Percobaan dengan nilai 0 milidetik berdurasi 13 menit 6 detik.
   \item Percobaan dengan nilai 500 milidetik berdurasi 13 menit 24 detik.
   \item Percobaan dengan nilai 1000 milidetik berdurasi 14 menit 30 detik.
   \item Percobaan dengan nilai 1500 milidetik berdurasi 15 menit 11 detik.
   \item Percobaan dengan nilai 2000 milidetik berdurasi 16 menit 34 detik.
\end{itemize}

\vspace{50mm}

Berdasarkan hasil pengujian yang telah didapatkan, diperoleh beberapa temuan sebagai berikut:
\begin{enumerate}
   \item Secara umum, seluruh kategori \textit{error} memiliki jumlah yang sama pada setiap percobaan, meskipun terdapat perbedaan kecil pada percobaan dengan nilai 500~milidetik, yaitu jumlah SSL \textit{Error} yang lebih rendah dan jumlah \textit{Timeout} yang lebih tinggi dibandingkan percobaan lainnya. Setelah ditelusuri, perbedaan ini disebabkan oleh satu tautan, yaitu \url{https://www.klikhotel.com/}, yang pada seluruh percobaan lain teridentifikasi sebagai SSL \textit{Error}, namun pada percobaan 500~milidetik menjadi \textit{Timeout}. Setelah dilakukan pemeriksaan secara terpisah pada tautan ini, didapati bahwa hasil pemeriksaan menghasilkan \textit{error} dengan kategori SSL \textit{Error}.

   \item Semakin besar nilai \texttt{INTERVAL}, semakin lama waktu yang dibutuhkan untuk menyelesaikan pemeriksaan. Peningkatan durasi ini bersifat konsisten dan linear, serta tidak menghasilkan perubahan pada jumlah hasil pemeriksaan.
\end{enumerate}

\vspace{2mm}

Berdasarkan temuan tersebut, dapat disimpulkan bahwa parameter \texttt{INTERVAL} tidak memberikan pengaruh terhadap akurasi hasil pemeriksaan, karena seluruh variasi nilai \texttt{INTERVAL} menghasilkan jumlah tautan rusak yang relatif identik. Dengan demikian, pemilihan nilai terbaik dilakukan berdasarkan efisiensi waktu, yaitu nilai interval yang memberikan durasi pemeriksaan tercepat. Nilai tersebut adalah \textbf{0~milidetik} dengan durasi pemeriksaan 13 menit 6 detik.


\setlength{\LTcapwidth}{\textwidth}
\renewcommand{\arraystretch}{1.4}

\begin{longtable}{
|>{\centering\arraybackslash}m{2.2cm}
|>{\centering\arraybackslash}m{2.2cm}
|>{\centering\arraybackslash}m{2.8cm}
|>{\centering\arraybackslash}m{2.8cm}
|>{\centering\arraybackslash}m{3.5cm}|}
\caption{Hasil Eksplorasi Parameter \texttt{INTERVAL}}
\vspace{-3mm}
\label{tab:hasil-pengujian-interval} \\
\hline
\textbf{Interval} &
\textbf{Durasi} &
\textbf{Total Tautan} &
\textbf{Tautan Halaman} &
\textbf{Tautan Rusak} \\
\hline
\endfirsthead

\multicolumn{5}{c}{Tabel~\ref{tab:hasil-pengujian-interval} dilanjutkan dari halaman sebelumnya}\\[4pt]
\hline
\textbf{Interval} &
\textbf{Durasi} &
\textbf{Total Tautan} &
\textbf{Tautan Halaman} &
\textbf{Tautan Rusak} \\
\hline
\endhead

\hline
\multicolumn{5}{|r|}{Bersambung ke halaman berikutnya} \\ \hline
\endfoot

\hline
\endlastfoot

% ====== ISI DATA DI SINI ======

0 & -- & -- & -- & -- (Lampiran~\ref{tab:percobaan-interval-0}) \\ \hline


500 & -- & -- & -- & -- (Lampiran~\ref{tab:percobaan-interval-500}) \\ \hline


1000 & 14m 38s & 603 & 369 & 75 (Lampiran~\ref{tab:percobaan-interval-1000}) \\ \hline


1500 & 15m 11s & 603 & 369 & 75 (Lampiran~\ref{tab:percobaan-interval-1500}) \\ \hline


2000 & 16m 34s & 603 & 369 & 75 (Lampiran~\ref{tab:percobaan-interval-2000}) \\ \hline


\end{longtable}


\vspace{10mm}

% ###########################################################################
% ###########################################################################
% ###########################################################################
\subsubsection{Pengujian 2: \texttt{CONNECTION\_TIMEOUT} pada \texttt{HttpClient}}
\label{subsubsec:05020302-pengujian-2-connection-timeout-httpclient}
Pengujian ini dilakukan untuk melihat pengaruh variasi nilai \texttt{CONNECTION\_TIMEOUT} pada \texttt{HttpClient} terhadap performa dan hasil pemeriksaan. Eksplorasi pada parameter ini menggunakan nilai dalam satuan detik, dengan nilai percobaan yaitu 5, 10, 15, 20, dan 25. Dalam pengujian ini parameter \texttt{INTERVAL} akan bernilai konstan 0~milidetik sesuai dengan pengujian sebelumnya dan \texttt{REQUEST\_TIMEOUT} bernilai konstan 20~detik. Hasil pengujian ini ditampilkan pada Tabel~\ref{tab:hasil-pengujian-connection-timeout}.


\setlength{\LTcapwidth}{\textwidth}
\renewcommand{\arraystretch}{1.4}

\begin{longtable}{
|>{\centering\arraybackslash}m{2.2cm}
|>{\centering\arraybackslash}m{2.2cm}
|>{\centering\arraybackslash}m{2.8cm}
|>{\centering\arraybackslash}m{2.8cm}
|>{\centering\arraybackslash}m{3.5cm}|}
\caption{Hasil Eksplorasi Parameter \texttt{CONNECTION\_TIMEOUT}}
\vspace{-3mm}
\label{tab:hasil-pengujian-connection-timeout} \\
\hline
\textbf{Connection Timeout} &
\textbf{Durasi} &
\textbf{Total Tautan} &
\textbf{Tautan Halaman} &
\textbf{Tautan Rusak} \\
\hline
\endfirsthead

\multicolumn{5}{c}{Tabel~\ref{tab:hasil-pengujian-connection-timeout} dilanjutkan dari halaman sebelumnya}\\[4pt]
\hline
\textbf{Connection Timeout} &
\textbf{Durasi} &
\textbf{Total Tautan} &
\textbf{Tautan Halaman} &
\textbf{Tautan Rusak} \\
\hline
\endhead

\hline
\multicolumn{5}{|r|}{Bersambung ke halaman berikutnya} \\ \hline
\endfoot

\hline
\endlastfoot

% ====== ISI DATA DI SINI ======

5 & 14m 50s & 585 & 343 & 91 (Lampiran~\ref{tab:percobaan-connection-timeout-5}) \\ \hline

10 & 14m 01s & 602 & 368 & 79 (Lampiran~\ref{tab:percobaan-connection-timeout-10}) \\ \hline

15 & 14m 01s & 602 & 368 & 79 (Lampiran~\ref{tab:percobaan-connection-timeout-15}) \\ \hline

20 & 17m 16s & 602 & 368 & 76 (Lampiran~\ref{tab:percobaan-connection-timeout-20}) \\ \hline

25 & 17m 16s & 602 & 368 & 76 (Lampiran~\ref{tab:percobaan-connection-timeout-25}) \\ \hline



\end{longtable}


\vspace{3mm}

Pada pengujian ini didapatkan durasi pemeriksaan pada setiap percobaan sebagai berikut:
\begin{itemize}[itemsep=2pt]
   \item Percobaan dengan nilai 5 detik berdurasi 12 menit 27 detik.
   \item Percobaan dengan nilai 10 detik berdurasi 12 menit 59 detik.
   \item Percobaan dengan nilai 15 detik berdurasi 13 menit 55 detik.
   \item Percobaan dengan nilai 20 detik berdurasi 14 menit 18 detik.
   \item Percobaan dengan nilai 25 detik berdurasi 13 menit 24 detik.
\end{itemize}

\vspace{10mm}

Berdasarkan hasil pengujian yang telah didapatkan, diperoleh beberapa temuan sebagai berikut:
\begin{enumerate}[itemsep=3pt]
   \item Nilai \texttt{CONNECTION\_TIMEOUT} memiliki pengaruh terhadap hasil pemeriksaan. Hal ini terlihat dari variasi jumlah \textit{Timeout}, yang cenderung menurun pada nilai \texttt{CONNECTION\_TIMEOUT} yang lebih besar. Nilai \texttt{CONNECTION\_TIMEOUT} yang terlalu kecil menyebabkan beberapa permintaan gagal terselesaikan dalam batas waktu, sehingga menghasilkan jumlah \textit{Timeout} yang lebih tinggi.

   \item Beberapa kategori \textit{error} lain, seperti SSL \textit{Error}, 403 \textit{Forbidden}, dan 404 \textit{Not Found}, menunjukkan pola yang lebih stabil pada nilai \texttt{CONNECTION\_TIMEOUT} yang lebih besar. Hal ini mengindikasikan bahwa nilai \texttt{CONNECTION\_TIMEOUT} yang terlalu rendah dapat menyebabkan pemeriksaan berhenti sebelum respons dari \textit{server} diterima sepenuhnya.

   \item Durasi pemeriksaan cenderung meningkat seiring bertambahnya nilai \texttt{CONNECTION\_TIMEOUT}, meskipun terdapat anomali pada nilai 25 detik yang menghasilkan durasi pemeriksaan lebih cepat dibandingkan nilai 15 dan 20 detik. Perbedaan ini diduga disebabkan oleh faktor eksternal seperti kestabilan jaringan internet.
   
   \item Pada kategori SSL \textit{Error}, nilai 5 dan 10 detik menghasilkan 5 kasus, sedangkan nilai lainnya menghasilkan 6 kasus. Setelah ditelusuri, satu tautan yang seharusnya teridentifikasi sebagai SSL \textit{Error} justru berpindah ke kategori \textit{Timeout} pada nilai \texttt{CONNECTION\_TIMEOUT} 5 dan 10 detik.

   \item Pada kategori 404 \textit{Not Found}, jumlah \textit{error} meningkat dari 27 (nilai 5 dan 10 detik) menjadi 28 (nilai 15 detik) dan kemudian 29 (nilai 20 dan 25 detik). Perbedaan ini muncul karena satu tautan yang seharusnya menghasilkan respons 404 masuk ke kategori \textit{Timeout} pada nilai \texttt{CONNECTION\_TIMEOUT} yang lebih rendah, sedangkan satu tautan tambahan baru dapat terdeteksi sebagai 404 ketika halaman sumbernya berhasil dimuat pada nilai \texttt{CONNECTION\_TIMEOUT} yang lebih besar.

   \item Pada kategori \textit{Timeout}, nilai 20 dan 25 detik menghasilkan 3 kasus yang sama. Setelah dilakukan pemeriksaan manual, keempat tautan tersebut terbukti memiliki waktu muat yang memang lambat. Sebaliknya, tambahan jumlah \textit{Timeout} pada nilai 5, 10, dan 15 detik berasal dari tautan yang tidak terlalu lambat, namun membutuhkan waktu muat yang lebih berat, sehingga gagal menyelesaikan koneksi pada batas waktu yang lebih kecil.

\end{enumerate}

\vspace{4mm}


Berdasarkan temuan tersebut, nilai \texttt{CONNECTION\_TIMEOUT} yang terlalu kecil menghasilkan jumlah \textit{Timeout} yang lebih tinggi dan menyebabkan pemeriksaan menjadi kurang akurat. Sebaliknya, nilai \texttt{CONNECTION\_TIMEOUT} yang terlalu besar tidak memberikan peningkatan berarti dan hanya menambah durasi pemeriksaan. Sesuai dengan prinsip pengujian bahwa akurasi hasil pemeriksaan lebih diprioritaskan dibandingkan durasi eksekusi, maka pemilihan nilai terbaik harus mempertimbangkan keakuratan hasil terlebih dahulu. Dengan demikian, nilai terbaik untuk parameter ini adalah \textbf{20 detik}, karena memberikan hasil yang lebih stabil dan masih sama dengan hasil pada pengujian \texttt{INTERVAL} dengan nilai 0~milidetik.


\vspace{20mm}

% ###########################################################################
% ###########################################################################
% ###########################################################################
\subsubsection{Pengujian 3: \texttt{REQUEST\_TIMEOUT} pada \texttt{HttpRequest}}
\label{subsubsec:05020303-pengujian-3-request-timeout-httprequest}
Pengujian ini dilakukan untuk melihat pengaruh variasi nilai \texttt{REQUEST\_TIMEOUT} pada \texttt{HttpRequest} terhadap performa dan hasil pemeriksaan. Eksplorasi pada parameter ini menggunakan nilai dalam satuan detik, dengan nilai percobaan yaitu 5, 10, 15, 20, dan 25. Dalam pengujian ini parameter \texttt{INTERVAL} akan bernilai konstan 0~milidetik dan \texttt{CONNECTION\_TIMEOUT} bernilai konstan 20~detik sesuai dengan penetapan nilai terbaik sebelumnya. Hasil pengujian ini ditampilkan pada Tabel~\ref{tab:hasil-pengujian-request-timeout}.

\vspace{3mm}

\setlength{\LTcapwidth}{\textwidth}
\renewcommand{\arraystretch}{1.4}

\begin{longtable}{
|>{\centering\arraybackslash}m{2.8cm}
|>{\centering\arraybackslash}m{2.8cm}
|>{\centering\arraybackslash}m{2.8cm}
|>{\centering\arraybackslash}m{2.8cm}
|>{\centering\arraybackslash}m{2.8cm}|}
\caption{Pengaruh Variasi \textit{Request Timeout} pada Hasil Pemeriksaan}
\vspace{-3mm}
\label{tab:hasil-pengujian-request-timeout} \\
\hline
\textbf{Request Timeout} &
\textbf{Durasi} &
\textbf{Total Tautan} &
\textbf{Tautan Halaman} &
\textbf{Tautan Rusak} \\
\hline
\endfirsthead

\multicolumn{5}{c}{Tabel~\ref{tab:hasil-pengujian-request-timeout} dilanjutkan dari halaman sebelumnya}\\[4pt]
\hline
\textbf{Request Timeout} &
\textbf{Durasi} &
\textbf{Total Tautan} &
\textbf{Tautan Halaman} &
\textbf{Tautan Rusak} \\
\hline
\endhead

\hline
\multicolumn{5}{|r|}{Bersambung ke halaman berikutnya} \\ \hline
\endfoot

\hline
\endlastfoot

% ====== ISI DATA DI SINI ======

5 & 14m 50s & 585 & 343 & 91 (Tabel~\ref{tab:percobaan-request-timeout-5}) \\ \hline

10 & 14m 01s & 602 & 368 & 79 (Tabel~\ref{tab:percobaan-request-timeout-10}) \\ \hline

15 & 14m 01s & 602 & 368 & 79 (Tabel~\ref{tab:percobaan-request-timeout-15}) \\ \hline

20 & 17m 16s & 602 & 368 & 76 (Tabel~\ref{tab:percobaan-request-timeout-20}) \\ \hline

25 & 17m 16s & 602 & 368 & 76 (Tabel~\ref{tab:percobaan-request-timeout-25}) \\ \hline

30 & 17m 16s & 602 & 368 & 76 (Tabel~\ref{tab:percobaan-request-timeout-30}) \\ \hline


\end{longtable}


Pada pengujian ini didapatkan durasi pemeriksaan pada setiap percobaan sebagai berikut:
\begin{itemize}[itemsep=2pt]
   \item Percobaan dengan nilai 5 detik berdurasi 7 menit 42 detik.
   \item Percobaan dengan nilai 10 detik berdurasi 12 menit 21 detik.
   \item Percobaan dengan nilai 15 detik berdurasi 14 menit 6 detik.
   \item Percobaan dengan nilai 20 detik berdurasi 13 menit 53 detik.
   \item Percobaan dengan nilai 25 detik berdurasi 13 menit 50 detik.
\end{itemize}

\vspace{20mm}

Berdasarkan hasil pengujian yang telah didapatkan, diperoleh beberapa temuan sebagai berikut:
\begin{enumerate}
   \item Nilai \texttt{REQUEST\_TIMEOUT} memiliki pengaruh yang jelas terhadap hasil pemeriksaan. Hal ini terlihat dari jumlah \textit{Timeout} yang sangat tinggi pada nilai 5 detik (22 kasus), kemudian menurun drastis pada nilai yang lebih besar hingga mencapai 3 kasus pada nilai 20 dan 25 detik. Nilai \texttt{REQUEST\_TIMEOUT} yang terlalu kecil menyebabkan permintaan gagal sebelum respons \textit{server} diterima, sehingga menghasilkan \textit{Timeout} yang tidak mencerminkan kondisi tautan yang sebenarnya.


   \item Pada kategori SSL \textit{Error}, nilai 5 detik menghasilkan 5 kasus, berbeda dengan nilai lainnya yang menunjukkan 6 kasus. Setelah ditelusuri, satu tautan yang seharusnya menghasilkan SSL \textit{Error} justru berpindah ke kategori \textit{Timeout}. Tautan tersebut tidak lambat dalam pembentukan koneksi, tetapi lambat dalam memberikan respons setelah koneksi berhasil terbentuk, sehingga dengan nilai timeout yang kecil tautan tersebut gagal menyelesaikan proses dan tercatat sebagai \textit{Timeout}.

   \item Pada kategori \textit{Timeout}, jumlah kasus pada nilai 5, 10, dan 15 detik lebih tinggi dibandingkan nilai 20 dan 25 detik. Setelah dilakukan penelusuran, tidak semua tambahan kasus tersebut merupakan tautan yang benar-benar lambat. Beberapa tautan hanya membutuhkan waktu pemuatan yang lebih berat, sehingga pada nilai timeout yang kecil pemeriksaan tidak sempat menerima respons. Kondisi jaringan yang tidak stabil selama pemeriksaan juga memperbesar kemungkinan terjadinya \textit{Timeout} pada nilai timeout yang rendah.

   \item Pada kategori 404 \textit{Not Found}, ditemukan bahwa satu tautan yang lambat pada pengujian \texttt{CONNECTION\_TIMEOUT} dengan nilai 5 dan 10 detik kembali teridentifikasi sebagai \textit{Timeout} pada pengujian ini. Hal ini menunjukkan bahwa tautan tersebut lambat baik dalam proses pembentukan koneksi maupun dalam proses pengiriman respons, sehingga gagal menghasilkan status \texttt{404} pada nilai timeout yang kecil.

   \item Pada kategori 403 \textit{Forbidden}, jumlah kasus pada nilai 5, 10, dan 15 detik lebih rendah dibandingkan nilai 20 dan 25 detik. Setelah dilakukan penelusuran, beberapa tautan yang seharusnya menghasilkan respons 403 gagal dimuat pada nilai timeout yang kecil. Hal ini menyebabkan tautan tersebut tidak sempat menerima respons 403 dari \textit{server} dan masuk ke kategori \textit{Timeout}. Fenomena ini juga muncul pada pengujian \texttt{CONNECTION\_TIMEOUT}, tetapi dijelaskan pada bagian ini agar tidak terjadi pengulangan.
\end{enumerate}

Sesuai dengan prinsip pengujian bahwa akurasi hasil pemeriksaan lebih diprioritaskan dibandingkan durasi eksekusi, maka penentuan nilai terbaik harus didasarkan pada konsistensi hasil yang diperoleh. Berdasarkan data pada Tabel~\ref{tab:hasil-pengujian-request-timeout}, nilai \texttt{REQUEST\_TIMEOUT} yang terlalu kecil seperti 5 dan 10 detik menghasilkan jumlah \textit{Timeout} yang jauh lebih tinggi, yaitu masing-masing 22 dan 12 kasus, sehingga tidak mencerminkan kondisi tautan yang sebenarnya. Mulai pada nilai 20 detik, jumlah \textit{Timeout} menurun dan stabil pada 4 kasus, serta kategori \textit{error} lainnya menunjukkan pola yang konsisten. Durasi pemeriksaan pada nilai 20 detik juga masih berada dalam rentang yang baik, yaitu 13 menit 53 detik. Oleh karena itu, berdasarkan stabilitas hasil pemeriksaan, nilai terbaik untuk parameter ini adalah \textbf{20 detik}.






% ###########################################################################
% ###########################################################################
% ###########################################################################
\subsubsection{Kesimpulan Akhir Pengujian}
\label{subsubsec:05020303-kesimpulan-akhir-pengujian}
Berdasarkan rangkaian pengujian yang telah dilakukan terhadap tiga parameter operasional, yaitu \texttt{INTERVAL}, \texttt{CONNECTION\_TIMEOUT}, dan \texttt{REQUEST\_TIMEOUT}, diperoleh kesimpulan bahwa setiap parameter memiliki pengaruh yang berbeda terhadap durasi pemeriksaan dan hasil identifikasi tautan rusak. Sesuai dengan prinsip pengujian bahwa akurasi hasil pemeriksaan lebih diprioritaskan dibandingkan durasi eksekusi, penetapan nilai terbaik pada setiap parameter dilakukan berdasarkan konsistensi hasil serta minimnya \textit{error} yang tidak mencerminkan kondisi tautan yang sebenarnya.


Pada pengujian \texttt{INTERVAL}, seluruh variasi nilai menghasilkan jumlah tautan rusak yang relatif identik sehingga parameter ini tidak berpengaruh terhadap akurasi pemeriksaan, dan nilai terbaik ditetapkan pada \textbf{0~milidetik} karena memberikan durasi tercepat. Pada pengujian \texttt{CONNECTION\_TIMEOUT}, nilai yang terlalu kecil menghasilkan banyak \textit{error} \textit{Timeout}, sedangkan nilai yang lebih besar memberikan hasil yang lebih stabil, sehingga nilai terbaik adalah \textbf{20 detik}. Pola yang sama juga ditemukan pada pengujian \texttt{REQUEST\_TIMEOUT}, di mana nilai yang rendah menyebabkan kegagalan permintaan HTTP dan meningkatnya jumlah \textit{error} \textit{Timeout}, sementara hasil mulai stabil pada nilai 20 detik. Dengan demikian, nilai terbaik untuk parameter ini juga adalah \textbf{20 detik}.



% ###########################################################################
% ###########################################################################
% ###########################################################################
\subsubsection{Perbandingan dengan Perangkat Lunak Serupa}
\label{subsubsec:05020303-perbandingan-dengan-perangkat-lunak-serupa}

Setelah dilakukan penetapan nilai terbaik pada setiap parameter operasional, tahap selanjutnya adalah membandingkan hasil pemeriksaan antara perangkat lunak yang dikembangkan (Broken Link Scanner), dengan dua perangkat lunak serupa lainnya, yaitu Broken Link Checker\footnote{\url{https://www.brokenlinkcheck.com}} dan Dead Link Checker\footnote{\url{https://www.deadlinkchecker.com}}. Pengujian ini dilakukan pada empat situs web, yaitu Informatika UNPAR, Informatika UNPAS, Informatika UNPAD, dan Informatika UNIKOM. Tujuan dari perbandingan ini adalah untuk mengevaluasi perbedaan hasil pemeriksaan pada setiap perangkat lunak sehingga dapat menjadi dasar dalam memberikan saran perbaikan dan arah pengembangan lebih lanjut.

Pada tahap perbandingan ini, penting untuk memahami karakteristik kedua perangkat lunak pembanding, yaitu Broken Link Checker dan Dead Link Checker. Kedua perangkat lunak tersebut memeriksa tautan tidak hanya pada elemen \texttt{<a>} HTML, tetapi juga pada elemen \texttt{img} dengan atribut \texttt{src}, sehingga cakupan pemeriksaannya lebih luas. Selain itu, Broken Link Checker memiliki batas maksimum 3000 halaman dan tidak menganggap \textit{error} terkait keamanan seperti 403 \textit{Forbidden}, 401 \textit{Unauthorized}, maupun SSL \textit{Error} sebagai tautan rusak sehingga kategori tersebut tidak dilaporkan dalam hasil pemeriksaannya. Dead Link Checker juga memiliki batas maksimum 2000 tautan secara keseluruhan, tanpa memisahkan antara halaman dan tautan umum. Sementara itu, Broken Link Scanner hanya memeriksa tautan pada elemen \texttt{<a>} HTML sehingga elemen lain seperti \texttt{img} berada di luar cakupan pemeriksaan. Untuk menjaga kesetaraan kondisi pengujian, jumlah tautan rusak yang diproses oleh Broken Link Scanner dibatasi hingga 2000 tautan agar selaras dengan batas yang dimiliki oleh Dead Link Checker. Perbedaan dalam cakupan elemen, perlakuan terhadap \textit{error} tertentu, dan batas maksimum tautan tersebut menjadi faktor penting yang memengaruhi variasi hasil pemeriksaan antarperangkat lunak.


\vspace{20mm}



Berikut adalah hasil pengujian pada keempat situs web menggunakan tiga perangkat lunak berbeda:

\begin{enumerate}

   \item \textbf{Pengujian pada Informatika UNPAR}\footnote{\url{https://informatika.unpar.ac.id} (Diakses pada 6 Desember 2025)} \\[2pt]
   Hasil pengujian pada situs web Informatika UNPAR ditampilkan pada Tabel~\ref{tab:hasil-pengujian-if-unpar}. 

   % \vspace{5mm}

   \setlength{\LTcapwidth}{\textwidth}
\renewcommand{\arraystretch}{1.4}

\begin{longtable}{
|>{\centering\arraybackslash}m{5cm}
|>{\centering\arraybackslash}m{3cm}
|>{\centering\arraybackslash}m{3cm}
|>{\centering\arraybackslash}m{3cm}|}
\caption{Pengujian pada Informatika UNPAR}
\vspace{-3mm}
\label{tab:hasil-pengujian-if-unpar} \\
\hline
\textbf{Error} &
\textbf{Broken Link Scanner} &
\textbf{Broken Link Checker} &
\textbf{Dead Link Checker} \\
\hline
\endfirsthead

\multicolumn{4}{c}{Tabel~\ref{tab:hasil-pengujian-if-unpar} dilanjutkan dari halaman sebelumnya}\\[4pt]
\hline
\textbf{Error} &
\textbf{Broken Link Scanner} &
\textbf{Broken Link Checker} &
\textbf{Dead Link Checker} \\
\hline
\endhead

\hline
\multicolumn{4}{|r|}{Bersambung ke halaman berikutnya} \\ \hline
\endfoot

\hline
\endlastfoot

% ====== ISI DATA DI SINI ======

Host Not Found & 14 & 11 & 7 \\ \hline

I/O Error & 1 & -- & -- \\ \hline

Invalid URL & 1 & -- & -- \\ \hline

SSL Error & 6 & -- & 5 \\ \hline

Timeout & 4 & 3 & 2 \\ \hline

400 Bad Request & 2 & 1 & 1 \\ \hline

403 Forbidden & 12 & -- & 3 \\ \hline

404 Not Found & 29 & 23 & 35 \\ \hline

405 Method Not Allowed & -- & -- & 1 \\ \hline

410 Gone & 1 & -- & -- \\ \hline

429 Too Many Requests & -- & -- & 3 \\ \hline

502 Bad Gateway & 1 & 1 & 1 \\ \hline

520 & 2 & 1 & 2 \\ \hline

999 & 2 & -- & -- \\ \hline



\end{longtable}


   
   % \vspace{30mm}

   Berikut adalah penjelasan dari perbedaan hasil pemeriksaan pada ketiga perangkat lunak berdasarkan Tabel~\ref{tab:hasil-pengujian-if-unpar}:

   % [itemsep=5pt]
   \begin{itemize}
      \item \textbf{Host Not Found}.  
      Broken Link Scanner mendeteksi 14 tautan dengan kategori \textit{Host Not Found}, lebih banyak dibandingkan Broken Link Checker yang mendeteksi 11 tautan dan Dead Link Checker yang mendeteksi 7 tautan. Setelah dilakukan pemeriksaan manual, seluruh 14 tautan yang ditemukan oleh Broken Link Scanner memang tidak dapat diakses. Selain itu, semua tautan yang teridentifikasi oleh Broken Link Checker dan Dead Link Checker merupakan bagian dari 14 tautan milik Broken Link Scanner.

      % \vspace{10mm}

      \item \textbf{I/O Error}.  
      Pada kategori ini, hanya Broken Link Scanner yang menghasilkan I/O \textit{Error}, sedangkan Broken Link Checker dan Dead Link Checker mengategorikan tautan yang sama sebagai SSL \textit{Error}. URL pada tautan tersebut adalah \url{https://kiri.travel/}. Berdasarkan pemeriksaan manual, situs ini menunjukkan perilaku yang tidak stabil, terkadang menampilkan halaman keamanan, terkadang mengarahkan ke situs yang tidak relevan seperti halaman perjudian, dan pada waktu lain membuka halaman unduhan berkas.

      \item \textbf{Invalid URL}.  
      Perbedaan pada kategori ini disebabkan oleh URL \url{https://itb.ac.id./}. Broken Link Scanner mengelompokkan URL ini sebagai \textit{Invalid} URL karena proses pembentukan objek \texttt{URI} pada Java gagal akibat adanya tanda titik pada akhir domain. Dead Link Checker mengategorikannya sebagai SSL \textit{Error}.

      \item \textbf{SSL Error}.  
      Broken Link Scanner mendeteksi 6 tautan dengan SSL \textit{Error}, sedangkan Dead Link Checker hanya mendeteksi 5 tautan. Setelah dilakukan penelusuran, kelima tautan yang ditemukan oleh Dead Link Checker merupakan bagian dari enam tautan yang ditemukan oleh Broken Link Scanner. 

      \item \textbf{Timeout}.  
      Broken Link Scanner dan Broken Link Checker menghasilkan jumlah \textit{Timeout} yang sama dan tautan yang teridentifikasi pun identik. Namun, Dead Link Checker tidak melaporkan URL \url{https://www.academynetriders.com/file.php/1/netriders_info/pdfs/Results_2015_NetRiders_APAC_CCENT_R2.pdf}, tautan ini berdasarkan pemeriksaan manual merupakan tautan yang memerlukan waktu akses lebih lama.

      \item \textbf{400 Bad Request}.  
      Pada kategori ini, Broken Link Scanner menemukan 2 tautan, sedangkan Broken Link Checker dan Dead Link Checker hanya menemukan 1 tautan. Pemeriksaan manual menunjukkan bahwa kedua tautan yang ditemukan oleh Broken Link Scanner memang mengembalikan status 400.


      \item \textbf{403 Forbidden}.  
      Broken Link Scanner menemukan jumlah 403 \textit{Forbidden} yang lebih tinggi dibandingkan kedua perangkat lunak lainnya. Pemeriksaan menggunakan \textit{browser} dan Postman memperlihatkan bahwa beberapa tautan yang dilaporkan sebagai 403 oleh Broken Link Scanner sebenarnya tidak mengembalikan status tersebut ketika diakses secara manual. Ini adalah salah satu contoh URL-nya \url{https://www.researchgate.net/profile/Pascal_Nugroho/research}

      \item \textbf{404 Not Found}.  
      Seluruh tautan yang teridentifikasi sebagai 404 \textit{Not Found} oleh masing-masing perangkat lunak memang benar-benar mengembalikan status 404. Namun, Dead Link Checker menghasilkan jumlah yang lebih tinggi karena memeriksa tidak hanya tautan pada elemen \texttt{<a>} HTML, tetapi juga tautan pada elemen \texttt{img} dengan atribut \texttt{src}.

      \item \textbf{405 Method Not Allowed}.  
      Hanya Dead Link Checker yang menemukan \textit{error} ini. URL pada tautan tersebut adalah \url{https://informatika.unpar.ac.id/xmlrpc.php}. Setelah diperiksa manual, server pada tautan ini mengharapkan permintaan dengan metode \texttt{POST}, sehingga ketika perangkat lunak menggunakan metode \texttt{GET}, server mengembalikan status 405.

      \item \textbf{410 Gone}.  
      Broken Link Scanner adalah satu-satunya perangkat lunak yang mendeteksi tautan yang mengembalikan status 410. Tautan tersebut adalah \url{https://pascalalfadian.wordpress.com/2018/09/29/catholicer-seminar-trip}. Hasil ini telah dikonfirmasi melalui pemeriksaan manual.

      \item \textbf{429 Too Many Requests}.  
      Hanya Dead Link Checker yang mengembalikan \textit{error} ini. Hal ini kemungkinan terjadi karena cara Dead Link Checker mengirim permintaan yang bersifat paralel, sehingga server mendeteksi permintaan berlebihan dari sumber yang sama dan membalas dengan status 429.

      \vspace{20mm}

      \item \textbf{520 Unknown Error}.  
      Pada kategori ini, Broken Link Scanner dan Dead Link Checker menemukan tautan yang sama, namun Broken Link Checker tidak menemukan tautan \url{https://godev.co/}.

      \item \textbf{999 Non-Standard Error}.
      Berdasarkan hasil pemeriksaan manual, tautan dengan status 999 yang ditemukan Broken Link Scanner memang benar mengembalikan kode status tersebut. Seluruh tautan tersebut berasal dari situs \url{https://www.linkedin.com}.
   \end{itemize}

   
   
   
   \vspace{5mm}
   

   \item \textbf{Pengujian pada Informatika UNPAD}\footnote{\url{https://informatika.unpad.ac.id} (Diakses pada 6 Desember 2025)} \\[2pt]
   Hasil pengujian pada situs web Informatika UNPAR ditampilkan pada Tabel~\ref{tab:hasil-pengujian-if-unpad}.

   
   \setlength{\LTcapwidth}{\textwidth}
\renewcommand{\arraystretch}{1.4}

\begin{longtable}{
|>{\centering\arraybackslash}m{6cm}
|>{\centering\arraybackslash}m{3cm}
|>{\centering\arraybackslash}m{3cm}
|>{\centering\arraybackslash}m{3cm}|}
\caption{Hasil pengujian pada Informatika UNPAD}
\vspace{-3mm}
\label{tab:hasil-pengujian-if-unpad} \\
\hline
\textbf{Error} &
\textbf{Broken Link Scanner} &
\textbf{Broken Link Checker} &
\textbf{Dead Link Checker} \\
\hline
\endfirsthead

\multicolumn{4}{c}{Tabel~\ref{tab:hasil-pengujian-if-unpad} dilanjutkan dari halaman sebelumnya}\\[4pt]
\hline
\textbf{Error} &
\textbf{Broken Link Scanner} &
\textbf{Broken Link Checker} &
\textbf{Dead Link Checker} \\
\hline
\endhead

\hline
\multicolumn{4}{|r|}{Bersambung ke halaman berikutnya} \\ \hline
\endfoot

\hline
\endlastfoot

% ====== ISI DATA DI SINI ======

Host Not Found & 10 & 10 & 11 \\ \hline

Connection Failed & 4 & 0 & 2 \\ \hline

Timeout & 10 & 2 & 2 \\ \hline

Invalid URL & 5 & 0 & 0 \\ \hline

SSL Error & 2 & 0 & 0 \\ \hline

Too many redirections & 0 & 0 & 2 \\ \hline

400 Bad Request & 0 & 0 & 1 \\ \hline

401 Unauthorized & 11 & 0 & 11 \\ \hline

403 Forbidden & 6 & 0 & 4 \\ \hline

404 Not Found & 214 & 20 & 397 \\ \hline

405 Method Not Allowed & 0 & 0 & 2 \\ \hline

429 Too Many Requests & 0 & 0 & 1 \\ \hline

502 Bad Gateway & 1 & 0 & 0 \\ \hline

503 Service Unavailable & 1 & 1 & 1 \\ \hline

520 & 1 & 2 & 2 \\ \hline

522 & 5 & 1 & 0 \\ \hline

530 & 3 & 3 & 3 \\ \hline

999 & 1 & 0 & 1 \\ \hline



\end{longtable}


   \vspace{30mm}

   Berikut adalah penjelasan dari perbedaan hasil pemeriksaan pada ketiga perangkat lunak berdasarkan Tabel~\ref{tab:hasil-pengujian-if-unpad}:

   % [itemsep=5pt]
   \begin{itemize}[itemsep=-1pt]
      \item \textbf{Host Not Found}.  
      Dead Link Checker mendeteksi 11 tautan, lebih banyak dibandingkan Broken Link Scanner dan Broken Link Checker yang masing-masing menemukan 10 tautan. Pemeriksaan manual menunjukkan bahwa 10 tautan pada kedua perangkat lunak tersebut merupakan bagian dari 11 tautan milik Dead Link Checker, sementara satu tautan tambahan adalah \url{http://www.copus.com/}.
      
      \item \textbf{Connection Failed}.  
      Broken Link Scanner menemukan 4 tautan, sementara Dead Link Checker menemukan 2 tautan. Dua tautan milik Dead Link Checker merupakan bagian dari 4 tautan yang ditemukan Broken Link Scanner. Dua tautan tambahan pada Broken Link Scanner masih dapat diakses namun membutuhkan waktu muat yang lebih lama.
      
      \item \textbf{Timeout}.  
      Broken Link Scanner menemukan 10 tautan, sedangkan dua perangkat lunak lainnya hanya menemukan 2 tautan. Kedua tautan tersebut terdapat pada hasil pemeriksaan Broken Link Scanner. Dari 8 tautan tambahan, dua di antaranya benar-benar tidak dapat diakses, sedangkan empat tautan lainnya masih dapat dibuka namun memiliki waktu muat yang lambat.
      
      \item \textbf{Invalid URL}.  
      Hanya Broken Link Scanner yang mendeteksi 5 tautan dengan kategori \textit{Invalid} URL. Salah satu contohnya adalah \url{http://Career Center/}, yang gagal diproses menjadi objek \texttt{URI} pada Java karena formatnya tidak valid.

      \item \textbf{SSL Error}.  
      Dua tautan dikategorikan sebagai SSL \textit{Error} oleh Broken Link Scanner, sedangkan perangkat lunak lain tidak melaporkannya. Pemeriksaan manual menunjukkan bahwa kedua tautan tersebut masih dapat diakses, namun seluruhnya berasal dari situs luar negeri, tautannya adalah \url{https://apps.univ-lr.fr/cgi-bin/WebObjects/Colloque.woa/1/wa/colloque?code=2141} dan \url{http://www.iaeng.org/}.

      \item \textbf{Too Many Redirections}.  
      Hanya Dead Link Checker yang mendeteksi dua tautan dengan kategori ini. Perbedaan ini dapat terjadi karena perangkat lunak tersebut menerapkan batas jumlah \textit{redirection} selama pemeriksaan, sedangkan Broken Link Scanner tidak menerapkan batasan serupa.

      \item \textbf{400 Bad Request}.  
      Satu tautan dikategorikan sebagai 400 \textit{Bad Request} oleh Dead Link Checker. Pemeriksaan menunjukkan bahwa tautan yang sama diidentifikasi sebagai \textit{Invalid} URL oleh Broken Link Scanner, yaitu \url{http://Career Center/}. Hal ini menunjukkan perbedaan perilaku dalam menangani URL dengan format tidak valid.

      \item \textbf{403 Forbidden}.  
      Broken Link Scanner mendeteksi 6 tautan, sedangkan Dead Link Checker mendeteksi 4 tautan. Dua tautan yang berbeda berasal dari domain \url{https://www.acm.org}.

      \item \textbf{404 Not Found}.  
      Dead Link Checker menghasilkan jumlah 404 \textit{Not Found} yang jauh lebih banyak dibandingkan dua perangkat lunak lainnya. Pemeriksaan menunjukkan bahwa sebagian besar tambahan ini berasal dari tautan pada elemen \texttt{img} HTML, yang memang termasuk dalam cakupan pemeriksaan Dead Link Checker.

      \item \textbf{405 Method Not Allowed}.  
      Dua tautan dikategorikan sebagai 405 \textit{Method Not Allowed} oleh Dead Link Checker. Pemeriksaan manual menunjukkan bahwa kedua tautan tersebut dapat diakses menggunakan metode HTTP \texttt{GET}. Salah satu contohnya adalah \url{https://www.dicoding.com/faq}.

      \vspace{20mm}
      \item \textbf{429 Too Many Requests}.  
      Hanya Dead Link Checker yang menemukan \textit{error} ini. Pemeriksaan menunjukkan bahwa tautan tersebut berasal dari domain \url{https://www.instagram.com}.

      \item \textbf{502 Bad Gateway}.  
      Hanya Broken Link Scanner yang mendeteksi \textit{error} ini. Pemeriksaan manual melalui Postman menghasilkan \texttt{200 OK}, sedangkan akses melalui browser menampilkan halaman dari Nginx dengan teks 502. URL tersebut adalah \url{http://cdc.unpad.ac.id/tracer-study}.

      \item \textbf{520}.  
      Broken Link Checker dan Dead Link Checker menemukan 2 tautan, sementara Broken Link Scanner hanya menemukan 1 tautan. Kedua tautan tersebut memang mengembalikan status 520, dan yang tidak muncul pada Broken Link Scanner adalah \url{https://gemastik13.telkomuniversity.ac.id/}.

      \item \textbf{522}.  
      Broken Link Scanner mendeteksi 5 tautan, Broken Link Checker mendeteksi 1 tautan, dan Dead Link Checker tidak mendeteksi tautan dengan status ini. Pemeriksaan manual menunjukkan bahwa kelima tautan tersebut benar-benar mengembalikan status non-standar 522.
   \end{itemize}

   \vspace{5mm}


   \item \textbf{Pengujian pada Informatika UNPAS}\footnote{\url{https://if.unpas.ac.id} (Diakses pada 6 Desember 2025)} \\[2pt]
   Hasil pengujian pada situs web Informatika UNPAR ditampilkan pada Tabel~\ref{tab:hasil-pengujian-if-unpas}. 
   
   \setlength{\LTcapwidth}{\textwidth}
\renewcommand{\arraystretch}{1.9}

\begin{longtable}{
|>{\centering\arraybackslash}m{6cm}
|>{\centering\arraybackslash}m{3cm}
|>{\centering\arraybackslash}m{3cm}
|>{\centering\arraybackslash}m{3cm}|}
\caption{Hasil pengujian perbandingan pada Informatika UNPAS}
\vspace{-3mm}
\label{tab:hasil-pengujian-if-unpas} \\
\hline
\textbf{Error} &
\textbf{Broken Link Scanner} &
\textbf{Broken Link Checker} &
\textbf{Dead Link Checker} \\
\hline
\endfirsthead

\multicolumn{4}{c}{Tabel~\ref{tab:hasil-pengujian-if-unpas} dilanjutkan dari halaman sebelumnya}\\[4pt]
\hline
\textbf{Error} &
\textbf{Broken Link Scanner} &
\textbf{Broken Link Checker} &
\textbf{Dead Link Checker} \\
\hline
\endhead

\hline
\multicolumn{4}{|r|}{Bersambung ke halaman berikutnya} \\ \hline
\endfoot

\hline
\endlastfoot

% ====== ISI DATA DI SINI ======

Host Not Found & 34 & 44 & 33 \\ \hline

Timeout & 2 & 2 & 2 \\ \hline

SSL Error & 1 & 0 & 1 \\ \hline

Too many redirections & 0 & 0 & 11 \\ \hline

400 Bad Request & 0 & 0 & 1 \\ \hline

404 Not Found & 8 & 20 & 20 \\ \hline

403 Forbidden & 15 & 0 & 6 \\ \hline

405 Method Not Allowed & 0 & 0 & 2 \\ \hline

429 Too Many Requests & 0 & 0 & 1 \\ \hline

502 Bad Gateway & 1 & 1 & 1 \\ \hline

522 & 0 & 1 & 0 \\ \hline


\end{longtable}

   
   \vspace{20mm}

   Berikut adalah penjelasan dari perbedaan hasil pemeriksaan pada ketiga perangkat lunak berdasarkan Tabel~\ref{tab:hasil-pengujian-if-unpas}:

   \vspace{2mm}

   \begin{itemize}[itemsep=4pt]
      \item \textbf{Host Not Found}.  
      Ketiga perangkat lunak menghasilkan jumlah \textit{Host Not Found} yang tinggi, meskipun jumlahnya berbeda-beda. Pemeriksaan manual menunjukkan bahwa seluruh tautan tersebut memang tidak dapat diakses, dan masih terdapat irisan tautan yang konsisten di antara ketiganya.

      \item \textbf{Timeout}.
      Meskipun pada ketiga perangkat lunak menunjukan jumlah yang sama pada kategori \textit{error} ini, setelah ditelusuri lebih lanjut terdapat perbedaan tautan antara Broken Link Checker dengan kedua perangkat lainnya. Perbedaanya ada pada tautan dengan URL \url{http://badanbahasa.kemdikbud.go.id/kbbi/}.

      \item \textbf{Too Many Redirections}.  
      Hanya Dead Link Checker yang mendeteksi 11 tautan dengan kategori ini. Perbedaan ini dapat terjadi karena perangkat lunak tersebut menerapkan batas maksimal jumlah \textit{redirection}, sedangkan Broken Link Scanner tidak menerapkan pembatasan serupa sehingga tidak menghasilkan \textit{error} ini.

      \item \textbf{400 Bad Request}.  
      Dead Link Checker mendeteksi satu tautan yang mengembalikan status 400, yaitu \url{https://if.unpas.ac.id/?na=s}. Pemeriksaan manual menunjukkan bahwa respons 400 tersebut valid. Tautan ini diperoleh dari elemen \texttt{form} pada halaman HTML, yang tidak termasuk dalam cakupan pemeriksaan Broken Link Scanner maupun Broken Link Checker.

      \item \textbf{404 Not Found}.  
      Broken Link Scanner hanya mendeteksi 8 tautan, sedangkan dua perangkat lunak lainnya mendeteksi 20 tautan. Perbedaan ini disebabkan oleh cakupan pemeriksaan: Broken Link Checker dan Dead Link Checker memeriksa tautan dari elemen HTML selain \texttt{<a>}, sehingga jumlah tautan 404 \textit{Not Found} yang ditemukan lebih banyak.

      \item \textbf{403 Forbidden}.  
      Broken Link Scanner mendeteksi lebih banyak tautan dengan status 403 dibandingkan Dead Link Checker. Pemeriksaan manual menunjukkan bahwa beberapa tautan yang dilaporkan sebagai 403 oleh Broken Link Scanner sebenarnya dapat diakses secara langsung. Salah satu contohnya adalah \url{https://www.rivalmind.com/what-are-the-benefits-of-seo}.

      \item \textbf{405 Method Not Allowed}.  
      Hanya Dead Link Checker yang mendeteksi dua tautan dengan kategori ini. Pemeriksaan manual menunjukkan bahwa salah satu tautan, yaitu \url{https://www.dicoding.com/blog/apa-itu-server/}, tidak mengembalikan status 405 ketika diakses menggunakan metode HTTP \texttt{GET}.

      \item \textbf{522}.  
      Hanya Broken Link Checker yang mendeteksi \textit{error} ini. Tautannya adalah \url{http://wargasaluyu.unpas.ac.id/}. Pemeriksaan manual melalui Postman menunjukkan respons yang sangat lambat dan HTML hasil permintaan menampilkan teks ``522: Connection timed out''. Setelah ditelusuri lebih jauh lagi, tautan ini teridentifikasi sebagai \textit{error} dengan kategori \textit{Timeout} pada perangkat lunak Broken Link Scanner dan Dead Link Checker.
   \end{itemize}


   \vspace{50mm}


   \item \textbf{Pengujian pada Informatika UNIKOM}\footnote{\url{https://if.unikom.ac.id} (Diakses pada 6 Desember 2025)} \\[2pt]
   Hasil pengujian pada situs web Informatika UNPAR ditampilkan pada Tabel~\ref{tab:hasil-pengujian-if-unikom}. 
   
   \setlength{\LTcapwidth}{\textwidth}
\renewcommand{\arraystretch}{1.4}

\begin{longtable}{
|>{\centering\arraybackslash}m{6cm}
|>{\centering\arraybackslash}m{3cm}
|>{\centering\arraybackslash}m{3cm}
|>{\centering\arraybackslash}m{3cm}|}
\caption{Pengujian pada Informatika UNIKOM}
\vspace{-3mm}
\label{tab:hasil-pengujian-if-unikom} \\
\hline
\textbf{Error} &
\textbf{Broken Link Scanner} &
\textbf{Broken Link Checker} &
\textbf{Dead Link Checker} \\
\hline
\endfirsthead

\multicolumn{4}{c}{Tabel~\ref{tab:hasil-pengujian-if-unikom} dilanjutkan dari halaman sebelumnya}\\[4pt]
\hline
\textbf{Error} &
\textbf{Broken Link Scanner} &
\textbf{Broken Link Checker} &
\textbf{Dead Link Checker} \\
\hline
\endhead

\hline
\multicolumn{4}{|r|}{Bersambung ke halaman berikutnya} \\ \hline
\endfoot

\hline
\endlastfoot

% ====== ISI DATA DI SINI ======

-- & -- & -- & -- \\ \hline

-- & -- & -- & -- \\ \hline

-- & -- & -- & -- \\ \hline

-- & -- & -- & -- \\ \hline

-- & -- & -- & -- \\ \hline

-- & -- & -- & -- \\ \hline



\end{longtable}


   Berikut adalah penjelasan dari perbedaan hasil pemeriksaan pada ketiga perangkat lunak berdasarkan Tabel~\ref{tab:hasil-pengujian-if-unikom}:

   \begin{itemize}[itemsep=-1pt]
      \item \textbf{SSL Error}.  
      Broken Link Scanner mendeteksi 3 tautan dengan SSL \textit{Error}, sedangkan Dead Link Checker hanya mendeteksi 1 tautan. Tautan yang ditemukan oleh Dead Link Checker juga terdapat pada hasil Broken Link Scanner. Dua tautan tambahan pada Broken Link Scanner, yaitu \url{https://smarttour.unikom.ac.id/tour-without-login} dan \url{https://jpi.unikom.ac.id/ifjudul}, telah diperiksa secara manual dan tidak valid sebagai SSL \textit{Error}. Perbedaan ini dapat terjadi karena variasi respons server terhadap pola permintaan dari masing-masing perangkat lunak.

      \item \textbf{404 Not Found}.  
      Dead Link Checker mendeteksi 13 tautan, lebih banyak daripada Broken Link Scanner dan Broken Link Checker yang masing-masing mendeteksi 10 tautan. Pemeriksaan manual menunjukkan bahwa seluruh tautan tersebut memang mengembalikan status 404. Sepuluh tautan yang ditemukan oleh kedua perangkat lunak lainnya merupakan bagian dari 13 tautan milik Dead Link Checker, sementara tiga tautan tambahan berasal dari URL yang diperoleh dari berkas CSS.

      \item \textbf{403 Forbidden}.  
      Hanya Dead Link Checker yang mendeteksi dua tautan dengan status 403. Setelah ditelusuri, kedua tautan tersebut \url{https://if.unikom.ac.id/xmlrpc.php?rsd} dan \url{https://jpi.unikom.ac.id/ifjudul/} teridentifikasi sebagai SSL \textit{Error} oleh Broken Link Scanner. Perbedaan ini kemungkinan disebabkan oleh perbedaan cara perangkat lunak mengelola permintaan HTTPS dan kegagalan \textit{handshake} SSL.

      \item \textbf{429 Too Many Requests}.  
      Hanya Dead Link Checker yang mendeteksi \textit{error} ini. Seluruh tautan berasal dari domain \url{https://www.instagram.com}. Hal ini dapat terjadi karena server membatasi frekuensi permintaan dari sumber yang sama, sehingga permintaan yang dikirim oleh Dead Link Checker ditolak dengan status 429.
   \end{itemize}

\end{enumerate}
