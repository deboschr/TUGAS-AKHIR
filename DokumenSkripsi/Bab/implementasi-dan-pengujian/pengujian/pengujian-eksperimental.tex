Pada pengujian ini akan dilakukan eksplorasi terhadap terhadap parameter-parameter operasional yang memengaruhi performa dan hasil pemeriksaan. Tujuan dari eksplorasi ini adalah untuk mendapatkan nilai terbaik pada setiap parameter sehingga durasi pemeriksaan dan hasil pemeriksaan dapat optimal. Meskipun demikin, hasil pemeriksaan akan lebih diprioritaskan dalam menentukan nilai terbaik pada masing-masing parameter. Setelah didapatkan nilai terbaik pada setiap paramater, akan dilakukan eksplorasi lanjutan dengan membandingan hasil pemeriksaan pada perangkat lunak yang dikembangkan dengan perangkat lunak serupa.

Setiap pengujian akan difokuskan pada satu parameter dan paramater lain akan dibuat konstan. Selain itu, seluruh percobaan akan menggunakan subjek yang sama, yaitu situs web Program Studi Informatika Universitas Katolik Parahyangan yang beralamat pada \url{https://informatika.unpar.ac.id}. Pendekatan ini dilakukan agar pengaruh masing-masing parameter dapat diamati secara terpisah dan memastikan bahwa setiap variasi hasil benar-benar disebabkan oleh perubahan pada nilai parameter yang sedang diuji. Pada setiap percobaan dilakukan evaluasi terhadap duriasi pemeriksaan, jumlah total tautan, jumlah tautan halaman, serta jumlah tautan rusak yang ditemukan.

\noindent
Berikut adalah daftar parameter yang akan di eksplorasi:
\begin{itemize}[itemsep=4pt]
   \item \textbf{Interval pada \texttt{RateLimiter}}: Parameter ini menentukan jarak antarpermintaan HTTP pada \textit{host} yang sama.
   
   \item \textbf{\textit{Connection timeout} pada \texttt{HttpClient}}: Parameter ini menentukan batas waktu pembentukan koneksi dalam permintaan HTTP.
   
   \item \textbf{\textit{Request timeout} pada \texttt{HttpRequest}}: Parameter ini menentukan batas waktu menunggu respons \textit{server} dalam permintaan HTTP

\end{itemize}


\subsubsection{Pengujian 1: Interval pada \texttt{RateLimiter}}
\label{subsubsec:05020301-pengujian-1-interval-rate-limiter}
Pengujian ini dilakukan untuk melihat pengaruh variasi nilai \texttt{interval} pada \texttt{RateLimiter} terhadap hasil pemeriksaan. Pada pengujian ini dilakukan serangkaian percobaan dengan rentang nilai 0--2000~milidetik dengan kenaikan 100~milidetik untuk setiap percobaan. Parameter lain dibuat konstan, yaitu \textit{connection timeout} sebesar 10~detik dan \textit{request timeout} sebesar 10~detik. Pemilihan nilai konstan tersebut dilakukan dengan mengambil nilai tengah dari rentang percobaan masing-masing parameter. Dengan demikian, setiap perubahan pada hasil pemeriksaan sepenuhnya disebabkan oleh variasi nilai \texttt{interval}. Hasil percobaan ditampilkan pada Tabel~\ref{tab:hasil-pengujian-interval}.

Tabel~\ref{tab:hasil-pengujian-interval}.

\subsubsection{Pengujian 2: \textit{Connection Timeout} pada \texttt{HttpClient}}
\label{subsubsec:05020302-pengujian-2-connection-timeout-httpclient}
Pengujian ini dilakukan untuk melihat pengaruh variasi nilai \textit{connection timeout} pada \texttt{HttpClient} terhadap hasil pemeriksaan. Pada pengujian ini dilakukan serangkaian percobaan dengan rentang nilai 1--20~detik dengan kenaikan 1~detik untuk setiap percobaan. Parameter lain dibuat konstan, yaitu \texttt{interval} pada \texttt{RateLimiter} sebesar 1000~milidetik serta \textit{request timeout} sebesar 10~detik. Pemilihan nilai konstan tersebut dilakukan dengan mengambil nilai tengah dari rentang percobaan masing-masing parameter. Dengan demikian, setiap perubahan pada hasil pemeriksaan sepenuhnya disebabkan oleh variasi \textit{connection timeout}. Hasil percobaan ditampilkan pada Tabel~\ref{tab:hasil-pengujian-connection-timeout}.

Tabel~\ref{tab:hasil-pengujian-connection-timeout}

\subsubsection{Pengujian 3: \textit{Request Timeout} pada \texttt{HttpRequest}}
\label{subsubsec:05020303-pengujian-3-request-timeout-httprequest}
Pengujian ini dilakukan untuk melihat pengaruh variasi nilai \textit{request timeout} pada \texttt{HttpRequest} terhadap hasil pemeriksaan. Pada pengujian ini dilakukan serangkaian percobaan dengan rentang nilai 1--20~detik dengan kenaikan 1~detik untuk setiap percobaan. Parameter lain dibuat konstan, yaitu \textit{connection timeout} sebesar 10~detik serta \texttt{interval} pada \texttt{RateLimiter} sebesar 1000~milidetik. Pemilihan nilai konstan tersebut dilakukan dengan mengambil nilai tengah dari rentang percobaan masing-masing parameter. Dengan demikian, setiap perubahan pada hasil pemeriksaan sepenuhnya disebabkan oleh variasi \textit{request timeout}. Hasil percobaan ditampilkan pada Tabel~\ref{tab:hasil-pengujian-request-timeout}.

Tabel~\ref{tab:hasil-pengujian-request-timeout}


\subsubsection{Kesimpulan Hasil Pengujian}
\label{subsubsec:05020306-kesimpulan-hasil-pengujian}

\subsubsection{Perbandingan dengan Perangkat Lunak Serupa}
\label{subsubsec:05020304-perbandingan-dengan-perangkat-lunak-serupa}

\setlength{\LTcapwidth}{\textwidth}
\renewcommand{\arraystretch}{1.4}

\begin{longtable}{
|>{\centering\arraybackslash}m{1cm}
|>{\raggedright\arraybackslash}m{10cm}
|>{\centering\arraybackslash}m{4cm}|}
\caption{Pengujian Dead Link Checker pada Informatika UNPAS}
\vspace{-3mm}
\label{tab:hasil-pengujian-if-unikom} \\
\hline
\multicolumn{1}{|c|}{\textbf{No}} &
\multicolumn{1}{|c|}{\textbf{URL}} &
\multicolumn{1}{|c|}{\textbf{Error}} \\
\hline
\endfirsthead

\multicolumn{3}{c}{Tabel~\ref{tab:hasil-pengujian-if-unikom} dilanjutkan dari halaman sebelumnya}\\[4pt]
\hline
\multicolumn{1}{|c|}{\textbf{No}} &
\multicolumn{1}{|c|}{\textbf{URL}} &
\multicolumn{1}{|c|}{\textbf{Error}} \\
\hline
\endhead

\hline
\multicolumn{3}{|r|}{Bersambung ke halaman berikutnya} \\ \hline
\endfoot

\hline
\endlastfoot

% ====== ISI DATA DI SINI ======

1 & \url{https://kp.if-unpas.org/wordpress/} & 404 Not Found \\ \hline
2 & \url{https://www.instagram.com/accounts/login/?next=https%3A%2F%2Fwww.instagram.com%2Finformatikaunpas%2F%3Fhl%3Den&is_from_rle} & 429 Too Many Requests \\ \hline
3 & \url{http://hmtif.unpas.ac.id/} & -1 Timeout \\ \hline
4 & \url{http://www.thscenter.com/images/career-icon.png} & 404 Not Found \\ \hline
5 & \url{https://lldikti4.kemdikbud.go.id/} & -1 Timeout \\ \hline
6 & \url{https://sinta.kemdikbud.go.id/authors/profile/6661974} & -1 Not found: The server name or address could not be resolved \\ \hline
7 & \url{https://sinta.kemdikbud.go.id/authors/profile/5987512} & -1 Not found: The server name or address could not be resolved \\ \hline
8 & \url{https://sinta.kemdikbud.go.id/authors/profile/6180947} & -1 Not found: The server name or address could not be resolved \\ \hline
9 & \url{https://sso.sevima.com/sessions/authorize?client_id=84f03a0e-a33a-461e-ba01-4eeb500bcf31&redirect_uri=https://situ2.unpas.ac.id/gate/authsso&response_type=code&check=1} & Too many redirections \\ \hline
10 & \url{https://www.scopus.com/authredirect.uri?txGid=49f05a868750ec3ad5eb68f70f709dd6&code=Rv6tcuf8UKLBzG3Wqn_4r-QTWCm0TSqNqUQAyU4P&state=autoLogin|txId%3D2B0E039171E18F3D2D6D1433DEF92D55.i-0825ac76ca18516fa%3A1} & Too many redirections \\ \hline
11 & \url{http://if.unpas.ac.id/wp-content/uploads/2016/10/cropped-unpas-copy-32x32.png} & 404 Not Found \\ \hline
12 & \url{http://if.unpas.ac.id/wp-content/uploads/2016/10/cropped-unpas-copy-192x192.png} & 404 Not Found \\ \hline
13 & \url{http://if.unpas.ac.id/wp-content/uploads/2016/10/cropped-unpas-copy-180x180.png} & 404 Not Found \\ \hline
14 & \url{https://www.scopus.com/authredirect.uri?txGid=9e1481b14d31344c85ab41a6099cc223&code=HJewI8UbiqkJV7QF5Rsu1OyJ5UclvcZmZ7ULmrsX&state=autoLogin|txId%3DC8AACD5A39BB90AF44AD58E0CD862F78.i-036b3148a22c61be4%3A1} & Too many redirections \\ \hline
15 & \url{https://www.scopus.com/authredirect.uri?txGid=8d78311d9441e403d0b1e4109a3e0cb3&code=jw5jxSC4HOMbKPwTKE13DLuQkP4cE2qom89qNEZ8&state=autoLogin|txId%3D2DA1ABE7CEFB9571BADEFAC6BA514584.i-0f64af503820b4197%3A1} & Too many redirections \\ \hline
16 & \url{https://sinta.kemdikbud.go.id/authors/profile/6801942} & -1 Not found: The server name or address could not be resolved \\ \hline
17 & \url{https://sinta.kemdikbud.go.id/authors/profile/6740641} & -1 Not found: The server name or address could not be resolved \\ \hline
18 & \url{https://sinta.kemdikbud.go.id/authors/profile/6004503} & -1 Not found: The server name or address could not be resolved \\ \hline
19 & \url{https://sinta.kemdikbud.go.id/authors/profile/6000960} & -1 Not found: The server name or address could not be resolved \\ \hline
20 & \url{https://sinta.kemdikbud.go.id/authors/profile/6000888} & -1 Not found: The server name or address could not be resolved \\ \hline
21 & \url{https://sinta.kemdikbud.go.id/authors/profile/6000654} & -1 Not found: The server name or address could not be resolved \\ \hline
22 & \url{https://sinta.kemdikbud.go.id/authors/profile/6649668} & -1 Not found: The server name or address could not be resolved \\ \hline
23 & \url{https://sinta.kemdikbud.go.id/authors/profile/6731958} & -1 Not found: The server name or address could not be resolved \\ \hline
24 & \url{https://sinta.kemdikbud.go.id/authors/profile/6022890} & -1 Not found: The server name or address could not be resolved \\ \hline
25 & \url{https://sinta.kemdikbud.go.id/authors/profile/6762861} & -1 Not found: The server name or address could not be resolved \\ \hline
26 & \url{https://sinta.kemdikbud.go.id/authors/profile/6157916} & -1 Not found: The server name or address could not be resolved \\ \hline
27 & \url{https://sinta.kemdikbud.go.id/authors/profile/6695534} & -1 Not found: The server name or address could not be resolved \\ \hline
28 & \url{https://sinta.kemdikbud.go.id/authors/profile/6000608} & -1 Not found: The server name or address could not be resolved \\ \hline
29 & \url{https://sinta.kemdikbud.go.id/authors/profile/6801396} & -1 Not found: The server name or address could not be resolved \\ \hline
30 & \url{https://sinta.kemdikbud.go.id/authors/profile/6103426} & -1 Not found: The server name or address could not be resolved \\ \hline
31 & \url{https://sinta.kemdikbud.go.id/authors/profile/5981043} & -1 Not found: The server name or address could not be resolved \\ \hline
32 & \url{https://www.teknik.unpas.ac.id/daftar} & -1 Not found: The server name or address could not be resolved \\ \hline
33 & \url{https://www.scopus.com/authredirect.uri?txGid=cd3202447e836a72ce1f8302519cd46c&code=nh3LGbJFwcU41yCrnU9PKYcMu6ihUt6QWa4LmrsX&state=autoLogin|txId%3D04BDF2329A67E4953C3988AD20DB3592.i-0f64af503820b4197%3A1} & Too many redirections \\ \hline
34 & \url{https://www.scopus.com/authredirect.uri?txGid=11e0f030d13b6937111011f1d59e7a4a&code=fKi3_8hyFxb1m_7O5U9GJNYC3S9Eu_B_ZtUAyU4P&state=autoLogin|txId%3D7B6D58871AFF6D5282CE6B1C4C1316F8.i-05ba7d12fc659f7d1%3A1} & Too many redirections \\ \hline
35 & \url{https://www.scopus.com/authredirect.uri?txGid=b2dae10edc018a959bcd7b9791d34e01&code=_OyChKDtA8iiYqSWn3yJ3NfAegIt7IZWabJytbsg&state=autoLogin|txId%3D7A5CF6DF0E1C4F62B1F6EFAEE92A5801.i-023afe6053fc6a195%3A1} & Too many redirections \\ \hline
36 & \url{https://www.scopus.com/authredirect.uri?txGid=11725f5ae3336790c2e7b796b1941282&code=uojpDHBY7bVSkGsvSv6Dw_9roqNCsekSOzAAyU4P&state=autoLogin|txId%3D5351FB7F78EA918C1C2CFD46A16FFCA7.i-0825ac76ca18516fa%3A1} & Too many redirections \\ \hline
37 & \url{https://www.scopus.com/authredirect.uri?txGid=434478ad9ffd79e357a62b333c1030f4&code=NAm6Ul8_UbiYiAnOqhIl6Zh8CT5M1mesQuxqNEZ8&state=autoLogin|txId%3DFC2434A7F79A0458B6AD49B5326507E5.i-0001141134d2e5469%3A1} & Too many redirections \\ \hline
38 & \url{https://demo.goodlayers.com/kingster/wp-content/uploads/2018/06/bird-s-eye-view-cars-daylight-1098820.jpg} & 404 Not Found \\ \hline
39 & \url{https://www.rivalmind.com/what-are-the-benefits-of-seo} & 403 Forbidden \\ \hline
40 & \url{https://searchengineland.com/guide/what-is-seo} & 403 Forbidden \\ \hline
41 & \url{https://searchengineland.com/guide/what-is-seo/embed} & 403 Forbidden \\ \hline
42 & \url{https://wartakota.tribunnews.com/2019/05/28/ini-petunjuk-lengkap-one-way-mudik-2019-di-tol-trans-jawa-download-brosur-korlantas?page=2} & 403 Forbidden \\ \hline
43 & \url{https://dynomapper.com/blog/21-sitemaps-and-seo/476-what-is-google-tag-manager-and-how-does-it-work} & 404 Not Found \\ \hline
44 & \url{https://evolve-systems.com/what-does-google-tag-manager-do-the-benefits-of-google-tag-manager-and-what-to-track/} & 404 Not Found \\ \hline
45 & \url{https://www.leadsquared.com/7-benefits-of-google-adwords/} & -1 Not found: An error occurred in the secure channel support \\ \hline
46 & \url{http://if.unpas.ac.id/wp-content/uploads/2022/03/google-ads-300x155.png} & 404 Not Found \\ \hline
47 & \url{http://if.unpas.ac.id/wp-content/uploads/2022/03/google-ads-300x221.png} & 404 Not Found \\ \hline
48 & \url{http://pena.belajar.kemdikbud.go.id/2020/10/cara-melakukan-backup-data-yang-benar-agar-data-kamu-aman/} & -1 Not found: The server name or address could not be resolved \\ \hline
49 & \url{https://www.beritateknologi.com/belajar-coding-dan-robotic-akan-semakin-menyenangkan-dengan-dji-robomaster-s1/embed/} & 404 Not Found \\ \hline
50 & \url{https://www.beritateknologi.com/belajar-coding-dan-robotic-akan-semakin-menyenangkan-dengan-dji-robomaster-s1/} & 404 Not Found \\ \hline
51 & \url{http://www.cyberlab-aoh.com/} & -1 Not found: The server name or address could not be resolved \\ \hline
52 & \url{http://jabar.aptikomfest.web.id/portal/activity} & -1 Not found: The server name or address could not be resolved \\ \hline
53 & \url{https://aptika.kominfo.go.id/2021/09/warganet-meningkat-indonesia-perlu-tingkatkan-nilai-budaya-di-internet/} & -1 Not found: The server name or address could not be resolved \\ \hline
54 & \url{https://aptika.kominfo.go.id/2021/08/waspada-rekam-jejak-digital-kita-di-internet/} & -1 Not found: The server name or address could not be resolved \\ \hline
55 & \url{https://www.dicoding.com/blog/apa-itu-server/} & 405 Method Not Allowed \\ \hline
56 & \url{https://www.pikiran-rakyat.com/teknologi/pr-012703939/apa-itu-nft-non-fungible-token-dan-mengapa-begitu-penting} & -1 Not found: Cannot find object or property. \\ \hline
57 & \url{https://blog.amartha.com/apa-itu-nft-bagaimana-cara-kerja-nft/} & 404 Not Found \\ \hline
58 & \url{https://www.hostinger.com/id/tutorial/penyebab-website-down/} & 404 Not Found \\ \hline
59 & \url{https://www.wartaekonomi.co.id/read354574/sistem-desentralisasi-pada-blockchain-apa-fungsinya} & -1 Not found: The server name or address could not be resolved \\ \hline
60 & \url{https://www.teknik.unpas.ac.id/2021/08/24/berbagi-cerita-tentang-bangkit-mahasiswa-mahasiwi-teknik-informatika-unpas/} & -1 Not found: The server name or address could not be resolved \\ \hline
61 & \url{https://www.teknik.unpas.ac.id/2021/08/01/dari-kampus-mengajar-angkatan-1/} & -1 Not found: The server name or address could not be resolved \\ \hline
62 & \url{http://www.teknik.unpas.ac.id/} & -1 Not found: The server name or address could not be resolved \\ \hline
63 & \url{http://www.kominfo.go.id/} & -1 Not found: The server name or address could not be resolved \\ \hline




\end{longtable}


\setlength{\LTcapwidth}{\textwidth}
\renewcommand{\arraystretch}{1.4}

\begin{longtable}{
|>{\centering\arraybackslash}m{5cm}
|>{\centering\arraybackslash}m{3cm}
|>{\centering\arraybackslash}m{3cm}
|>{\centering\arraybackslash}m{3cm}|}
\caption{Pengujian pada Informatika UNPAR}
\vspace{-3mm}
\label{tab:hasil-pengujian-if-unpar} \\
\hline
\textbf{Error} &
\textbf{Broken Link Scanner} &
\textbf{Broken Link Checker} &
\textbf{Dead Link Checker} \\
\hline
\endfirsthead

\multicolumn{4}{c}{Tabel~\ref{tab:hasil-pengujian-if-unpar} dilanjutkan dari halaman sebelumnya}\\[4pt]
\hline
\textbf{Error} &
\textbf{Broken Link Scanner} &
\textbf{Broken Link Checker} &
\textbf{Dead Link Checker} \\
\hline
\endhead

\hline
\multicolumn{4}{|r|}{Bersambung ke halaman berikutnya} \\ \hline
\endfoot

\hline
\endlastfoot

% ====== ISI DATA DI SINI ======

Host Not Found & 14 & 11 & 7 \\ \hline

I/O Error & 1 & -- & -- \\ \hline

Invalid URL & 1 & -- & -- \\ \hline

SSL Error & 6 & -- & 5 \\ \hline

Timeout & 4 & 3 & 2 \\ \hline

400 Bad Request & 2 & 1 & 1 \\ \hline

403 Forbidden & 12 & -- & 3 \\ \hline

404 Not Found & 29 & 23 & 35 \\ \hline

405 Method Not Allowed & -- & -- & 1 \\ \hline

410 Gone & 1 & -- & -- \\ \hline

429 Too Many Requests & -- & -- & 3 \\ \hline

502 Bad Gateway & 1 & 1 & 1 \\ \hline

520 & 2 & 1 & 2 \\ \hline

999 & 2 & -- & -- \\ \hline



\end{longtable}


\setlength{\LTcapwidth}{\textwidth}
\renewcommand{\arraystretch}{1.9}

\begin{longtable}{
|>{\centering\arraybackslash}m{6cm}
|>{\centering\arraybackslash}m{3cm}
|>{\centering\arraybackslash}m{3cm}
|>{\centering\arraybackslash}m{3cm}|}
\caption{Hasil pengujian perbandingan pada Informatika UNPAS}
\vspace{-3mm}
\label{tab:hasil-pengujian-if-unpas} \\
\hline
\textbf{Error} &
\textbf{Broken Link Scanner} &
\textbf{Broken Link Checker} &
\textbf{Dead Link Checker} \\
\hline
\endfirsthead

\multicolumn{4}{c}{Tabel~\ref{tab:hasil-pengujian-if-unpas} dilanjutkan dari halaman sebelumnya}\\[4pt]
\hline
\textbf{Error} &
\textbf{Broken Link Scanner} &
\textbf{Broken Link Checker} &
\textbf{Dead Link Checker} \\
\hline
\endhead

\hline
\multicolumn{4}{|r|}{Bersambung ke halaman berikutnya} \\ \hline
\endfoot

\hline
\endlastfoot

% ====== ISI DATA DI SINI ======

Host Not Found & 34 & 44 & 33 \\ \hline

Timeout & 2 & 2 & 2 \\ \hline

SSL Error & 1 & 0 & 1 \\ \hline

Too many redirections & 0 & 0 & 11 \\ \hline

400 Bad Request & 0 & 0 & 1 \\ \hline

404 Not Found & 8 & 20 & 20 \\ \hline

403 Forbidden & 15 & 0 & 6 \\ \hline

405 Method Not Allowed & 0 & 0 & 2 \\ \hline

429 Too Many Requests & 0 & 0 & 1 \\ \hline

502 Bad Gateway & 1 & 1 & 1 \\ \hline

522 & 0 & 1 & 0 \\ \hline


\end{longtable}

\setlength{\LTcapwidth}{\textwidth}
\renewcommand{\arraystretch}{1.4}

\begin{longtable}{
|>{\centering\arraybackslash}m{6cm}
|>{\centering\arraybackslash}m{3cm}
|>{\centering\arraybackslash}m{3cm}
|>{\centering\arraybackslash}m{3cm}|}
\caption{Hasil pengujian pada Informatika UNPAD}
\vspace{-3mm}
\label{tab:hasil-pengujian-if-unpad} \\
\hline
\textbf{Error} &
\textbf{Broken Link Scanner} &
\textbf{Broken Link Checker} &
\textbf{Dead Link Checker} \\
\hline
\endfirsthead

\multicolumn{4}{c}{Tabel~\ref{tab:hasil-pengujian-if-unpad} dilanjutkan dari halaman sebelumnya}\\[4pt]
\hline
\textbf{Error} &
\textbf{Broken Link Scanner} &
\textbf{Broken Link Checker} &
\textbf{Dead Link Checker} \\
\hline
\endhead

\hline
\multicolumn{4}{|r|}{Bersambung ke halaman berikutnya} \\ \hline
\endfoot

\hline
\endlastfoot

% ====== ISI DATA DI SINI ======

Host Not Found & 10 & 10 & 11 \\ \hline

Connection Failed & 4 & 0 & 2 \\ \hline

Timeout & 10 & 2 & 2 \\ \hline

Invalid URL & 5 & 0 & 0 \\ \hline

SSL Error & 2 & 0 & 0 \\ \hline

Too many redirections & 0 & 0 & 2 \\ \hline

400 Bad Request & 0 & 0 & 1 \\ \hline

401 Unauthorized & 11 & 0 & 11 \\ \hline

403 Forbidden & 6 & 0 & 4 \\ \hline

404 Not Found & 214 & 20 & 397 \\ \hline

405 Method Not Allowed & 0 & 0 & 2 \\ \hline

429 Too Many Requests & 0 & 0 & 1 \\ \hline

502 Bad Gateway & 1 & 0 & 0 \\ \hline

503 Service Unavailable & 1 & 1 & 1 \\ \hline

520 & 1 & 2 & 2 \\ \hline

522 & 5 & 1 & 0 \\ \hline

530 & 3 & 3 & 3 \\ \hline

999 & 1 & 0 & 1 \\ \hline



\end{longtable}

\setlength{\LTcapwidth}{\textwidth}
\renewcommand{\arraystretch}{1.4}

\begin{longtable}{
|>{\centering\arraybackslash}m{6cm}
|>{\centering\arraybackslash}m{3cm}
|>{\centering\arraybackslash}m{3cm}
|>{\centering\arraybackslash}m{3cm}|}
\caption{Pengujian pada Informatika UNIKOM}
\vspace{-3mm}
\label{tab:hasil-pengujian-if-unikom} \\
\hline
\textbf{Error} &
\textbf{Broken Link Scanner} &
\textbf{Broken Link Checker} &
\textbf{Dead Link Checker} \\
\hline
\endfirsthead

\multicolumn{4}{c}{Tabel~\ref{tab:hasil-pengujian-if-unikom} dilanjutkan dari halaman sebelumnya}\\[4pt]
\hline
\textbf{Error} &
\textbf{Broken Link Scanner} &
\textbf{Broken Link Checker} &
\textbf{Dead Link Checker} \\
\hline
\endhead

\hline
\multicolumn{4}{|r|}{Bersambung ke halaman berikutnya} \\ \hline
\endfoot

\hline
\endlastfoot

% ====== ISI DATA DI SINI ======

-- & -- & -- & -- \\ \hline

-- & -- & -- & -- \\ \hline

-- & -- & -- & -- \\ \hline

-- & -- & -- & -- \\ \hline

-- & -- & -- & -- \\ \hline

-- & -- & -- & -- \\ \hline



\end{longtable}

