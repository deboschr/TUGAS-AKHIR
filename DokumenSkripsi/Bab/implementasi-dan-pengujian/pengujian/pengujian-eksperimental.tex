Berbeda dengan pengujian fungsional yang berfokus pada verifikasi implementasi fitur berdasarkan hasil perancangan, pengujian eksperimental menekankan eksplorasi terhadap parameter-parameter operasional yang memengaruhi performa dan hasil pemeriksaan. Pada pengujian ini, kualitas hasil pemeriksaan dievaluasi berdasarkan tiga aspek utama, yaitu efisiensi, akurasi, dan konsistensi. Pemeriksaan dikatakan efisien apabila proses diselesaikan dalam waktu yang relatif singkat namun identifikasi tautan rusak tetap akurat. Pemeriksaan dikatakan akurat apabila hasil pemeriksaan mencerminkan kondisi tautan yang sebenarnya tanpa menghasilkan \textit{false positive}, yaitu tautan yang tidak rusak namun diidentifikasi sebagai tautan rusak. Sementara itu, pemeriksaan dikatakan konsisten apabila hasil pemeriksaan yang diperoleh tetap sama atau sangat mendekati pada percobaan yang berbeda dalam kondisi pengujian tidak berubah.

\vspace{-3mm}
% ####################### TUJUAN PENGUJIAN #######################
\subsubsection{Tujuan Pengujian}
\label{subsubsec:05020301-tujuan-pengujian}
\vspace{-2mm}
Pengujian eksperimental ini memiliki beberapa tujuan sebagai berikut:
\begin{itemize}

   \item Menentukan nilai interval pada \texttt{RateLimiter} untuk mendapatkan kecepatan \textit{crawling} terbaik tanpa menyebabkan lonjakan permintaan HTTP terhadap \textit{host} yang sama yang dapat memicu pemblokiran sementara atau menghasilkan kode status HTTP 429 (\textit{Too Many Requests}).
   
   \item Menentukan nilai \textit{connection timeout} pada \texttt{HttpClient} untuk memperoleh batas waktu pembentukan koneksi yang optimal, sehingga tidak terjadi \textit{connection error} pada tautan yang responsif jika nilainya terlalu kecil ataupun menyebabkan waktu pemeriksaan menjadi terlalu lama jika nilainya terlalu besar. 
   
   \item Menentukan nilai \textit{request timeout} pada \texttt{HttpRequest} untuk memperoleh batas waktu permintaan HTTP yang optimal, sehingga tidak terjadi identifikasi \textit{false positive} pada tautan yang respons \textit{server}-nya lambat jika nilainya terlalu kecil ataupun menyebabkan waktu pemeriksaan menjadi terlalu lama jika nilainya terlalu besar. 

   \item Membandingkan hasil pemeriksaan dengan perangkat lunak serupa untuk melihat perbedaan pada hasil pemeriksaan.

\end{itemize}

\vspace{-3mm}
% ####################### DESAIN PERCOBAAN #######################
\subsubsection{Desain Percobaan}
\label{subsubsec:05020302-desain-percobaan}
\vspace{-2mm}
Seluruh percobaan pada setiap pengujian menggunakan subjek yang sama, yaitu situs web Program Studi Informatika Universitas Katolik Parahyangan yang beralamat pada \url{https://informatika.unpar.ac.id}. Penggunaan subjek yang sama memastikan bahwa setiap variasi hasil benar-benar disebabkan oleh perubahan pada nilai parameter yang sedang diuji, bukan disebabkan oleh perbedaan struktur atau jumlah tautan pada subjek pengujian.

Setiap pengujian difokuskan pada satu parameter tertentu, dan di dalamnya dilakukan beberapa percobaan dengan variasi nilai parameter tersebut, sementara parameter lainnya dibuat konstan. Pendekatan ini digunakan agar pengaruh masing-masing parameter dapat diamati secara terpisah. Tiga parameter yang diuji beserta rentang nilai percobaannya ditunjukkan pada Tabel~\ref{tab:parameter-pengujian-eksperimantal}.

Pada setiap percobaan dilakukan evaluasi terhadap indikator berikut:
\begin{itemize}
   \item Durasi pemeriksaan.
   \item Jumlah total tautan.
   \item Jumlah tautan halaman.
   \item Jumlah tautan rusak.
\end{itemize}

\renewcommand{\arraystretch}{1.8}
\begin{table}[H]
\centering
\caption{Parameter Pengujian Eksperimantal}
\vspace{3mm}
\label{tab:parameter-pengujian-eksperimantal}

\begin{tabular}{|
    >{\raggedright\arraybackslash}m{3.5cm}|
    >{\raggedright\arraybackslash}m{4.5cm}|
    >{\centering\arraybackslash}m{3.3cm}|
    >{\centering\arraybackslash}m{3.3cm}|}
\hline

\multicolumn{1}{|c|}{\textbf{Parameter}} &
\multicolumn{1}{c|}{\textbf{Deskripsi}} &
\multicolumn{1}{c|}{\textbf{Rentang Nilai}} &
\multicolumn{1}{c|}{\textbf{Interval Percobaan}} \\ \hline

Interval pada \texttt{RateLimiter} & 
Jarak antar permintaan HTTP per \textit{host} URL & 
0--2000~milidetik &
100~milidetik \\ \hline

\textit{Connection Timeout} pada \texttt{HttpClient} & 
Batas waktu pembentukan koneksi dalam permintaan HTTP & 
1--30~detik &
2~detik \\ \hline

\textit{Request Timeout} pada \texttt{HttpRequest} & 
Batas waktu menunggu respons \textit{server} dalam permintaan HTTP & 
1--30~detik &
2~detik \\ \hline


\end{tabular}
\end{table}


% ####################### PENGUJIAN 1 #######################
\subsubsection{Pengujian 1: Interval pada \texttt{RateLimiter}}
\label{subsubsec:05020303-pengujian-1-interval-rate-limiter}

Tabel~\ref{tab:hasil-pengujian-interval}

\begin{table}[H]
\centering
\caption{Pengaruh Variasi Parameter Interval (\textit{RateLimiter}) pada Hasil Pemeriksaan}
\label{tab:hasil-pengujian-interval}
\vspace{6pt}

\begin{tabular}{
| >{\centering\arraybackslash}m{2cm}
| >{\centering\arraybackslash}m{2.5cm}
| >{\centering\arraybackslash}m{2cm}
| >{\centering\arraybackslash}m{2cm}
| >{\centering\arraybackslash}m{1cm}
| >{\centering\arraybackslash}m{4cm}|}
\hline

% =========== HEADER TABLE ===========

\textbf{Interval} &
\textbf{Waktu Pemeriksaan} &
\textbf{Total Tautan} &
\textbf{Jumlah Halaman} &
\multicolumn{2}{c|}{\textbf{Jumlah Tautan Rusak}} \\ \hline

% =========== BODY TABLE ===========


\multirowcell{3}{0 ms} & 
\multirowcell{3}{16m 54s} & 
\multirowcell{3}{602} & 
\multirowcell{3}{302} & 
\multirowcell{3}{79} &
47 Client Error \\ \cline{6-6}
& & & & & 1 Server Error \\ \cline{6-6}
& & & & & 27 Connection Error \\ \hline

\multirowcell{3}{100 ms} & 
\multirowcell{3}{} & 
\multirowcell{3}{} & 
\multirowcell{3}{} & 
\multirowcell{3}{} &
Client Error \\ \cline{6-6}
& & & & & Server Error \\ \cline{6-6}
& & & & & Connection Error \\ \hline

\multirowcell{3}{200 ms} & 
\multirowcell{3}{} & 
\multirowcell{3}{} & 
\multirowcell{3}{} & 
\multirowcell{3}{} &
Client Error \\ \cline{6-6}
& & & & & Server Error \\ \cline{6-6}
& & & & & Connection Error \\ \hline

\multirowcell{3}{300 ms} & 
\multirowcell{3}{} & 
\multirowcell{3}{} & 
\multirowcell{3}{} & 
\multirowcell{3}{} &
Client Error \\ \cline{6-6}
& & & & & Server Error \\ \cline{6-6}
& & & & & Connection Error \\ \hline

\multirowcell{3}{400 ms} & 
\multirowcell{3}{} & 
\multirowcell{3}{} & 
\multirowcell{3}{} & 
\multirowcell{3}{} &
Client Error \\ \cline{6-6}
& & & & & Server Error \\ \cline{6-6}
& & & & & Connection Error \\ \hline

\multirowcell{3}{500 ms} & 
\multirowcell{3}{} & 
\multirowcell{3}{} & 
\multirowcell{3}{} & 
\multirowcell{3}{} &
Client Error \\ \cline{6-6}
& & & & & Server Error \\ \cline{6-6}
& & & & & Connection Error \\ \hline

\multirowcell{3}{600 ms} & 
\multirowcell{3}{} & 
\multirowcell{3}{} & 
\multirowcell{3}{} & 
\multirowcell{3}{} &
Client Error \\ \cline{6-6}
& & & & & Server Error \\ \cline{6-6}
& & & & & Connection Error \\ \hline

\multirowcell{3}{700 ms} & 
\multirowcell{3}{} & 
\multirowcell{3}{} & 
\multirowcell{3}{} & 
\multirowcell{3}{} &
Client Error \\ \cline{6-6}
& & & & & Server Error \\ \cline{6-6}
& & & & & Connection Error \\ \hline

\multirowcell{3}{800 ms} & 
\multirowcell{3}{} & 
\multirowcell{3}{} & 
\multirowcell{3}{} & 
\multirowcell{3}{} &
Client Error \\ \cline{6-6}
& & & & & Server Error \\ \cline{6-6}
& & & & & Connection Error \\ \hline

\multirowcell{3}{900 ms} & 
\multirowcell{3}{} & 
\multirowcell{3}{} & 
\multirowcell{3}{} & 
\multirowcell{3}{} &
Client Error \\ \cline{6-6}
& & & & & Server Error \\ \cline{6-6}
& & & & & Connection Error \\ \hline

\multirowcell{3}{1000 ms} & 
\multirowcell{3}{} & 
\multirowcell{3}{} & 
\multirowcell{3}{} & 
\multirowcell{3}{} &
Client Error \\ \cline{6-6}
& & & & & Server Error \\ \cline{6-6}
& & & & & Connection Error \\ \hline


\end{tabular}
\end{table}




% \renewcommand{\arraystretch}{1.3}

% \captionsetup{singlelinecheck=off, justification=centering}
% \begin{longtable}{
% | >{\centering\arraybackslash}m{2cm}
% | >{\centering\arraybackslash}m{2.5cm}
% | >{\centering\arraybackslash}m{2cm}
% | >{\centering\arraybackslash}m{2cm}
% | >{\centering\arraybackslash}m{1cm}
% | >{\centering\arraybackslash}m{4cm}|}
% \caption{Pengaruh Variasi Parameter Interval pada Hasil Pemeriksaan}
% \label{tab:hasil-pengujian-interval} \\

% \hline
% \textbf{Interval} &
% \textbf{Waktu Pemeriksaan} &
% \textbf{Total Tautan} &
% \textbf{Jumlah Halaman} &
% \multicolumn{2}{c|}{\textbf{Jumlah Tautan Rusak}} \\
% \hline
% \endfirsthead

% % ---------- HEADER REPEAT DI HALAMAN SELANJUTNYA ----------
% \hline
% \textbf{Interval} &
% \textbf{Waktu Pemeriksaan} &
% \textbf{Total Tautan} &
% \textbf{Jumlah Halaman} &
% \multicolumn{2}{c|}{\textbf{Jumlah Tautan Rusak}} \\
% \hline
% \endhead

% % ---------- FOOTER PER HALAMAN ----------
% \hline
% \multicolumn{6}{r}{\textit{bersambung ke halaman berikutnya}} \\
% \endfoot

% % ---------- FOOTER TERAKHIR ----------
% \hline
% \endlastfoot

% % ==========================================================
% % ==================== BODY TABLE ==========================
% % ==========================================================

% \multirowcell{3}{0 ms} &
% \multirowcell{3}{16m 54s} &
% \multirowcell{3}{602} &
% \multirowcell{3}{302} &
% \multirowcell{3}{79} &
% 47 Client Error \\ \cline{6-6}
% & & & & & 1 Server Error \\ \cline{6-6}
% & & & & & 27 Connection Error \\ \hline

% \multirowcell{3}{100 ms} &
% \multirowcell{3}{} &
% \multirowcell{3}{} &
% \multirowcell{3}{} &
% \multirowcell{3}{} &
% Client Error \\ \cline{6-6}
% & & & & & Server Error \\ \cline{6-6}
% & & & & & Connection Error \\ \hline

% \multirowcell{3}{200 ms} &
% \multirowcell{3}{} &
% \multirowcell{3}{} &
% \multirowcell{3}{} &
% \multirowcell{3}{} &
% Client Error \\ \cline{6-6}
% & & & & & Server Error \\ \cline{6-6}
% & & & & & Connection Error \\ \hline

% \multirowcell{3}{300 ms} &
% \multirowcell{3}{} &
% \multirowcell{3}{} &
% \multirowcell{3}{} &
% \multirowcell{3}{} &
% Client Error \\ \cline{6-6}
% & & & & & Server Error \\ \cline{6-6}
% & & & & & Connection Error \\ \hline

% \multirowcell{3}{400 ms} &
% \multirowcell{3}{} &
% \multirowcell{3}{} &
% \multirowcell{3}{} &
% \multirowcell{3}{} &
% Client Error \\ \cline{6-6}
% & & & & & Server Error \\ \cline{6-6}
% & & & & & Connection Error \\ \hline

% \multirowcell{3}{500 ms} &
% \multirowcell{3}{} &
% \multirowcell{3}{} &
% \multirowcell{3}{} &
% \multirowcell{3}{} &
% Client Error \\ \cline{6-6}
% & & & & & Server Error \\ \cline{6-6}
% & & & & & Connection Error \\ \hline

% \multirowcell{3}{600 ms} &
% \multirowcell{3}{} &
% \multirowcell{3}{} &
% \multirowcell{3}{} &
% \multirowcell{3}{} &
% Client Error \\ \cline{6-6}
% & & & & & Server Error \\ \cline{6-6}
% & & & & & Connection Error \\ \hline

% \multirowcell{3}{700 ms} &
% \multirowcell{3}{} &
% \multirowcell{3}{} &
% \multirowcell{3}{} &
% \multirowcell{3}{} &
% Client Error \\ \cline{6-6}
% & & & & & Server Error \\ \cline{6-6}
% & & & & & Connection Error \\ \hline

% \multirowcell{3}{800 ms} &
% \multirowcell{3}{} &
% \multirowcell{3}{} &
% \multirowcell{3}{} &
% \multirowcell{3}{} &
% Client Error \\ \cline{6-6}
% & & & & & Server Error \\ \cline{6-6}
% & & & & & Connection Error \\ \hline

% \multirowcell{3}{900 ms} &
% \multirowcell{3}{} &
% \multirowcell{3}{} &
% \multirowcell{3}{} &
% \multirowcell{3}{} &
% Client Error \\ \cline{6-6}
% & & & & & Server Error \\ \cline{6-6}
% & & & & & Connection Error \\ \hline

% \multirowcell{3}{1000 ms} &
% \multirowcell{3}{} &
% \multirowcell{3}{} &
% \multirowcell{3}{} &
% \multirowcell{3}{} &
% Client Error \\ \cline{6-6}
% & & & & & Server Error \\ \cline{6-6}
% & & & & & Connection Error \\ \hline

% \end{longtable}


% ####################### PENGUJIAN 2 #######################
\subsubsection{Pengujian 2: \textit{Connection Timeout} pada \texttt{HttpClient}}
\label{subsubsec:05020304-pengujian-2-connection-timeout-httpclient}

Tabel~\ref{tab:hasil-pengujian-connection-timeout}

\setlength{\LTcapwidth}{\textwidth}
\renewcommand{\arraystretch}{1.4}

\begin{longtable}{
|>{\centering\arraybackslash}m{2.5cm}
|>{\centering\arraybackslash}m{2.5cm}
|>{\centering\arraybackslash}m{2.5cm}
|>{\centering\arraybackslash}m{2.5cm}
|>{\centering\arraybackslash}m{5cm}|}
\caption{Pengaruh Variasi \textit{Connection Timeout} pada Hasil Pemeriksaan}
\vspace{-3mm}
\label{tab:hasil-pengujian-connection-timeout} \\
\hline
\textbf{Connection Timeout (detik)} &
\textbf{Durasi Pemeriksaan} &
\textbf{Jumlah Total Tautan} &
\textbf{Jumlah Tautan Halaman} &
\textbf{Jumlah Tautan Rusak} \\
\hline
\endfirsthead

\multicolumn{5}{c}{Tabel~\ref{tab:hasil-pengujian-connection-timeout} dilanjutkan dari halaman sebelumnya}\\[4pt]
\hline
\textbf{Connection Timeout (detik)} &
\textbf{Durasi Pemeriksaan} &
\textbf{Jumlah Total Tautan} &
\textbf{Jumlah Tautan Halaman} &
\textbf{Jumlah Tautan Rusak} \\
\hline
\endhead

\hline
\multicolumn{5}{|r|}{Bersambung ke halaman berikutnya} \\ \hline
\endfoot

\hline
\endlastfoot

% ====== ISI DATA DI SINI ======
1  & -- & -- & -- & -- \\ \hline
3  & -- & -- & -- & -- \\ \hline
5  & -- & -- & -- & -- \\ \hline
10 & -- & -- & -- & -- \\ \hline
15 & -- & -- & -- & -- \\ \hline

\end{longtable}


% ####################### PENGUJIAN 3 #######################
\subsubsection{Pengujian 3: \textit{Request Timeout} pada \texttt{HttpRequest}}
\label{subsubsec:05020305-pengujian-3-request-timeout-httprequest}

Tabel~\ref{tab:hasil-pengujian-request-timeout}

\setlength{\LTcapwidth}{\textwidth}
\renewcommand{\arraystretch}{1.4}

\begin{longtable}{
|>{\centering\arraybackslash}m{2.8cm}
|>{\centering\arraybackslash}m{2.8cm}
|>{\centering\arraybackslash}m{2.8cm}
|>{\centering\arraybackslash}m{2.8cm}
|>{\centering\arraybackslash}m{2.8cm}|}
\caption{Pengaruh Variasi \textit{Request Timeout} pada Hasil Pemeriksaan}
\vspace{-3mm}
\label{tab:hasil-pengujian-request-timeout} \\
\hline
\textbf{Request Timeout} &
\textbf{Durasi} &
\textbf{Total Tautan} &
\textbf{Tautan Halaman} &
\textbf{Tautan Rusak} \\
\hline
\endfirsthead

\multicolumn{5}{c}{Tabel~\ref{tab:hasil-pengujian-request-timeout} dilanjutkan dari halaman sebelumnya}\\[4pt]
\hline
\textbf{Request Timeout} &
\textbf{Durasi} &
\textbf{Total Tautan} &
\textbf{Tautan Halaman} &
\textbf{Tautan Rusak} \\
\hline
\endhead

\hline
\multicolumn{5}{|r|}{Bersambung ke halaman berikutnya} \\ \hline
\endfoot

\hline
\endlastfoot

% ====== ISI DATA DI SINI ======
% 1  & -- & -- & -- & -- \\ \hline

2 & 13m 30s & 466 & 256 & 125 (Tabel~\ref{tab:percobaan-request-timeout-2}) \\ \hline

3 & 13m 23s & 602 & 364 & 90 (Tabel~\ref{tab:percobaan-request-timeout-3}) \\ \hline

4 & 14m 40s & 589 & 337 & 107 (Tabel~\ref{tab:percobaan-request-timeout-4}) \\ \hline

5 & 14m 50s & 585 & 343 & 91 (Tabel~\ref{tab:percobaan-request-timeout-5}) \\ \hline

6 & 35m 18s & 515 & 265 & 142 (Tabel~\ref{tab:percobaan-request-timeout-6}) \\ \hline

7 & 14m 27s & 602 & 368 & 77 (Tabel~\ref{tab:percobaan-request-timeout-7}) \\ \hline

8 & 14m 00s & 602 & 368 & 82 (Tabel~\ref{tab:percobaan-request-timeout-8}) \\ \hline

9 & 13m 17s & 602 & 368 & 80 (Tabel~\ref{tab:percobaan-request-timeout-9}) \\ \hline

10 & 14m 01s & 602 & 368 & 79 (Tabel~\ref{tab:percobaan-request-timeout-10}) \\ \hline

12 & 14m 17s & 602 & 368 & 79 (Tabel~\ref{tab:percobaan-request-timeout-12}) \\ \hline

14 & 13m 28s & 602 & 368 & 75 (Tabel~\ref{tab:percobaan-request-timeout-14}) \\ \hline

16 & 13m 59s & 602 & 368 & 77 (Tabel~\ref{tab:percobaan-request-timeout-16}) \\ \hline

18 & 14m 35s & 602 & 368 & 79 (Tabel~\ref{tab:percobaan-request-timeout-18}) \\ \hline

20 & 17m 16s & 602 & 368 & 76 (Tabel~\ref{tab:percobaan-request-timeout-20}) \\ \hline

% 22 & -- & -- & -- & -- \\ \hline
% 24 & -- & -- & -- & -- \\ \hline
% 26 & -- & -- & -- & -- \\ \hline
% 28 & -- & -- & -- & -- \\ \hline
% 30 & -- & -- & -- & -- \\ \hline


\end{longtable}


% ####################### KESIMPULAN HASIL PENGUJIAN #######################
\subsubsection{Kesimpulan Hasil Pengujian}
\label{subsubsec:05020306-kesimpulan-hasil-pengujian}

% ####################### PERBANDINGAN #######################
\subsubsection{Perbandingan dengan Perangkat Lunak Serupa}
\label{subsubsec:05020307-perbandingan-dengan-perangkat-lunak-serupa}

% ####################### KESIMPULAN HASIL PERBANDINGAN #######################
\subsubsection{Kesimpulan Hasil Perbandingan}
\label{subsubsec:05020306-kesimpulan-hasil-perbandingan}