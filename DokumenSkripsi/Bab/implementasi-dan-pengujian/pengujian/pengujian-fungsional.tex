Pengujian fungsional dilakukan untuk memastikan setiap fitur pada aplikasi BrokenLink Checker berjalan sesuai dengan kebutuhan yang telah ditetapkan pada Bab~\ref{chap:040000-perancangan}. Pengujian dilakukan secara manual menggunakan berbagai skenario input dan interaksi antarmuka untuk memverifikasi bahwa proses pemeriksaan tautan, pengelolaan tampilan, serta fitur pendukung seperti ekspor dan filter berjalan dengan benar.

\begin{enumerate}
   \item \textbf{Pengujian pada Input URL}\\
   Pengujian ini bertujuan memastikan bahwa aplikasi dapat memvalidasi URL yang dimasukkan pengguna, termasuk menangani kondisi URL valid, URL tidak valid, maupun kolom input yang dibiarkan kosong. Hasil pengujian dapat dilihat pada Tabel. Hasil pengujian dapat dilihat pada Tabel~\ref{tab:uji-input}.

   \renewcommand{\arraystretch}{1.4}
\begin{table}[H]
\centering
\caption{Pengujian pada Input URL}
\label{tab:uji-input}
\begin{tabular}{|c|>{\raggedright\arraybackslash}p{6cm}|>{\raggedright\arraybackslash}p{6cm}|c|}
\hline
\textbf{Kasus} & \textbf{Skenario} & \textbf{Hasil Diharapkan} & \textbf{Hasil Uji} \\ \hline

1 &
Memasukkan URL dengan struktur lengkap: https://example.com/page?id=1 &
Input diterima dan proses pemeriksaan dimulai. &
Sesuai \\ \hline

2 &
Memasukkan URL tanpa skema: example.com/page &
Window notifikasi \textit{warning} terbuka. &
Sesuai \\ \hline

3 &
Memasukkan URL tanpa host: https:///page &
Window notifikasi \textit{warning} terbuka. &
Sesuai \\ \hline

4 &
Memasukkan URL dengan skema tidak valid: ftp://unpar.com &
Window notifikasi \textit{warning} terbuka. &
Sesuai \\ \hline

5 &
Memasukkan URL dengan sintaks tidak valid: http://exa mple..com &
Window notifikasi \textit{warning} terbuka. &
Sesuai \\ \hline

6 &
Tidak memasukkan URL atau hanya spasi &
Window notifikasi \textit{warning} terbuka. &
Sesuai \\ \hline

7 &
Memasukkan URL dengan dot-segment: https://unpar.com/.././index.html &
URL dinormalisasi dan proses pemeriksaan dimulai. &
Sesuai \\ \hline

\end{tabular}
\end{table}


   \item \textbf{Pengujian pada Tombol}\\
   Pengujian dilakukan untuk memastikan bahwa ketiga tombol utama, yaitu \textit{Start}, \textit{Stop}, dan \textit{Export}, merespons sesuai dengan kondisi aplikasi pada saat tombol ditekan. Pengujian mencakup perilaku tombol \textit{Start} ketika memulai proses pemeriksaan, tombol \textit{Stop} ketika menghentikan proses yang sedang berjalan maupun ketika tidak ada proses yang aktif, serta tombol \textit{Export} ketika data hasil pemeriksaan tersedia maupun belum tersedia. Hasil pengujian dapat dilihat pada Tabel~\ref{tab:pengujian-tombol}.

   \renewcommand{\arraystretch}{1.4}
\begin{table}[H]
\centering
\caption{Pengujian pada Tombol}
\label{tab:uji-tombol}
\begin{tabular}{|c|>{\raggedright\arraybackslash}p{6cm}|>{\raggedright\arraybackslash}p{6cm}|c|}
\hline
\textbf{Kasus} & \textbf{Skenario} & \textbf{Hasil Diharapkan} & \textbf{Hasil Uji} \\ \hline

1 &
Menekan tombol Start ketika field URL berisi URL valid. &
Proses pemeriksaan dimulai; status berubah menjadi CHECKING; tombol Stop aktif; tombol Export nonaktif. &
Sesuai \\ \hline

2 &
Menekan tombol Start ketika field URL kosong atau tidak valid. &
Window notifikasi warning terbuka; proses tidak dimulai; tombol Stop dan Export tetap nonaktif. &
Sesuai \\ \hline

3 &
Menekan tombol Stop ketika proses pemeriksaan belum berjalan (status IDLE). &
Tidak terjadi apa-apa; tidak ada status atau UI yang berubah. &
Sesuai \\ \hline

4 &
Menekan tombol Stop saat proses pemeriksaan sedang berjalan. &
Proses dihentikan; status berubah menjadi STOPPED; tombol Start aktif kembali; tombol Export tetap nonaktif. &
Sesuai \\ \hline

5 &
Menekan tombol Export ketika tidak ada data hasil pemeriksaan. &
Window notifikasi warning terbuka; tidak ada berkas yang diekspor. &
Sesuai \\ \hline

6 &
Menekan tombol Export setelah proses pemeriksaan selesai dan data telah tersedia. &
Jendela penyimpanan muncul; berkas hasil pemeriksaan berhasil diekspor sesuai format yang dipilih. &
Sesuai \\ \hline

7 &
Menekan tombol Start setelah proses sebelumnya dihentikan dengan tombol Stop. &
Proses pemeriksaan baru dimulai; status kembali menjadi CHECKING; UI reset dan tombol Export nonaktif. &
Sesuai \\ \hline

\end{tabular}
\end{table}


   \item \textbf{Pengujian pada Summary}\\
   Pengujian ini dilakukan untuk memastikan bahwa komponen ringkasan menampilkan informasi proses secara akurat, meliputi status pemeriksaan, jumlah seluruh tautan yang ditemukan, jumlah tautan halaman, serta jumlah tautan rusak. Pengujian juga memverifikasi bahwa nilai pada Summary selalu mencerminkan hasil crawling sebenarnya dan tidak berubah ketika pengguna menerapkan filter pada tabel hasil. Hasil pengujian dapat dilihat pada Tabel~\ref{tab:pengujian-summary}.

   \renewcommand{\arraystretch}{1.4}
\begin{table}[H]
\centering
\caption{Pengujian pada Summary}
\label{tab:pengujian-summary}
\begin{tabular}{|c|>{\raggedright\arraybackslash}p{6cm}|>{\raggedright\arraybackslash}p{6cm}|c|}
\hline
\textbf{Kasus} & \textbf{Skenario} & \textbf{Hasil Diharapkan} & \textbf{Hasil Uji} \\ \hline

1 &
Memulai proses pemeriksaan dengan URL valid. &
Status berubah menjadi \texttt{CHECKING} dan seluruh nilai Summary mulai dihitung. &
Sesuai \\ \hline

2 &
Proses pemeriksaan selesai tanpa dihentikan. &
Status berubah menjadi \texttt{COMPLETED}, seluruh nilai Summary menampilkan hasil akhir crawling. &
Sesuai \\ \hline

3 &
Menghentikan proses pemeriksaan dengan menekan tombol Stop. &
Status berubah menjadi \texttt{STOPPED}, nilai Summary tidak lagi bertambah. &
Sesuai \\ \hline

4 &
Menerapkan filter URL pada tabel hasil. &
Nilai Summary tidak berubah dan tetap menampilkan total data crawling sebenarnya. &
Sesuai \\ \hline

5 &
Menerapkan filter Status Code pada tabel hasil. &
Nilai Summary tetap, tidak dipengaruhi filter apa pun. &
Sesuai \\ \hline

6 &
Melakukan pemeriksaan ulang setelah proses sebelumnya dihentikan. &
Summary di-reset ke kondisi awal dan menampilkan nilai baru sesuai proses crawling berikutnya. &
Sesuai \\ \hline

\end{tabular}
\end{table}

   
   \item \textbf{Pengujian pada Pagination}\\
   Pengujian ini dilakukan untuk memastikan bahwa pengguna dapat menavigasi halaman hasil secara benar menggunakan kontrol pagination, baik ketika seluruh data ditampilkan maupun ketika filter sedang aktif. Pengujian mencakup perpindahan halaman maju dan mundur, kondisi ketika berada pada halaman pertama atau terakhir, serta perubahan jumlah data yang ditampilkan akibat filter. Hasil pengujian dapat dilihat pada Tabel~\ref{tab:pengujian-pagination}.
   
   \renewcommand{\arraystretch}{1.4}

\begin{longtable}{
|>{\centering\arraybackslash}p{1cm}
|>{\raggedright\arraybackslash}p{4.5cm} 
|>{\raggedright\arraybackslash}p{4.5cm}
|>{\raggedright\arraybackslash}p{4.5cm}|}
\caption{Hasil Pengujian Funsional Pagination}
\vspace{-3mm}
\label{tab:pengujian-funsional-pagination} \\

\hline
\multicolumn{1}{|c|}{\textbf{Kasus}} &
\multicolumn{1}{c|}{\textbf{Skenario}} &
\multicolumn{1}{c|}{\textbf{Hasil yang Diharapkan}} &
\multicolumn{1}{c|}{\textbf{Hasil Uji}} \\
\hline
\endfirsthead

\multicolumn{4}{c}{Tabel~\ref{tab:pengujian-funsional-pagination} dilanjutkan dari halaman sebelumnya}\\[4pt]

\hline
\multicolumn{1}{|c|}{\textbf{Kasus}} &
\multicolumn{1}{c|}{\textbf{Skenario}} &
\multicolumn{1}{c|}{\textbf{Hasil yang Diharapkan}} &
\multicolumn{1}{c|}{\textbf{Hasil Uji}} \\
\hline
\endhead

\hline
\multicolumn{4}{r}{Bersambung ke halaman berikutnya} \\
\endfoot

\hline
\endlastfoot


1 &
Menekan tombol Start ketika status aplikasi berada pada status IDLE. &
Proses crawling dimulai, data lama dibersihkan, dan status berubah menjadi CHECKING. &
Aplikasi merubah status menjadi CHECKING dan memulai proses pemeriksaan. \\ \hline

2 &
Menekan tombol Start ketika status aplikasi berada pada status CHECKING, STOPPED atau COMPLETED. &
Proses pemeriksaan dimulai ulang dan status berubah menjadi CHECKING. &
Aplikasi membersihkan data lama, merubah status menjadi CHECKING dan memulai ulang proses pemeriksaan. \\ \hline

3 &
Menekan tombol Stop ketika status aplikasi berada pada status CHECKING. &
Proses crawling dihentikan dan status berubah menjadi STOPPED. &
Aplikasi menghentikan proses pemeriksaan yang sedang berjalandan dan merubah status menjadi STOPPED. \\ \hline

4 &
Menekan tombol Stop ketika status aplikasi berada pada status IDLE, STOPPED atau COMPLETED. &
Tombol Stop tidak bisa ditekan. &
Tombol Stop tidak merespons karena dalam status disabled. \\ \hline

5 &
Menekan tombol Export ketika status aplikasi berada pada status IDLE atau CHECKING. &
Tombol Export tidak bisa ditekan. &
Tombol Export tidak merespons karena dalam status disabled. \\ \hline

6 &
Menekan tombol Export ketika status STOPPED atau COMPLETED dan data di tabel tersedia. &
Dialog penyimpanan file muncul dan proses ekspor dapat dilakukan. &
Aplikasi menampilkan dialog penyimpanan file, lalu menampilkan notifikasi SUCCESS saat selesai. \\ \hline

7 &
Menekan tombol Export ketika status STOPPED atau COMPLETED namun data di tabel tidak tersedia. &
Muncul notifikasi WARNING bahwa tidak ada data yang dapat diekspor. &
Aplikasi menampilkan notifikasi WARNING \textit{“There are no broken links to export.”}. \\ \hline

8 &
Menekan tombol Export lalu membatalkan dialog penyimpanan file &
Proses ekspor dibatalkan tanpa ada perubahan status aplikasi. &
Aplikasi tidak memproses ekspor \\ \hline



\end{longtable}

   

   \item \textbf{Pengujian pada Fitur Pemeriksaan}\\
   Pengujian ini mencakup verifikasi terhadap seluruh mekanisme pemeriksaan tautan yang dijalankan oleh sistem, mulai dari proses \textit{crawling} hingga pengecekan status setiap tautan. Hasil pengujian ditampilkan pada Tabel~\ref{tab:pengujian-fitur-pemeriksaan}.

   
   \renewcommand{\arraystretch}{1.4}
\begin{longtable}{
|>{\centering\arraybackslash}p{1cm}
|>{\raggedright\arraybackslash}p{4.5cm} 
|>{\raggedright\arraybackslash}p{4.5cm}
|>{\raggedright\arraybackslash}p{4.5cm}|}
\caption{Hasil Pengujian Fungsional Fitur Pemeriksaan}
\vspace{-3mm}
\label{tab:hasil-pengujian-fungsional-fitur-pemeriksaan} \\

\hline
\multicolumn{1}{|c|}{\textbf{Kasus}} &
\multicolumn{1}{c|}{\textbf{Skenario}} &
\multicolumn{1}{c|}{\textbf{Hasil yang Diharapkan}} &
\multicolumn{1}{c|}{\textbf{Hasil Uji}} \\
\hline
\endfirsthead

\multicolumn{4}{c}{Tabel~\ref{tab:hasil-pengujian-fungsional-fitur-pemeriksaan} dilanjutkan dari halaman sebelumnya}\\[4pt]

\hline
\multicolumn{1}{|c|}{\textbf{Kasus}} &
\multicolumn{1}{c|}{\textbf{Skenario}} &
\multicolumn{1}{c|}{\textbf{Hasil yang Diharapkan}} &
\multicolumn{1}{c|}{\textbf{Hasil Uji}} \\
\hline
\endhead

\hline
\multicolumn{4}{|r|}{Bersambung ke halaman berikutnya} \\ \hline
\endfoot

\hline
\endlastfoot

1 &
Memeriksa satu per satu tautan rusak yang ditemukan dengan menekan URL pada tabel. &
Aplikasi mengarahkan tautan untuk terbuka ke \textit{browser} dan menampilkan \textit{error} yang sama. &
Tautan terbuka ke \textit{browser} dan menunjukkan \textit{error} yang sama dengan hasil pemeriksaan aplikasi. \\ \hline

2 &
Mengekspor hasil pemeriksaan lalu memeriksa satu per satu daftar halaman sumber. &
Seluruh daftar URL halaman sumber memiliki \textit{host} yang sama dengan \textit{host} URL awal. &
Seluruh daftar URL halaman menunjukkan bahwa \textit{host}-nya ``informatika.unpar.ac.id'', sama dengan \textit{host} URL awal. \\ \hline

3 &
Membuat \textit{counter} untuk masing-masing URL yang ditemukan, lalu tampilkan jendela notifikasi jika \textit{counter} lebih dari satu. &
Aplikasi tidak membuka jendela notifikasi untuk menampilkan pesan tautan diperiksa lebih dari satu kali. &
Pemeriksaan berjalan sampai proses selesai tanpa menampilkan jendela notifikasi apapun. \\ \hline


4 &
Menambahkan log waktu pemeriksaan pada setiap permintaan HTTP. &
Tautan dengan \textit{host} yang sama menampilkan jeda waktu yang konsisten antar permintaan, sedangkan tautan dari \textit{host} yang berbeda tidak mengikuti jeda yang sama. &
Log menunjukkan bahwa permintaan ke \textit{host} ``informatika.unpar.ac.id'' muncul dengan selang waktu tetap, sementara permintaan ke \textit{host} lain dapat muncul di antara jeda tersebut. \\ \hline


\end{longtable}


% ###################################################################################
% ###################################################################################
% ###################################################################################

\renewcommand{\arraystretch}{1.5}
\begin{table}[H]
   \centering
   \caption{Ringkasan Proses Pemeriksaan}
   \vspace{3mm}
   \label{tab:ringkasan-proses-pemeriksaan-fitur-pemeriksaan}
   
   \begin{tabular}{|p{7cm}|p{5cm}|}
      \hline      
      \textbf{Status Pemeriksaan} & COMPLETED \\ \hline
      \textbf{Waktu Mulai Pemeriksaan} & 26-11-2025 06:41:36 \\ \hline
      \textbf{Waktu Selesai Pemeriksaan} & 26-11-2025 06:54:43 \\ \hline
      \textbf{Durasi Pemeriksaan} & 13~menit 6~detik \\ \hline
      \textbf{Jumlah Total Tautan} & 602 \\ \hline
      \textbf{Jumlah Tautan Halaman} & 302 \\ \hline
      \textbf{Jumlah Tautan Rusak} & 80 \\ \hline
   \end{tabular}
\end{table}

% ###################################################################################
% ###################################################################################
% ###################################################################################

\renewcommand{\arraystretch}{1.5}
\begin{longtable}{
   |>{\raggedright\arraybackslash}m{4cm}
   |>{\raggedright\arraybackslash}m{6cm}
   |>{\centering\arraybackslash}m{2cm}|}
\caption{Ringkasan Tautan Rusak}
\vspace{-3mm}
\label{tab:ringkasan-tautan-rusak-fitur-pemeriksaan} \\
\hline

{\centering\arraybackslash\textbf{Kategori}} &
{\centering\arraybackslash\textbf{Error}} &
{\centering\arraybackslash\textbf{Jumlah}} \\ \hline
\endfirsthead

\multicolumn{3}{c}{Tabel~\ref{tab:ringkasan-tautan-rusak-fitur-pemeriksaan} dilanjutkan dari halaman sebelumnya}\\[4pt]

\hline
{\centering\arraybackslash\textbf{Kategori}} &
{\centering\arraybackslash\textbf{Error}} &
{\centering\arraybackslash\textbf{Jumlah}} \\ \hline
\endhead

\hline
\multicolumn{3}{|r|}{Bersambung ke halaman berikutnya} \\
\hline
\endfoot

\hline
\endlastfoot

% ********************************************

\multirow{5}{*}{\textbf{Connection Error}}
 & ConnectException & 13 \\ \cline{2-3}
 & HttpConnectTimeoutException & 5 \\ \cline{2-3}
 & IOException & 1 \\ \cline{2-3}
 & IllegalArgumentException & 1 \\ \cline{2-3}
 & SSLHandshakeException & 5 \\ \hline

\multicolumn{2}{|c|}{{\centering\textbf{Total Connection Error}}} &
{\centering\arraybackslash\textbf{25}} \\ \hline

% ********************************************

\multirow{5}{*}{\textbf{4XX Client Error}}
 & 400 Bad Request & 2 \\ \cline{2-3}
 & 403 Forbidden & 12 \\ \cline{2-3}
 & 404 Not Found & 31 \\ \cline{2-3}
 & 410 Gone & 1 \\ \cline{2-3}
 & 429 Too Many Requests & 4 \\ \hline

\multicolumn{2}{|c|}{{\centering\textbf{Total 4XX Client Error}}} &
{\centering\arraybackslash\textbf{50}} \\ \hline

% ********************************************

\multirow{1}{*}{\textbf{5XX Server Error}}
 & 503 Service Unavailable & 1 \\ \hline

\multicolumn{2}{|c|}{{\centering\textbf{Total 5XX Server Error}}} &
{\centering\arraybackslash\textbf{1}} \\ \hline

% ********************************************

\multirow{2}{*}{\textbf{Non-Standard Error}}
 & 520 & 2 \\ \cline{2-3}
 & 999 & 2 \\ \hline

\multicolumn{2}{|c|}{{\centering\textbf{Total Non-Standard Error}}} &
{\centering\arraybackslash\textbf{4}} \\ \hline

\end{longtable}

   
   \item \textbf{Pengujian pada Fitur Filter}\\
   Pengujian ini dilakukan untuk memastikan bahwa pengguna dapat menyaring daftar tautan berdasarkan kondisi URL maupun kode status HTTP. Pengujian mencakup penggunaan filter tunggal, kombinasi filter, dan kondisi ketika filter tidak menghasilkan data apa pun. Hasil pengujian dapat dilihat pada Tabel~\ref{tab:pengujian-fitur-filter}.

   
   \renewcommand{\arraystretch}{1.4}

\begin{longtable}{
|>{\centering\arraybackslash}p{1cm}
|>{\raggedright\arraybackslash}p{4.5cm} 
|>{\raggedright\arraybackslash}p{4.5cm}
|>{\raggedright\arraybackslash}p{4.5cm}|}
\caption{Hasil Pengujian Funsional Fitur \textit{Filter}}
\vspace{-3mm}
\label{tab:pengujian-funsional-fitur-filter} \\

\hline
\multicolumn{1}{|c|}{\textbf{Kasus}} &
\multicolumn{1}{c|}{\textbf{Skenario}} &
\multicolumn{1}{c|}{\textbf{Hasil yang Diharapkan}} &
\multicolumn{1}{c|}{\textbf{Hasil Uji}} \\
\hline
\endfirsthead

\multicolumn{4}{c}{Tabel~\ref{tab:pengujian-funsional-fitur-filter} dilanjutkan dari halaman sebelumnya}\\[4pt]

\hline
\multicolumn{1}{|c|}{\textbf{Kasus}} &
\multicolumn{1}{c|}{\textbf{Skenario}} &
\multicolumn{1}{c|}{\textbf{Hasil yang Diharapkan}} &
\multicolumn{1}{c|}{\textbf{Hasil Uji}} \\
\hline
\endhead

\hline
\multicolumn{4}{r}{Bersambung ke halaman berikutnya} \\
\endfoot

\hline
\endlastfoot
1 &
Menerapkan \textit{filter} URL dengan kondisi \texttt{Equals} menggunakan sebuah URL tertentu (``https://godev.co/''). &
Tabel hanya menampilkan satu baris yang URL-nya sama persis dengan kata kunci \textit{filter}. &
Tabel hanya menampilkan baris yang URL-nya ``https://godev.co/''. \\ \hline

2 &
Menerapkan \textit{filter} URL dengan kondisi \texttt{Contains} menggunakan kata kunci tertentu (``informatika'') &
Tabel hanya menampilkan baris yang URL-nya mengandung kata kunci \textit{filter}. &
Tabel hanya menampilkan baris yang URL-nya mengandung teks ``informatika''. \\ \hline

3 &
Menerapkan \textit{filter} URL dengan kondisi \texttt{Starts With} menggunakan awal URL tertentu (``https://informatika''). &
Tabel hanya menampilkan baris yang URL-nya diawali dengan kunci \textit{filter}. &
Tabel hanya menampilkan baris yang URL-nya dimulai dengan ``https://informatika''. \\ \hline

4 &
Menerapkan \textit{filter} URL dengan kondisi \texttt{Ends With} menggunakan akhir URL tertentu (``.pdf''). &
Tabel hanya menampilkan baris yang URL-nya diakhiri dengan kunci \textit{filter}. &
Tabel hanya menampilkan baris yang URL-nya diakhiri dengan ``.pdf''. \\ \hline

5 &
Menerapkan \textit{filter} Kode Status dengan kondisi \texttt{Equals} menggunakan sebuah nilai tertentu (0). &
Tabel hanya menampilkan baris yang kode statusnya sama dengan kunci \textit{filter}. &
Tabel hanya menampilkan baris dengan error koneksi. \\ \hline

6 &
Menerapkan \textit{filter} Kode Status dengan kondisi \texttt{Greater Than} menggunakan sebuah nilai tertentu (404). &
Tabel hanya menampilkan baris yang kode statusnya lebih besar dari kunci \textit{filter}. &
Tabel hanya menampilkan baris dengan kode status yang lebih besar dari 404. \\ \hline

7 &
Menerapkan \textit{filter} Kode Status dengan kondisi \texttt{Less Than} menggunakan sebuah nilai tertentu (404). &
Tabel hanya menampilkan baris yang kode statusnya lebih kecil dari kunci \textit{filter}. &
Tabel hanya menampilkan baris dengan kode status yang lebih kecil dari 404. \\ \hline

8 &
Menerapkan \textit{filter} gabungan URL \texttt{Contains} (``informatika'') dan Kode Status \texttt{Equals} (404). &
Tabel menampilkan hanya baris yang memenuhi kedua kondisi sekaligus. &
Tabel hanya menampilkan baris yang URL-nya mengandung teks ``informatika'' dan \textit{error}-nya \textit{404 Not Found} \\ \hline

9 &
Menerapkan \textit{filter} ketika proses pemeriksaan masih berlangsung. &
Tabel hanya menampilkan baris sesuai \textit{filter} &
\textit{Filter} tetap bekerja secara \textit{real time } dan hasil tabel menyesuaikan setiap kali ada tautan baru masuk. \\ \hline

10 &
Menghapus penerapan \textit{filter}. &
Tabel kembali menampilkan seluruh tautan rusak. &
Tabel langsung menampilkan semua baris tautan rusak seperti sebelum \textit{filter} diterapkan. \\ \hline

\end{longtable}
   
   \item \textbf{Pengujian pada Fitur Ekspor}\\
   Pengujian ini dilakukan untuk memastikan bahwa aplikasi dapat mengekspor data hasil pemeriksaan ke berkas Excel dengan benar. Pengujian juga memverifikasi bahwa hasil ekspor mengikuti data yang sedang ditampilkan pada tabel, termasuk ketika filter aktif sehingga hanya data hasil penyaringan yang diekspor. Hasil pengujian dapat dilihat pada Tabel~\ref{tab:pengujian-fitur-export}.
   
   \renewcommand{\arraystretch}{1.4}

\begin{longtable}{
|>{\centering\arraybackslash}p{1cm}
|>{\raggedright\arraybackslash}p{4.5cm} 
|>{\raggedright\arraybackslash}p{4.5cm}
|>{\raggedright\arraybackslash}p{4.5cm}|}
\caption{Hasil Pengujian Fungsional Fitur \textit{Filter}}
\vspace{-3mm}
\label{tab:hasil-pengujian-fungsional-fitur-export} \\

\hline
\multicolumn{1}{|c|}{\textbf{Kasus}} &
\multicolumn{1}{c|}{\textbf{Skenario}} &
\multicolumn{1}{c|}{\textbf{Hasil yang Diharapkan}} &
\multicolumn{1}{c|}{\textbf{Hasil Uji}} \\
\hline
\endfirsthead

\multicolumn{4}{c}{Tabel~\ref{tab:hasil-pengujian-fungsional-fitur-export} dilanjutkan dari halaman sebelumnya}\\[4pt]

\hline
\multicolumn{1}{|c|}{\textbf{Kasus}} &
\multicolumn{1}{c|}{\textbf{Skenario}} &
\multicolumn{1}{c|}{\textbf{Hasil yang Diharapkan}} &
\multicolumn{1}{c|}{\textbf{Hasil Uji}} \\
\hline
\endhead

\hline
\multicolumn{4}{|r|}{Bersambung ke halaman berikutnya} \\ \hline
\endfoot

\hline
\endlastfoot

1 &
Menekan tombol \texttt{Export} ketika tabel kosong. &
Tidak ada berkas diekspor dan aplikasi menampilkan notifikasi bahwa tidak ada data yang dapat diekspor. &
Aplikasi menampilkan peringatan \textit{``There are no broken links to export.''} dan dialog penyimpanan tidak muncul. \\ \hline

2 &
Menekan tombol \texttt{Export} ketika tabel memiliki data. &
Dialog penyimpanan file muncul dan pengguna dapat memilih lokasi penyimpanan. &
Aplikasi menampilkan dialog penyimpanan file, dan proses ekspor berhasil menghasilkan file Excel. \\ \hline

3 &
Melakukan \textit{filter} pada tabel setelah ekspor pertama, lalu melakukan ekspor kembali. &
Ekspor berhasil dan data hasil ekspor sesuai dengan data terkini. &
File Excel kedua hanya berisi data hasil \textit{filter}. \\ \hline

4 &
Membatalkan dialog penyimpanan file (menekan tombol \texttt{Cancel}). &
Tidak ada berkas yang dibuat. &
Dialog langsung tertutup dan tidak ada berkas baru dihasilkan. \\ \hline


\end{longtable}
   
   \item \textbf{Pengujian pada Jendela Detail Tautan}\\
   Pengujian ini dilakukan untuk memastikan bahwa URL, \textit{final} URL, jenis konten, pesan error, serta daftar halaman sumber ditampilkan dengan benar dan sesuai data pemeriksaan. Hasil pengujian dapat dilihat pada Tabel~\ref{tab:pengujian-jendela-detail-tautan}.


   \renewcommand{\arraystretch}{1.4}
\begin{table}[H]
\centering
\caption{Pengujian pada Jendela Detail Tautan}
\label{tab:pengujian-jendela-detail-tautan}
\begin{tabular}{|c|>{\raggedright\arraybackslash}p{6cm}|>{\raggedright\arraybackslash}p{6cm}|c|}
\hline
\textbf{Kasus} & \textbf{Skenario} & \textbf{Hasil Diharapkan} & \textbf{Hasil Uji} \\ \hline


1 &
Menekan salah satu baris tidak kosong pada tabel tautan rusak. &
Jendela detail tautan terbuka dan menampilkan informasi tautan yang sesuai. &
Sesuai \\ \hline

2 &
Menekan salah satu baris kosong pada tabel tautan rusak. &
Jendela detail tautan tidak. &
Sesuai \\ \hline

3 &
Menekan kolom URL pada jendela detail tautan &
\textit{Browser} bawaan OS membuka URL yang ditekan &
Sesuai \\ \hline

4 &
Menekan kolom \textit{final} URL yang tidak kosong pada jendela detail tautan &
\textit{Browser} bawaan OS membuka URL yang ditekan &
Sesuai \\ \hline

5 &
Menekan kolom \textit{final} URL yang kosong pada jendela detail tautan &
\textit{Browser} bawaan OS tidak terbuka &
Sesuai \\ \hline


\end{tabular}
\end{table}

   
   \item \textbf{Pengujian pada Jendela Notifikasi}\\
   Pengujian ini dilakukan untuk memastikan bahwa jendela notifikasi untuk pesan \textit{success}, \textit{warning}, dan \textit{error} muncul pada kondisi yang tepat serta menampilkan pesan yang sesuai. Hasil pengujian dapat dilihat pada Tabel~\ref{tab:pengujian-jendela-notifikasi}.
   
   \renewcommand{\arraystretch}{1.3}

\begin{longtable}{
|>{\centering\arraybackslash}p{1cm}
|>{\raggedright\arraybackslash}p{4.5cm} 
|>{\raggedright\arraybackslash}p{4.5cm}
|>{\raggedright\arraybackslash}p{4.5cm}|}
\caption{Hasil Pengujian Fungsional Jendela Notifikasi}
\vspace{-3mm}
\label{tab:hasil-pengujian-fungsional-jendela-notifikasi} \\

\hline
\multicolumn{1}{|c|}{\textbf{Kasus}} &
\multicolumn{1}{c|}{\textbf{Skenario}} &
\multicolumn{1}{c|}{\textbf{Hasil yang Diharapkan}} &
\multicolumn{1}{c|}{\textbf{Hasil Uji}} \\
\hline
\endfirsthead

\multicolumn{4}{c}{Tabel~\ref{tab:hasil-pengujian-fungsional-jendela-notifikasi} dilanjutkan dari halaman sebelumnya}\\[4pt]

\hline
\multicolumn{1}{|c|}{\textbf{Kasus}} &
\multicolumn{1}{c|}{\textbf{Skenario}} &
\multicolumn{1}{c|}{\textbf{Hasil yang Diharapkan}} &
\multicolumn{1}{c|}{\textbf{Hasil Uji}} \\
\hline
\endhead

\hline
\multicolumn{4}{|r|}{Bersambung ke halaman berikutnya} \\ \hline
\endfoot

\hline
\endlastfoot

1 &
Berhasil melakukan ekspor data. &
Jendela notifikasi bertipe \texttt{SUCCESS} ditampilkan dan berisi pesan keberhasilan. &
Notifikasi \texttt{SUCCESS} ditampilkan dengan pesan \textit{``Data has been exported to''}. \\ \hline

2 &
Memicu \textit{exception} pada proses pemeriksaan. &
Jendela notifikasi bertipe \texttt{ERROR} ditampilkan dan berisi pesan kesalahan. &
Notifikasi \texttt{ERROR} ditampilkan dengan pesan error yang sesuai. \\ \hline

3 &
Memasukkan input URL tidak valid &
Jendela notifikasi bertipe \texttt{WARNING} ditampilkan dan berisi pesan peringatan. &
Notifikasi \texttt{WARNING} ditampilkan dengan pesan \textit{``Please enter a valid seed URL …''} \\ \hline

4 &
Menutup jendela notifikasi. &
Jendela notifikasi tertutup sepenuhnya. &
Jendela notifikasi langsung menutup. \\ \hline

5 &
Memicu dua jenis notifikasi berurutan (\texttt{ERROR} dan \texttt{WARNING}). &
Hanya satu jendela notifikasi yang tampil, tidak ada duplikasi jendela meskipun dua notifikasi dipicu. &
Aplikasi hanya menampilkan satu jendela notifikasi pada satu waktu. \\ \hline


\end{longtable}

   \item \textbf{Perbandingan dengan Perangkat Lunak Sejenis}\\
   

   \item \textbf{Perbandingan dengan Perangkat Lunak Sejenis}\\
   Pada bagian ini akan dilakukan perbandingan hasil pemeriksaan antara perangkat lunak yang dikembangkan pada tugas akhir ini dengan dua perangkat lunak serupa yang tersedia secara daring, yaitu Broken Link Checker\footnote{\url{https://brokenlinkcheck.com}} dan Dead Link Checker\footnote{\url{https://deadlinkchecker.com}}. Seluruh percobaan akan dilakukan pada situs Informasika Unpar (\url{https://informatika.unpar.ac.id}) dengan jumlah percobaan sebanyak lima kali. Parameter yang akan digunakan dalam perbandingan adalah durasi pemeriksaan, jumlah seluruh tautan yang ditemukan, jumlah tautan halaman yang ditemukan dan jumlah tautan rusak yang ditemukan. Perbandingan ini bertujuan menilai kelengkapan pendeteksian tautan rusak, performa proses pemeriksaan, serta kualitas informasi yang dihasilkan.
   
   Berikut adalah hasil pemeriksaan dari setiap perangkat lunak:
   \begin{itemize}
      \item Perangkat lunak yang dikembangkan
      \begin{itemize}
         \item Durasi pemeriksaan: \textbf{15,48 menit}
         \item Jumlah seluruh tautan: \textbf{602}
         \item Jumlah tautan halaman: \textbf{302}
         \item Jumlah tautan rusak: \textbf{92}
      \end{itemize}

      \item Perangkat lunak Broken Link Checker:
      \begin{itemize}
         \item Durasi pemeriksaan: \textbf{--}
         \item Jumlah seluruh tautan: \textbf{--}
         \item Jumlah tautan halaman: \textbf{--}
         \item Jumlah tautan rusak: \textbf{--}
      \end{itemize}

      \item Perangkat lunak Dead Link Checker:
      \begin{itemize}
         \item Durasi pemeriksaan: \textbf{--}
         \item Jumlah seluruh tautan: \textbf{--}
         \item Jumlah tautan halaman: \textbf{--}
         \item Jumlah tautan rusak: \textbf{--}
      \end{itemize}
      
   \end{itemize}


   \begin{figure}[H]
      \centering
      \includegraphics[width=0.85\textwidth]{Gambar/050202-hasil-pengujian-aplikasi.png}
      \caption{Hasil Pemeriksaan Menggunakan Aplikasi yang Dikembangkan}
      \label{fig:hasil-blc-penulis}
   \end{figure}

   \begin{figure}[H]
      \centering
      \includegraphics[width=0.85\textwidth]{Gambar/050202-hasil-pengujian-aplikasi.png}
      \caption{Hasil Pemeriksaan Menggunakan Broken Link Checker}
      \label{fig:hasil-blc-web}
   \end{figure}

   \begin{figure}[H]
      \centering
      \includegraphics[width=0.85\textwidth]{Gambar/050202-hasil-pengujian-aplikasi.png}
      \caption{Hasil Pemeriksaan Menggunakan Dead Link Checker}
      \label{fig:hasil-dlc-web}
   \end{figure}
   
\end{enumerate}