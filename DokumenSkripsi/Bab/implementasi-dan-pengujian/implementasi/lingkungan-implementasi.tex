Lingkungan implementasi menjelaskan konfigurasi perangkat keras dan perangkat lunak yang digunakan dalam proses pengembangan aplikasi desktop pemeriksa tautan rusak.
Berikut ini merupakan rincian lingkungan implementasi yang digunakan:

\begin{enumerate}
    \item \textbf{Lingkungan Perangkat Keras}\\
    Perangkat keras yang digunakan dalam proses implementasi adalah sebuah komputer dengan spesifikasi sebagai berikut:
    \begin{enumerate}
        \item Komputer: Lenovo Legion 5 (Model 82B5)
        \item \textit{Processor}: AMD Ryzen 5 4600H (6 \textit{Cores}, 12 \textit{Threads}), 3.0\,GHz
        \item \textit{Random Access Memory} (RAM): 16\,GB DDR4
        \item \textit{Solid State Drive} (SSD): 512\,GB
    \end{enumerate}

    \item \textbf{Lingkungan Perangkat Lunak}\\
    Perangkat lunak yang digunakan dalam proses implementasi adalah sebagai berikut:
    \begin{enumerate}
        \item Sistem Operasi: Windows 11 64-bit
        \item Bahasa Pemrograman: Java 21
        \item \textit{Build System}: Gradle 8.8
        \item \textit{Framework} dan pustaka yang digunakan:
        \begin{itemize}
            \item JavaFX 21: Digunakan untuk pengembangan antarmuka pengguna berbasis FXML.
            \item Jsoup 1.17.2: Digunakan untuk melakukan pengambilan dan pemrosesan konten HTML.
            \item Java \texttt{HttpClient}: Digunakan untuk melakukan permintaan HTTP terhadap tautan.
            \item Apache POI 5.5.0: Digunakan untuk ekspor hasil ke berkas Excel.
        \end{itemize}
    \end{enumerate}
\end{enumerate}