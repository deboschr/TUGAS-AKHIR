Lingkungan implementasi menjelaskan konfigurasi perangkat keras dan perangkat lunak yang digunakan dalam proses pengembangan aplikasi pemeriksa tautan rusak pada situ web. 
Berikut merupakan rincian lingkungan implementasi yang digunakan:

\begin{enumerate}
    \item \textbf{Lingkungan Perangkat Keras}\\
    Perangkat keras yang digunakan dalam proses implementasi aplikasi pemeriksa tautan rusak pada situs web adalah sebuah laptop dengan spesifikasi sebagai berikut:
    \begin{enumerate}
        \item Laptop: Lenovo Legion 5 (Model 82B5)
        \item \textit{Processor}: AMD Ryzen 5 4600H (12 \textit{Cores} CPU), 3.0\,GHz
        \item \textit{Random Access Memory} (RAM): 16\,GB DDR4
        \item \textit{Solid State Drive} (SSD): 512\,GB
    \end{enumerate}

    \item \textbf{Lingkungan Perangkat Lunak}\\
    Perangkat lunak yang digunakan untuk mengimplementasikan aplikasi terdiri atas sistem operasi, bahasa pemrograman, pustaka pendukung, dan alat bantu pengembangan, dengan rincian sebagai berikut:
    \begin{enumerate}
        \item Sistem Operasi: Windows 11 64-bit
        \item Bahasa Pemrograman: Java 21
        \item \textit{Build System}: Gradle 8.8
        \item Framework dan Pustaka yang Digunakan:
        \begin{itemize}
            \item JavaFX 21 : Digunakan untuk pengembangan antarmuka pengguna berbasis FXML.
            \item Jsoup 1.17.2 : Digunakan untuk melakukan pengambilan dan pemrosesan konten HTML.
            \item Java HttpClient : Digunakan untuk melakukan permintaan HTTP terhadap tautan.
            % \item Apache POI 5.2.5 : Digunakan untuk ekspor hasil pemeriksaan tautan ke dalam format \texttt{.xlsx}.
            % \item Gson 2.11.0 : Digunakan untuk pemrosesan data berformat JSON.
        \end{itemize}
    \end{enumerate}
\end{enumerate}

