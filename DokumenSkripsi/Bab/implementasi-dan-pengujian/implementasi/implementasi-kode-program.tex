Subbab ini menjelaskan implementasi kode program dari aplikasi pemeriksa tautan rusak pada situs web berdasarkan yang telah dirancang pada Bab~\ref{chap:040000-perancangan}.
Berikut merupakan hasil implementasi kode program yang digunakan dalam aplikasi:

\begin{enumerate}
   \item \textbf{Application.java}\\
   Kelas ini berfungsi sebagai titik masuk program sekaligus bertugas untuk membuka seluruh jendela aplikasi. 
   Kelas ini memiliki satu atribut \texttt{mainStage} yang menyimpan referensi ke jendela utama agar dapat dijadikan induk bagi jendela lain yang dibuka dari aplikasi.
   Implementasi kode program dari kelas ini dapat dilihat pada Lampiran~\ref{code:application-java}.

   \item \textbf{MainController.java, main-window.fxml dan main-style.css}\\

   \item \textbf{LinkController.java, link-window.fxml dan link-style.css}\\

   \item \textbf{NotificationController.java, notification-window.fxml dan notification-style.css}\\

   \item \textbf{Crawler.java}\\

   \item \textbf{Link.java}\\

   \item \textbf{Summary.java}\\

   \item \textbf{Status.java}\\

   \item \textbf{Exporter.java}\\

   \item \textbf{UrlHandler.java}\\

   \item \textbf{HttpStatus.java}\\

   \item \textbf{RateLimiter.java}\\

\end{enumerate}
