Bagian ini menjelaskan hasil implementasi kode program aplikasi berdasarkan perancangan yang telah dibuat pada perancangan kelas (lihat Subbab~\ref{sec:040100-perancangan-kelas}) dan perancangan alur (lihat Subbab~\ref{sec:040200-perancangan-alur}). Berikut adalah penjelasan dari setiap hasil implementasi kode program aplikasi:

\vspace{2mm}

\begin{enumerate}

   \item \textbf{Berkas \texttt{Application.java}}\\
   Kelas \texttt{Application} merupakan titik awal aplikasi JavaFX serta bertugas untuk membuka jendela aplikasi seperti jendela utama, jendela detail tautan, dan jendela notifikasi. Kode lengkap untuk kelas ini terdapat pada Lampiran~\ref{code:application-java}, namun pada bagian ini akan dijelaskan lebih rinci untuk metode \texttt{openMainWindow}, yaitu metode yang digunakan untuk membuka jendela utama aplikasi. Berikut adalah penjelasan lebih rinci untuk metode \texttt{openMainWindow} (lihat Kode~\ref{code:application-open-main-window-java}):

   \begin{enumerate}
      \item (Baris 3) Menentukan lokasi berkas \texttt{main-scene.fxml} yang berisi struktur tampilan.

      \item (Baris 4) Membaca berkas \texttt{main-scene.fxml}.
      
      \item (Baris 5) Membentuk tampilan antarmuka dari berkas \texttt{main-scene.fxml} untuk digunakan sebagai isi jendela utama.
      
      \item (Baris 7) Memasukkan tampilan antarmuka ke dalam jendela utama.
      
      \item (Baris 8) Menghilangkan dekorasi jendela bawaan sistem operasi agar tampilan dapat dikustomisasi sepenuhnya.
      
      \item (Baris 9) Menempatkan jendela pada posisi tengah layar.
      
      \item (Baris 10) Membuat jendela tampil dalam ukuran maksimal saat pertama kali dibuka.
      
      \item (Baris 11--12) Menetapkan ukuran minimum jendela agar tata letak tidak rusak ketika jendela diperkecil.
      
      \item (Baris 13) Menampilkan jendela ke layar.
      
      \item (Baris 15--16) Jika terjadi \textit{exception}, maka tampilkan informasi \textit{error} ke konsol.
      
   \end{enumerate}

   \vspace{5mm}

   \lstinputlisting[
      language=Java, 
      caption=Kelas \texttt{Application} metode \texttt{openMainWindow},
      label={code:application-open-main-window-java}
   ]{Lampiran/implementasi-kode-program/Application_openMainWindow.java}

   \vspace{8mm}

   \item \textbf{Berkas \texttt{MainController.java}, \texttt{main-scene.fxml} dan \texttt{main-style.css}}\\
   Ketiga kode program ini bertanggung jawab untuk mengelola antarmuka jendela utama (lihat Gambar~\ref{fig:implementasi-jendela-utama}). Kelas \texttt{MainController} digunakan untuk mengendalikan interaksi pada jendela utama (lihat Lampiran~\ref{code:main-controller-java}). Berkas \texttt{main-scene.fxml} digunakan untuk mendefinisikan struktur, hierarki, dan komponen antarmuka pada jendela utama (lihat Lampiran~\ref{code:main-scene-fxml}). Sementara itu, berkas \texttt{main-style.css} digunakan untuk mengatur gaya visual pada jendela utama (lihat Lampiran~\ref{code:main-style-css}). Pada kelas \texttt{MainController} terdapat metode \texttt{onStartClick} yang digunakan untuk menjalankan proses pemeriksaan ketika pengguna menekan tombol \textit{Start}, berikut adalah penjelasan lebih rinci untuk metode ini (lihat Kode~\ref{code:main-controller-on-start-click-java}):

   \begin{enumerate}

      \item (Baris 3--8) Mendapatkan URL masukan lalu menormalisasi URL tersebut, jika tidak valid maka hentikan proses dan tampilkan jendela notifikasi bertipe \textit{warning}.
      
      \item (Baris 9--11) Menampilkan URL hasil normalisasi kembali ke GUI lalu membersihkan data lama dan memperbarui status pemeriksaan.

      \item (Baris 12) Menjalankan proses pemeriksaan di \textit{thread} terpisah agar tampilan GUI tidak membeku.
      
      \item (Baris 14--16) Mencatat waktu mulai pemeriksaan, lalu menjalankan pemeriksaan, setelah pemeriksaan selesai catat waktunya.
      
      \item (Baris 17) Jika proses pemeriksaan tidak dihentikan oleh pengguna, maka perbarui status menjadi \texttt{COMPLETED}.
      
      \item (Baris 18--24) Jika terjadi \textit{exception}, maka tampilkan \textit{error} ke jendela notifikasi.

   \end{enumerate}

   \lstinputlisting[
      language=Java, 
      caption=Kelas \texttt{MainController} metode \texttt{onStartClick},
      label={code:main-controller-on-start-click-java}
   ]{Lampiran/implementasi-kode-program/MainController_onStartClick.java}


   \vspace{5mm}

   \item \textbf{Berkas \texttt{LinkController.java}, \texttt{link-scene.fxml}, dan \texttt{link-style.css}}\\
   Ketiga kode program ini bertanggung jawab untuk mengelola antarmuka jendela detail tautan (lihat Gambar~\ref{fig:implementasi-jendela-detail-tautan}). Kelas \texttt{LinkController} bertugas sebagai pengendali interaksi pada jendela detail tautan (lihat Lampiran~\ref{code:link-controller-java}). Berkas \texttt{link-scene.fxml} digunakan untuk mendefinisikan struktur, hierarki, dan komponen antarmuka pada jendela detail tautan (lihat Lampiran~\ref{code:link-scene-fxml}). Sementara itu, berkas \texttt{link-style.css} digunakan untuk mengatur gaya visual pada jendela detail tautan (lihat Lampiran~\ref{code:link-style-css}). Pada kelas \texttt{LinkController} terdapat metode \texttt{setFieldValue} yang digunakan untuk menetapkan nilai dari atribut objek \texttt{Link} ke komponen GUI jendela detail tautan, berikut adalah penjelasan lebih rinci untuk metode ini (lihat Kode~\ref{code:link-controller-set-link-java}):

   \vspace{2mm}

   \begin{enumerate}[itemsep=5pt]
      \item (Baris 2--8) Memasukkan nilai atribut dari objek \texttt{Link} seperti URL, URL akhir setelah \textit{redirect}, tipe konten tautan, serta pesan kesalahan ke komponen GUI.

      \item (Baris 10--12) Membuat komponen \texttt{urlField} dan \texttt{finalUrlField} dapat diklik untuk membuka URL tersebut di \textit{browser}.

   \end{enumerate}

   \vspace{5mm}

   \lstinputlisting[
      language=Java, 
      caption=Kelas \texttt{LinkController} metode \texttt{setFieldValue},
      label={code:link-controller-set-link-java}
   ]{Lampiran/implementasi-kode-program/LinkController_setFieldValue.java}

   \vspace{7mm}

   \item \textbf{Berkas \texttt{NotifController.java}, \texttt{notif-scene.fxml}, dan \texttt{notif-style.css}}\\
   Ketiga kode program ini bertanggung jawab untuk mengelola antarmuka jendela notifikasi (lihat Gambar~\ref{fig:implementasi-jendela-notifikasi}). Kelas \texttt{NotifController} bertugas sebagai pengendali interaksi pada jendela notifikasi (lihat Lampiran~\ref{code:notif-controller-java}). Berkas \texttt{notif-scene.fxml} digunakan untuk mendefinisikan struktur, hierarki, dan komponen antarmuka pada jendela notifikasi (lihat Lampiran~\ref{code:notif-scene-fxml}). Sementara itu, berkas \texttt{notif-style.css} digunakan untuk mengatur gaya visual pada jendela notifikasi (lihat Lampiran~\ref{code:notif-style-css}). Pada kelas \texttt{NotifController} terdapat metode \texttt{setNotifValue} yang digunakan untuk menetapkan pesan dan jenis pesan yang ingin ditampilkan ke jendela notifikasi, berikut adalah penjelasan lebih rinci untuk metode ini (lihat Kode~\ref{code:notif-controller-set-notif-value-java}):

   \vspace{2mm}

   \begin{enumerate}[itemsep=5pt]
      \item (Baris 2) Memasukkan pesan notifikasi ke komponen GUI.
      
      \item (Baris 3--8) Menetapkan \textit{styling} pada jendela berdasarkan tipe pesan. 
   \end{enumerate}

   \vspace{10mm}

   \lstinputlisting[
      language=Java, 
      caption=Kelas \texttt{NotifController} metode \texttt{setNotifValue},
      label={code:notif-controller-set-notif-value-java}
   ]{Lampiran/implementasi-kode-program/NotifController_setNotifValue.java}

   \vspace{4mm}
   

   \item \textbf{Berkas \texttt{Crawler.java}}\\
   Kelas \texttt{Crawler} bertugas untuk melakukan proses pemeriksaan tautan rusak dengan cara menjalankan proses \textit{crawling}. Kode lengkap untuk kelas ini terdapat pada Lampiran~\ref{code:crawler-java}, namun pada bagian ini akan dijelaskan lebih rinci untuk metode \texttt{start}, \texttt{checkLink} dan \texttt{extractLink}. Kelas ini juga memiliki beberapa atribut sebagai berikut:

   \begin{enumerate}
      
      \item \texttt{frontier}: Atribut ini menyimpan antrean halaman situs web yang akan dilakukan \textit{crawling}. Struktur data yang digunakan adalah \texttt{Queue<Link>} agar halaman situs web diproses menggunakan urutan \textit{First-In-First-Out} (FIFO), sesuai dengan kebutuhan strategi \textit{Breadth-First Crawler}. Implementasi dari antrean ini menggunakan kelas \texttt{ConcurrentLinkedQueue} agar setiap pembacaan dan penulisan \textit{entry} tidak mengalami \textit{race condition}.
      
      \item \texttt{repositories}: Atribut ini menyimpan seluruh URL unik yang diperiksa selama proses \textit{crawling} dan digunakan untuk memastikan tidak ada duplikasi dalam pemeriksaan. Struktur data yang digunakan adalah \texttt{Map<String, Link>}, dengan \textit{key} berupa URL dan \textit{value} berupa objek \texttt{Link} yang mewakili URL tersebut. Implementasinya menggunakan \texttt{ConcurrentHashMap} agar setiap pembacaan dan penulisan \textit{entry} tidak mengalami \textit{race condition}.
      
      \item \texttt{rateLimiters}: Atribut ini menyimpan daftar objek \texttt{RateLimiter} yang digunakan untuk membatasi frekuensi permintaan HTTP ke setiap \textit{host}. Struktur data yang digunakan adalah \texttt{Map<String, RateLimiter>}, dengan \textit{key} berupa \textit{host} URL dan \textit{value} berupa objek \texttt{RateLimiter} yang terkait dengan \textit{host} tersebut. Implementasinya menggunakan \texttt{ConcurrentHashMap} agar setiap pembacaan dan penulisan \textit{entry} tidak mengalami \textit{race condition}.
      
      \item \texttt{receiver}: Atribut ini menyimpan objek dari antarmuka \texttt{LinkReceiver} dan digunakan untuk mengirim tautan yang ditemukan selama proses pemeriksaan ke kelas \texttt{MainController} yang melakukan \textit{implement} terhadap antarmuka tersebut.

      \item \texttt{isStopped}: Atribut ini merupakan penanda untuk mengetahui apakah proses \textit{crawling} sedang berlangsung atau telah dihentikan oleh pengguna.

      \item \texttt{rootHost}: Atribut ini menyimpan \textit{host} dari URL awal yang akan digunakan untuk menentukan apakah sebuah tautan adalah tautan halaman situs web atau bukan.
      
      \item \texttt{MAX\_LINKS}: Atribut ini menyimpan batas maksimal tautan yang boleh diperiksa untuk mencegah proses pemeriksaan yang terlalu lama pada situs web yang memiliki tautan yang banyak.

   \end{enumerate}

   \vspace{5mm}

   Metode \texttt{start} merupakan metode utama dari kelas \texttt{Crawler} dan digunakan untuk menjalankan proses \textit{crawling}. Masukan metode ini adalah URL yang menjadi titik awal proses \texttt{crawling}. Berikut adalah penjelasan lebih rinci untuk metode \texttt{start} (lihat Kode~\ref{code:crawler-start-java}):


   \begin{enumerate}[itemsep=2pt]
      \item (Baris 2--5) Mengatur ulang status penghentian dan membersihkan seluruh struktur data internal agar proses \textit{crawling} dimulai dalam keadaan bersih.
      
      \item (Baris 6--7) Jika \textit{executor service} masih ada maka hentikan semua \textit{thread} lama, lalu Membuat \textit{executor service} baru berbasis \textit{virtual thread} untuk menjalankan pemeriksaan tautan eksternal secara pararel.
      
      \item (Baris 8--9) Mendapatkan \textit{host} dari URL awal, lalu masukkannya ke dalam antrean \texttt{frontier} sebagai titik awal pemeriksaan.
      
      \item (Baris 10) Menjalankan \textit{looping} untuk \textit{crawling} halaman situs web. \textit{Loop} akan terus berlangsung selama \texttt{frontier} belum kosong, tidak dihentikan oleh pengguna dan jumlah tautan belum melebihi batas.
      
      \item (Baris 11--12) Mengambil elemen pertama dari antrean \texttt{frontier}. Jika nilai yang diambil adalah \texttt{null}, maka proses dihentikan.
      
      \item (Baris 14) Melakukan pemeriksaan terhadap tautan saat ini dan mengirim perintah agar dilakukan \textit{parsing} pada \textit{response body} untuk mendapatkan dokumen HTML.
      
      \item (Baris 16) Jika tautan saat ini bukan merupakan tautan halaman situs web atau dokumen HTML kosong, maka lanjutkan ke iterasi berikutnya.
      
      \item (Baris 18) Melakukan ekstraksi seluruh tautan yang terdapat pada tautan halaman.
      
      \item (Baris 20) Membuat daftar tugas yang akan dijalankan secara pararel untuk memeriksa tautan.
      
      \item (Baris 21) Melakukan \textit{looping} terhadap setiap tautan hasil ekstraksi.
      
      \item (Baris 22) Jika proses dihentikan oleh pengguna maka keluar dari metode.
      
      \item (Baris 24-25) Mengambil objek \texttt{Link} hasil ekstraksi beserta teksnya.
      
      \item (Baris 27--31) Jika tautan sudah pernah dikunjungi, maka tambahkan relasi antara tautan yang lama dengan tautan halaman saat ini, lalu lanjutkan ke iterasi berikutnya.
      
      \item (Baris 32) Tambahkan tautan halaman saat ini sebagai sumber halaman dari tautan.
      
      \item (Baris 34--35) Jika tautan memiliki \textit{host} yang sama dengan \texttt{rootHost} maka ditambahkan ke dalam \texttt{frontier} karena tautan tersebut berpotensi menjadi tautan halaman.
      
      \item (Baris 37) Menambahkan tugas baru yang akan dieksekusi secara pararel.
      
      \item (Baris 38--39) Melakukan pemeriksaan terhadap tautan dan mengirim perintah agar tidak melakukan \textit{parsing} pada \textit{response body}. Setelah pemeriksaan selesai, keluar dari tugas.
      
      \item (Baris 43-45) Jika ada tugas, maka jalankan seluruh tugas tersebut secara paralel dan menunggu sampai semuanya selesai sebelum melanjutkan halaman berikutnya.
      
      \item (Baris 46--47) Jika terjadi \textit{interrupt} pada \textit{thread} yang sedang berjalan, maka tandai kembali \textit{thread} tersebut sudah di hentikan.
   \end{enumerate}

   \vspace{15mm}

   \lstinputlisting[
      language=Java, 
      caption=Kelas \texttt{Crawler} metode \texttt{start},
      label={code:crawler-start-java}
   ]{Lampiran/implementasi-kode-program/Crawler_start.java}
   
   Metode \texttt{checkLink} digunakan untuk memeriksa tautan dengan cara melakukan permintaan HTTP lalu memperbarui atribut dari objek tautan, serta melakukan \textit{parsing} terhadap \textit{response body} menjadi dokumen HTML. Masukan metode ini adalah objek \texttt{Link} yang atributnya akan diperbarui dan sebuah penanda untuk melakukan \textit{parsing}. Keluaran dari metode ini dokumen HTML jika dilakukan \textit{parsing} dan \texttt{null} jika tidak. Berikut adalah penjelasan lebih rinci untuk metode \texttt{checkLink} (lihat Kode~\ref{code:crawler-check-link-java}):

   \begin{enumerate}[itemsep=4pt]

      \item (Baris 2--3) Jika tautan sudah pernah diperiksa atau jumlah tautan keseluruhan sudah melebihi batas, maka keluar dari metode.
      
      \item (Baris 6) Mengambil atau membuat objek \texttt{RateLimiter} berdasarkan \textit{host} dari URL tautan. Sehingga penerapan \textit{rate limiting} dapat dilakukan per \textit{host}.
      
      \item (Baris 7) Menerapkan \textit{rate limiting} per \textit{host}.
      
      \item (Baris 9) Membuat objek \texttt{HttpRequest} untuk melakukan permintaan HTTP.
      
      \item (Baris 11) Mempersiapkan variabel untuk menyimpan hasil permintaan HTTP.
      
      \item (Baris 12--13) Jika tidak diminta untuk \textit{parsing} maka lakukan permintaan HTTP tanpa mengambil \textit{request body}.
      
      \item (Baris 14--15) Jika diminta untuk \textit{parsing} maka lakukan permintaan HTTP dengan mengambil \textit{request body}.
      
      \item (Baris 18--20) Memperbarui nilai dari atribut objek \texttt{Link}, seperti \textit{final} URL, tipe konten, kode status HTTP.
      
      \item (Baris 23) Menetapkan syarat \textit{parsing}. Jika permintaan HTTP berhasil dan \textit{response body} tidak kosong.
      
      \item (Baris 24) Menetapkan syarat \textit{parsing}. Jika \textit{final} URL memiliki \textit{host} yang sama dengan \textit{host} dari URL awal.
      
      \item (Baris 26) Jika diminta untuk \textit{parsing} dan semua syarat \textit{parsing} terpenuhi.
      
      \item (Baris 27) Mengambil \textit{response body} dalam bentuk \texttt{String} agar bisa dilakukan \textit{parsing}.

      \item (Baris 29) Melakukan \textit{parsing} \textit{response body} menjadi objek \texttt{Document} HTML menggunakan pustaka \texttt{Jsoup}.
      
      \item (Baris 30) Jika \textit{parsing} berhasil maka buat tautan saat ini menjadi tautan halaman.
      
      \item (Baris 32) Jika \textit{parsing} gagal, nilai \texttt{html} tetap \texttt{null}.
      
      \item (Baris 35) Mengembalikan dokumen HTML hasil \textit{parsing}.
      
      \item (Baris 37) Jika terjadi kesalahan dalam proses permintaan HTTP, maka simpan nama tipe kesalahan ke atribut \texttt{error} pada objek \texttt{Link}.
      
      \item (Baris 38) Kembalikan \texttt{null} jika gagal.
      
      \item (Baris 40) Tambahkan tautan yang sudah diperiksa ke \texttt{repositories}.
      
      \item (Baris 41--43) Jika tautan baru berhasil ditambahkan ke \texttt{repositories}, maka kirim tautan ke \texttt{MainController} untuk ditampilkan ke GUI.
      
   \end{enumerate}

   \vspace{12mm}

   \lstinputlisting[
      language=Java, 
      caption=Kelas \texttt{Crawler} metode \texttt{checkLink},
      label={code:crawler-check-link-java}
   ]{Lampiran/implementasi-kode-program/Crawler_checkLink.java}

   % \vspace{12mm}

   Metode \texttt{extractLink} digunakan untuk mendapatkan seluruh tautan yang diambil dari \textit{tag} \texttt{<a>} yang memiliki atribut \texttt{href} pada sebuah dokumen HTML. Masukan metode ini adalah dokumen HTML yang akan diekstrak tautannya, sedangkan keluaran dari metode ini adalah sebuah daftar tautan yang dipetakan dengan teks yang berada pada tautan tersebut. Berikut adalah penjelasan lebih rinci untuk metode \texttt{extractLink} (lihat Kode~\ref{code:crawler-extract-link-java}):
   
   \begin{enumerate}[itemsep=3pt]

      \item (Baris 2) Membuat objek \texttt{Map} untuk menyimpan hasil ekstraksi tautan, dengan \textit{key} berupa objek \texttt{Link} dan \textit{value} berupa teks yang berada pada tautan tersebut.
      
      \item (Baris 3) Melakukan iterasi untuk setiap elemen HTML pada \textit{tag} \texttt{<a>} yang memiliki atribut \texttt{href}.
      
      \item (Baris 4) Mengambil URL dari atribut \texttt{href}, lalu ubah menjadi URL absolut menggunakan metode \texttt{absUrl} dari Jsoup.
      
      \item (Baris 6) Menormalisasi URL menggunakan metode \texttt{normalizeUrl} untuk memastikan URL konsisten secara format.
      
      \item (Baris 7) Lanjutkan ke iterasi berikutnya jika hasil normalisasi \texttt{null}, yang berarti URL tidak memenuhi aturan normalisasi.
      
      \item (Baris 9) Membuat objek \texttt{Link} baru berdasarkan URL yang telah dinormalisasi.
      
      \item (Baris 10) Mengambil teks yang berada di dalam \textit{tag} \texttt{<a>} dan menghapus spasi berlebih pada bagian ujung teks.
      
      \item (Baris 12) Jika objek \texttt{Link} saat ini belum ada di dalam map hasil maka tambahkan.
      
      \item (Baris 14) Mengembalikan seluruh tautan yang telah berhasil diekstraksi dari dokumen HTML.
      
   \end{enumerate}

   \lstinputlisting[
      language=Java, 
      caption=Kelas \texttt{Crawler} metode \texttt{extractLink},
      label={code:crawler-extract-link-java}
   ]{Lampiran/implementasi-kode-program/Crawler_extractLink.java}

   \vspace{12mm}

   \item \textbf{Berkas \texttt{Exporter.java}}\\
   Kelas \texttt{Exporter} bertugas untuk melakukan ekspor hasil pemeriksaan ke berkas Excel (\texttt{xlsx}). Kode lengkap untuk kelas ini terdapat pada Lampiran~\ref{code:exporter-java}, namun pada bagian ini akan dijelaskan lebih rinci untuk metode \texttt{save}, yaitu metode utama dari kelas ini yang digunakan untuk mempersiapkan data sebelum diekspor dan menyimpannya ke berkas lokal komputer. Berikut adalah penjelasan lebih rinci untuk metode \texttt{save} (lihat Kode~\ref{code:exporter-save-java}):

   \begin{enumerate}[itemsep=3pt]
      
      \item (Baris 2) Membuat objek \texttt{Workbook} menggunakan \texttt{XSSFWorkbook} agar bisa menulis berkas dengan format \texttt{xlsx}.
      
      \item (Baris 4--8) Membuat \textit{styling} untuk masing-masing jenis baris pada tabel. Baris ganjil dan genap dibuat memiliki \textit{styling} yang berbeda untuk meningkatkan keterbacaan tabel.
      
      \item (Baris 10) Membuat \textit{sheet} untuk menyimpan data ringkasan.

      \item (Baris 11) Menulis tabel ringkasan proses pemeriksaan ke dalam \textit{sheet} khusus ringkasan.
      
      \item (Baris 12) Menulis tabel ringkasan tautan rusak ke dalam \textit{sheet} khusus ringkasan.
      
      \item (Baris 14) Membuat \textit{sheet} untuk menyimpan data tautan rusak.
      
      \item (Baris 15) Menulis tabel tautan rusak ke dalam \textit{sheet} khusus tautan rusak.
      
      \item (Baris 17--19) Menulis data Excel ke berkas lokal yang sudah dipilih oleh pengguna.
      
   \end{enumerate}

   \vspace{5mm}

   \lstinputlisting[
      language=Java, 
      caption=Kelas \texttt{Exporter} metode \texttt{save},
      label={code:exporter-save-java}
   ]{Lampiran/implementasi-kode-program/Exporter_save.java}

   \vspace{10mm}

   \item \textbf{Berkas \texttt{RateLimiter.java}}\\
   Kelas \texttt{RateLimiter} bertugas membatasi frekuensi permintaan HTTP agar pemeriksaan tidak dilakukan terlalu cepat dan tidak membebani \textit{server} situs web tujuan. Kelas ini memiliki dua atribut, yaitu atribut \texttt{INTERVAL} yang digunakan untuk mengatur jarak antar permintaan HTTP dan atribut \texttt{lastRequestTime} yang digunakan untuk menyimpan waktu terakhir permintaan HTTP dilakukan. 

   Atribut \texttt{lastRequestTime} dideklarasikan menggunakan \texttt{volatile} untuk memastikan setiap \textit{thread} yang membaca dan menulis nilai atribut ini secara langsung ke memori utama, bukan dari \texttt{cache} CPU masing-masing. Dengan demikian, perubahan nilai yang dilakukan oleh satu \textit{thread} selalu terlihat oleh \textit{thread} lain, sehingga menghindari inkonsistensi saat kelas ini digunakan dalam lingkungan \textit{multi-thread}.

   Kode lengkap untuk kelas ini terdapat pada Lampiran~\ref{code:rate-limiter-java}, namun pada bagian ini akan dijelaskan secara rinci untuk metode \texttt{delay}. Metode \texttt{delay} digunakan untuk mengatur jeda minimal antar permintaan HTTP. Prinsip kerjanya adalah membandingkan waktu saat ini dengan waktu permintaan sebelumnya, kemudian menunda eksekusi jika jeda antar permintaan belum mencapai nilai \texttt{INTERVAL}. Agar tidak terjadi \textit{race condition}, maka metode ini didefinisikan menggunakan \texttt{synchronized}, untuk memastikan hanya satu \textit{thread} yang dapat menjalankan metode ini pada satu waktu. Berikut adalah penjelasan lebih rinci untuk metode \texttt{delay} (lihat Kode~\ref{code:rate-limiter-delay-java}):
   
   \begin{enumerate}[itemsep=3pt]
      \item (Baris 2) Mengambil waktu saat ini dalam satuan milidetik.

      \item (Baris 3) Menghitung waktu tunggu, jika hasilnya bernilai positif, artinya jeda antar permintaan sebelumnya belum terpenuhi.

      \item (Baris 5-7) Jika \texttt{waitTime} bernilai positif, maka eksekusi \textit{thread} dihentikan sementara selama \texttt{waitTime} milidetik.

      \item (Baris 13) Memperbarui nilai \texttt{lastRequestTime} dengan waktu saat ini setelah jeda selesai, sehingga permintaan HTTP berikutnya dapat dihitung jedanya berdasarkan nilai terbaru.
   \end{enumerate}

   \vspace{5mm}

   \lstinputlisting[
      language=Java, 
      caption=Kelas \texttt{RateLimiter} metode \texttt{delay},
      label={code:rate-limiter-delay-java}
   ]{Lampiran/implementasi-kode-program/RateLimiter_delay.java}

   \vspace{5mm}

   \item \textbf{Berkas \texttt{UrlHandler.java}}\\
   Kelas \texttt{UrlHandler} bertugas untuk menyediakan metode untuk menangani segala kebutuhan terkait URL. Pada kelas ini terdapat tiga metode, yaitu \texttt{normalizeUrl} untuk menetapkan aturan pada URL yang ditangani, metode \texttt{normalizePath} untuk menetapkan aturan pada \textit{path} dari URL, dan metode \texttt{getHost} untuk mendapatkan \textit{host} dari URL. Kode lengkap untuk kelas ini terdapat pada Lampiran~\ref{code:url-handler-java}, namun pada bagian ini yang akan dijelaskan lebih rinci adalah metode \texttt{normalizeUrl}.

   Metode \texttt{normalizeUrl} menerima dua masukan, yaitu \texttt{rawUrl} yang merupakan URL target normalisasi dan \texttt{isStrict} untuk menandakan apakah normalisasi dilakukan secara ketat atau tidak. Kembalian dari metode ini memiliki tiga kemungkinan bentuk, yaitu URL awal apabila tidak valid dibentuk sebagai objek URI, \texttt{null} apabila URL tidak memenuhi aturan normalisasi, dan URL baru yang sudah dinormalisasi jika seluruh aturan berhasil diterapkan. Berikut adalah penjelasan lebih rinci untuk metode \texttt{normalizeUrl} (lihat Kode~\ref{code:url-handler-normalize-url-java}):

   \vspace{2mm}
   
   \begin{enumerate}[itemsep=4pt]

      \item (Baris 2--4) Melakukan pemeriksaan awal terhadap nilai masukan. Jika bernilai \texttt{null} atau hanya berisi spasi, maka URL dianggap tidak valid dan kembalikan \texttt{null}.

      \item (Baris 7) Mencoba membentuk objek \texttt{URI} dari URL masukan, jika gagal maka akan terlempar ke \textit{exception}.

      \item (Baris 9--17) Mengambil komponen-komponen URL dari objek \texttt{URI} yang sudah dibuat, seperti \texttt{scheme}, \texttt{host}, \texttt{port}, \texttt{path}, dan \texttt{query}, sedangkan \texttt{userInfo} dan \texttt{fragment} karena tidak dibutuhkan.
      
      \item (Baris 19--26) Jika normalisasi diminta untuk dilakukan secara ketat, maka \texttt{scheme} dan \texttt{host} tidak boleh \texttt{null} atau \texttt{String} kosong.

      \item (Baris 28--31) Melakukan validasi terhadap skema URL. Jika skema bukan \texttt{http} atau \texttt{https}, maka URL dianggap tidak valid dan kembalikan \texttt{null}. Tindakan ini dilakukan untuk menghindari \textit{false positive}, karena tautan dengan skema ini tidak dapat diperiksa melalui protokol HTTP dan akan mengembalikan \textit{error} jika dilakukan pemeriksaan menggunakan \texttt{HttpClient}.

      \item (Baris 33--36) Menghapus port standar dari URL dengan mengubah nilainya menjadi tidak valid.

      \item (Baris 38) Menormalisasi nilai \texttt{path} menggunakan metode \texttt{normalizePath}, untuk menghilangkan \textit{dot-segment} dan duplikasi garis miring.

      \item (Baris 40) Membentuk kembali objek \texttt{URI} baru. Komponen \texttt{userInfo} dan \texttt{fragment} tidak disertakan karena keduanya tidak diperlukan dalam proses pemeriksaan tautan.

      \item (Baris 42) Mengembalikan URL hasil normalisasi dalam format ASCII untuk menghindari karakter non-ASCII yang tidak kompatibel dengan \texttt{HttpClient}.

      \item (Baris 44--46) Jika terjadi \textit{error} maka kembalikan URL awal sehingga dapat teridentifikasi sebagai tautan rusak pada tahap \textit{fetching}. Tindakan ini dilakukan untuk menghindari \textit{false negative}, karena jika dikembalikan \texttt{null} maka tautan ini tidak akan diperiksa.
      
   \end{enumerate}

   \vspace{20mm}

   \lstinputlisting[
      language=Java, 
      caption=Kelas \texttt{UrlHandler} metode \texttt{normalizeUrl},
      label={code:url-handler-normalize-url-java}
   ]{Lampiran/implementasi-kode-program/UrlHandler_normalizeUrl.java}

   \vspace{10mm}

   \item \textbf{Berkas \texttt{ErrorHandler.java}}\\
   Kelas \texttt{ErrorHandler} bertugas untuk menyediakan metode untuk menangani segala kebutuhan terkait pesan kesalahan pada kode program. Pada kelas ini terdapat tiga metode, yaitu \texttt{getHttpError} untuk mendapatkan pesan kesalahan berdasarkan kode status HTTP, metode \texttt{getExceptionError} untuk mendapatkan pesan kesalahan dari \textit{exceoption} yang terjadi pada kode program, dan metode \texttt{isHttpError} untuk menentukan apakah sebuah bilangan bulat merupakan kode status HTTP yang dikategorikan sebagai \textit{error}. Kode lengkap untuk kelas ini terdapat pada Lampiran~\ref{code:error-handler-java}.

   \vspace{1mm}

   \item \textbf{Berkas \texttt{Link.java}}\\
   Kelas \texttt{Link} merupakan model yang digunakan untuk merepresentasikan tautan yang ditemukan selama proses pemeriksaan. Atribut dari kelas ini terdiri atas \texttt{url} untuk menyimpan URL awal, \texttt{finalUrl} untuk menyimpan URL hasil \textit{redirect}, \texttt{statusCode} untuk menyimpan kode status HTTP, \texttt{contentType} untuk menyimpan jenis konten yang dikembalikan URL, \texttt{error} untuk menyimpan pesan \textit{error}, \texttt{isWebpage} untuk menentukan apakah URL merupakan halaman situs web atau bukan, serta \texttt{webpageSources} untuk menyimpan daftar tautan halaman yang menjadi sumber dari tautan saat ini. Seluruh atribut pada kelas ini kecuali \texttt{webpageSources}, didefinisikan menggunakan JavaFX \textit{properties} agar nilai pada atribut dapat diperbarui secara otomatis di antarmuka. Kode lengkap untuk kelas ini terdapat pada Lampiran~\ref{code:link-java}.

   \vspace{1mm}

   \item \textbf{Berkas \texttt{Summary.java}}\\
   Kelas \texttt{Summary} merupakan model yang digunakan untuk merepresentasikan data ringkasan dari proses pemeriksaan yang akan ditampilkan pada bagian \textit{Summary} di jendela utama. Data tersebut dikonversi menjadi atribut kelas yang mencakup jumlah total tautan, jumlah tautan halaman, jumlah tautan rusak, serta status proses pemeriksaan. Atribut pada kelas ini didefinisikan menggunakan JavaFX \textit{properties} agar nilai pada atribut dapat diperbarui secara otomatis di antarmuka. Kode lengkap untuk kelas ini terdapat pada Lampiran~\ref{code:summary-java}.

   \vspace{1mm}
   
   \item \textbf{Berkas \texttt{Status.java}}\\
   Enumerasi \texttt{Status} merupakan model yang digunakan untuk mendefinisikan status dari proses pemeriksaan, yaitu \texttt{IDLE} untuk keadaan diam, \texttt{CHECKING} untuk keadaan sedang dalam proses pemeriksaan, \texttt{STOPPED} untuk keadaan proses dihentikan oleh pengguna, dan \texttt{COMPLETED} untuk keadaan proses selesai. Enumerasi ini digunakan oleh kelas \texttt{Summary} dan kelas \texttt{MainController} untuk menampilkan status proses kepada pengguna. Kode lengkap untuk enumerasi ini terdapat pada Lampiran~\ref{code:status-java}.
   
   \vspace{1mm}
   
   \item \textbf{Berkas \texttt{LinkReceiver.java}}\\
   Antarmuka \texttt{LinkReceiver} digunakan untuk melakukan mekanisme komunikasi satu arah dari kelas yang menggunakan antarmuka ini menuju kelas lain yang melakukan \textit{implement} terhadap antarmuka ini. Kelas yang melakukan \textit{implement} terhadap antarmuka ini diwajibkan untuk melakukan \textit{override} terhadap metode \texttt{receive}, yaitu metode yang menerima masukan berupa objek \texttt{Link} dan tidak mengembalikan nilai apapun. Kode lengkap untuk antarmuka ini terdapat pada Lampiran~\ref{code:link-receiver-java}.
   
\end{enumerate}
