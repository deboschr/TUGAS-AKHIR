Bagian ini memaparkan hasil implementasi antamuka pengguna berdasarkan perancangan yang telah dibuat pada perancangan antarmuka (lihat Subbab~\ref{sec:040300-perancangan-antarmuka}). Antarmuka pengguna dikembangkan menggunakan JavaFX dengan struktur tampilan yang diatur melalui berkas FXML dan gaya visual yang diatur melalui berkas CSS. Berikut adalah penjelasan dari setiap hasil implementasi antarmuka pengguna:

\begin{enumerate}
    \item \textbf{Jendela Utama}\\
    Hasil implementasi jendela utama aplikasi dapat dilihat pada Gambar~\ref{fig:implementasi-jendela-utama}, implementasi ini didasarkan pada perancangan jendela utama yang telah dibuat sebelumnya (lihat Gambar~\ref{fig:rancangan-antarmuka-jendela-utama}). Struktur tampilan dari jendela ini diatur melalui berkas \texttt{main-scene.fxml} (lihat Lampiran~\ref{code:main-scene-fxml}), gaya visualnya diatur melalui berkas \texttt{main-style.css} (lihat Lampiran~\ref{code:main-style-css}), serta logika pengendali antarmuka diatur melalui kelas \texttt{MainController.java} (lihat Lampiran~\ref{code:main-controller-java}).
    
    Seluruh rancangan komponen antamuka jendela utama telah berhasil diimplementasi pada jendela ini. Pengguna dapat memasukan URL situs web yang ingin diperiksa, memulai serta menghentikan proses pemeriksaan. Hasil pemeriksaa ditampilkan dalam bentuk tabel pada bagian \textit{Result} dengan ringkasannya yang ditampilkan pada bagian \textit{Summary}. Selain itu, pengguna juga dapat melakukan penyaringan terhadap hasil pemeriksaan dan menyimpan hasil pemeriksaan ke berkas lokal.

    \begin{figure}[H]
        \centering
        \includegraphics[width=0.85\textwidth]{Gambar/050104-implementasi-jendela-utama.png}
        \caption{Implementasi Antarmuka Jendela Utama}
        \label{fig:implementasi-jendela-utama}
    \end{figure}

    \item \textbf{Jendela Detail Tautan}\\
    Hasil implementasi jendela detail tautan dapat dilihat pada Gambar~\ref{fig:implementasi-jendela-detail-tautan}, implementasi ini didasarkan pada perancangan jendela detail tautan yang telah dibuat sebelumnya (lihat Gambar~\ref{fig:rancangan-antarmuka-jendela-detail-tautan}). Struktur tampilan dari jendela ini diatur melalui berkas \texttt{link-scene.fxml} (lihat Lampiran~\ref{code:link-scene-fxml}), gaya visualnya diatur melalui berkas \texttt{link-style.css} (lihat Lampiran~\ref{code:link-style-css}), serta logika pengendali antarmuka diatur melalui kelas \texttt{LinkController.java} (lihat Lampiran~\ref{code:link-controller-java}).

    Seluruh rancangan komponen antamuka jendela detail tautan telah berhasil diimplementasi pada jendela ini, meskipun terdapat beberapa perbedaan dalam gaya visual. Pada jendela ini ditampilkan informasi lengkap dari sebuah tautan yaitu, URL, \textit{final} URL, \textit{content type}, \textit{error} serta daftar sumber halaman dimana tautan ditemukan yang ditampilkan dalam bentuk tabel.

    \begin{figure}[H]
        \centering
        \includegraphics[width=0.85\textwidth]{Gambar/050104-implementasi-jendela-detail-tautan.png}
        \caption{Implementasi Antarmuka Jendela Detail Tautan}
        \label{fig:implementasi-jendela-detail-tautan}
    \end{figure}

    \item \textbf{Jendela Notifikasi}\\
    Jendela notifikasi digunakan untuk menampilkan pesan singkat kepada pengguna. Pesan yang ditampilkan dikategorikan menjadi empat jenis pesan, yaitu pesan \textit{error}, peringatan, dan berhasil. Hasil implementasi jendela notifikasi dapat dilihat pada Gambar~\ref{fig:implementasi-jendela-notifikasi}. Implementasi ini tidak didasarkan pada rancangan antarmuka karena kebutuhan terhadap jendela ini baru teridentifikasi pada tahap implementasi. Struktur tampilan dari jendela ini diatur melalui berkas \texttt{notification-scene.fxml} (lihat Lampiran~\ref{code:notification-scene-fxml}), gaya visualnya diatur melalui berkas \texttt{notification-style.css} (lihat Lampiran~\ref{code:notification-style-css}), serta logika pengendali antarmuka diatur melalui kelas \texttt{NotificationController.java} (lihat Lampiran~\ref{code:notification-controller-java}).

    \begin{figure}[H]
        \centering
        \includegraphics[width=0.85\textwidth]{Gambar/050104-implementasi-jendela-notifikasi.png}
        \caption{Implementasi Antarmuka Jendela Notifikasi}
        \label{fig:implementasi-jendela-notifikasi}
    \end{figure}
\end{enumerate}
