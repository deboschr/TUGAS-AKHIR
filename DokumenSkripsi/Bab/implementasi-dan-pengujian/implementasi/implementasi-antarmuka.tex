Bagian ini memaparkan hasil implementasi antamuka pengguna berdasarkan perancangan yang telah dibuat pada perancangan antarmuka (lihat Subbab~\ref{sec:040300-perancangan-antarmuka}). Antarmuka pengguna dikembangkan menggunakan JavaFX dengan struktur tampilan yang diatur melalui berkas \texttt{FXML} dan gaya visual yang diatur melalui berkas \texttt{CSS}. Berikut merupakan hasil implementasi antarmuka pengguna yang terbagi menjadi tiga jendela aplikasi.

\begin{enumerate}
    \item \textbf{Jendela Utama}\\
    Hasil impelemtasi jendela utama aplikasi dapat dilihat pada Gambar~\ref{fig:implementasi-jendela-utama}, implementasi ini didasarkan pada perancangan jendela utama yang telah dilakukan sebelumnya (lihat Gambar~\ref{fig:rancangan-antarmuka-jendela-utama}).
    Struktur tampilan dari jendela ini diatur melalui berkas \texttt{main-window.fxml} (lihat Lampiran~\ref{code:main-fxml}) dan gaya visualnya diatur melalui berkas \texttt{main-window.css} (lihat Lampiran~\ref{code:main-css}). 
    
    Seluruh komponen antamuka jendela utama yang telah dirancang sebelumnya telah berhasil diimplementasi, pengguna dapat memasukan URL situs web yang ingin diperiksa, memulai dan menghentikan proses pemerik

    
    Jendela ini merupakan pusat interaksi pengguna dalam melakukan pemeriksaan tautan rusak pada situs web. Melalui jendela ini, pengguna dapat memasukan URL situs web yang ingin diperiksa, memulai serta menghentikan proses pemeriksaan. Hasil pemeriksaa ditampilkan dalam bentuk tabel pada bagian \textit{Result} dengan ringkasannya yang ditampilkan pada bagian \textit{Summary}. Pada bagian \textit{Filters} disediakan fitur untuk melakukan penyaringan hasil pemerisaan berdasarkan URL dan kode status HTTP, serta disediakan pula fitur untuk menyimpan hasil pemeriksaan ke berkas lokal dengan cara menekan tombol \textit{Export}.

    \begin{figure}[H]
        \centering
        \includegraphics[width=0.85\textwidth]{Gambar/050104-implementasi-jendela-utama.png}
        \caption{Implementasi Antarmuka Jendela Utama}
        \label{fig:implementasi-jendela-utama}
    \end{figure}

    \item \textbf{Jendela Detail Tautan}\\
    Jendela detail tautan digunakan untuk menampilkan informasi lengkap mengenai satu tautan rusak tertentu yang dipilih dari tabel hasil pemeriksaan pada jendela utama. 
    Jendela ini menampilkan atribut \texttt{URL}, \texttt{Final URL}, \texttt{Content Type}, dan \texttt{Error} untuk menunjukkan hasil pemeriksaan tautan yang bersangkutan. 
    Selain itu, tabel pada bagian bawah jendela menampilkan daftar halaman asal (\textit{webpage URL}) tempat tautan tersebut ditemukan beserta dengan teks yang menjadi penanda tautan tersebut pada sumber halaman (\textit{anchor text}). 
    Implementasi dari jendela detail tautan ditunjukan pada Gambar~\ref{fig:implementasi-jendela-detail-tautan}, implementasi ini didasarkan pada perancangan yang telah dibuat berdasarkan Gambar~\ref{fig:rancangan-antarmuka-jendela-detail-tautan}.

    \begin{figure}[H]
        \centering
        \includegraphics[width=0.85\textwidth]{Gambar/050104-implementasi-jendela-detail-tautan.png}
        \caption{Implementasi Antarmuka Jendela Detail Tautan}
        \label{fig:implementasi-jendela-detail-tautan}
    \end{figure}

    \item \textbf{Jendela Notifikasi}\\
\end{enumerate}
