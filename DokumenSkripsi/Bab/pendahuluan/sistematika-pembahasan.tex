Sistematika pembahasan dalam penelitian ini disusun sebagai berikut:

\begin{enumerate}
    \item Bab 1 Pendahuluan \\
    Bab ini berisi latar belakang dilakukannya penelitian, rumusan masalah, tujuan penelitian, batasan masalah, metodologi yang digunakan, serta sistematika pembahasan.

    \item Bab 2 Landasan Teori \\
    Bab ini membahas teori-teori yang menjadi dasar dalam pengembangan perangkat lunak pemeriksa tautan rusak pada situs web. Pembahasan mencakup protokol HTTP, URI, HTML, \textit{web crawling}, serta teknologi yang digunakan seperti JavaFX, Jsoup, Java HTTP Client API, dan Apache POI.

    \item Bab 3 Analisis \\
    Bab ini membahas analisis terhadap permasalahan dalam pemeriksaan tautan rusak pada situs web, peninjauan perangkat lunak serupa, perumusan kebutuhan fungsional dan non-fungsional, serta analisis teknologi yang digunakan dalam pengembangan perangkat lunak.

    \item Bab 4 Perancangan \\
    Bab ini membahas perancangan perangkat lunak berdasarkan hasil analisis yang telah dilakukan. Perancangan yang dilakukan meliputi perancangan kelas, perancangan alur, dan perancangan antarmuka pengguna.

    \item Bab 5 Implementasi dan Pengujian \\
    Bab ini memaparkan hasil implementasi perangkat lunak pemeriksa tautan rusak berdasarkan perancangan yang telah dilakukan. Selain itu, bab ini juga memaparkan hasil pengujian fungsional dan eksperimental serta analisis terhadap hasil pengujian.
    
    \item Bab 6 Kesimpulan dan Saran \\
    Bab ini berisi kesimpulan dari hasil pengembangan perangkat lunak pemeriksa tautan rusak pada situs web, serta saran pengembangan lebih lanjut yang didasarkan pada analisis hasil pengujian yang telah dilakukan.
\end{enumerate}

