Situs web merupakan salah satu sarana utama dalam penyebaran informasi dan representasi identitas institusi di era digital. Dalam berbagai sektor, termasuk perguruan tinggi, situs web digunakan sebagai kanal resmi untuk menyampaikan informasi secara luas. Perguruan tinggi memanfaatkan situs web untuk menyediakan beragam konten seperti kalender akademik, publikasi ilmiah, dokumentasi kegiatan kampus, serta informasi administratif lainnya. Situs web juga menjadi titik awal interaksi antara institusi dengan calon mahasiswa, mitra kerja, dan masyarakat umum.


Pentingnya situs web perguruan tinggi juga tercermin dengan adanya sistem pemeringkatan non-akademis yang menjadikan situs web resmi perguruan tinggi sebagai objek penilaian. Webometrics\footnote{\url{https://webometrics.info} (Diakses pada 7 April 2025)} adalah sistem pemeringkatan yang menilai kehadiran dan visibilitas situs web akademik di internet sebagai cerminan dari kinerja dan dampak institusi dalam ranah digital. Namun, sejak April 2025 laman resmi Webometrics sudah tidak lagi dapat diakses. Sementara itu, uniRank\footnote{\url{https://www.unirank.org/about} (Diakses pada 20 Juli 2025)} (University Ranking) adalah sistem pemeringkatan yang berfokus pada popularitas dan kehadiran daring situs web perguruan tinggi. Penilaian pada uniRank menggunakan empat metrik yang diperoleh dari tiga sumber \textit{web intelligence} independen, yaitu Majestic, SimilarWeb, dan MOZ. Metrik dengan bobot tertinggi adalah metrik \textit{Referring Domains} dari Majestic dengan bobot 55\% yang mengukur jumlah domain eksternal unik yang menautkan ke situs web perguruan tinggi, serta metrik \textit{Global Rank} dari SimilarWeb dengan bobot penilaian 35\% yang merepresentasikan tingkat popularitas berdasarkan estimasi trafik global pengunjung situs web. Kedua metrik tersebut menunjukkan bahwa aspek aksesibilitas halaman yang dapat ditautkan dan aspek kelancaran kunjungan pengguna memiliki peran penting dalam menentukan peringkat institusi perguruan tinggi pada uniRank.

Namun, kedua aspek tersebut dapat terganggu oleh keberadaan tautan rusak pada situs web. Jika suatu halaman tidak dapat diakses karena tautan yang mengarah ke halaman tersebut rusak, maka domain eksternal lain akan kesulitan atau enggan untuk menautkan ke halaman tersebut, sehingga berpotensi menurunkan nilai \textit{Referring Domains}. Selain itu, tautan rusak juga dapat menghambat pengguna dalam mengakses atau menelusuri konten situs web, menyebabkan kunjungan gagal, penjelajahan terhenti, atau pengguna meninggalkan situs web lebih awal. Kondisi ini dapat menurunkan trafik kunjungan pada situs web yang menjadi dasar penilaian \textit{Global Rank}. Oleh karena itu, pemeriksaan tautan rusak pada situs web perlu dilakukan secara berkala untuk menjaga kesehatan situs dan peringkat dalam sistem pemeringkatan seperti uniRank.

Untuk mendeteksi tautan rusak pada sebuah situs web, terdapat dua pendekatan yang dapat digunakan. Pendekatan pertama adalah dengan melakukan pemeriksaan secara manual, namun pendekatan ini menjadi kurang efektif jika situs web memiliki ratusan bahkan ribuan halaman, karena setiap tautan harus diuji satu per satu. Pendekatan kedua adalah dengan memanfaatkan perangkat lunak pemeriksa tautan rusak yang tersedia secara daring seperti Broken Link Checker\footnote{\url{https://www.brokenlinkcheck.com} (Diakses pada 20 Juli 2025)}. Gambar~\ref{fig:contoh-broken-link-checker} menunjukkan hasil pemeriksaan menggunakan Broken Link Checker pada situs web Informatika UNPAR\footnote{\url{https://informatika.unpar.ac.id} (Diakses pada 20 Juli 2025)}. Hasil pemeriksaan tersebut ditampilkan dalam bentuk tabel yang berisi informasi seperti URL dari tautan rusak yang ditemukan, teks pada tautan tersebut, halaman sumber tautan tersebut ditemukan, serta kode status HTTP sebagai respons dari \textit{server}. 


\begin{figure}[H]
    \centering
    \includegraphics[width=1\textwidth]{Gambar/010100-broken-link-checker.png}
    \caption{Contoh Hasil Pemeriksaan Menggunakan Broken Link Checker}
    \label{fig:contoh-broken-link-checker}
\end{figure}


Dalam penelitian ini dikembangkan sebuah perangkat lunak berbasis desktop yang dapat digunakan untuk memeriksa keberadaan tautan rusak pada sebuah situs web. Perangkat lunak ini menerima sebuah masukan berupa URL situs web, kemudian melakukan penelusuran terhadap halaman pada situs web tersebut. Pada setiap halaman yang dikunjungi dilakukan ekstraksi tautan dan setiap tautan yang ditemukan kemudian diperiksa dengan melakukan permintaan HTTP. Respons dari \textit{server} dicatat dan dievaluasi untuk menentukan apakah tautan tersebut teridentifikasi sebagai tautan rusak atau tidak. Selanjutnya, setiap tautan rusak akan dikumpulkan dan disajikan sebagai hasil pemeriksaan. Untuk mempermudah pengguna dalam menjalankan proses pemeriksaan dan membaca hasil pemeriksaan, perangkat lunak ini dilengkapi dengan antarmuka pengguna grafis yang intuitif.